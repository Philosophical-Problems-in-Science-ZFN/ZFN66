% This file was converted to LaTeX by Writer2LaTeX ver. 1.6
% see http://writer2latex.sourceforge.net for more info
\documentclass[a4paper]{article}
\usepackage[utf8]{inputenc}
\usepackage[noenc]{tipa}
\usepackage{tipx}
\usepackage[geometry,weather,misc,clock]{ifsym}
\usepackage{pifont}
\usepackage{eurosym}
\usepackage{amsmath}
\usepackage{wasysym}
\usepackage{amssymb,amsfonts,textcomp}
\usepackage[T3,T1]{fontenc}
\usepackage[varg]{txfonts}\usepackage[english]{babel}
\usepackage{color}
\usepackage[top=1.251cm,bottom=1.499cm,left=2cm,right=2cm,includehead,head=0.654cm,headsep=0.554cm,includefoot,foot=0.501cm,footskip=0.901cm]{geometry}
\usepackage{array}
\usepackage{hhline}
\usepackage{hyperref}
\hypersetup{pdftex, colorlinks=true, linkcolor=blue, citecolor=blue, filecolor=blue, urlcolor=blue, pdftitle=}
% Footnote rule
\setlength{\skip\footins}{0.119cm}
\renewcommand\footnoterule{\vspace*{-0.018cm}\setlength\leftskip{0pt}\setlength\rightskip{0pt plus 1fil}\noindent\textcolor{black}{\rule{0.25\columnwidth}{0.018cm}}\vspace*{0.101cm}}
\title{}
\begin{document}
{\centering\bfseries\color{black}
 Parallel Worlds of ~Faith and Science~ in the Russian Intellectual Milieu
\par}

Obolevitch T., \textit{Faith and Science in Russian Religious Thought}, Oxford University Press, Oxford-New York 2019,
pp.240.

{\color{black}
The new book written by Sister Teresa Obolevich SBDNP carries on the details and nuances of the thoroughness of the
perception of the world by the orient society in the relation of science and religion. Admitted emphasis opens a door
to the peculiar development of the scientific and religious spheres as well as a tendency of accepting the world by the
Eastern society from the 10\textsuperscript{th}  century up to our times. So, the book opens the door for understanding
the specificity of development of the world by the mind which managed to fight with secular intentions. As author
shows, that the starting point of the outlook in the East was spirituality which through ages had improved its wise
nature and was ready to accept the philosophical tendencies along with scientific ones. Notably, that geographically
the most complete would be say that the orient or the Eastern Christian society here means the lands which in the
18\textsuperscript{th} century were encompassed by the Russian Empire but could be started as separated centers.}

{\color{black}
The unique clue of the book is that the author pays attention to the relationship of faith and science, not to faith and
reason. It is crucial to admit while the represented relation has a special stress, is actual and with precise nuances,
while former is more narrow than the later. Hence, this postulation describes the picture of how the scientists
contributed into their field keeping God as be a source of inspiration.}

{\color{black}
It worth noting that the author successfully selects the parts of the curriculum giving access to reach the full extent
of the possibility to enter to the relationship between two disciplines. A reader can admit that this linkage is not
monotonous and obtains a very different character with numerous interpretations and meanings, as well as that religious
attitude, still, which sometimes even reached fundamentalist approach. Another thing, it becomes clear from the text
the contrast of the developing of this approach in the West and the East.}

{\color{black}
In the represented orient world the author pays most attention to two eras: the Enlightenment and the Silver Age each of
which got here a special signification. The former associated with the enlightenment of Christ (and here is clear
contradiction to Western enlightenment), the latter is admired by the new brief of philosophy as a pillar for
development of scientific view too.}

{\color{black}
During the history the interest in the relationship between science and religion in the East had been included in the
interest of researchers from different fields, church and religious figures, and experts in the natural sciences. Such
effort demonstrates that the Eastern world was open to the western influences, yet usually it accepted them through
spiritual filter. Even a “pure materialistic” position (for instance, represented by Tsiolkovsky) signs about a
like-God’s presence in the world. Generally speaking, any materialistic wind came from the secularized West mostly was
described as a quasi-religious. }

{\color{black}
Sister Teresa Obolevich establishes a few incentives of such cause grounded first of all on theological and consequently
philosophical background among which are Biblical recourses and patristic tradition, sacred art and literature and
which mainly originated\textbf{ }the Eastern thought. Taking them into consideration, thereafter a reader can find out
in the numerous proposed ideas and concepts the presence the controversy of St. Basil the Great and St. Gregory Palamas
formula which originally distinguishes\textbf{ }between God’s essence and the presence of His energies in the world.
What follows from this standpoint, is that the researchers in this case work not with just matter, but with the
expressions of God’s potentiality and through which it is able to recognize Him. Thus, field of science does not
contradict to the existence of Creator but through the acknowledgment of creation is opened as one of the ways to
recognize God. Therefore, science is a kind of common deal which could be also called as a cosmological liturgy.}

{\color{black}
Still, it would be a mistake to say about uniformity of the outlooks in the relationship between science and religion in
the Russian philosophical perspective, while in the book the author takes a clear attempt to show the diversity of
approaches. Even they are organized in the streams with numerous representatives including concordism, cosmism, the
Neopatristic synthesis, panpsychism, Pantheist and panentheism concepts, etc. }

{\color{black}
 The book invites to make one view wider accepting the ability to look at the science as well as on religion from a
distinctive prospect showing the presupposition of rational and supra-rational order. On the one hand, such recognized
persons like Mikhail Tugan-Baranovsky, Mikhail Lomonosov, Nikolai Fedorov, Vladimir Vernadsky demonstrated their
ability to combine science and spiritual experiences. On the other, such the famous religious figures as Vladimir
Soloviev, Fr. Pavel Florensky, Nikolai Lossky, and others thinkers had a tendency to keep in mind the discoveries in
the scientific fields and correlate them with the religious understanding of the world. Then, the book is useful for
those who are open to go deeper into the understanding of the Russian Religious Philosophy and get precise and correct
knowledge about it unique ability and quest to harmonize faith and science representing them as two polls of humane
activity. These numerous projects could saturate the curiosity of the reader prompting to continue to prospect the
Eastern Christian religious tradition.}

{\raggedleft\color{black}
 \textit{Nataliya Petreshak}
\par}
\end{document}
