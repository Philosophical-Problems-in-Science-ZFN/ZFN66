\begin{recplenv}{Nataliya Petreshak}
	{Parallel worlds of faith and science in the Russian intellectual milieu}
	{Parallel worlds of faith and science in the Russian intellectual\\milieu}
	{Teresa Obolevitch, \textit{Faith and Science in Russian Religious Thought}, Oxford University Press, Oxford-New York 2019,
		pp.240.}
 





The new book written by Teresa Obolevich carries on the details and nuances of the thoroughness of the
perception of the world by the orient society in the relation between science and religion. Admitted emphasis opens a door
to the peculiar development of the scientific and religious spheres as well as a tendency of accepting the world by the
Eastern society from the 10\textsuperscript{th} century up to our times. So, the book reveals
the specificity of development of the world by the mind which managed to fight with secular intentions. As the author
shows, the starting point of the outlook in the East was spirituality which has improved through ages its wise
nature and was ready to accept the philosophical tendencies along with scientific ones. It is worth noting
that geographically speaking
the orient or the Eastern Christian society means here the lands which in the
18\textsuperscript{th} century were encompassed by the Russian Empire but could start as separate centers.


The unique clue of the book is that the author pays attention to the relationship of faith and science rather than to faith and
reason. It is crucial to note that while the former relation has a special tone, precise nuances and is actual,
it is more narrow than the latter. Hence, this postulation describes the picture of how the scientists
contributed into their fields keeping God as a source of their inspiration.


It is worth noting that the author successfully selects the parts of the curriculum giving access to the full extent
of the possibility to enter the relationship between these two disciplines. A reader can admit that this linkage is not
monotonous and obtains a very different character with numerous interpretations and meanings, as well as religious
attitude, which sometimes even took a fundamentalist approach. Another thing is that
the contrast in the development of this approach in the West and the East
is very evident, as it is clear from the book.


In the orient world pictured in the book the author draws the most attention to two eras: the Enlightenment and the Silver Age,
each of
which got here a special significance. The former is associated with the enlightenment of Christ (and as such it
clearly
contradicts the Enlightenment in the West), the latter is admired by the new brief of philosophy as a pillar for
development of scientific view too.


During the history the topic of the relationship between science and religion in the East had been included in the
scope of interest of researchers from different fields, church and religious figures, and experts in the natural sciences. Such
effort demonstrates that the Eastern world was open to the western influences, yet usually they were accepted through
spiritual filter. Even a ``pure materialistic'' position (represented by Tsiolkovsky, for instance)
was interpreted in terms of the evidence of
God-like presence in the world. Generally speaking, any materialistic ``wind'' that came from the secularized West was mostly
described as a quasi-religious.


Obolevich establishes a few incentives of such cause grounded first of all on theological and consequently
philosophical background, among which are Biblical recourses and patristic tradition, sacred art and literature and
which mainly originated from the Eastern thought.
Taking them into consideration, in the numerous proposed ideas and concepts a reader can recognize
the controversy of St. Basil the Great and St. Gregory Palamas
formula, which originally distinguishes between God’s essence and the presence of His energies in the world.
What follows from this standpoint is that in this case the researchers deal not with just the matter, but with the
expressions of God’s potentiality through which He can be recognized. Thus, a field of science does not
contradict the existence of the Creator, but the acknowledgment of the creation is revealed as one of the ways to
recognize God. Therefore, science is a kind of a common deal which could be also called a cosmological liturgy.


Nevertheless, it would be a mistake to talk about uniformity of the outlooks on the relationship between science and religion in
the Russian philosophical perspective, while in the book the author takes a clear attempt to show the diversity of
approaches. Even if they are organized in the streams with numerous representatives including concordism, cosmism, the
neopatristic synthesis, panpsychism, pantheistic and panentheistic concepts, etc.


The book invites one to make a wider look at the science as well as religion from a
distinctive prospect showing the presupposition of rational and supra-rational order. On the one hand, such recognized
persons like Mikhail Tugan-Baranovsky, Mikhail Lomonosov, Nikolai Fedorov, Vladimir Vernadsky demonstrated their
ability to combine science and spiritual experiences. On the other, such famous religious figures as Vladimir
Soloviev, Fr. Pavel Florensky, Nikolai Lossky, and others thinkers had a tendency to keep in mind the discoveries in
the scientific disciplines and correlate them with the religious understanding of the world. Then, the book is useful for
those who are open to go deeper into the understanding of the Russian Religious Philosophy and get precise and correct
knowledge about its unique ability and quest to harmonize faith and science representing them as two forms of human
activity. These numerous projects could saturate the curiosity of the reader prompting to continue to prospect the
Eastern Christian religious tradition.


\autorrec{Nataliya Petreshak}

\end{recplenv}

