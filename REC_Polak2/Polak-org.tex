% This file was converted to LaTeX by Writer2LaTeX ver. 1.6
% see http://writer2latex.sourceforge.net for more info
\documentclass[a4paper]{article}
\usepackage[utf8]{inputenc}
\usepackage[noenc]{tipa}
\usepackage{tipx}
\usepackage[geometry,weather,misc,clock]{ifsym}
\usepackage{pifont}
\usepackage{eurosym}
\usepackage{amsmath}
\usepackage{wasysym}
\usepackage{amssymb,amsfonts,textcomp}
\usepackage[T3,T1]{fontenc}
\usepackage[varg]{txfonts}\usepackage[polish]{babel}
\usepackage{color}
\usepackage[top=2cm,bottom=2cm,left=2cm,right=2cm,nohead,nofoot]{geometry}
\usepackage{array}
\usepackage{hhline}
\usepackage{hyperref}
\hypersetup{pdftex, colorlinks=true, linkcolor=blue, citecolor=blue, filecolor=blue, urlcolor=blue, pdftitle=}
% Footnote rule
\setlength{\skip\footins}{0.119cm}
\renewcommand\footnoterule{\vspace*{-0.018cm}\setlength\leftskip{0pt}\setlength\rightskip{0pt plus 1fil}\noindent\textcolor{black}{\rule{0.25\columnwidth}{0.018cm}}\vspace*{0.101cm}}
\title{}
\begin{document}
{\bfseries
Nauka w oczach erudytów}

Kochański A.A., Leibniz G.W., \textit{Korespondencja Adama Adamandego Kochańskiego i Gottrieda Wilhelma Leibniza z lat
1670-1698}, D. Sieńko (tłum.), wyd. Muzeum Pałacu Króla Jana III w Wilanowie, Warszawa 2019, ss. 256.

Postaci Gottfrieda Wilhelma Leibniza nie trzeba z pewnością przedstawiać czytelnikom ZFN, wszak w archiwum czasopisma
można znaleźć wiele artykułów na jego temat, wymieńmy chociaż trzy ostatnio opublikowane: \label{ref:RNDrwfEocyqF0}(M.
Heller, 2016; D. Bubula, 2011; M. Heller, 2008). Zupełnie inaczej rzecz ma się z postacią polskiego jezuity Adama
Adamandego Kochańskiego – ten wybitny uczony XVII wieku pojawił się w kręgu zainteresowania tylko raz i to na
marginesie rozważań o recepcji myśli Galileusza \label{ref:RND9yqTsZs0te}(K. Targosz, 2003). Najwyższy więc czas, aby
przypomnieć znaczenie również i tej postaci dla rozwoju refleksji filozoficznej w kontekście nauki.

Naukowa korespondencja polskiego jezuity Adama Adamandego Kochańskiego oraz słynnego filozofa Gottfrieda Wilhelma
Leibniza należy bowiem do jednych z najciekawszych i najważniejszych w historii polskiej nauki (a także związanej z nią
filozofii). Obejmuje okres prawie trzydziestu lat i choć niestety dotrwała do naszych czasów w niekompletnym stanie, to
i tak zachowany zbiór 39 listów stanowi unikalne świadectwo rozwoju nauki pod koniec XVII wieku oraz kształtowania się
poglądów związanych z nowoczesną metodą naukową. Listy przynoszą niezwykłe bogactwo treści ubranej w kunsztowną formę
epistolarną, zaświadczając, że mamy do czynienia z najwybitniejszymi przedstawicielami \textit{respublica eruditorum}.
Korespondencję zapoczątkował młody, liczący 24 lata Leibniz, który zabiegał o kontakt z Kochańskim, sławnym już wówczas
uczonym. Tematykę listów można podzielić za tłumaczką, Dorotą Sieńko, na pięć głównych grup: a) zagadnienia nauk
ścisłych (matematyka, astronomia) oraz przyrodniczych; b) zagadnienia konstruktorsko-inżynieryjne; c) językoznawstwo
oraz badania dotyczące pochodzenia ludów i narodów; d) sprawy związane z misjami w Chinach i tamtejszą kulturą; e)
problematykę alchemiczną i medyczną (zob. s. 25). Korespondencja ukazuje z jednej strony wielkie znaczenie, jakie
przykładano wówczas w nauce do metod matematycznych, z drugiej zaś istne \textit{silva rerum}, które próbowali w jakiś
sposób przy pomocy teorii opanować ówcześni uczeni. Lektura korespondencji jest wyjątkową okazją do ujrzenia nauki
\textit{in statu nascendi}, a koniec wieku XVII stanowi bardzo ważny okres kształtowania się nowożytnej nauki
(wspomnijmy choćby wczesne wzmianki Leibniza o istocie rachunku różniczkowego doskonale ukazujące intuicje genialnego
twórcy). Historyk i filozof nauki odnajdzie tam wiele interesujących wątków – nie sposób wymienić tu wszystkich, więc z
własnej perspektywy chciałbym wskazać trzy: kwestie z zakresu metodologii nauk, epistemicznej roli przyrządów naukowych
oraz maszyn liczących.

Niniejsze wydanie korespondencji nie jest bynajmniej pierwszą edycją. Dzieje edycji dokonanych przez polskich uczonych
są dobrym świadectwem trudności w rozwoju rodzimej historiografii nauki. Pierwsze wydanie w języku oryginalnym zostało
dokonane przez Samuela Dicksteina na łamach „Prac Matematyczno-Fizycznych” \label{ref:RNDQZ5P2qjRIF}(S. Dickstein,
1901, 1902) na podstawie odpisów Eduarda Bodemana. Jedynie dwa listy z tego zbioru zostały przetłumaczone przez
warszawskiego matematyka i opublikowane nieco wcześniej na łamach (notabene!) „Przeglądu Filozoficznego”
\label{ref:RNDPWVI1402FW}(S. Dickstein, 1897). W edycji Dicksteina brakowało dwóch listów, które po wielu latach
opublikował również w odpisach z oryginałów Stanisław Dobrzycki \label{ref:RNDZro5CQEWai}(1967). Komplet łacińskich
odpisów Bodemana i Dobrzyckiego bez zmian przedrukował Bogdan Lisiak SJ \label{ref:RNDH2bcoL4kp0}(2005). Po stu latach
od pierwszej edycji Dicksteina konieczne było opracowanie nowego wydania uwzględniającego nowoczesny aparat krytyczny,
wciąż nie posiadaliśmy również tłumaczenia całości korespondencji. W nowym opracowaniu i tłumaczeniu należało
uwzględnić również wiedzę z zakresu historii i filozofii nauki. Niestety dzieła tego Lisiak nie mógł dokonać, pisząc z
żalem we wstępie, że „[…] wątków było zbyt wiele, a podawanie tłumaczeń przekraczało możliwości redaktorów i
wydawnictwa” \label{ref:RNDCfWVJWMSCO}(B. Lisiak, 2005, s. 9). Wyjaśnił również, że u podstaw współczesnych problemów
leżało niezrozumienie znaczenia edycji korespondencji w polskim środowisku naukowym. Recenzent Komitetu Badań Naukowych
oceniał, że „edycja listów nie jest podstawą do premiowania” \label{ref:RNDWLWa7Ayb7H}(B. Lisiak, 2005, s. 10
[przypis]). Nie jest to niestety jedyny współczesny przykład braku zrozumienia znaczenia edycji korespondencji naukowej
w polskim środowisku naukowym. Tym bardziej więc przyjąć należy z wielkim uznaniem nowoczesne tłumaczenie listów
Kochańskiego i Leibniza, wydane niezwykle starannie i z wielką dbałością o wytworną oprawę graficzną.

Na marginesie trzeba dodać, że dominujący w Polsce brak zrozumienia dla znaczenia edycji korespondencji naukowej
powoduje, że wiele naszych historycznych skarbów może jeszcze długo spoczywać na półkach archiwów – ani reprezentanci
humanistyki, ani reprezentanci nauk często nie są w stanie dostrzec ich wartości. Oczywiście są chlubne wyjątki, a
jednym z dowodów na to że są w naszym kraju naukowcy, jak i wydawcy rozumiejący ten stan rzeczy, jest omawiana
publikacja.

Tłumaczenie dokonane przez Dorotę Sieńko jest, jak wspomniałem, pierwszym tłumaczeniem całej korespondencji na język
polski. Tłumaczka uzupełniła tekst źródłowy wcześniejszych wydań \label{ref:RND7xLlLu4TfG}(S. Dickstein, 1901, 1902; S.
Dobrzycki, 1967; B. Lisiak, 2005) w oparciu o krytyczne wydanie listów Leibniza: \textit{Sämtliche Schriften und
Briefe}. Do zbioru dodała również w „Dodatku” trzy listy z korespondencji z innymi uczonymi, które pozwoliły lepiej
naświetlić dyskutowane przez Kochańskiego i Leibniza problemy zawarte w \textit{Liście pytań do o. Bouveta}.

O znaczeniu edycji decyduje przede wszystkim poziom tłumaczenia, który oceniam wysoko, ponieważ tłumaczka starała się
bardzo wiernie oddać sens wypowiedzi uczonych jednocześnie unikając wielu pułapek, które czyhają na tłumacza nie
znającego dobrze kontekstu historycznego i naukowego. Drugim aspektem decydującym o mojej wysokiej ocenie tej
publikacji jest bardzo dobre opracowanie krytyczne. Liczne cenne przypisy ukazują doskonałą orientację tłumaczki w
niezwykle bogatym tle historycznym korespondencji. Jej uwagi wyraźnie nawiązują do erudycyjnego stylu Kochańskiego i
Leibniza, co nadaje publikacji wysoką wartość, a komentarzom pewną spójność stylistyczną z tekstem źródłowym. Godne
uwagi jest również wprowadzenie, ukazujące czytelnikowi szeroki kontekst historyczny oraz najważniejsze wątki
korespondencji\footnote{ Jako ciekawostkę można powiedzieć, że praca nad publikacją trwała około dekady, a częściowo
związana była z działalnością grupy historii nauki Centrum Kopernika Badań Interdyscyplinarnych.}. Szkoda, że część ta
nie została bardziej rozbudowana, gdyż pozostawia pewien niedosyt, który jednak dla dociekliwego czytelnika może stać
się inspiracją do własnych studiów.

Na zakończenie można jeszcze zapytać, dlaczego korespondencję dwóch uczonych wydało niezwykle starannie i pięknie Muzeum
Pałacu Króla Jana III w Wilanowie. Za wyjaśnienie niech wystarczy urywek listu Leibniza, wychwalający (nie jeden raz
zresztą!) króla Sarmacji, Jana III Sobieskiego: „Nie bez racji bowiem nazywa się Wielkim waszego Króla, który – jedyny
spośród wszystkich, którzy ćwiczą się w cnocie Bohaterstwa, nie zabiega o podziw dla siebie i otrzymał panowanie w
królestwie wzniosłej myśli, zanim i Fortuna ogłosiła go królem [Polski]” (s. 96). Szczere świadectwo jednego z
największych uczonych tamtych czasów powinno wejść do kanonu wiedzy każdego wykształconego Polaka. Niestety tak nie
jest, a nawet wśród historyków nauki wiedza ta nie jest powszechna. Wciąż nazbyt mało wiemy o tamtych czasach i w
związku z tym nie możemy doceniać naszego wkładu naukowo-filozoficznego. Książki takie, jak recenzowana, mogą jednak
skutecznie odwrócić ten nieszczęśliwy stan rzeczy.

{\raggedleft\itshape
Paweł Polak
\par}

{\bfseries
Bibliografia}

Bubula, D., 2011. Woluntaryzm w ujeciu Gottfrieda Wilhelma Leibniza i Samuela Clarke’a. \textit{Philosophical Problems
in Science (Zagadnienia Filozoficzne w Nauce)}, [online] (48), s. 77–94. Dostępne na:
{\textless}http://zfn.edu.pl/index.php/zfn/article/view/140{\textgreater} [dostęp 9.07.2019].

Dickstein, S., 1897. Wyjątek z korespondencji Kochańskiego z Leibnizem. \textit{Przegląd Filozoficzny}, 1, s. 71–85.

Dickstein, S., 1901. Korespondencya Kochańskiego i Leibniza według odpisów D-ra E. Bodemanna z oryginałów, znajdujących
się w Bibliotece Królewskiej w Hanowerze, po raz pierwszy do druku podana przez S. Dicksteina. \textit{Prace
Matematyczno-Fizyczne}, [online] 12, s. 225–273. Dostępne na:
{\textless}http://pldml.icm.edu.pl/pldml/element/bwmeta1.element.bwnjournal-article-pmfv12i1p8bwm{\textgreater} [dostęp
9.07.2019].

Dickstein, S., 1902. Korespondencya Kochańskiego i Leibniza, według odpisów D-ra E. Bodemanna z oryginałów, znajdujących
się w Bibliotece królewskiej w Hanowerze, po raz pierwszy do druku podana przez S. Dicksteina (ciąg dalszy i
dokończenie). \textit{Prace Matematyczno-Fizyczne}, [online] 13, s. 237–283. Dostępne na:
{\textless}http://pldml.icm.edu.pl/pldml/element/bwmeta1.element.bwnjournal-article-pmfv13i1p6bwm{\textgreater} [dostęp
9.07.2019].

Dobrzycki, S., 1967. Deux lettres inédites de Leibniz à Kochański. \textit{Organon}, 4, s. 217–228.

Heller, M., 2008. Stworzenie świata według Leibniza. \textit{Philosophical Problems in Science (Zagadnienia Filozoficzne
w Nauce)}, [online] (42), s. 3–14. Dostępne na:
{\textless}http://zfn.edu.pl/index.php/zfn/article/view/243{\textgreater} [dostęp 9.07.2019].

Heller, M., 2016. Wyzwanie dla racjonalności: Leibniz i początek ery Newtona. \textit{Philosophical Problems in Science
(Zagadnienia Filozoficzne w Nauce)}, [online] (61), s. 5–22. Dostępne na:
{\textless}http://zfn.edu.pl/index.php/zfn/article/view/377{\textgreater} [dostęp 9.07.2019].

Lisiak, B. red., 2005. \textit{Korespondencja Adama Adamandego Kochańskiego SJ (1657-1699)}. Źródła do Dziejów Kultury.
Kroniki i Listy. Kraków: Wyższa Szkoła Filozoficzno-Pedagogiczna „Ignatianum”[202F?]: Wydawnictwo WAM.

Targosz, K., 2003. Polski wątek w życiu i sprawie Galileusza „Galileo Galilei e il mondo polacco” Bronisława Bilińskiego
(1969) z uzupełnieniami. \textit{Philosophical Problems in Science (Zagadnienia Filozoficzne w Nauce)}, (32), s. 45–90.
\end{document}
