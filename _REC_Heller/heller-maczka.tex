\begin{recplenv}{Michał Heller, Janusz Mączka}
	{Uczniowie Platona}
	{Uczniowie Platona}
	{Bogdan Dembiński, \textit{Stara Akademia Platona. W~początkach epoki hellenistycznej
			(ostatni okres)}, Wydawnictwo Marek Derewiecki, Kęty 2018, ss.~183.}


Podboje mogą być różne -- wymuszone siłą i~intelektualne. Wymuszone siłą są
nietrwałe, intelektualne niekiedy trwają stulecia. Aleksander Macedoński (356--323 p.n.e.) podbił cały ówczesny świat,
ale stworzone przez niego imperium uległo rozpadowi wraz z~jego śmiercią. Wojska Aleksandra niosły również ze sobą
elementy greckiej myśli i~kultury. Okazały się one skuteczniejsze niż greccy hoplici. Rozpad macedońskiego imperium nie
stał się dla nich przeszkodą lecz, przeciwnie, czynnikiem sprzyjającym myślowemu fermentowi, który zaczął stopniowo
przetwarzać ówczesny świat. Dziś o~dokonaniach Aleksandra pamiętają tylko podręczniki historii, natomiast myśli zasiane
przez greckich filozofów nadal wydają owoce. Ale nie jest tak, że filozofia pozostaje nieczuła na polityczne przemiany.
Gdy grecka filozofia zetknęła się z~lokalnymi kulturami podbitych narodów, sama przyjęła niektóre ich aspekty. Stało
się to możliwe dzięki pewnemu uniwersalizmowi, który był jej charakterystyczną cechą.


Mówiąc o~filozofii greckiej, nie sposób nie pomyśleć o~Akademii Platońskiej. Po
podbojach Aleksandra, Ateny straciły polityczną samodzielność, ale w~nowej sytuacji nie przestały odgrywać swojej
kulturalnej a~nawet politycznej roli. Wraz ze śmiercią Aleksandra Wielkiego nie tylko świat śródziemnomorski wkroczył
 w~hellenistyczną epokę, ale dobiegł także końca pewien etap rozwoju Akademii Platońskiej. W~roku 315 p.n.e. umarł trzeci
scholarcha Akademii, Ksenokrates i~jej prowadzenie przejęli myśliciele, którzy nie byli bezpośrednimi uczniami Platona.
Akademia Platońska, chcąc nie chcąc, uległa duchowi czasu -- z~helleńskiej stała się hellenistyczną.


Symbolem nowego świata stała się Filozoficzna Szkoła Aleksandryjska, ale znamienne
jest to, że związki z~Akademią Platońską i~Arystotelesowskim Likeonem pozostały nadal żywe. Istnieją nawet podejrzenia,
że myśl zorganizowania ośrodka naukowego w~Aleksandrii posunął Ptolemeuszowi I~uciekinier z~Aten, Demetriusz
 z~Faleronu.


Właśnie w~tym dziejowym momencie rozpoczyna się akcja omawianej książki. Jest ona
kontynuacją monografii \parencite{dembinski_pozny_2010} poświęconej późnemu Platonowi i~Starej
Akademii (tzn. bezpośrednim uczniom
Platona)\footnote{Warto również wspomnieć dwie inne książki tego autora poświęcone
Platonowi i~jego Szkole (\cite{dembinski_teoria_1997}, \cite*{dembinski_pozna_2003}).}. Niniejsza książka analizuje przeobrażenia
Akademii związane z~tworzeniem się hellenistycznego świata i~doprowadza ją do scholarchatu Arkezylaosa z~Pitane, który
zmarł w~241 roku p.n.e.


Jak pisze Bogdan Dembiński, po podbojach Aleksandra „cały świat stanął szeroko
otworem przed helleńskimi wpływami” (s.~23). Z~tej konfrontacji helleńskich wpływów z~resztą ówczesnego świata narodził
się okres hellenistyczny. „Człowiek, reprezentujący najwyższy stopień złożoności świata, nie przestaje nigdy do świata
tego należeć. Nie sposób traktować go jako autonomicznego bytu. Z~drugiej strony, rozumienie świata jest zawsze dziełem
podmiotu, i~nie sposób nie uwzględnić jego podmiotowej natury” (s.~26). Następcy bezpośrednich uczniów Platona
podtrzymywali tę tezę, ale kładli większy nacisk na jej antropologiczne niż kosmologiczne aspekty. Nic więc dziwnego,
że etyka stała się przedmiotem ich zainteresowań w~stopniu większym niż dotychczas. Bogdan Dembiński utrzymuje, że
 w~tych poglądach leży źródło przekonania uczniów Platona, iż rozum powinien być przewodnikiem postępowania, a~co za tym
idzie zasadą etyki. Powodem tego był również, między innymi, wpływ stoików. Ta zmiana akcentów okazała się też zgodna
 z~ogólnymi tendencjami: nowe ludy, które weszły w~orbitę wpływów myśli greckiej, były bardziej zainteresowane sprawami
człowieka niż kosmologicznymi spekulacjami.


Przyjrzyjmy się nieco dokładniej tej tak zwanej Starej Akademii Platońskiej. Jak
wspomnieliśmy, główną trudnością jest ubóstwo źródeł. Dobrym tego przykładem są poglądy
Polemona z~Aten, który został scholarchą akademii
w~315 roku p.n.e. po śmierci Ksenokratesa. Do naszych czasów nie dotarły żadne dzieła Polemona, jedynie nieliczne
fragmenty jego wypowiedzi przechowały się u~innych autorów. Dembiński pisze: „Biorąc pod uwagę te ograniczenia,
zdecydowałem się podjąć próbę rekonstrukcji jego myśli poprzez umieszczenie zachowanych wypowiedzi Polemona i~opinii
 o~jego filozofii, w~szerokim kontekście tradycji filozoficznej, która poprzedzała jego koncepcję, oraz w~kontekście tych
systemów, które z~myśli tej się wyłoniły” (s.~80). Realizując przyjętą strategię, Bogdan Dembiński nie ma łatwego
zadania, gdyż rekonstrukcję poglądów Polemona wysnuwa z~czterech krótkich wypowiedzi i~jednego zachowanego tytułu
dzieła. Pierwszą wypowiedź znamy dzięki Stobajosowi: „Wszechświat jest boski i~ma boską naturę” (s.~82), drugą
wypowiedź relacjonuje Diogenes Leartios: „Polemon zwykł mawiać, iż należy się zaprawiać w~czynach, a~nie dialektycznych
subtelnościach; kto by postępował inaczej, będzie podobny do człowieka, który się wyuczył z~podręcznika zasad harmonii,
ale nie umie grać, i~potrafi tylko wysuwać godne podziwu problemy, sprzeczne z~własną
jego postawą moralną” (s.~95). Zachowany tytułu dzieła brzmi: \textit{O życiu zgodnym z~naturą. }Z tym tematem związane
są dwie wypowiedzi, jedna przechowana przez Cycerona: „najwyższym dobrem jest życie zgodne z~naturą, tzn. korzystanie
z pierwotnych darów natury, zgodnie z~moralnymi zasadami” (s.~87) i~druga, którą zawdzięczamy Klemensowi
Aleksandryjskiemu: „Bez cnoty nie ma szczęścia; ale też może wystarczy ona sama, bez cielesnych i~zewnętrznych dóbr,
cnota sama dla siebie wystarczająca do szczęścia” (s.~87).


Wychodząc z~tych szczątkowych informacji i~umieszczając je w~kontekście poglądów
poprzedników i~następców, Dembiński rekonstruuje hipotetyczne poglądy Polemona z~Aten na ponad dziesięciu stronach.
Pierwszy cytat otwiera szerokie pole domysłów, ponieważ temat boskości świata był szeroko eksploatowany w~filozofii
greckiej. Również problem życia zgodnego z~naturą (z~Logosem świata) ma głębokie zakorzenienie w~myśli greckiej, żeby
wymieniać choćby poglądy stoików. Z~kolei zainteresowania Polemona dialektyką wydają się być „charakterystyczne dla
koncepcji filozofii hellenistycznej, w~której filozofia zaczęła być traktowana nie tylko jako koncept teoretyczny, ale
przede wszystkim, jako działalność praktyczna, sztuka życia” (s.~93). Nie jest jasne, czy Polemon zrezygnował
 z~dialektyki i~przeszedł na pozycje czysto praktyczne, czy też podkreślał jedynie świadomość ograniczeń w~stosowaniu
metody dialektycznej.

\enlargethispage{-.5\baselineskip}
Stosunki w~Akademii były, można powiedzieć, przyjacielskie. Następcą Polemona
został Krates, jego uczeń i~do końca życia bliski przyjaciel. Również jego uczniem był Krantor, przyjaciel kolejnego
scholarchy, Arkezylaosa. Nie zachowały się żadne informacje dotyczące nauki Kratesa. Uzasadnione wydaje się
przypuszczenie, iż „bliskość z~Polemonem i~obecność w~Akademii może wskazywać, że jego poglądy zbliżone były do
poglądów innych akademików” (s.~97). O~Krantorze wiemy więcej. Diogenes Leartios informuje, że był on autorem licznych
prac filozoficznych oraz komentarzy do innych filozofów, a~całość jego pism obejmowała trzydzieści tysięcy wierszy.
 Z~kolei Proklos i~Plutarch przekazują informację, że Krantor pisał komentarze do dzieł Platona. Komentując \textit{Timajosa},
 w~związku z~koncepcją duszy świata, rozwinął teorię proporcji muzycznych.


W omawianym okresie Akademii Platońskiej nie zapomniano o~wątku matematycznego
przyrodoznawstwa. Jego przedstawicielami byli Euklides i~Eratostenes. Wprawdzie obaj związali się ze Szkołą
Aleksandryjską, ale obaj studiowali w~Akademii Platońskiej i~pozostawali pod wpływem platońskich idei. Proklos pisze
wprost: „Euklides był platonikiem i~kontynuował prace nad tzw. bryłami platońskimi” (s.~114).
Zdaniem Bogdana Dembińskiego, o~platońskim kształcie \textit{Elementów} Euklidesa zdecydowały trzy
czynniki: po pierwsze; założenia filozoficzne, po drugie; założenia metodologiczne i~po trzecie; „cała filozoficzna
tradycja poprzedników, którzy zajmowali się podobną problematyką” (s.~116). Na temat różnych aspektów naukowego dorobku
Euklidesa istnieje ogromna literatura. Nic więc dziwnego, że analizy Dembińskiego, chociaż dość obszerne, mają
charakter wybiórczy.


Bogdan Dembiński z~dziejami starszej Akademii Platońskiej wiąże także Eratostenesa
 z~Cyreny. Mimo, iż urodził się on w~roku 275 p.n.e., kiedy to okres Starej Akademii dobiegał końca, to -- zdaniem
Dembińskiego -- należy „zaliczać go do kręgu myślicieli związanych ściśle z~nauką późnej fazy Starej Akademii” (s.~135).
Jest faktem historycznym, że Eratostenes był uczniem Arkezylaosa, na którym kończy się okres Starej Akademii i~zaczyna
się okres następny.


Eratostenes jest autorem komentarza do \textit{Timajosa} i~kontynuatorem idei
matematycznego przyrodoznawstwa, tak charakterystycznej dla Akademii. Ideę tę zwięźle ujął Arystoteles: ,,matematyka
bada własności matematyczne bez odwoływania się do fizycznego świata, zaś nauki przyrodnicze badają fizyczność świata,
korzystając z~narzędzi matematycznych'' (s.~135).


W roku 255 p.n.e. Eratostenes został zaproszony przez Ptolomeusza III do Aleksandrii,
by objąć funkcję nauczyciela syna królewskiego, późniejszego Ptolomeusza IV; został także kierownikiem Biblioteki
Aleksandryjskiej. Odtąd związał się na stałe ze Szkołą Aleksandryjską. Naukowe zainteresowania Eratostenesa były
wszechstronne. Zajmował się matematyką, astronomią, geografią, historią, a~także filologią. Ta wszechstronność zjednała
mu przydomek pięcioboisty. Zalicza się go do najwybitniejszych uczonych starożytnej Grecji, żeby wspomnieć tylko jego
matematyczną teorię proporcji oraz niezwykle pomysłowy pomiar obwodu ziemi.


Mistrz Eratostenesa, Arkezylaos z~Pitane był kolejnym scholarchą Akademii. Pod jego
rządami Akademia powróciła do dawnej świetności i~ponownie stała się wiodącą szkołą Aten. Zajmował się on szerokim
wachlarzem zagadnień: od typowych dla Akademii rozważań ontologicznych i~epistemologicznych po matematykę, muzykę
i literaturę. Na podstawie zachowanych przekazów trudno jednoznacznie zinterpretować poglądy Arkezylaosa. Według jednych,
był on sceptykiem bardziej radykalnym niż Pirron, według innych, używał argumentów sceptycznych jedynie po to, aby przy
ich pomocy obronić naukę Platona. Według relacji Sykstusa Empiryka, Ariston miał o~nim powiedzieć, że był „z przodu
Platonem, Pirronem z~tyłu, a~w~środku Diodorem, gdyż korzystał z~dialektyki Diodora,
a z~wyznania był platonikiem” (s.~162), Dembiński skłania się do tej opinii, chociaż zauważa, że niejednoznaczność
interpretacji doktryny Arkezylaosa „pozostanie jej trwałym składnikiem” (s.~160).


Arkezylaosa uważa się za ostatniego przedstawiciela Starej Akademii i~założyciela
Akademii średniego okresu.


Powoływanie się na Platona jest czymś nagminnym we współczesnej filozofii. Bardzo
często jednak poglądy przypisywane Platonowi są, w~gruncie rzeczy, poglądami niektórych jego późniejszych uczniów lub
tylko luźno związanymi z~tym, co Platon naprawdę głosił. Dlatego też opracowania Bogdana Dembińskiego zarówno dotyczące
samego Platona, jak i~przedstawicieli jego szkoły są niezwykle cenne i~niemal unikatowe w~polskiej literaturze
filozoficznej. Także i~omawiana książka wypełnia lukę w~naszej znajomości okresu dziejów Starszej Akademii
Platońskiej. Jest to okres szczególnie ważny, gdyż pokrywa się z~czasem przemian, które stworzyły epokę hellenistyczną.
Jak widzieliśmy, filozofia tej epoki skierowała się bardziej w~stronę filozofii człowieka, ale również tematyka
kosmologii i~matematycznego przyrodoznawstwa była w~niej rozwijana, chociaż środek ciężkości tych zainteresowań
przeniósł się z~Akademii Platońskiej do Szkoły Aleksandryjskiej. Postaciami pomostowymi między tymi ośrodkami byli
Euklides i~Eratostenes.


Widzieliśmy również, że główną trudnością w~opracowaniu tego okresu jest ubóstwo
źródeł. Należy podziwiać zręczność Autora, który z~kilku przechowanych cytatów potrafił wydobyć całe pokłady możliwych
interpretacji. To ubóstwo źródeł stało się w~pewnym sensie atutem książki. Poruszanie się bowiem wśród poglądów różnych
autorów i~szukanie między nimi powiązań, celem wydobycia głębszej myśli z~jakiegoś resztkowego cytatu, pozwala lepiej
uchwycić różne podskórne prądy epoki.


Pewien niedosyt sprawia brak wyraźnego zakończenia książki, na przykład, jakiegoś
podsumowania całości. Książka kończy się na poglądach Arkezylaosa, założyciela Średniej Akademii Platońskiej; chciałoby
się o~tym nowym etapie dziejów Akademii dowiedzieć czegoś więcej, choćby w~wielkim skrócie. Chyba, że Autor planuje
następną książkę\ldots

\smallskip
\flushright{Łeba, 7 września 2018 roku}

\autorrec{Michał Heller}
\autorrec{Janusz Mączka}



\subsubsection{Bibliografia}\nopagebreak[4]
\end{recplenv}
