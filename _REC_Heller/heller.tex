\begin{recplenv}{Michał Heller}
	{Filozoficznie prowokująca teoria kategorii}
	{Filozoficznie prowokująca teoria kategorii}
	{Elaine Landry (red.), \textit{Categories for the Working Philosopher}, Oxford University Press, Oxford 2017, ss.~xiv+471.}







Niewiele jest matematycznych teorii tak filozoficznie prowokujących jak teoria kategorii. Nie tylko stwarza ona nową perspektywę w~spojrzeniu na matematykę, lecz również dostarcza skutecznych narzędzi, które aż proszą się, by je zastosować do szeregu filozoficznych zagadnień. Wiele cennych wyników, dotyczących tych zagadnień, jest rozproszonych po różnych czasopismach, często takich, które nie bywają regularnie nawiedzane przez filozofów. Dlatego wydanie tomu poświęconego wprost filozofii teorii kategorii przez Oxford University Press jest kolejnym wydawniczym ,,strzałem w~dziesiątkę''. Tytuł w~oczywisty sposób nawiązuje do znanej książki Saudersa Mac Lane'a \textit{Categories for the Working Mathematician} i, podobnie jak ona, tom ten nie zniża się do poziomu łatwej popularyzacji, lecz stawia duże wymagania pod względem matematycznego przygotowania czytelnika. Chcąc filozofom przybliżyć zawartość recenzowanego tomu, dokonam przeglądu jego treści na tyle dokładnego, na ile pozwalają rozmiary recenzji.

Teoria kategorii często jest uważana za główną rywalkę teorii mnogości w~roli podstawowej teorii matematycznej, nic więc dziwnego, że tom rozpoczyna się od spojrzenia na rolę teorii mnogości w~matematyce. Collin McLarty, w~rozdziale 1. zatytułowanym \textit{The Roles of Set Theories in Mathematics}, zwraca uwagę, że nie ma jednej teorii mnogości. Logicy zwykle za standardowe uważają ujęcie ZFC (Zermelo-Fraenkla z~aksjomatem wyboru), ale ZFC nie jest po prostu synonimem teorii mnogości. W~praktyce matematycznej (algebra, topologia, teoria homotopii...) za bardziej naturalne uważa się inne ujecia, o~aksjomatyce ZFC niekiedy w~ogóle nie wspominając. To, co jest wspólne wszystkim tym ujęciom (i jeszcze więcej), łączy w~sobie ETCS (\textit{Elementary Theory of the Category of Sets}), po polsku zwykle zwana po prostu kategorią zbiorów. W~pewnym sensie jednak ZFC mówi więcej o~zbiorach niż ETCS, chociaż to ,,więcej'' rzadko bywa używane w~innych działach matematyki. Dzieje się tak dlatego, że w~ZFC wszystko daje się zredukować do zbiorów, podczas gdy dla ETCS niektóre ze struktur, które dla ZFC są zbiorami, muszą być traktowane w~innych kategoriach. Nie ma oczywiście nic takiego, co dałoby się zrobić w~ZFC, a~czego nie dałoby się zrobić w~ETCS lub innych kategoriach. Mc Larty utrzymuje, i~pokazuje na przykładach, że wszystko, co matematycy z~teorii mnogości wykorzystują w~swoich pracach, lepiej ujmuje ETCS.

Od siebie dodałbym jeszcze jedną ,,przewagę'' ETCS nad ZFC. ETCS włącza teorię zbiorów w~styl myślenia teorii kategorii, a~jest to styl daleko bardziej perspektywiczny niż styl ,,zwykłej'' teorii mnogości.

Teoria kategorii ma silny aspekt unifikujący: łączy ona algebrę, logikę i~geometrię. David Cornfield (rozdział 2., \textit{Reviving the Philosophy of Geometry}) zwraca uwagę na jej aspekt geometryczny. Podkreśla, że dominujący dotychczas kierunek w~filozofii geometrii wyznaczały poglądy Koła Wiedeńskiego. Wedle tych poglądów, należy rozróżnić geometrię matematyczną i~fizyczną. Pierwsza jest po prostu pewnym systemem aksjomatycznym, a~druga jego interpretacją, określoną przez ,,reguły pomostowe'' pomiędzy aksjomatyką a~wynikami eksperymentów. Zdaniem Cornfielda ta formalistyczna koncepcja odwróciła uwagę badaczy od bardziej zgodnej z~praktyką matematyczną linii myślenia, reprezentowanej m.in. przez Weyla i~Cassirera, którzy rozwijali matematyczną koncepcję przestrzeni w~duchu dziewiętnastowiecznej tradycji, wywodzącą się bardziej ,,z~wnętrza matematyki''. Cornfield uważa, że w~pojęciowym środowisku teorii kategorii kultywuje się właśnie tę zaniedbaną linię myślenia. Prace Grothediecka i~jego szkoły dały potężny impuls temu kierunkowi. Jego przyszłość Cornfield wiąże z~najnowszymi osiągnięciami w~dziedzinie tzw. HOTT/UF (\textit{Homotopy Type Theory and Univalent Foundations}). Programowi temu poświęcone są dwa następne rozdziały.

Michael Shulman, w~rozdziale~3. \textit{Homotopy Type Theory: A~Synthetic Approach to Higher Equalities}, kreśli zarys programu \mbox{HOTT/UF}. Jego podstawą, podobnie jak teorii mnogości, jest pojęcie klasy lub zbioru składającego się z~elementów, ale elementy te zachowują się inaczej niż w~teorii mnogości. Mogą mianowicie być równe między sobą na różne sposoby (w teorii mnogości tylko na jeden sposób). To tworzy ,,sieć'' rozmaitych równości między elementami, co kreuje strukturę, zwaną w~matematyce grupoidem. Ale różne sposoby, na jakie dwa elementy mogą być sobie równe, też mogą być ze sobą równe na różne sposoby, co również tworzy sieć i~ta sieć także jest grupoidem (2-grupoidem). Możemy tak postępować w~nieskończoność, otrzymując pojęcie \mbox{$\infty$-grupoidu}. Już ta konstrukcja sugeruje, że matematyka, na jakiej oparty jest program HOTT/UF, nie jest prosta.

Druga część skrótu HOTT/UF (\textit{Univalent Foundations}) odnosi się do pewnej wersji programu, w~której istotną rolę odgrywa tzw. aksjomat uniwalencji (\textit{Univalent Axiom}). Waga tego programu (oprócz tego, że stanowi on istotny postęp w~teorii kategorii) polega na tym, iż proponuje on nową, atrakcyjną wersję podstaw matematyki. Zauważyć jednak wypada, że propozycja ta ciągle jeszcze znajduje się w~stadium początkowym.

Program HOTT/UF ma także wymowę filozoficzną. Jeden z~jego współtwórców, Steve Avodey (rozdział 4. \textit{Structuralism, Invariance, and Univalence}), przekonuje czytelnika, iż program ten jest urzeczywistnieniem idei strukturalizmu.

Bezpośrednio zagadnieniu teorii kategorii i~podstaw matematyki poświęcony jest kolejny rozdział pt. \textit{Category Theory and Foundations}. Michael Ernst skupia uwagę na ETCS i~na kategorii wszystkich kategorii CCAF (\textit{Category of Categories as a~Foundation}) jako ewentualnych kandydatkach do zastąpienia ZFC w~roli podstaw matematyki i~dokonuje obszernego przeglądu polemik na ten temat. Strony przytaczają ważkie argumenty na poparcie swojego stanowiska i~przeciw stanowisku oponentów, ale wydaje się, że kategoryjne podstawy bardziej uzasadniają roboczą praktykę matematyków.

Równie ważne jak problem podstaw jest zagadnienie globalnego spojrzenia na matematykę. Temu poświęcony jest rozdział~6. \textit{Canonical Maps}. Jean-Pierre Mar\-quis omawia w~nim rolę tak zwanych przez niego odwzorowań kanonicznych w~teorii kategorii. Nie istnieje żadna formalna ich definicja (podobnie jak nie istniała formalna definicja przekształceń naturalnych przed powstaniem teorii kategorii), ale matematyk-ekspert natychmiast je rozpozna, gdy na nie natrafi. Ich charakterystyczną cechą jest to, że nie trzeba ich ,,wymyślać''; wymusza je sama konstrukcja danej struktury. Jako przykład może posłużyć odwzorowanie ze zbioru do zbioru jego klas abstrakcji (w tym przypadku oficjalnie nazywa się ono odwzorowaniem kanonicznym) lub jedyne odwzorowanie, jakie występuje  przy definiowaniu dowolnych własności uniwersalnych w~teorii kategorii.

Odwzorowania, które Marquis nazwał kanonicznymi, w~zwykłej teorii mnogości nie wyróżniają się niczym szczególnym, natomiast w~teorii kategorii odgrywają ważną rolę, stanowią jakby ,,mapę drogową'', po której podróżują pojęcia, ale, co więcej, ,,stanowią one rusztowanie, na którym inne pojęcia są konstruowane lub budowane'' (s. 92). Odwzorowania kanoniczne powstają w~sposób naturalny podczas konstruowania obiektów lub struktur, ale z~chwilą gdy już raz zostaną powołane do życia, ,,matematyka rozwija się wokół nich w~sposób całkowicie organiczny'' (s. 93).

Odwzorowania kanoniczne dotychczas nie były przywoływane w~dyskusjach na temat filozofii matematyki, ale na pewno warto im się przyjrzeć dokładniej pod kątem globalnych charakterystyk matematyki. Tworzą one -- jak pisze Mar\-quis -- ,,architekturę matematyki'' (s. 104). W~filozofii nauki coraz częściej słyszy się głosy, że fundacjonizm (szukanie podstaw) należy porzucić na rzecz strukturalnego, globalnego ujęcia teorii naukowych i~nauki jako całości. Przy takim podejściu do filozofii matematyki rola teorii kategorii jest niekwestionowalna.

J.-P. Marquis jest także autorem rozdziału 8. \textit{Unfolding FOLDS. A~Foundational Framework for Abstract Mathematical Concepts}. Podejmuje w~nim problem abstrakcyjnego charakteru współczesnej matematyki. W~podejściu teoriomnogościowym matematykę konstruuje się z~czystych zbiorów. Znaczy to z~grubsza tyle, że konstrukcję matematyki zaczynamy od zbioru czystego $\{\emptyset \}$; następnie tworzymy zbiór $\{ \{\emptyset \}, \{ \emptyset , \{ \emptyset \} \} \} $, który też jest zbiorem czystym; następnie konstruujemy... itd., itd. W~efekcie potem każdy zbiór czysty możemy rozłożyć na rodzinę zbiorów czystych. Ale tak ,,skonstruowana'' matematyka zatraca swój abstrakcyjny charakter. Jej podstawowe ,,cegiełki'' są konkretami a~nie abstrakcyjnymi zbiorami.

Pojęcia abstrakcyjne, w~przeciwieństwie do konkretnych, po pierwsze, są dane przez przykłady i, po drugie, nie są dane indywidualnie lecz w~powiązaniu z~innymi. Próbę skonstruowania formalnej teorii, która realizowałaby tego rodzaju abstrakcyjne podejście do matematyki podjął M. Makkai. Jego system nazywa się FOLDS (\textit{First Order Logic with Dependent Sorts})\footnote{\textit{The Theory of Abstract Sets Based on First-Order Logic with Dependent Types}, \url{http://www.math.mgill.ca/makkai/Various/MateFest2013.pdf}.}. Jest to hierarchiczny system kategorii, funktorów między kategoriami i~naturalnych transformacji między funktorami, ale kategorie z~wyższego poziomu (z odpowiadającymi im funktorami i~transformacjami naturalnymi) są ,,bytami innego rodzaju'' niż kategorie z~niższego poziomu. Całemu systemowi FOLDS odpowiada pewien formalny język, ale charakter (sygnatura) tego języka zmienia się w~zależności od ,,miejsca'' w~systemie. Odpowiada temu fakt, że pojęcie identyczności, które w~teorii mnogości jest ,,sztywne'' (raz na zawsze ustalone), tu zmienia się zależnie od tego ,,miejsca''. Jest to interesująca próba stworzenia ścisłych podstaw do ujęcia abstrakcyjnego charakteru matematyki. Mar\-quis nie uważa jej za konkurencję w~stosunku do teoriomnogościowych podstaw matematyki. Po prostu ujmuje ona inne aspekty królowej nauk.

Niewątpliwie wielkie znaczenie filozoficzne teorii kategorii polega na jej związku z~logiką. Ten właśnie aspekt wybrał John Bell za przedmiot swoich rozważań w~rozdziale 7. pt. \textit{Categorical Logic and Model Theory}. Dokonał on wnikliwego przeglądu niektórych problemów z~tej dziedziny, jakie powinny zainteresować filozofa. Celem ,,zachęty'' zasygnalizuję tylko jeden fragment z~rozdziału Bella. Rozważmy pewien język $L$ i~zwiążmy z~nim kategorię $C$ w~następujący sposób: obiektami kategorii $C$ niech będą zdania języka $L$. Jeżeli $p$ i~$q$ są dwoma takimi obiektami (zdaniami), to strzałką między nimi niech będzie dedukcja $q$ z~$p$. Kategorię  $C$ nazywamy kategorią syntaktyczną języka $L$ (została tu naszkicowana jedynie idea jej konstrukcji). Co więcej, jeżeli $L$ jest językiem jakiegoś sformalizowanego systemu dedukcyjnego, to okazuje się, że nie tylko ten system dedukcyjny można przedstawić jako pewną kategorię, ale każda kategoria odpowiada pewnego rodzaju systemowi dedukcyjnemu. 

Problem wzajemnego oddziaływania syntaktyki i~semantyki podejmuje Kohei Kishida w~rozdziale~9. \textit{Categories and Modalities}. Jak widomo, logika modalna powstaje przez dodanie spójników modalnych (najczęściej  ,,jest możliwe, że'' i~,,jest konieczne, że''). Semantyką takiej logiki jest zwykle semantyka Kripkego (jest to semantyka na wzór semantyki Tarskiego, ale dla nieklasycznych logik, na przykład właśnie modalnych), lecz i~tu perspektywa kategoryjna istotnie wzbogaca  problematykę. Ukazuje ona strukturalne własności wielu aspektów modalności. Czytelnikowi bez dobrego treningu w~logice i~teorii kategorii Autor nie daje szans.

Nie trzeba nikogo przekonywać, jak ważną rolę w~matematyce odgrywają dowody. Teoria dowodów jest bardzo blisko podstaw matematyki. Właśnie tą problematyką zajęli się J.R.B. Cockett i~R.A.G. Seeley w~10. rozdziale \textit{Proof Theory of the Cut Rule}. Tytułowa \textit{cut rule} jest pewną regułą z~rachunku sekwentów. Autorzy na jej (bardzo rozbudowanym) przykładzie pokazują, w~jaki sposób wykorzystuje się teorię kategorii w~teorii dowodów. Jedną z~ciekawych własności tego podejścia jest możliwość prezentowania skomplikowanych dowodów w~postaci przejrzystych diagramów.

Rozdział 11. pióra Samsona Abramsky'ego pt. \textit{Contextuality: At the Borders of Paradox} otwiera szereg rozdziałów poświęconych zastosowaniom teorii kategorii do różnych działów nauki. Abramsky zajął się pojęciem kontekstualności, które pojawiło się w~mechanice kwantowej w~związku z~problematyką nielokalności i~stanów splątanych. Najogólniej mówiąc, kontekstualność występuje wówczas, gdy dysponujemy rodziną danych (wyników pomiarów), które lokalnie są konsystentne, ale globalnie nie. Autor formalizuje to pojęcie przy pomocy pojęć pochodzących z~teorii kategorii (głównie pojęcie snopu Grothendiecka) i~uważa, że jest to krok w~kierunku stworzenia matematycznej teorii kontekstualności. Żmudne analizy, poparte wielu przykładami, prowadzą Autora do wniosku, że niezwykle płodne obszary badawcze leżą na granicy niekonsystencji: ,,bogate pola zjawisk z~logiki i~teorii informacji, ściśle związane z~kluczowymi zagadnieniami podstaw fizyki, powstają na granicach paradoksu'' (s. 282).

\textls[-5]{Rozdział 12. \textit{Categorical Quan\-tum Mechanics I: Causal Quan\-tum Processes}}, autorstwa Boba Coecke i~Aleksa Kissingera, jest pierwszą częścią zamierzonego cyklu trzech artykułów poświęconych ,,kategorialnej mechanice kwantowej''. Tak Autorzy nazywają swoją koncepcję reprezentacji mechaniki kwantowej przy pomocy diagramów oraz prostych reguł ich łączenia i~przekształcania. Wiele podstawowych praw i~twierdzeń (wraz z~dowodami) mechaniki kwantowej daje się przedstawić przy pomocy ,,gry diagramów''. Diagramy te tłumaczy się następnie na język teorii kategorii. Na przykład, standardowe ujęcie mechaniki kwantowej przy pomocy przestrzeni Hilberta przekłada się na symetryczne kategorie monoidalne. Niektóre skomplikowane dowody matematyczne stają się przejrzyste i~znacznie prostsze, gdy się je przeprowadza, manipulując diagramami. Związek między metodą diagramów a~teorią kategorii pozwala samo pojęcie kategorii zdefiniować w~języku diagramów. Warto dodać, że -- jak wyznają Autorzy -- pomysł tej metody został zainspirowany ontologią procesu. Podstawowa konstrukcja diagramu miała stanowić jedną z~możliwych formalizacji idei procesu.

Następne dwa rozdziały są poświęcone fizyce relatywistycznej. Rozdział 13. \textls[-5]{\textit{Category Theory and the Foundations of Classical Space-Times Theories}} napisał \mbox{James} Owen Weatherall. Zajmuje się w~nim strukturami różnych czasoprzestrzeni. Istnieją prace Ehlersa i~Trautmana z~lat siedemdziesiątych ubiegłego stulecia (niewymienione w~tym rozdziale) dotyczące porównywania struktur czasoprzestrzeni zakładanych przez różne teorie fizyczne. Wówczas robiono to metodą konstrukcyjno-geometryczną. Weatherall, idąc za Johnem Baezem, zastosował metodę kategoryjną: teorie fizyczne należy rozpatrywać jako kategorie i~rozważać odpowiednie funktory między nimi. Metoda okazuje się skuteczna, a~nawet pod niektórymi względami bardziej szczegółowa niż metoda klasyczna. Pozwala na przykład rozstrzygać, czy dana teoria dysponuje większą strukturą niż taką, która by wystarczyła do zrealizowania celu postawionego przed nią. Takie teorie ,,z nadmiarem'' nazywają się teoriami cechowania (\textit{gauge}).

Joachim Lambek jest autorem kolejnego rozdziału zatytułowanego \textit{Six-Dimensional Lorentz Category}. W~związku z~badaniami dotyczącymi równania Diraca Autor wprowadza kategorię, którą nazywa kategorią Lorentza (ma ona trzy obiekty a~morfizmy są macierzami). Ten krótki rozdział ma postać komunikatu w~fachowym czasopiśmie. Autor zakłada, że czytelnik zna całe techniczne zaplecze. Z~filozofią łączy się on o~tyle, że czas w~zaproponowanym modelu ma dwa wymiary, a~-- jak wiadomo -- wszystko, co dotyczy czasu, winno interesować filozofa.

Andr\'ee Ehresmann w~rozdziale 15., noszącym tytuł \textit{Applications of Categories to Biology and Cognition}, przechodzi do biologicznych zastosowań teorii kategorii. Konstruuje on szereg kategorii, których celem jest modelowanie procesów życiowych i~poznawczych. Rozpoczyna od prostego modelu przedstawiającego ogólnie rozumianą ewolucję. Model ten składa się z~następujących elementów konstrukcyjnych:
\begin{enumerate}
\item
Interwał $T$ prostej rzeczywistej -- czas życia układu.
\item
Dla każdej chwili $t$ czasu $T$ istnieje kategoria $K_t$ -- konfiguracja układu w~chwili $t$.
\item
Dla każdych dwu chwil $t_1$ i~$t_2$ istnieje funktor (spełniający pewne proste warunki) -- przejście układu od stanu $K_{t_1}$ do stanu $K_{t_2}$.
\end{enumerate}
Cały ten model jest szczególnym przypadkiem tzw. \textit{semi-sheaf category}. Następnie Autor w~szeregu kroków wzbogaca ten model, aby modelowane w~ten sposób procesy coraz bardziej upodobnić do procesów życiowych. I~tak najpierw wprowadza pewną hierarchiczność struktur, odpowiadającą przystosowaniu układów biologicznych do zmieniających się warunków otoczenia; dalej -- strukturalne zmiany, odpowiadające wzrostowi złożoności, potem -- elementy pamięci historii układu, i~wreszcie -- pewne integrujące cechy, mające modelować ,,układy  neuronalne i~mentalne''. Wszystko to jest definiowane przy pomocy teoriokategoryjnych pojęć.

Wynik swojej pracy Autor nazywa ,,raczej rozwijającą się metodologią niż modelem''. Metodologia ta wygląda imponująco, brak jedynie w~całym rozdziale wzmianki o~jej zgodności z~rzeczywistymi procesami biologicznymi, czyli o~przewidywaniach, które by z~konstruowanych modeli wynikały i~je uwiarygodniały.

Rolę tego rodzaju modeli (lub ,,metodologii'') trafnie wyjaśnia David I. Spivak we wstępie do następnego rozdziału \textit{Categories as Mathematical Models}. U~podstaw dobrze pracujących modeli matematycznych zawsze leży operowanie liczbami (np. w~układach dynamicznych, statystyce itp.), natomiast w~dziedzinach wiedzy, takich jak biologia, mamy również do czynienia ze zjawiskami, których nie dało się (jeszcze) sprowadzić do operowania liczbami. W~takich przypadkach także usiłujemy tworzyć ich modele, ale pozostają one ,,na poziomie idei''. Do konstruowania tego rodzaju modeli doskonale nadaje się teoria kategorii, ale jest ona czymś znacznie więcej. Autor stwierdza, że teorię kategorii należy traktować jako ,,matematyczny model matematycznego modelowania'' (s. 385).

\enlargethispage{-1.5\baselineskip}
\textls[3]{Rozdział 17. \textit{Cathegories of Scientific Theories}, którego autorami są Hans Halvorson i~Dimitris Tsementzis, podejmuje wątek -- poruszony przez Johna Bella w~rozdziale 7., a~także przez Kohei Kishidę w~rozdziale 8. -- związku teorii kategorii z~logiką, ale stosuje go do zupełnie innej dziedziny -- do filozofii nauki. Jak wiadomo, w~filozofii nauki istnieją dwie rodziny koncepcji teorii naukowych: koncepcje syntaktyczne i~semantyczne. Pierwsze wywodzą się z~pozytywizmu logicznego i~przez kilka dekad były paradygmatem w~filozofii nauki. Drugie powstały jako opozycja w~stosunku do poprzednich i~spór między nimi trwa do dziś. Autorzy tego rozdziału angażują teorię kategorii do jego rozstrzygnięcia. Jeżeli mamy sformalizowaną teorię~$T$, to można skonstruować kategorię $C_T$ (zwaną także kategorią syntaktyczną), której wewnętrzna logika dokładnie odpowiada logice teorii $T$. I~odwrotnie, mając kategorię $C$ można skonstruować odpowiadającą jej sformalizowaną teorię $T$ (por. rozdział 7.). Pomiędzy tymi teoriami i~kategoriami istnieje specyficzne sprzężenie, które można ściśle wyrazić przy pomocy pary  sprzężonych funktorów (\textit{ad\-joint functors}). Sprzężenie to wyraża oddziaływanie między syntaktyką i~semantyką danej teorii~$T$. Autorzy proponują zastosować te rozważania do filozofii nauki, by wykazać, że między syntaktyczną i~semantyczną koncepcją teorii nie musi zachodzić wykluczanie. Oczywiście, żeby cały ten teorio-kategoryjny schemat działał, trzeba dopracować wiele szczegółów technicznych, co Autorzy starannie czynią.}

W ostatnim rozdziale, zatytułowanym \textit{Structural Realism and Category Mistakes}, autorka (równocześnie redaktorka całej książki), Elaine Landry, obszernie referuje spory, jakie toczą się wśród filozofów nauki na temat tzw. realizmu strukturalistycznego czyli poglądu, wedle którego realnie istnieją nie obiekty lecz struktury i~-- w~mocniejszych wersjach -- że struktury mogą istnieć bez obiektów. Niektórzy filozofowie wsparcia takiego poglądu poszukują w~teorii kategorii. Autorka przekonuje, że teoria kategorii nie wnosi niczego istotnego do tej dyskusji. Teoria ta bowiem rozgrywa się na poziomie konceptualnym s~nie dotyczy realnie istniejących struktur.

Jak widać z~tego przeglądu, otrzymaliśmy solidną porcję wiedzy i~analiz z~pobrzeży teorii kategorii i~filozofii. Poruszając się po tych pobrzeżach, niekiedy zapuszczaliśmy się w~głąb matematycznej teorii kategorii i~wówczas gubiliśmy się nieco w~zbyt technicznych analizach. Niekiedy wchodziliśmy w~bardziej filozoficzne obszary. Przeważnie były to obszary kontrolowane przez logikę lub filozofię nauki. Uprzywilejowany region stanowiła filozofia matematyki, rozumiana głównie jako podstawy matematyki, co jest rzeczą o~tyle zrozumiałą, że teoria kategorii niemal od samego początku rościła sobie pretensje do odgrywania istotnej roli w~tym regionie. Autorzy, występujący w~tym tomie, unikali jednak choćby krótkich wycieczek w~bardziej odległe dziedziny metafizyki lub ontologii. A~szkoda, gdyż teoria kategorii i~tu mogłaby wnieść świeży powiew. Choćby problem wielości logik, który z~taką ostrością pojawia się w~teorii kategorii. Czy więc istnieje jedna, ,,nadrzędna'' logika, czy trzeba przyjąć logiczny pluralizm? Jakie miałoby to konsekwencje dla filozofii (ontologii, epistemologii), która prawie cała jest oparta na rozumowaniach wykorzystujących jedynie logikę klasyczną?

Po przeczytaniu ważnej książki, zwykle rodzą się w~stosunku do niej  pretensje -- że zostało pominięte coś, czegośmy się po niej spodziewali. Tylko książki błahe odkłada się bez żalu. Książka zredagowana przez panią Elaine Landry jest książką ważną, choć trzeba niemało wysiłku, by przedrzeć się przez jej lekturę.

\smallskip
\flushright{październik 2018}

\autorrec{Michał Heller}


\end{recplenv}




