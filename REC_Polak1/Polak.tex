\begin{recplenv}{Paweł Polak}
	{Wychodzenie z~sarmackiej kopalni, czyli teologia nauki w~działaniu}
	{Wychodzenie z~sarmackiej kopalni, czyli teologia nauki\\w~działaniu}
	{Michał Heller, \textit{Ważniejsze niż Wszechświat}, Copernicus Center Press, Kraków 2018,
		ss.~128.}

\enlargethispage{1\baselineskip}


Przeglądając ogromny dorobek pisarski Michała Hellera łatwo zauważyć, że Wszechświat jest bodajże najważniejszym
tematem, który przewija się w~rozmaitych kontekstach w~jego publikacjach. Autor samym tytułem książki wyraźnie pragnie
nas zaintrygować, zapowiadając podjęcie tematyki, która jest dla niego ważniejsza niż w~zasadzie wszystko, o~czym do
tej pory pisał.

Niewielka książeczka swym pieczołowitym opracowaniem i~charakterystyczną piękną oprawą graficzną przypomina dwie
inne książki Michała Hellera\footnote{Notabene wszystkie trzy książki opracował graficznie Olgierd Chmielewski.}. Mam tu
na myśli pierwsze wydania książek \textit{Podróże z~filozofią w~tle} oraz
\textit{Jak być uczonym?}. Obie wcześniejsze książki na swój sposób są bardzo
osobiste. Niniejsza publikacja tworzy z~nimi spójną trylogię, ujawniając kolejne bardzo intymne przemyślenia Michała
Hellera. Tym razem mamy możliwość zapoznania się z~jego poglądami z~zakresu szeroko rozumianej teologii, o~których
dotychczas stosunkowo niewiele można było się dowiedzieć z~jego publikacji. Nikogo nie powinno dziwić, że prezentowane
przez Michała Hellera refleksje teologiczne ściśle powiązane są z~nauką, a~Autor operuje językiem bliższym nowoczesnej
nauce i~filozofii nauki niż tradycyjnej teologii spetryfikowanej językiem średniowiecznej filozofii. Choć poglądy
Michała Hellera brzmią na pozór odmiennie od tego, co możemy usłyszeć z~ust zdecydowanej większości współczesnych
teologów, to w~istocie jesteśmy w~ścisłym centrum ortodoksji. Co więcej~-- pewne na pozór zaskakujące tezy, jak np.
apofatyzm logiczny, są tylko konsekwencją zastosowania nowoczesnej wiedzy z~zakresu nauk i~filozofii do wyjaśniania
klasycznych problemów. Michał Heller jawi się tutaj jako twórczy kontynuator idei leibnizjańskich. Religijna refleksja
w duchu leibnizjańskiego teistycznego oświecenia jest wyzwaniem, które warto podjąć, niezależnie od indywidualnego
stosunku do filozofii Leibniza i~do samej religii.


Recenzowana pozycja ma poniekąd charakter \textit{silva rerum}, a~z~pewnością
można ją określić jako \textit{Miscellanea} z~teologii nauki. Składa się z~siedmiu
niedługich rozdziałów, z~których cztery stanowią teksty dotychczas niepublikowane. Książka rozpoczyna się od rozważań
nad zagadnieniem Wielkich Pytań, aby przejść do kwestii rozważań nad racjonalnością i~sensem. Trzeci rozdział przynosi
najbardziej „techniczną” część całości. „Logika Pana Boga” to intrygujący tekst poświęcony zagadnieniu logiki, jaka
może być odkryta w~stwórczym zamyśle Boga (\textit{Mind of the God})\footnote{Na
ironię losu może zakrawać fakt, że rozdział ten ilustrują wyidealizowane obrazy ukazujące piękno budowy świata
ożywionego
\parencite{haeckel_kunstformen_1904}.
%(Haeckel, 1904).
Ich autor, niemiecki przyrodnik Ernst Haeckel (1834-1919), zaangażowany był w~promocję
skrajnego darwinizmu i~materialistycznego monizmu, ale jego obrazy doskonałości natury Olgierd Chmielewski celnie
wykorzystał do ilustracji rozważań o~logice boskiego zamysłu widocznego w~kunsztownej budowie rzeczywistości.}.
\enlargethispage{1\baselineskip}
Tekst
ten jest jednym z~ważniejszych zastosowań teorii kategorii w~filozofii i~bez wątpienia~-- najważniejszym w~teologii.
Michał Heller wskazuje, że kategoryzując fizykę możemy zauważyć, że na różnych poziomach opisu stosowana jest różna
logika. Logika staje się zatem jedną ze zmiennych w~strukturach wyjaśniających rzeczywistość. Na zupełnie inny poziom
dyskurs przenosi się w~kolejnych paragrafach, w~których postawione zostało pytanie nie tylko o~logikę rządzącą danym
poziomem rzeczywistości, ale ostatecznie~-- o~logikę Boga. Michał Heller odrzuca absolutyzowanie (ontologizowanie)
zasady niesprzeczności, stąd godzi się na możliwość użycia logik parakonsystentnych, ale tylko takich, które nie
prowadzą do „przepełnienia systemu”, czyli nie pozwalają na wykazanie czegokolwiek (w klasycznej logice obowiązuje
bowiem zasada \textit{ex contradictione quodlibet}). Założenie, że Bóg musi
posługiwać się logiką klasyczną, jest szkodliwym antropomorfizmem, którego należy unikać w~teologicznych rozważaniach o~Bogu
(s.~51). Najważniejsza teologiczna teza Michała Hellera wynika więc z~ostrzeżenia przed bezkrytycznym stosowaniem
logiki klasycznej w~teologii (inną sprawą jest, na ile jest ona tam konsekwentnie stosowana). Autor nazwał ją
\textit{zasadą logicznego apofatyzmu}~-- twierdzi ona, że logika Boga nie musi być
logiką spełniającą nasze kryteria racjonalności. Najważniejszym zaś wnioskiem jest, że nie można w~teologii żadnej z~logik
absolutyzować. Otwiera to przed badaniami logicznymi w~teologii nowe, ogromne pola dociekań.

\enlargethispage{1\baselineskip}

Kolejna część „Usprawiedliwienie historii Wszechświata” próbuje ukazać rozwój Wszechświata z~perspektywy pytań o~Dobro i~zło.
Ta część wraz z~poprzednim rozdziałem jest najbardziej „leibnizjańska” z~całej książki i~ukazuje
nieskończonościową perspektywę teologii Michała Hellera. Teologia musi na serio wziąć pod uwagę nieskończoność Boga, a~co
za tym idzie musi oprzeć się na koncepcjach nieskończonościowych. Lei\-bniz jako pierwszy konsekwentnie próbował
budować taką wizję teologiczną, a~omawiany rozdział popularyzuje to podejście. Mimo przystępnej prezentacji czytelnik
łatwo może zauważyć, jak wymagające koncepcyjnie i~wyobrażeniowo mogą być takie wyjaśnienia oraz jak dalekie są od
\textit{folk theology}. Nasze intuicyjne przekonania teologiczne są silnie obarczone
myśleniem w~kategoriach pojęć skończonych. Leibniz ukuł nawet piękną metaforę dla takiego złudzenia, nazywając ją
kopalnią sarmatów\footnote{Notabene 'sarmacka kopalnia' u~Leibniza miała za swój realny pierwowzór kopalnię soli w~Wieliczce.
Leibniz żywo się interesował kopalniami i~szczególnie dużo wiedział o~słynnych żupach krakowskich. Świadczy o~tym
jego korespondencja z~ks. A.A. Kochańskim SJ z~lat 1691-1692
\parencite{kochanski_korespondencja_2005,kochanski_korespondencja_2019}.
%(Kochański, 2005, Kochański i~Leibniz, 2019).
Interesujące wyjaśnienia Roberta Latty
można znaleźć również w~angielskim wydaniu \textit{Monadologii}
\parencite[s.~346]{leibniz_monadology_1898}.
%(Leibniz, 1898, s. 346).
Więcej na ten temat, zob.
\url{https://filozoficznykrakow.wordpress.com/2019/04/13/zupy-krakowskie-a-filozofia-czyli-o-zwiazkach-wielickiej-kopalni-z-filozofia}.
}.
Tak jak osoba urodzona i~wychowana w~kopalni sarmatów zna tylko nikłe światło kaganka i~uważa, że wzrok może sięgać
tylko najbliższej okolicy wyznaczonej przez krąg tego światła, tak jest najczęściej z~naszymi niekrytycznymi poglądami
teologicznymi. Teologia z~perspektywy nieskończoności wymaga przebudowy pojęciowej filozofii leżącej u~jej podstaw.
Filozofia scholastyczna, stanowiąca do dziś ukrytą bazę koncepcyjną teologii, operowała bowiem pojęciami i
konstrukcjami finitystycznymi z~mistrzowską wirtuozerią omijając problemy nieskończonościowe\footnote{Doskonałym
przykładem jest pojęcie wieczności np. w~ujęciu św. Tomasza z~Akwinu. W~tym ujęciu nie pojawiają się żadne pojęcia
nieskończonościowe, ale odpowiedni charakter wieczności jest oddany poprzez przesunięcie tego pojęcia na inną
płaszczyznę niż pojęcia czasowe. Jest to konsekwencją przyjęcia starożytnej zasady głoszącej, że nieskończoność jest
niedoskonała. Adaptacja finitystycznej filozofii greckiej do teologii chrześcijańskiej opartej na koncepcji
nieskończonego Boga musiała więc rodzić nieuniknione problemy, które zasadniczo omijano za pomocą pomysłowych
konstrukcji pojęciowych. Po raz pierwszy przed Leibnizem pojęcia nieskończonościowe próbował wprowadzić do teologii
Mikołaj z~Kuzy (1401-1464).}. Na gruncie teologii nowożytne nauki mogą więc jeszcze rozjaśnić tak wiele spraw.
Mentalnie wciąż bowiem tkwimy w~kopalni sarmatów.

Następne trzy rozdziały są niepublikowanymi wcześniej tekstami, łączy je również bardziej osobisty ton.
„Czasoprzestrzeń i~życie wieczne” rozpoczyna się od przybliżenia koncepcji czasoprzestrzeni, aby później na jej
podstawie zbudować wizję teologiczną. Otrzymujemy próbę nakreślenia eschatologii pisanej w~języku zaczerpniętym ze
współczesnych nauk. Mamy więc próbę racjonalizowania przekazu wiary, podjętej bez kompleksów wobec nauk i~ich
wewnętrznej filozofii. Owszem, w~niektórych aspektach może być ona dyskusyjna, ale tylko dzięki takim zabiegom
teologia może zostać wyrażona w~języku zrozumiałym dla współczesnego intelektualisty. W~języku, który pozwoli bez
niepotrzebnych kompromisów intelektualnych krytycznie myśleć o~swojej wierze. Szkoda tylko, że dziś tak rzadko ktoś
podejmuje się ujęcia wizji teologicznej w~rozsądne, krytyczne ramy filozoficzne\footnote{Na gruncie eschatologii we
współczesnej literaturze polskiej udało mi się znaleźć jedynie jeden przykład porównywalny poziomem do omawianej tu
próby. Mam tu na myśli pankomputacjonistyczną interpretację Witolda Marciszewskiego
\parencite*{marciszewski_wszechswiat_1999}.
%(1999).
}.  

Przedostatni rozdział „Logos Wszechświata i~człowieka” jest bardzo ważny dla wszystkich, którzy czytają publikacje
Michała Hellera. Po raz pierwszy wyszedł bowiem daleko poza bezpieczne granice filozofii w~nauce i~ukazał swoją
dopracowaną wizję teologiczną. Jak sam opisuje, jest to wynik długiego i~powolnego procesu tworzenia spójnej i~co
najważniejsze racjonalnej oraz krytycznej wizji teologicznej (s.~99). Bardzo ważne są metodologiczne deklaracje
Autora~-- wizja jest stale otwarta na rewizję, zatem nigdy nie będzie ona w~stanie ostatecznym. Wizja teologiczna dziedziczy
więc antyfundacjonistyczny i~rewidowalny charakter z~samego rdzenia filozofii Michała Hellera. Autor ma również
świadomość, że ostatecznie rzecz biorąc jest to „hipoteza filozoficzna”, choć wychodzi ona z~perspektywy Objawienia,
spełnia więc kryteria nauk teologicznych
\parencite[s.~54]{dzidek_poznanie_2006}.
%(Dzidek, Kamykowski i~Napiórkowski, 2006, s. 54).

Logos Wszechświata w~ujęciu Michała Hellera dla niektórych czytelników może wydawać się kontrowersyjny. Są to jednak
kontrowersje, które zmuszają nas do myślenia, do mierzenia się rozumowego z~wiarą. Michał Heller jest bardzo
ostrożny~-- wszak łatwo tu o~nieporozumienia. Nie można bowiem przykładać do wizji teologicznej tej samej miary, co do filozofii,
choć łączy je wiele. Wizja teologiczna, przez ideę Boga transcendującego radykalnie ludzką rzeczywistość, w~zasadzie
jest zamierzeniem źle uwarunkowanym epistemologicznie. Egzystencjalnie jednak nie możemy uciec od Wielkich Pytań, nie
możemy uciec od konieczności przyjęcia jakichś rozstrzygnięć. Każdy z~nas ma własną wizję filozoficzną i~teologiczną,
często są to jednak wizje niskiej jakości, które są przypadkowym i~bezkrytycznym zlepkiem różnych obrazów. Konfrontacja
ze spójną, racjonalną wizją, jak ta ukazana w~książce, może być dla czytelnika rodzajem \textit{katharsis}. A~może zamiast
wstrząsu stanie się inspiracją do własnych przemyśleń?

%\enlargethispage{-.5\baselineskip}

Logos jako Pole Racjonalności brzmi być może na początku dość egzotycznie, ale to tylko próba wysłowienia w~języku
polskim tej intuicji, którą w~starożytnym Kościele nazwano greckim pojęciem filozoficznym Logos. Pole racjonalności
nabiera jednak w~tym tekście nie tylko charakteru metafizycznego, logicznego podłoża~-- ma ono również charakter
agatologiczny. Wyraźne są tu inspiracje platońskie i~lei\-bnizjańskie.

%\textls[-10]{
Intrygująca jest zarysowana przez Michała Hellera geneza rzeczywistości~-- ewolucja od stanu Alfa ku stanowi Omega.
Intryguje, ponieważ powraca ona (nieświadomie?) do idealistycznych pomysłów z~XIX wieku, stworzonych przez genialnego
matematyka Józefa Marię Hoene-Wrońskiego (porównanie poglądów M. Hellera z~poglądami Pierra Teilharda de Chardin wydaje
się jednak banalizacją tych pierwszych). Pomijając ewidentne różnice metafilozoficzne, a~także~-- równie ważne~-- różnice
stylu, u~Hellera i~Hoene-Wrońskiego odnajdziemy ten sam hiperracjonalizm Boga-Logosu (u Hellera) czy Absolutu
(u Hoene-Wrońskiego). Używając typowych narzędzi historii filozofii należałoby zakwalifikować wizję teologiczną Michała
Hellera jako odmianę idealizmu filozoficznego. Czy idealizm ten jest zharmonizowany z~pozostałymi obszarami filozofii
Autora? Trudno dziś odpowiedzieć na to pytanie, najlepiej by było, gdybyśmy mogli przeczytać jeszcze kolejne teksty na
ten intrygujący temat.
%}

Ostatni rozdział stanowi swoiste „zejście na ziemię”. Wielkie pytania nie mogą nam przesłonić naszej egzystencji.
Celem jest bowiem dobre życie. Ujawnia się tu jeszcze jedna twarz filozofii Michała Hellera, inna od „analitycznej”
filozofii w~nauce. Filozofia jest również sposobem na dobre życie. Motyw ten pojawił się już w~innej części wspomnianej
trylogii
\parencite{heller_jak_2009},
%(Heller, 2009),
tutaj pełni rolę klamry spinającej rozważania. Są bowiem rzeczy ważniejsze niż Wszechświat i~nauka.
Dla ich głębszego zrozumienia warto sięgnąć po tę książkę, nawet choćbyśmy nie chcieli się zgodzić z~wieloma jej
tezami.


\autorrec{Paweł Polak}


\subsubsection{Bibliografia}\nopagebreak[4]
\end{recplenv}
