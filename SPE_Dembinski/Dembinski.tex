\begin{artengenv}{Bogdan Dembiński}
	{The theory of ideas and Plato’s philosophy of mathematics}
	{The theory of ideas and Plato’s philosophy of mathematics}
	{The theory of ideas and Plato’s philosophy of mathematics}
	{Silesian Universisty}
	{In this article I analyze the issue of many levels of reality that are studied by natural sciences. A particularly
		interesting level is the level of mathematics and the question of the relationship between mathematics and the
		structure of the real world. The mathematical nature of the world has been considered since ancient times and is the
		ongoing subject of research for philosophers of science to this day. One of the viewpoints in this field is
		mathematical Platonism.
		
		In contemporary philosophy it is widely accepted that according to Plato mathematics is the domain of ideal beings
		(ideas) that are eternal and unalterable and exist independently from the subject's beliefs and decisions. Two issues
		seem to be important here. The first issue concerns the question: was Plato really a proponent of present-day
		mathematical Platonism? The second one is of greater importance: how mathematics influences our understanding of the
		nature of the world on its many ontological levels?
		
		In the article I consider three issues: the Platonic theory of ``two worlds'', the method of
		building a mathematical structure, and the ontology of mathematics.}
	{mathematical Platonism, ontology, Platonic Academy.}







\lettrine[loversize=0.13,lines=2,lraise=-0.05,nindent=0em,findent=0.2pt]%
{E}{}volution has provided us with the ability to describe the world appropriately, at least on%
%Używane jest określenie „describe on scale'' (np. W ocenianiu)
%nieznany
%July 10, 2019 5:54 PM
 our scale. However, from the point of view of evolution, there is no reason for us to be able to describe it on other
scales. Science and philosophy convince us that these different scales really exist. The problem is, however, that we
are forced to perceive the world on many scales from the perspective of one particular scale. The questions thus arise:
what does the world look like  on our scale, and what does it look like on other scales? If we focus our attention on
natural sciences, I think that one can agree with the statement that the most accurate description of reality at
various scales is provided by the language of mathematics. If so, then the image of the world depends essentially on
the description that is made in this language. This means, however, that mathematics determines what the world looks
like, especially at those levels that are not directly accessible for us. But is mathematics the same as the structure
of the world, or is it only its subjective image? Already in Antiquity attempts were made to answer
these questions. Particular attention was paid to them by Plato and members of his Academy who initiated the program of
mathematical natural sciences---a program that has become the basis for the development of the technical civilization.
In this way, the Academy has become not only the first university in the history of the world, but also the place where
the most important projectcrucial for the development of science and civilization was created. Its creators, apart from
Plato, were such eminent thinkers as Eudoxos of Knidos, Theaetetus, Menaichmos, Theudius,%
%Theudius of Magnesia (an early member of Academy)
%nieznany
%July 10, 2019 6:15 PM
Leon, Speusippus, Xenocrates, Heraklides of Pontus, 

It is widely accepted in the contemporary philosophy of mathematics, that according to Plato, mathematics is the domain
of ideal beings (ideas) that are eternal, unchanging and exist independently of the subject's decisions
\parencite[pp.12–16]{brown_philosophy_2008}.
%\label{ref:RND4CBsKhvbOj}(Brown, 2008, pp.12–16).
Two issues seem important. The first
one concerns the question: Was Plato really thinking that way? The second one is more important: How does mathematics
influence our understanding of the nature of the world on its many levels of being?

Three things need consideration. The first one is related to the Platonic theory of ``two
worlds.'' The second one---with the method of building a mathematical structure. The third one
concerns the ontology of mathematics.

The popular belief is that in Plato’s philosophy we have two worlds separate from each other, one of which is the world
of phenomena, and the other is the world of ideas
\parencite{dembinski_streit_2007}.
%\label{ref:RNDr6IA7g3JQB}(Dembiński, 2007).
The conclusion derived
from this assumption is that mathematics belongs to the Platonic world of ideas, which is eternal and unchanging. In
this world, mathematical objects obtain the status of ideal beings and their existence is independent from the
subject and phenomena. One can only acquire knowledge about this self-existing world of mathematical objects through
the intellect. Unfortunately, this popular belief is in fact inconsistent with Plato’s philosophy and his understanding
of mathematics. Nevertheless, it has persevered in the history of the interpretation of Platonic philosophy and
has been transferred to the philosophy of mathematics. The reason lies in the adaptation of a simplified
vision of Platonism, which assumes the existence of only two levels of reality, the level of ideas and the level of
phenomena, with the mathematical objects situated among the former. Meanwhile, in his analysis of mathematical objects'
mode of existence, Plato concluded that such objects are related neither to the world of ideas nor to the world of
phenomena. He claimed that they occupy an intermediate position and do not belong to any of them.
\footnote{``Further, he states that besides sensible things and the Forms there exists an intermediate class,
the objects of mathematics, which differ from sensible things in being eternal and immutable,
and from the Forms in that there
are many similar objects of mathematics, whereas each Form is itself unique.''
\parencite[987b]{aristotle_aristotles_1924}.
%\label{ref:RNDgQfk1ifbz1}(Aristotle, 1924, 987b).
}
Instead, they are merely the product of our mind. This position requires a further explanation, because it is of
crucial importance for Plato’s understanding of mathematics, as well as on the contemporary discussions about
mathematical Platonism.

Plato tries to explain his position, first and foremost, in books VI and VII of the \textit{Republic} and in
\textit{Letter VII}. The original starting point is the conviction that the first stage of cognition (including
mathematical cognition) emerges from the observation of nature in its phenomenal form. Then the sensory imaginations
arise, which provide only the images of the phenomena (\textit{eikasia}). We see, says Plato, the reflections of
reality that are born in our senses. They are imperfect and often illusive.\footnote{``first, shadows, and then
reflections in water and on surfaces of dense, smooth and bright texture, and everything of that kind, if you
apprehend.''
\parencite[\textit{Republic}, 510a]{plato_platonis_1955}.
%\label{ref:RND1JxaZbngpl}(Plato, 1955, Republic 510a).
} At the next stage, we try to make these images
credible (\textit{pistis}). We try to confirm the data of sensual experience by examining and observing certain states
of things, from as many perspectives as possible
\parencite[\textit{Republic}, 509d-511e]{plato_platonis_1955}.
%\label{ref:RNDE93aAEVZ5r}(Plato, 1955, Republic, 509d-511e).
Today,
such behavior would correspond to empirical tests. The most important element of this study %
%enabled Plato to discern 
%nieznany
%July 10, 2019 6:32 PM
has given Plato the ability to discern patterns present in phenomena. These patterns indicate the order according to
which the phenomena are organized, as well as the existence of regularities that this order defines. Plato
discusses movement patterns, harmony patterns, or (in relation to human activities) ethical and aesthetic patterns. In
the context of the Platonic philosophy of mathematics, the most important role is played by the movement patterns of
the celestial bodies, based on the man’s ability to recognize them. Plato considers this ability to be the supreme gift
of the gods. Observation of patterns in nature: rhythms, motifs, harmony, symmetry or proportion, directs the subject’s
attention towards their source
\parencite[\textit{Timaeus}, 47a-e]{plato_platonis_1955}.
%\label{ref:RNDnLWaOZKKpl}(Plato, 1955, Timaeus, 47a-e.).
But the source itself is no
longer available for sensual cognition. It is available only for intellectual cognition. On the border between sensual
and intellectual cognition, there is a kind of intuition that Plato describes as ``the suspicion of truth.'' The point is
that from the sensory data, basing on the perceived regularities, patterns and proportions, one is able to formulate
hypotheses regarding their source. Plato calls these hypotheses ``true opinion (\textit{al\=eth\=es doxa})''
\parencite[\textit{Meno}, 85c.98a.97c]{plato_platonis_1955}.
%\label{ref:RNDwzlgx2EGmj}(Plato, 1955, Meno, 85c.98a.97c).
To confirm their value, one is asked to verify them. This is
the task of yet another higher cognitive power, which is the reason (\textit{dianoia}). To put it
simply, the task of reason is to conduct logical chains of inference and to analyze causal relationships, what Plato
calls ``causal splicing (\textit{symploke})''
\parencite[\textit{Meno}, 97e-98a]{plato_platonis_1955}.
%\label{ref:RND1wSQpANotM}(Plato, 1955, Meno, 97e-98a)
 Reason is the
authority of the subject, who plays the essential role in the Platonic philosophy of mathematics. His role boils down
to creating intellectual models of sensually given states of affairs and patterns perceived in nature. These models are
the representation of phenomena and patterns at the level of intellect, based on the abstraction skills
(\textit{aphairesis}) and they constitute the product of the activities confirming the permanent occurrence of a
certain set of features in a certain class of objects.\footnote{„For I think you are aware that students of geometry
and reckoning and such subjects first postulate the odd and the even and the various figures and the three kinds of
angles and other things akin to these in each branch of science, regard them as known, and, treating them as absolute
assumptions, do not deign to render any further account of them to themselves or others, taking it for granted that
they are obvious to everybody.''
\parencite[\textit{Republic}, 510c]{plato_platonis_1955}.
%\label{ref:RNDSFOMeyrE9b}(Plato, 1955, Republic 510c).
} Abstraction concerns the
regularities observed in nature, which in turn were derived at the levels of \textit{eikasia} and \textit{pistis}
(\textit{doxa}). Such models are also mathematical %
%objects? Tu i poniżej
%Piotr Urbańczyk
%July 8, 2019 4:12 PM
objects, according to Plato. The examples are numbers. Euclid, who was educated at the Plato Academy, complies with
Plato’s intuition by defining the number as ``a multitude made up of monads'' (\textit{Arithmos de, to ek monad\=on
sygkeimenon pl\=ethos}). A number is defined by a monad (multiple monads). But what is a monad? It is a model created
at the level of intellect, allowing to describe the infinite multitudes appearing at the level of the senses (things
and their reflections). In contrast to sensual images, which are always different, a monad is ``always equal to every
other, and no different from any other, and has no part in it''
\parencite[\textit{Republic}, 526a]{plato_platonis_1955}.
%\label{ref:RNDEWaOwRPmyD}(Plato, 1955, Republic 526a).
It is a model created by the intellect, presenting structural features of every sensual multitude. The model understood
in this way has a completely different status than counted items and, most importantly, it is the creation of the
object. In this sense too, the numbers are the objects of the object. The same applies to geometrical objects. A line
is a length without a width, a surface possesses only length and width, and a circle is a plane figure contained by one
line comprised of points equally distant from the circle’s center. None of the sensory objects has such properties. One
can consider the internal structure of the model, one can also analyze the relationship of a given model to other
models, one can finally examine which models are possible, which are necessary and which are completely excluded. The
analysis of these models is the subject of the work of mathematicians. However, a mathematician, let alone a
philosopher, cannot avoid asking questions concerning the legitimacy of creating such models. The question arises: What
makes mathematical objects, which are human creations, not arbitrary?

Searching for the answer, Plato appealed to the concept of ideas-measures. He claimed that mathematics can neither
derive its own justification from the phenomena, which are variable and temporary, nor from the arbitrary decisions of
the subject. However, there must be something that guarantees the functioning of the cosmic order, and also ensures the
correctness of the mathematical models constructed. In this context, Plato proposes the %
%Adaptation?
%nieznany
%July 10, 2019 11:08 PM
adoption of the eternal model of the organization of the world, which is created according to unchanging regularities
that define the order of the Cosmos. It was these norms, these measures establishing the model of the cosmic order,
which he called ideas. Today, their equivalent would be the laws of nature and the laws of physics. It is not difficult
to notice that these laws exist differently than phenomena do. A physical law and its implementation have different
ontological statuses. As far as we know, the former is immutable, has the feature of unity (there are no two identical
laws), and does not depend on the decision of the subject. We can only say about such laws that they are, and that they
are always as they are. Plato attributed to this way of existence the name of being, and their mode of being he called
``really real'' (\textit{ont\=os on}). He claimed that beings, ideas, inhabit a separate
reality which constitutes the (eternal) organizational model of the Cosmos, and is the essence of existence of all
phenomenal structures and processes. This model manifests itself in the form of symmetry, proportions, various types of
harmony, which can be understood as defining the essence and behavior of phenomenal structures. The goal of philosophy
and science, says Plato, is to reach the idea-measures, that condition a particular kind of order, regardless of
whether it is cosmic, ethical or aesthetic. Of course, this also applies to the mathematical order. That is why Plato
also postulates the existence of mathematical ideas, that form the basis and cause of mathematical order. However, the
most important is, and what must be always remembered, that these ideas are not mathematical objects themselves.
Mathematical ideas constitute a completely autonomous world, existing outside the world of our mathematics, just as the
laws of physics exist outside the world of physical theories we create. As Aristotle comments, no mathematical
operations can be performed on ideas
\parencite[Metaphysica, 1081a]{aristotle_aristotles_1924}.
%\label{ref:RND7l8LNNPyp8}(Aristotle, 1924, Metaphysica, 1081a).
One can only
examine the relationships that exist between them. However, for such a study the mathematical method with its axiomatic
approach is inapplicable. It is rather the dialectical method, whose purpose is exactly to study the relationship
between the ideas, which is appropriate. Confusion among of mathematical Platonism stems from the unawareness of the
difference between mathematical ideas and mathematical objects. Plato tries to explain precisely that issue in
\textit{Letter VII}.\footnote{For an in-depth discussion of this issue, see:
\parencite[pp.55–110]{dembinski_pozna_2003}.
%	\label{ref:RNDgH1femURQM}(Dembiński, 2003, pp.55–110).
}

Plato considers a simple mathematical object---a circle. We can assign a name to it. It could be changed, because---as he
argues---``none of the objects, we affirm, has any fixed name, [...] nor is there anything to prevent forms which are now
called ‘round’ from being called ‘straight,’ and the ‘straight’ ‘%
%Plato. Plato in Twelve Volumes, Vol. 7 translated by R.G. Bury. Cambridge, MA, Harvard University Press; London, William Heinemann Ltd. 1966. Wzięte z: http://www.perseus.tufts.edu/hopper/text?doc=Perseus\%3Atext\%3A1999.01.0164\%3Aletter\%3D7\%3Asection\%3D343b, ale nie wiem, czy autorowi spodoba się zmiana tłumaczenia.
%Piotr Urbańczyk
%July 9, 2019 1:25 PM
round’; and men will find the names no less firmly fixed when they have shifted them and apply them in an opposite
sense.''
\parencite[\textit{Letters}, 343b]{plato_platonis_1955}.
%\label{ref:RNDS5RcrfAAN2}(Plato, 1955,  Letters , 343b).
Next, we try to formulate a fairly precise definition
of a circle. It should cover everything that is round and circular. Most often, according to Plato, an imperfect
definition is formulated, based on specific wording, which ``inasmuch as it is compounded of names and verbs, it is in
no case fixed with sufficient firmness.''
\parencite[\textit{Letters}, 343b]{plato_platonis_1955}.
%\label{ref:RND4n4JdpjLkY}(Plato, 1955, Letters , 343b).
In the further course
of the procedure, an attempt may be made to build a model or a schema, corresponding to what has been defined. We can
do this by creating thoughtful constructions, presenting drawings or spatial visualizations. Later, the analysis of
thus obtained model and its relationship to other models (mathematical objects) is developed into a special theory,
which includes all previous stages. Theory is the highest degree of cognition.  That level a cognizing subject can
attain thanks to his abilities, i.e. sensual perception, abstraction, and logical analysis. Plato, however, strongly
believes that regardless of the degree of precision available at the above-mentioned levels of cognition, one should be
aware that ``their inaccuracy is an endless topic''
\parencite[\textit{Letters}, 343b]{plato_platonis_1955},
%\label{ref:RNDcxMLzs2MfW}(Plato, 1955,  Letters , 343b),
how much
arbitrariness and uncertainty is associated with them, and how much they depend on the cognizing subject and its
limitations. Meanwhile, mathematical cognition is required to be certain; to be characterized by necessity,
universality and truthfulness. Therefore, there must be some basis, upon which we could justify and validate the four
existing procedures of cognition (name, definition, model and theory). We find it, according to Plato, in%
%Wydaje się Ok
%nieznany
%July 10, 2019 11:48 PM
 the idea of the circle, ``the circle as such'' (\textit{autos o kyklos}). Such an idea must be called a real being
(\textit{alethés òn}), the essence of a thing (\textit{tode ti}). Plato describes it as ``%
%Jeśli mielibyśmy się trzymać przejętego tłumaczenia.
%Piotr Urbańczyk
%July 9, 2019 1:35 PM
the Fifth (\textit{to pempton})''
\parencite[\textit{Letters}, 342a-343d]{plato_platonis_1955}.
%\label{ref:RNDJLB7hWnVGd}(Plato, 1955, Letters 342a-343d).
The circle ``in itself''
exists differently than the one which is the intellectual model or the circle we draw, which the wheelwright creates,
or which we observe in phenomena. The circle ``in itself'' is the highest, unchanging and only measure of all
circularity, a condition for the possibility of creating theories, models, definitions and names %
%Nie jest do końca jasne, o co autor chciał przekazać.
%Piotr Urbańczyk
%July 9, 2019 1:37 PM
involving the circular. The circle ``in itself'', as a regularity, as the measure of the specificity of everything that is
circular, is unique, unchangeable and independent of the subject’s %
%knowledge, convictions?
%Piotr Urbańczyk
%July 9, 2019 1:40 PM
beliefs.%
%Odpowiedź do Piotr Urbańczyk (09.07.2019, 13:40): {\textquotedbl}...{\textquotedbl}
%Beliefs chyba może pozostać w głównym znaczeniu „the feeling of being certain that something exists or is true''
%nieznany
%July 10, 2019 11:55 PM
The same applies to numbers. If we take a numerical idea, for example the ideal number two, then with its help we are
able to determine the essence of each mathematical two. Using the descriptive language, one can say that the ideal
number two defines the structural features of each mathematical two. And the mathematical two is only an intellectual
model created by the subject. If we add two plus two, we are not in the world of ideas, but at the level of our
mathematics. A good description of this situation was presented by Michael Heller. He accepts the distinction between
mathematics with a lowercase ``m'', and Mathematics with a capital ``M''. The former is the mathematics created by man. The
latter is the mathematics which is inherent to nature, and to which we have no direct access. Indirect access is
provided only by means of representation, which is our mathematics, the mathematics with a lowercase ``m''. The
Mathematics with a capital ``M'' corresponds to the level of Platonic, mathematical ideas. This situation is explained by
Plato in the ``Metaphor of the Cave''. We humans are only able to see the world of shadows
\parencite[\textit{Republic}, 514-518d]{plato_platonis_1955}.
%\label{ref:RNDwjC6IKbU9w}(Plato, 1955, Republic, 514-518d).
We know, however, that these shadows are shadows of
something we do not directly see (real figures, fire). Therefore, we are forced to create images, models of what we do
not see. These models come with all the disadvantages and limitations that arise in the subjective process of
cognition. However, according to Plato, there are moments (being a gift of the gods), when we are temporarily given a
somewhat vague, intuitive ``seeing'' of the outlines of regularity, organizing the order of nature.  This are the moments
when someone in the cave suddenly ``senses'' that there is ``something'' which is the cause and condition of the existence
of shadows. Plato describes this moment as the ``conversion of the soul'' (\textit{periagoge tes psyches}). In
such a ``foresight'', however, we are not able to stay for long, instead we return quickly to our shadows, and again
continue the arduous inference from the representation. We return to our mathematical world of shadows, to our
mathematics named with the lowercase ``m''. 

The proponents of mathematical Platonism claim that---according to Plato–mathematical %
%Wcześniej też się pojawiały takie miejsca, gdzie zarówno Tomek, jak i ja, byliśmy pewni, że idzie o „objects'', nie „subjects'', ale nie zawsze zmieniałem, bo pojęcie podmiotu matematycznego też jest istotne dla tych rozważań i nie byłem pewien, co autor miał na myśli. Tu byłoby to już nieco dziwne, więc zmieniam.
%Piotr Urbańczyk
%July 9, 2019 1:56 PM
objects exist independently of the subject. We still need to answer the question: Where did this conviction  %
%Początek tego akapitu prosi się, żeby go przearanżować a z tego zdania zrobić co najmniej dwa.
%Piotr Urbańczyk
%July 9, 2019 2:01 PM
originate?%
%Odpowiedź do Piotr Urbańczyk (09.07.2019, 14:01): {\textquotedbl}...{\textquotedbl}
%popieram
%nieznany
%July 11, 2019 12:04 AM
The answer is simple. Such a belief was born not in the thoughts of Plato but in the thoughts of his successors,
Speusippus and Xenocrates
\parencite[see][]{dembinski_pozny_2010,dillon_heirs_2003}.
%\label{ref:RNDpbYvB44iPJ} (see Dembiński, 2010; Dillon, 2003).
Speusippus decided that the Platonic world of ideas should be inhabited by mathematical
objects.\footnote{``Those who recognize
only the objects of mathematics as existing besides sensible things, abandoned Ideal number and posited mathematical
number because they perceived the difficulty and artificiality of the Ideal theory.''
\parencite[\textit{Methaphysics}, 1086a]{aristotle_aristotles_1924}.
%\label{ref:RNDv7b1rw2c8M}(Aristotle, 1924, Methaphysics, 1086a).
} He assigned to
them all the attributes of an idea: separate existence, eternity, immutability and independence from the subject. He
decided that it is unnecessary to double the worlds and postulated the existence of something beyond just mathematics.
In this way, he put mathematics at the top of the %
%reality?
%Piotr Urbańczyk
%July 9, 2019 2:06 PM
world of Beings. Mathematics took the place of the Platonic world of ideas. In this way, he expected to eliminate the
difficulties associated with explaining the relationship between ideas (ideal numbers, ideal figures) and the objects
of mathematics. He thought that it is sufficient to recognize the objects of mathematics themselves as ideals and that
there is no need to introduce difficult notions of ideal numbers and figures. Aristotle did not consider such a
solution as a good one. Probably because he thought that Speusippus wanted to replace in this way philosophical
cognition of the world---and even its very existence---with mathematics. Aristotle attributed to Speusippus the
claim that the whole philosophy of his time can be reduced to mathematics
\parencite[\textit{Methaphysics}, 992a30]{aristotle_aristotles_1924}.
%\label{ref:RNDjZZARv4iL3}(Aristotle, 1924, Metaphysics, 992a30).
Speusippus stand was strengthened by Xenocrates, another Academy scholar, who decided to replace
ontology with mathematics. He claimed that mathematics is the only acceptable ontology, because the world is in fact
created and constituted according to mathematical patterns and structures.

Ideas are, according to Xenocrates, identical with mathematical numbers. Their geometrical form constitutes geometric
ideas. Aristotle considered this solution to be the worst one. Perhaps he thought so because he accepted Plato’s
conviction that ideas are independently existing entities on which no mathematical operations can be carried out. As he
believed, treating objects of mathematics as ideas would exclude the possibility of the existence of mathematics.%
%Nie wiem, czy dobrze rozumiem to zdanie. Tomek też nie rozumie do końca.
%Piotr Urbańczyk
%July 9, 2019 2:13 PM
However, the proposal of Xenocrates could have a different meaning. Recognizing mathematical objects as ideas,
Xenocrates wanted to draw attention to the deep relationship of ontology with mathematics, where mathematics is
understood as the only acceptable ontology. If the world were ultimately created according to mathematical structures
and patterns, mathematics would be its proper ontology. %
%???
%Piotr Urbańczyk
%July 9, 2019 2:16 PM
In similar manner, for example, R. Penrose and many mathematicians admitting mathematical Platonism. Thus, modern
advocates of mathematical Platonism must remember that by adopting Platonism, they essentially adopt the position of
Speusippus and Xenocrates, not of Plato himself. This, of course, is still Platonism in a broad sense. Nevertheless, it
is not a Platonism understood as Plato’s position.

Above considerations lead to the following conclusions: first of all, Plato’s concept of mathematics is not the one that
is usually referred to by the adherents of mathematical Platonism. Secondly, the mathematics we use is always our human
mathematics created by us for the sake of representing nature, whereas nature itself uses different mathematics, the
outline of which we can see only fragmentarily through intellectual intuition. We will never fully see the Mathematics
of nature, because it is directly inaccessible for us. We know, however, that this ``capital-M'' Mathematics is there,
and that it justifies the existence of our mathematics, the ``lowercase-M'' mathematics of the shadow world. This also
applies to logic that organizes the world on its many levels. We, the people who see only shadows, want to describe and
understand other levels of reality with this view. But these levels exist differently, have a completely different
structure, and different logic. Getting to know them requires different methods. Plato suggested that this situation
should be taken into account, without confusing the %
%?
%Piotr Urbańczyk
%July 9, 2019 2:26 PM
modes of existence at various levels. The most common type of error we make is the attempt to describe other levels
using the methods valid at our own level.  Yet ideas, mathematical objects, and other time-space phenomena %
%Nie wiem, czy dobrze oddałem sens tego zdania, bo trudno go zrozumieć.
%Piotr Urbańczyk
%July 9, 2019 2:32 PM
exist differently. %
%Odpowiedź do Piotr Urbańczyk (09.07.2019, 14:31): {\textquotedbl}...{\textquotedbl}
%Lepiej usunąć to zdanie.
%nieznany
%July 11, 2019 12:17 AM
Today, we begin to understand that the logic valid at the microscale is different from the one at our scale, and
different from the logic at the macroscale. Perhaps it is worth to resort to the intuition of the ancient thinkers who,
as the poets say: are closer to the gods, and see better than us. 




\end{artengenv}
