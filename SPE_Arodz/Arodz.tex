\begin{artengenv}{Henryk Arod\'z}
	{Ehrenfest's Theorem Revisited}
	{Ehrenfest's Theorem Revisited}
	{Ehrenfest's Theorem Revisited}
	{Marian Smoluchowski Institute of Physics \\ Jagiellonian University, Krak\'ow\label{arodz-start}}
	{Historically, Ehrenfest's theorem \parencite*{ehr} is the first one which shows that classical physics can emerge from quantum physics as a kind of approximation. We recall the theorem in its original form, and we highlight its generalizations to the relativistic Dirac particle and to a particle with spin and izospin. We argue that aparent classicality of the macroscopic world can probably be explained within the framework of standard quantum mechanics.}
	{Ehrenfest's theorem, wave packets, quantum vs. classical world.}



%{\large Ehrenfest's Theorem Revisited\footnote{Article based on talk given at the 22\textsuperscript{th} Krak\'ow Methodological Conference ``Emergence of the Classical'', Krak\'ow, October 2018. }}




\section{Introduction}


\lettrine[loversize=0.13,lines=2,lraise=-0.05,nindent=0em,findent=0.2pt]%
{T}{}he principal aim of both classical and quantum mechanics is to describe motions of certain physical objects. Both theories can be very successfully applied to various physical objects, but the sets of these objects do not coincide. As is well known, classical mechanics gives wrong predictions when applied to microscopic objects such as atoms. On the other hand, it seems that quantum mechanics is capable to correctly describe motions of the elementary particles as well as motions of macroscopic bodies, hence it has a wider range of applicability. Nevertheless, there are phenomena description of which requires a theory still more general than quantum mechanics. For example, scattering of elementary particles can lead to creation or annihilation of particles---here quantum field theory is needed. Such a generalization of quantum mechanics to quantum field theory is well-known since the middle of 20\textsuperscript{th} century. There are still some problems with it, but the prevailing opinion is that they concern more its mathematical side than foundations. Another departure from standard quantum mechanics seems to be necessary when an elementary particle interacts with a very complex, perhaps even randomly fluctuating or unstable, environment. Understanding of this case is rather poor. To a certain degree the situation is analogous to the well known partition of classical electrodynamics into the theory of the electromagnetic field in vacuum and the electrodynamics of 
continua with constitutive relations and other additional ingredients. An effective quantum mechanics in continua is still 
under construction. 





 It turns out that classical mechanics can be derived from quantum mechanics as a kind of approximate theory. Such derivations are usually called classical limit of quantum mechanics. There exist several of them, including the discussed below Ehrenfest type classical limit, which dates back to \cite*{ehr}, and is likely the oldest one. Its main feature is that it links solutions of the pertinent fundamental evolution equations, which are the Schroedinger wave equation in quantum mechanics and the Newton equation of motion in classical mechanics. Other kinds of classical limits are carried out on more advanced levels of theory. For instance, one may derive from quantum mechanics the classical Hamilton-Jacobi equation \parencite[see, e.g.][chap.8, sec.34]{WKB}, the Lagrange formalism \parencite[see, e.g.][chap.1, sec.2]{pathint}, or distributions on phase space \parencite[see, e.g.][]{siegel,wignerdistr}.



Our main goal here is to recall the seminal paper \parencite{ehr}, and to show, using modern examples, how fruitful is the invented by Paul Ehrenfest method for deriving classical mechanics from quantum mechanics. It leads to very interesting extended versions of classical mechanics featuring, e.g., a non relativistic particle with spin and izospin, or a relativistic particle with spin, which all emerge from quantum mechanics. Furthermore, Ehrenfest's theorem provides a tantalizing suggestion that perhaps whole classical physics can be recovered as certain approximations to quantum theories. Considering wave packets, we find some arguments that corroborate this idea. 

 

The present article is addressed to audience wider than just theoretical physicists. Nevertheless, certain familiarity with basic equations of classical and quantum mechanics is assumed. 






The plan of the article is as follows. First, we briefly discuss description of states of a particle in quantum mechanics with emphasis on the so called wave packets. Section~3 
is devoted to the original form of Ehrenfest's theorem. In Section~4 we sketch a solution of the main problem with the Ehrenfest method: the lack of relativistic covariance. 
Next, in the Section~5 we discuss certain extension of that theorem, which leads to a less known example of classical mechanics of a point-like particle with spin and izospin. Section~6 contains remarks on applicability of quantum mechanics to macroscopic bodies, including a new argument for practical nonexistence of so called Schroedinger's cats. 












\section{Quantum states of a particle }



Classical mechanics and quantum mechanics address the same issue: description of motions of a set of particles. Such set could consists of just one particle, or a finite number of them. The restriction to finite number of particles is important, because otherwise one would have to use a field theory which is regarded as different from mechanics for several important reasons. Classical and quantum mechanics are structurally similar to each other in the sense that in both theories we introduce a space of states of the particle and we postulate an equation of motion. They differ in the form of equation of motion: in classical mechanics this can be, for example, the Newton equation, while in quantum mechanics the Schr\"odinger equation. Also
 the spaces of states are very different. For instance, for the simplest single, point-like particle it is six dimensional phase space in classical mechanics, and infinite dimensional Hilbert space in quantum mechanics. The different equations of motion, and different sets of measurable properties (called observables) for the same set of particles are possible because the spaces of states in classical and quantum mechanics are not identical. Therefore, we regard this latter difference as the most important one. 

In this article we consider the simplest particles, which we describe as elementary. Particles which possess constituents, for example, hadrons, nuclei, or atoms, are excluded. Physical incarnations of the elementary particles are, e.g. electrons, photons, quarks, or the Higgs particle. 

The term `point-like particle' used above is well justified only in classical mechanics. It refers to the fact that in the simplest case the state of a single particle at fixed time $t$ is given by the position and velocity of the particle. The position is represented by a point in the $R^3$ space. In quantum mechanics the complete description of the state of the particle at a given time $t$ is provided by a smooth wave function $\psi(\vec{x}, t)$ defined on the $R^3$ space.\footnote{For simplicity, we consider here only so called pure states, omitting more general mixed states.} 
There is no reason to relate such a quantum particle with a material point moving in the space. Rather, it should better be pictured as a cloud of matter of a very special kind, which is present at all points where the wave function does not vanish. In particular, it does not have any constant shape or size. The most peculiar feature of the elementary quantum particle is that it can not be destroyed or created in parts in spite of its spatial extension, while, for example, a drop of water can be divided into parts, and one part evaporated without disturbing the remaining parts. Physical processes always involve whole elementary quantum particles, which are single indivisible entities, albeit spatially extended.\footnote{In literature this feature is often referred to as the unitarity.} With such picture of the quantum particle, the often discussed and experimentally verified nonlocality of quantum mechanics is natural and rather obvious feature. We shall return to the question what is the best intuitive picture of the quantum particle in the last section. 

Certain special clouds of quantum matter are called wave packets. Roughly speaking, a wave packet is compact and it consists of a single bit, as opposed to more general quantum states of the particle which, for example, can consist of several non-overlapping compact bits. 
Change in time of any state is described by the Schr\"odinger equation.
It turns out that in the case of particle in empty space typical wave packet expands. For example, the width $l(t)$ of a
 three dimensional (spherical Gaussian) wave packet for a particle at rest is given by the formula [p.66]\parencite{qmtext}
\[ l(t) = \sqrt{l^2_0 + \frac{\hbar^2 t^2}{ m^2 l_0^2}}, \]
where $l_0$ is the initial width at $t=0$, $m$ is the mass of the particle, $\hbar$ is the Planck constant. This formula implies that the velocity of the expansion monotonically increases to the asymptotic value
\[ v_0 = \frac{\hbar}{ m l_0}.\]

 The value of Planck's constant is
$ \hbar = 1.0545 \cdot 10^{-34} J \cdot s$ , and the masses of electron and proton are, respectively, $m_e= 9.1 \cdot 10^{-28} g, \;\;\; m_p = 1.67 \cdot 10^{-24} g $. We would like to draw attention of the reader to the exceedingly small values of these masses. The hydrogen atom $H$---one proton plus one electron, and the hydrogen molecule $H_2$---two hydrogen atoms, also are very very light. If we would like to have hand-picked one milligram of hydrogen gas,\footnote{About 11 ccm at $0^{o}$C and the normal pressure.} adding one molecule $H_2$ per second, it would take about $10^{13}$ years, while the estimated age of our Universe is about $1.4 \cdot 10^{10}$ years. One should be very cautious when extrapolating our picture of macroscopic particles to such tiny objects.

It is instructive to compute the asymptotic velocity $v_0$ for various masses and initial widths. 
Let us first take as the initial width $l_0 = 10^{-8} cm $, which is the typical atomic size. Then, for the electron we find 
 $v_0 \approx 1160 \: \frac{km}{s}.$ For a nucleus with the mass $m= 100 m_p$, $v_0 \approx 6.4 \: \frac{m}{s}.$ However, already for a `speck of dust' of size $l_0=10^{-6} cm$ and mass 
 $m=10^{-4} g$ the velocity is $v_0 \approx 10^{-13} \: \frac{cm}{s} \approx 3.2 \cdot 10^{-10} \frac{cm}{year}.$ This means that the wave packet will increase by 1\% during 30 years. For a drop of water in a fog, $l_0 = 10^{-1} cm$, $m=10^{-2}g,$ and $v_0\approx 3\cdot 10^{-17} \frac{cm}{year}$. Thus we see that the electron in empty space expands very rapidly. On the other hand, the size of the wave packet for the `speck of dust', and also for larger and heavier particles at rest, remains practically constant---the wave packet of appears as a `frozen' blob of quantum matter. 


What happens to the wave packet when we switch on interactions of our quantum particle with other particles? P.~Ehrenfest considered relatively simple case when the interaction is described by a smooth potential $V(\bf{x})$ of a fixed form (thus he neglected backreaction of the particle on the other particles). He proved the theorem which quite often is summarized by saying that in such circumstances the wave packet moves in the space along a trajectory $\bf{x}(t)$ which obeys Newton's equation of motion
\begin{equation} \ddot{\bf{x}}(t) = - \nabla \:V(\bf{x}(t)). \end{equation} 
Strictly speaking, the actual content of the theorem is a bit weaker. Nevertheless, classical equations of the form (1) can be obtained from the theorem after some further steps. 


Our notation is as follows. The dot denotes the derivative with respect to the time $t$. 
The boldface denotes three-dimensional vectors, for
example $\ddot{\bf{x}} = (\ddot{x}^1, \ddot{x}^2, \ddot{x}^3 )$, where $x^1, x^2, x^3 $ are Cartesian coordinates in the space, and ${\bf x} = (x^1, x^2, x^3)$. $\nabla$ is the vector composed of derivatives with respect to the coordinates, i.e., $\nabla = (\partial_1, \partial_2, \partial_3)$, where $\partial_1 = \partial/\partial x^1$, etc., and $\nabla \:V = (\partial_1 V, \partial_2 V, \partial_3 V).$ Summation over repeated indices is understood irrespectively of the level of indices. ${\bf a}{\bf b} = a^i b^i$ denotes the scalar product of the three-dimensional vectors ${\bf a}= (a^1, a^2, a^3)$ and $ {\bf b}= (b^1, b^2, b^3)$. 


\section{The original form of Ehrenfest's theorem}

The seminal paper \parencite{ehr} is entitled ``Bemerkung \"uber die angen\"aherte G\"ultigkeit der klassischen Mechanik innerhalb der Quantenmechanik''. It counts merely two and half pages including the title, abstract and references. In its first half the Schr\"odinger equation for the wave function $\Psi$ is quoted \footnote{In the present paragraph I copy the original notation from \parencite{ehr} in which no special symbol is used for the three dimensional vectors. The Abstract in \parencite{ehr} clearly indicates that the three dimensional case is considered. In particular, $\partial/ \partial x $ above should be identified with $\nabla$, and $\partial^2/ \partial x^2 $ with the Laplacian $\Delta$.}, 
\[ - \frac{\hbar^2}{2m} \frac{\partial^2 \Psi}{\partial x^2} + V(x) \Psi = i \hbar \frac{\partial \Psi}{\partial t}, \]
as well as its complex conjugation. 
 Next it is stated that these equations imply the following relations 
\begin{equation} \frac{d Q}{dt} = \frac{1}{m} P, \;\;\; m \frac{d^2 Q}{dt^2} = \frac{d P}{dt}= \int \! dx \:\Psi \Psi^* (- \frac{\partial V}{\partial x}), \end{equation}
where \[Q(t) \equiv \int\! dx \: x \Psi \Psi^*, \;\;\;\mbox{and} \;\;\;\; P(t) \equiv i \hbar \int \! dx \:\Psi \frac {\partial \Psi^* }{\partial x}. \]
Details of the derivation are omitted, except for the remark that the second relation in formulas (2) is obtained with the help of integration by parts. P. Ehrenfest assumes that the spatial extension of the wave packet is small compared with macroscopic distances 
(nota bene, he uses the name `wave packet' for the product $\Psi \Psi^*$). 

Commenting on his results, P. Ehrenfest underlines similarity of the second relation in (2) to Newton's equation of classical mechanics. He is satisfied with such qualitative correspondence, and does not attempt to make it more precise. In particular, he does not even mention the approximation
\[ \int \! dx \:\Psi \Psi^* (- \frac{\partial V}{\partial x}) \approx - \frac{\partial V(Q)}{\partial Q}, \]
probably because he knew that it would be a hard task to formulate it in a rigorous manner. In fact, this approximation is the subject of numerous nontrivial investigations till nowadays. Only with this approximation the second relation (2) turns into Newton's equation (1) 
if we identify $ Q(t)$ with ${\bf x}(t).$ 


The second part of the paper has the subtitle `Bemerkungen'. It is devoted to the one dimensional Gaussian wave packet for a free particle ($V=0$). Its explicit form is presented, and the spreading out discussed. The paper ends with the observation that in the case of a very heavy particle the Gaussian wave packet expands very slowly, while for proton very rapidly. 

Paper \parencite{ehr} is very important, indeed, for at least two reasons. First, P.~Ehrenfest has shown that quantum mechanics does not contradict classical mechanics, but rather generalizes it---the latter can be regarded as a very good approximation to the former for a large class of physical phenomena. 
Second, he pioneered derivations of various kinds of classical equations of motion from underlying quantum mechanical models. Two important examples of this kind are outlined below. 





\section{Lorentz covariant formulation of the Ehrenfest's method }

There is a problem with Lorentz covariance in the Ehrenfest approach to classical limit: because the standard expectation values do not have clear relativistic transformation law, the classical mechanics derived from Lorentz covariant quantum mechanics based on, e.g., the Dirac equation, is not covariant. Hence, it can hardly be accepted as the correct classical limit. This problem is explicitely pointed out in \parencite{HilWou}. 

It turns out that there exists a modification of the Ehrenfest method which yields Lorentz covariant result \parencite{aro1}. Below we give a description of the results. Our main point here is that there is no single classical mechanics that follows from the underlying quantum theory. Instead, we obtain an infinite sequence of classical theories, which approximate the quantum theory with better and better accuracy and, unfortunately, with a complexity rapidly increasing to the level that renders such classical theories impractical. 



This paragraph contains certain technical details given here for the readers interested in the theoretical physics aspects of the work \parencite{aro1}---it can be omitted by not interested ones. In the improved approach, we start from a new definition of expectation values, which respects the Lorentz covariance. In this definition, the integral over the three Cartesian coordinates $x^1, x^2, x^3$ is replaced by an integral over 
three new spatial coordinates in a special coordinate system in the Minkowski space-time. In this system, the time axis is replaced by 
a time-like line $X^{\mu}(s)$ in the Minkowski space-time. This line will ultimately coincide with the classical trajectory associated with the wave packet. The three new spatial coordinates parameterize the directions perpendicular to this line (in the Minkowski sense).
The Cartesian time coordinate $t$ is replaced by the proper time coordinate $s$ on that line. Next, the Dirac equation is transformed to these new coordinates. The evolution parameter is not the laboratory time $t$, but the proper time $s$. There are certain consistency conditions for the new expectation values which result from the requirement that 
the line $X^{\mu}(s)$ stays close to the wave packet, which evolves according to the Dirac equation. The explicit form of the wave packet is not needed. The consistency conditions imply the classical 
equations of motion for $X^{\mu}(s)$, and for other quantities like spin. Their form is approximate one in the sense that all terms proportional to $1/m^2$ or to higher powers of $1/m$ have been neglected. This is justified because $m$ is assumed to have a large value. We use the Foldy-Wouthuysen transformation. 


The starting point---the Dirac equation for a single electron---has the form: 
\[ \gamma^{\mu} \left(\frac{\partial}{\partial x^{\mu}} + i A_{\mu}\right)\psi + i m \: \psi =0. \]
It replaces the Schr\"odinger equation considered by P. Ehrenfest. 
 $A_{\mu}(x)$ in the Dirac equation denotes the so called four-potential of electromagnetic field. It encodes information about electric and magnetic fields in which the electron moves. By assumption, it does not include the field generated by the considered electron. Furthermore, we assume that the mass $m$ is large, in accordance with the discussion of spreading out of wave packets in Section 2. For convenience, we use the notation in which the Planck constant $\hbar$ and the velocity of light in vacuum $c$ are not visible---as if $c=\hbar =1$ (the notation commonly referred to as `the natural units'). 
We also assume that the particle has unit electric charge. Summation over repeated indices is understood. We use the standard relativistic four dimensional notation as explained in, e.g., \parencite[chap.1--2]{relat}.


The modified Ehrenfest method yields the classical equations of motion which read: 
\[\hspace*{-1.3cm} m \ddot{X}_{\mu} = F_{\mu\nu}\dot{X}^{\nu}+ \frac{1}{2m} \epsilon_{\nu\lambda\sigma\alpha}\dot{X}^{\lambda} (\delta^{\beta}_{\mu}- \dot{X}^{\beta} \dot{X}_{\mu}) \:W^{\sigma} \:\partial_{\beta}F^{\alpha\nu} \] 

\vspace*{-0.7cm}
\begin{equation} \hspace*{1.0cm} + \frac{1}{2m} (\delta^{\sigma}_{\mu}- \dot{X}^{\sigma} \dot{X}_{\mu}) \:C^{\rho\nu}\: \partial_{\rho}F_{\nu\sigma} + \frac{1}{2m} \dot{X}^{\rho} \dot{X}_{\nu} \: C_{\mu\sigma}\:\partial_{\rho} F^{\nu\sigma}, \end{equation}

\vspace*{-0.4cm}
\[ \frac{dW^{\lambda}}{ds} = - \dot{X}^{\lambda} \ddot{X}_{\mu}W^{\mu} + \frac{1}{m} (\delta^{\lambda}_{\mu}- \dot{X}^{\lambda} \dot{X}_{\mu}) \:W_{\nu}\:F^{\mu\nu} \hspace*{2cm} \] 

\vspace*{-1cm}
 \begin{equation} \hspace*{1cm} + \frac{1}{m} (\delta^{\lambda}_{\mu}- \dot{X}^{\lambda} \dot{X}_{\mu})\:Z_{\sigma \rho}\: F^{\mu\sigma,\rho} + \frac{1}{m} (\ddot{X}^{\lambda} P^{\nu}_{\nu} + \ddot{X}_{\nu}P^{\nu\lambda}). \end{equation}
Technical details again: the dot denotes the derivative $d/ ds$, where $s$ is the proper time along the classical trajectory $X^{\mu}(s)$. The proper time replaces the time $t$ present in Eqs.\ (2). Furthermore, $\partial_{\mu}$ stands for the partial derivative $\partial/ \partial x^{\mu}$, and 
$F_{\mu\nu} = \partial_{\mu} A_{\nu} - \partial_{\nu} A_{\mu} $ is the electromagnetic field strength tensor. It is composed of the electric and magnetic fields. $\epsilon_{\nu\lambda \sigma \alpha}$ (the so called totally antisymmetric symbol) is equal to $0, 1, -1$ depending on the values of the Greek indices, for instance, $\epsilon_{0123} =1$. 
The spin four-vector $ W^{\sigma}$ is related to the expectation value of the quantum spin operator. In the particular case of constant electric and magnetic fields, equations (3) and (4) reduce to the well-known Bargmann-Michel-Telegdi equations for a relativistic particle with spin. 

 

The relativistic classical equation of motion for a point-like particle with the unit electric charge ($e=1$) in the external electromagnetic field that is usually given in textbooks has the form:
\begin{equation} m \ddot{X}_{\mu} = F_{\mu\nu}\dot{X}^{\nu}. \end{equation}
It precedes the quantum mechanics and also the concept of spin. We see that 
it is a small part of equation (3) above. Moreover, equation (5) does not take into account the spin of the particle, which in equations (3), (4) is represented by $W^{\mu}$. In many important tasks, for example, in calculations of trajectories of electrons or protons in accelerators, one has to use equations which take into account the spin in order to achieve the desired accuracy---equation (5) is not good enough. In practice, certain simplified version of equations obtained with the Ehrenfest method is used. Such nontrivial and successful applications corroborate the correctness of the attitude that classical equations of motion should be derived from underlying quantum theory. On the other hand, there are many problems in which the spin is not important. In such cases the old equation (5) gives very good predictions for trajectories of the particle. 






The classical variables $ C^{\rho\nu}(s), \;Z_{\rho\sigma}(s), \;P^{\nu\lambda}(s)$ are related to entanglement of quantum observables: position with momentum, position with spin, and momentum with spin, respectively \parencite{aro1}. In principle, also equations of motion for $ C^{\rho\nu}(s), \;Z_{\rho\sigma}(s)$ and $P^{\nu\lambda}(s)$ are needed for the mathematical completeness of the system of equations. They can be obtained with the help of the (modified) Ehrenfest method, but in practice one usually eliminates these variables by making certain simplifying assumptions. For example, in most situations all terms in the second line of equation (3), and also in the second line of (4), can be omitted. Then we do not need equations of motion for $ C^{\rho\nu}(s), \;Z_{\rho\sigma}(s), \;P^{\nu\lambda}(s)$. If the equations of motion for these classical variables were included, one would get even more accurate 
classical approximation to the quantum mechanics, but at the price of having to deal with a much larger set of equations. 



\section{Classical mechanics of a point-like particle with spin and color }


This example of derivation of classical mechanics is interesting because prior to the pertinent quantum theory such a classical theory had not been known at all. Once derived, it has turned out to be a useful tool for theoretical investigations of quark matter. Quarks have special charges, called color and weak izospin, which make them sensitive to the so called non-Abelian gauge fields. Both the non-Abelian gauge fields and the quarks as constituents of the material world were discovered in 1960's and 1970's. Certain particular
version of the non-Abelian gauge field is called the Yang-Mills field. Below we outline the basic features of the classical limit for a quantum particle that interacts with the Yang-Mills field. The resulting classical theory describes motion of a point particle, known as the particle with color or izospin, in certain fixed Yang-Mills field.

Historically, the first attempt to derive classical equations of motion for a point particle interacting with the Yang-Mills field was made by S.K. Wong \parencite*{wong}. 
 Classical state of this particle at given time $t$ is represented jointly by: the so called classical izospin vector $I^a(t)$, where the index $a$ takes values 1, 2 and 3; the position $\mathbf{x}(t)$; and the velocity $\dot{\mathbf{x}}(t)$. The derivation given by Wong does not use the Ehrenfest method. For that matter, it should rather be described as an educated guess based on symmetry principles and algebraic structure of the Dirac equation. In consequence, his equations respect the Lorentz invariance as well as the so called gauge invariance, but they miss the spin of the particle and certain less obvious classical variable, as explained below. We will not present here these equations in order to avoid overloading this article with technical details. Interested reader may consult the original paper by Wong or \parencite{aro2}. 


More systematic derivation is based on the Ehrenfest method \parencite{aro2}. We consider expectation values of the following quantum observables: the position $\hat{\mathbf{x}}$, the so called kinetic momentum $ \hat{\mathbf{p}} - \mathbf{A}^a \hat{T}^a$, the spin $\hat{\mathbf{S}}= (\hat{S}^i)$, the izospin $\hat{\vec{T}}= (\hat{T}^a)$, and the product of the spin and izospin operators $\hat{J}^{ai} = \hat{T}^a \hat{S}^i$. The hat $\hat{}$ means that these objects are operators in pertinent Hilbert space. The indices $a, \:i $ take values 1,2, and 3. The three vectors $\mathbf{A}^a$ represent the Yang-Mills field. They are counterparts of the electromagnetic vector potential $\mathbf{A},$ which is a part of the four-potential $A_{\mu}$ introduced in the previous section, $(A_{\mu}) = ( A_0, \:\mathbf{A})$. 



Expectation values of these observables become the classical dynamical variables. Furthermore, the pertinent quantum evolution equation yields the counterpart of the Newton equation and a few other equations. In this manner we obtain the classical mechanics with the following classical variables that characterize the point particle with izospin: the position $\mathbf{x}(t)$, the velocity $\dot{\mathbf{x}}(t)$, the classical izospin $I^a(t)$, the classical spin vector $\mathbf{S}(t)$, and a novel classical variable 
 $ J^{ai}(t)$. 
 
The novel dynamical variable $J^{ai}(t)$ is the expectation value of the operator $\hat{J}^{ai} $. It can be regarded as the three vectors $\mathbf{J}^a(t)$, $a=1,2,3$, with their components enumerated by the index $i$. In spite of the fact that the operator $\hat{J}^{ai}$ is the product of operators 
$\hat{T}^a$ and $\hat{S}^i$, its expectation value does not have to be equal to the product $I^a(t) S^i(t)$, because in general expectation value of product of operators is not equal to product of expectation values of the operators. 

The Ehrenfest method not only reveals the new classical variable---it also shows that there are relations, traditionally called constraints, between the classical variables, which reflect the fact that the classical variables are defined as the expectation values in the same quantum state $\psi({\bf x}, t).$ These constraints have the following form
\[ 4 J^{i a} S^i = I^a, \;\;\; 4 J^{ia} J^{ib} = (\frac{1}{4} - \mathbf{S}^2) \: \delta_{ab} + I^a I^b. \]




To summarize, applying the Ehrenfest method we have discovered that Wong's equations of motion for the classical point particle with izospin are rather oversimplified version of the more adequate equations. In particular, we have found the new classical variable $J^{ai}(t)$, which appears because the particle possesses both spin and izospin. 








\section{Conclusion and remarks} 

{\bf 1.} Ehrenfest's theorem and its generalizations show that classical mechanics of particles can be reinterpreted in terms of expectation values, with pertinent quantum states being the wave packets. In this way, the relation between classical and quantum mechanics, viewed as the relation between old and new theories, acquires the perfect form: the new theory is more general and more accurate, and it rather encompasses the old one instead of contradicting it in all respects. Furthermore, the method used by Ehrenfest---the emphasis on properties and evolution of expectation values---has turned out to be very fruitful as the tool for improving existing classical theories. In Section 4 we have seen such improvement in the case of classical particle in electromagnetic field. The method can also provide completely new classical mechanics, unknown prior to quantum theory, as discussed in section 5 on the example of particle with spin and izospin. 




{\bf 2.} The enormous success of the Ehrenfest method suggests that perhaps no part of the material world is purely classical, that quantum mechanics embraces all physical phenomena,\footnote{With possible exception for gravitational phenomena. So far there is no experimental evidence for quantum nature of gravitation.} and that the classical world is fictitious in the sense that it exists only as certain theoretical approximation to the real world.\footnote{Here we touch the philosophical problem to what extent it really does not exist. Interesting philosophical analysis of a related problem can be found in \parencite{heller}.}
 Such assumption of absolute quantumness of the seemingly classical macroscopic world leads to the following question: why we do not see in nature isolated macroscopic bodies in typical quantum states such as, e.g., wave packets spatially extended over sizable distances (in literature dubbed `Schr\"odinger's cats'). To explain their absence, one can propose a new theory which deviates from quantum mechanics in the macroscopic world, and essentially coincides with it in the micro-world. The recently popular Continuous Spontaneous Localization theory \parencite{CSL} is of this kind. 
 One should also mention 
the decoherence phenomenon \parencite{zeh,zurek}, in which states of a quantum system are very quickly transformed into the so called mixed states, due to strong interactions with environment. Here the absence of widely extended wave packets of macroscopic particles is explained by the presence of interactions with an environment. Which mixed state (`pointer state') appears at the end of the process of decoherence of a concrete wave packet still is the matter of many investigations. It is a difficult problem, and there are many related hypotheses, some with picturesque names, e.g., `quantum Darwinism' \parencite{zurek2}. The decoherence phenomenon belongs to the realm of effective quantum mechanics in continua, mentioned in the Introduction. 


The author prefers another viewpoint: we think that one can provide an explanation for the apparent absence of quantum phenomena in the macroscopic world using the standard quantum mechanics. An interesting possibility is that such extended quantum states of heavy isolated particles are possible in principle, but that they are hardly achievable in reality. The main difficulty is that a spatially extended state has to be produced as such, because wave packets of very heavy particles practically do not expand. This can be rather difficult task. For illustration, let us consider the following thought experiment. Suppose that we can produce a kind of hydrogen-like `atom' in which the electron is replaced by a heavy (in comparison with electron) particle of the mass $M= 10^{-6} g$, and the proton with an even more massive particle. Next, let us excite it in order to increase its spatial size. Highly excited states close to ionization threshold have a macroscopic size---there is no theoretical upper bound on the size of excited atoms. Finally, we ionize that `atom'---this would provide the heavy particle (`electron') in an extended quantum state of the size of the `atom'. The trouble is that the energy needed for the ionization is of the order $10^{13}$ GeV, as a simple calculation shows, while the highest achievable at present energies of particles are of the order $10^{4}$ GeV only.

Another thought experiment involves quantum harmonic oscillator. This system is ubiquitous in physics---it arises as a very good approximation to many complex systems. Classical harmonic oscillator consists of a particle of mass $M$ subject to a force which increases proportionally to the distance from a fixed point, called the center, to the particle. The strength of the force is characterized by a constant $k$. Quantum theory of such object predicts that the least energy state has the form of a wave packet of the size $l = \sqrt{2 \hbar/ \sqrt{M k}}$. Now, let us take the particle roughly of the size of a droplet of water from a fog. Its radius is $r= 10^{-1} cm$ and the mass $M= 10^{-2} g$. We are interested in situations such that $l$ is much larger than $r$---then the wave packet will be much larger than the classical radius of the particle. Simple calculation shows that the constant $k$ has to be exceedingly small, namely $k \ll 4\cdot 10^{-48} g/s^2$. Sizable force appears only when the distance from the center is of the order $10^{40} cm$. Let us recall that the light year counts about $10^{18} cm$. Construction of such a feeble harmonic oscillator is far beyond the present day engineering. On the other hand, if we take a more realistic value $k =1 \:g/s^2$, the condition $ l \gg r $ is satisfied only if $M \ll 4 \cdot 10^{-50} g$---the mass incomparably smaller than the mass of electron. Such particle certainly is not macroscopic. 





{\bf 3.} Let us return to the question from section 2: what is the best intuitive picture of elementary quantum particle. Such a picture can be very helpful if it is adequate, or very misleading when wrong. In our opinion, many mysteries, controversies, and so called paradoxes that are discussed in literature on quantum mechanics arise in a large part from inadequate images of the quantum particle. As we have written in section 2, we prefer to regard
 the quantum particle as a cloud of quantum matter. Its main feature is that it can be created or annihilated as a whole---it is impossible to have one half of electron. Notwithstanding our views, we admit that there exist other pictures as well. It seems that the most popular one is that actually there exists exactly point-like material particle which has a concrete position in space at each time, but we do not know that position. What is known is merely the probability of finding this point-like particle in a chosen volume of the space. It is calculated as the integral of the modulus squared of the wave function over that volume. We think that by adopting such image of the quantum particle one simply carries over to quantum mechanics the picture from classical mechanics.\footnote{This is precisely what was done in the prequantum theory of atoms with the Bohr-Sommerfeld rules. This theory is in fact classical one. The Bohr-Sommefeld rules serve only as a tool for selecting particular classical trajectories.} This can not be justified, especially if we regard classical mechanics of point-like particles as a secondary theory which is derived from quantum mechanics. Therefore we should base our intuitions solely on the Schr\"odinger equation, and on the actual mathematical representation of the states of the particle as wave functions, forgetting completely about the classical mechanics. 

The picture of point-like quantum particle with concrete yet unknown location in the space may be motivated also by unjustified enhancement of the probabilistic interpretation of quantum mechanics. It is known for sure from numerous experiments that outcomes of measurements are distributed with certain probability, which can be calculated with the help of quantum mechanics if we assume the so called Born rule. The point is that there is no experimental evidence for the probabilistic character of quantum mechanics without invoking an experiment. Thus, we may suppose that it is a specific coupling between the two systems: the quantum particle and a very special physical macroscopic apparatus---the measuring apparatus---that is responsible for the probabilistic nature of outcomes of experiments. We adhere precisely to this view. 

 To summarize, we prefer the picture of elementary particle as a cloud of quantum matter. The probabilistic outcomes of measurements are due to interaction of the particle with a macroscopic measuring apparatus. For us, such views are quite natural corollaries to Ehrenfest's theorem. 




\paragraph{Acknowledgement}
This article is based on a talk given at the 22\textsuperscript{th} Krak\'ow Methodological Conference ``Emergence of the Classical'', Krak\'ow, October 2018. I would like to thank the Organizers of the conference for the invitation to this very pleasant and stimulating meeting. 



 


\end{artengenv}\label{arodz-stop}