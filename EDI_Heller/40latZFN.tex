\begin{editorial}{Michał Heller}
	{40 lat~-- sprężystość młodości i~doświadczenie wieku}
	{40 lat~-- sprężystość młodości i~doświadczenie wieku}
	{40 lat~-- sprężystość młodości i~doświadczenie wieku}
%	{Copernicus Center for Interdisciplinary Studies}
	{40 years of ZFN – the flexibility of youth and the experience of age}
%	{Abstrakt lorem ipsum}
%	{słowo, słowo.}




\lettrine[loversize=0.13,lines=2,lraise=-0.03,nindent=0em,findent=0.2pt]%
{R}{}óżne rocznice skłaniają do historycznych refleksji. Wprawdzie czterdzieści lat od jakiegoś wydarzenia nie zwykło
się specjalnie świętować, ale w~przypadku naukowego czasopisma jest to już na tyle długi okres, że warto mu poświęcić
chwilę zamyślenia.

\textit{Zagadnienia Filozoficzne w~Nauce} zaczynały skromnie, ale ambitnie: od zgrzebnego formatu typu ,,samizdat''
(bibułowy papier, powielacz), poprzez pierwsze przymiarki do komputerowego druku (nadal bibuła), aż do postaci naprawdę
drukowanej. Kolejne numery \textit{Zagadnień} są wiernym świadkiem stopniowych postępów w~polskiej, pokomunistycznej
sztuce drukarskiej: najpierw druk oszczędny, na miarę dostępnych technik, potem stopniowe ulepszenia, by wreszcie dojść
do wysmakowanego, nawet trochę snobistycznego, układu graficznego.

Tytuł \textit{Zagadnienia Filozoficzne w~Nauce} od samego początku zapowiadał pewien filozoficzny program. Że ma to
być coś o~wzajemnych relacjach nauki i~filozofii~-- było oczywiste, ale akcent nie był położony ani na ,,zagadnieniach
filozoficznych'', ani na ,,nauce'', lecz na przyimku ,,w''. Filozofia nauki jest~-- i~była już wtedy~-- dobrze rozwiniętą
dyscypliną filozoficzną. Miała swoje liczne szkoły i~odmiany. Najbardziej wpływową do dziś pozostaje filozofia nauki
zwana (nie całkiem merytorycznie poprawnie) anglosaską lub analityczną, ale analizy metodologiczne wywodzące się z~filozoficznej
szkoły lwowsko-warszawskiej także cieszyły się~-- i~nadal cieszą~-- niemałym poważaniem. Założycielom
czasopisma chodziło o~coś innego~-- o~to, jak tradycyjne pytania filozoficzne są obecne w~badaniach naukowych i~ich
rezultatach. Tradycyjna filozofia, stworzona przez Greków i~przetworzona przez europejską myśl średniowieczną nie tylko
wydała z~siebie nowożytne nauki empiryczne, lecz również wycisnęła na nich swoje ślady. Odczytywanie tych śladów i~odcyfrowywanie
ich znaczeń jest pasjonującym zadaniem, którego zaniedbanie byłoby niepowetowaną stratą dla współczesnej
kultury. Z~czasem dla tego typu uprawiania filozofii przyjęła się nazwa ,,filozofia w~nauce''~-- nie ,,filozofia nauki'',
lecz istotny jest właśnie ten mały przyimek~,,w''.

Jest rzeczą oczywistą, że do realizowania w~ten sposób zarysowanego programu niezbędne jest wykorzystywanie środków
poznawczych i~metodologicznych narzędzi wypracowanych przez filozofię nauki. To jednak nie wszystko. Ślady
filozoficznych pytań rzadko leżą na powierzchni wytworów nauki. Nie jest więc tak, że aby je odczytać, wystarczy
przywołać znajomość tradycyjnej filozofii i~zastosować odpowiednie metodologiczne narzędzia. Filozoficzne tropy
prowadzą często w~głąb naukowych teorii, ich inspiracji i~wniosków. Do tego niezbędna jest dogłębna znajomość samej
nauki (najlepiej, jeżeli wynika ona z~twórczego jej uprawiania). Tylko wnikając głęboko w~tkankę naukowych teorii,
można zidentyfikować ich filozoficzne uwarunkowania i~poddać je trafnej interpretacji

Potrzebne jest także jeszcze inne wsparcie. Historia nauki nie tylko wiąże naukę, poprzez jej rodowód, z~tradycją
filozoficzną, lecz także bardzo skutecznie naprowadza na filozoficzne pozostałości w~dzisiejszych naukowych
dokonaniach. Dlatego też ,,filozofia w~nauce'' ściśle wiąże się z~historią nauki. W~tym mariażu historia nauki nie
sprowadza się do odtwarzania dziejów naukowych odkryć, lecz staje się aktywnym narzędziem badania.

Nawet pobieżne przejrzenie spisów treści poszczególnych numerów \textit{Zagadnień Filozoficznych w~Nauce}
przekonuje, że wśród autorów tego pisma pojawiają się: filozofowie, filozofowie nauki, historycy nauki, matematycy,
fizycy, astronomowie, biologowie i~przedstawiciele innych nauk. Dobre czasopismo to nie tylko kolejne numery drukowane
na papierze lub pojawiające się w~Internecie, lecz także środowisko, jakie wokół niego się skupia~-- kształtuje je i~samo
jest przez nie kształtowane. W~Krakowie, przynajmniej od końca dziewiętnastego wieku, żywe są tradycje
,,filozofujących uczonych'' i~dialogu między przedstawicielami różnych nauk a~filozofami. Kluczowymi pod tym względem są
takie postacie jak: Tadeusz Garbowski, Władysław Heinrich, Joachim Metallmann, Marian Smoluchowski, Władysław
Natanson\mydots\ \textit{Zagadnienia} wpisują się w~zapoczątkowaną przez nich tradycję i~pozwalają jej promieniować poza
Kraków.

Czasopisma tym różnią się od książek, że przeczytaną książkę po prostu odkłada się na półkę, a~czasopismo odradza
się z~każdym nowym numerem i, jeżeli jest dobrze wrośnięte w~środowisko, mimo iż przybywa mu lat, zachowuje sprężystość
młodości i~wzbogaca ją doświadczeniem dojrzałego wieku.
%\begin{flushright}
%	Kraków, 22 lutego 2019 roku
%\end{flushright}

{\raggedleft Kraków, 22 lutego 2019 roku\par}%



%Michał Heller

\end{editorial}