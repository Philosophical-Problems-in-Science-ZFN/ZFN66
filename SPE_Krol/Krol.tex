\begin{artengenv2auth}{Jerzy Kr\'ol, Torsten Asselmeyer-Maluga}
	{Topology and models of ZFC at early Universe}
	{Topology and models of ZFC at early Universe}
	{Topology and models of ZFC at early Universe}
	{\textsuperscript{1}University of Information Technology and Management, Rzesz\'ow, Poland\\
		\textsuperscript{2}German Aerospace Center (DLR), Berlin, Germany}
	{Recently the cosmological evolution of the universe has been considered where 3-dimensional spatial topology undergone drastic changes. The process can explain, among others, the observed smallness of the neutrino masses and the speed of inflation. However, the entire evolution is perfectly smooth from 4-dimensional point of view. Thus the raison d'{\^e}tre for such topology changes is the existence of certain non-standard 4-smoothness on $\mathbb{R}^4$ already at very early stages of the universe. 
		We show that the existence of such smoothness can be understood as a byproduct of the quantumness of the origins of the universe. Our analysis is based on certain formal aspects of the quantum mechanical lattice of projections of infinite dimensional Hilbert spaces where formalization reaches the level of models of axiomatic set theory. }
	{Cosmological model, Exotic $R^4$ and $S^3 \times \mathbb{R}$ in cosmology, 4-exotic smoothness, Models of ZFC, Topological model for inflation, Topological model for neutrino masses, Forcing, QM lattice of projections.}
	{%
		{\flushright\subbold{Jerzy Kr\'ol}\\\subsubsectit\small{University of Information Technology and Management, Rzesz\'ow, Poland}\par}%
		{\flushright\subbold{Torsten Asselmeyer-Maluga}\\\subsubsectit\small{German Aerospace Center (DLR), Berlin, Germany}\par}%
	}









\section{Introduction}
\lettrine[loversize=0.13,lines=2,lraise=-0.03,nindent=0em,findent=0.2pt]%
{T}{}here exist many free parameters of physics which can be determined experimentally though fundamental theoretic derivation of them is still missing. Moreover, knowing such derivation presumably will lead to the fundamental revolution in physics which would rely both on the extension of the standard model of particles (e.g. \cite{Weinberg2018}) and understanding gravity at quantum regime through cosmological data (e.g. \cite{Woodard2014}). Among the parameters in question there are masses of elementary particles, mixing angles, coupling constants and in cosmology the value of the cosmological constant, the speed of inflation or the $\alpha$ parameter in the Starobinsky model. For example we have experimental bounds on the neutrinos masses from PLANCK, Baryon Acoustic Oscillations (BAO) and KamLAND-Zen (Majorana neutrino) experiments \parencite{Neutrino-mass-KmLAND-Zen2016,PlanckCosmoParam2015,Neutrino2015}. The smallest experimental bound for the sum of the three neutrino masses reads
\[ \sum_i m_{\nu_i}\leq 0,12eV\,.\]
A way how these bounds were obtained indicate strongly that successful predicting true values for the masses should deal with cosmology. That is why the following question is in order.

\myquote{
Q1: Do we know any candidate for the model of cosmological evolution which would help determining theoretically the realistic (bound for) neutrino masses?    
}

We do know the seesaw mechanism of generating small neutrino masses, however, the derivation deals with two energy scales as free parameters which are not, however, fundamentally fixed. The Q1 has been answered in the affirmative in the series of papers \parencite{AK2018,AK2014,AK2019} where a suitable cosmological model has been constructed. Hence the immediate additional problem emerges:

\myquote{
Q2: Does the model of Q1 predicts realistic values of some inflationary parameters, like a speed of inflation?
}

The affirmative answer for Q2 has been indeed given recently. The model is based on a smooth differential structure on $\mathbb{R}^4$ which is not standard, i.e. is not any smooth product $\mathbb{R}\times\mathbb{R}^3$. We call such a structure an exotic smooth structure and $\mathbb{R}^4$ with it an exotic $R^4$. There exist infinitely continuum many such different, pairwise nondiffeomorphic, exotic $R^4$'s. Thus mathematics favours dimension 4 in this respect, i.e. any other $\mathbb{R}^n$ for any $n\neq 4$ carries {\em unique} standard smooth structure. The physics of the proposed cosmological model distinguishes one of the structures, namely the one which embeds in, also exotic, K3 surface, i.e.
\[R^4\hookrightarrow K3\oplus \overline{{\rm CP}}^2\,.  \]

\myquote{
Q3: K3 is compact. Does it have any physical, probably cosmological, meaning which would extend the embedded noncompact $R^4$ representing spacetime?
}
After presenting some details of the model we will present certain speculative ideas regarding this issue.

In the second part of the paper we will deal with quantum origins of the smoothness required by the model. Provided, the initial state of the Universe is quantum mechanical, i.e. formulated as usual by a Hilbert space of states, we will analyze the following problem:

\myquote{
Q4: Does the QM formalism know that the universe at large scales is smooth, 4-dimensional and exotic? 
}

Similar question has been recently addressed in \parencite{JKuniverse17,JK2017a}. More thorough analysis will be given elsewhere. We show that the smoothness on $\mathbb{R}^4$, which agrees with QM formalism, has to be exotic. This result strongly supports the proposed model. Questions Q1 and Q2 along with the details of the model will be presented in the next section. The results and arguments concerning Q4 will appear in the subsequent section. We close the paper with the discussion which covers also Q3.

\section{The smooth cosmological model for inflation and neutrino masses}\label{sec:2}
%It is quite successful scenario to apply the Friedman-Robertson-Walker (FRW) cosmology for describing large scale universe with its homogeneous and isotropic structure.
The Friedman-Robertson-Walker (FRW) model has proven to be very successful in modelling a homogeneous and isotropic universe.
The time-like slices define 3-geometries which, in the case of the closed universe, are 3-spheres $S^3$ ($k=+1$) giving rise to the model based on $S^3\times \mathbb{R}$. Since 1979 seminal work by M. H. Freedman \parencite*{Freedman1979} mathematicians have become aware of the existence and basic constructions of smooth manifolds which all are topologically $S^3\times \mathbb{R}$, however, smoothly they are not diffeomorphic neither to each other nor to $S^3\times \mathbb{R}$. Soon after in 1980s mathematicians again found that similar open 4-manifolds exist also for $\mathbb{R}^4$---exotic $R^4$'s. They all are homeomorphic to $\mathbb{R}^4$ being pairwise nondiffeomorphic. Moreover, there is a continuum of mutually nondiffeomorphic classes of exotic open \mbox{4-manifolds} each being homeomorphic with a given open 4-manifold (see e.g. \cite{GS1999}). This and the existence of exotic $R^4$ make the dimension 4 completely distinguished in mathematics unlike the other dimensions where the usual tools of differential geometry and topology work well. Many new techniques have been found and developed within the recent years to understand and explore the phenomenon of exotic 4-smoothness. Among which Casson handles, Akbuluth corks, tools of hyperbolic geometry, handle calculus and many others have become an everyday toolkit of researchers in the field. We do not know any way how to avoid these constructions and replace them by known techniques from other dimensions. That is why it is not a surprise that the variety of methods are being applied in order to recognize physical applications of 4-exotic smooth structures. Two aspects are particularly promising---the dimension 4 is also distinguished by physics and several parameters in cosmology and particle physics call for their fundamental derivation and explanation. 

On the way of searching for such an explanation we have recently proposed the cosmological model based on exotic $S^3\times \mathbb{R}$, i.e. $S^3\times_{\Theta}\mathbb{R}$ \parencite{AK2018,AK2014,AK2019}. Such smooth open 4-manifolds (infinite continuum many of them) can be seen as submanifolds (exotic ends) of exotic $R^4$'s
\[ S^3\times_{\Theta} \mathbb{R}\subset R^4\,. \]
Here $\Theta$ refers to certain homology 3-sphere smoothly embedded in exotic $S^3\times_{\Theta} \mathbb{R}$
and allows for distinguishing between these exotic smooth manifolds. When $\Theta = S^3$ the product is globally smooth and the 4-manifold becomes the unique standard smooth $S^3\times \mathbb{R}$. 
So the base for the cosmological model is to refer to these exotic smooth $S^3\times_{\Theta}\mathbb{R}$ rather than to the standard smooth $S^3\times \mathbb{R}$.
\begin{quotation}
Exotic $S^3\times_{\Theta}\mathbb{R}$ is a smooth 4-manifold, so the cosmic evolution seen from dimension 4 can be considered perfectly smooth. However, the 3-dimensional slices undergoes drastic topology changes
\begin{equation}\label{S3}S^3\overset{1.}{\to} \Sigma(2,5,7) \overset{2.}{\to}P\# P\,,  \end{equation}
which will determine the values of certain physical parameters like neutrino masses.

The 3-dimensional evolution within the standard $S^3\times \mathbb{R}$ is trivially $S^3\to S^3\to S^3$.
\end{quotation}
Here $\Sigma(2,5,7)$ is the Brieskorn homology 3-sphere and $P\# P$ is the connected sum of two copies of the Poincar{\'e} 3-sphere \parencite{AK2018,AK2019}. To pinpoint physics into the model we are taking the radius of $S^3$ in (\ref{S3}) to be of order of the Planck length and the natural energy of that epoch to be Planck energy. Such a choice is natural since the evolution of the universe should start with the quantum Planck era. The topology change 1. is the 4-dimensional cobordism $W(S^3,\Sigma(2,5,7))$ between 3-sphere and the Brieskorn sphere. To make it smooth we need to glue a Casson handle. A Casson handle (Ch) is the infinite geometric construction which becomes the main player in investigating of 4-exotic smoothness. As shown by Freedman every Ch is topologically the ordinary 2-handle, $D^2\times \mathbb{R}^2$, with the attaching region $S^1\times \mathbb{R}^2$ while smoothly there are infinitely many of different exotic Ch's (e.g. \cite{GS1999}). The infinite geometric construction present within any Ch is naturally grouped into the layers indexed by $n\in \mathbb{N}$ and each layer corresponds to the level of a labeled tree defining Ch. Each level $n$ contributes topologically to the change of the length scale by the expression $\sim \frac{\theta^n}{n!}$. $\theta$ is the function of purely topological invariant of the 3-manifold $\Sigma(2,5,7)$, i.e. $\theta =\frac{3}{2\cdot CS(\Sigma(2,5,7))}$ where in the denominator stands Chern-Simons invariant of the Brieskorn homology sphere. The contribution of the entire infinite Casson handle is thus given by \parencite{AK2018,AK2014,AK2019}
\begin{equation}\label{exp1} a=a_0\sum_{n=0}^{\infty}\frac{\theta^n}{n!}\,. \end{equation}
To determine the energy scale of the first topology change 1. in (\ref{S3}) relies on finding the minimal segment of Ch which has to be develop in order to make the cobordism $W(S^3,\Sigma(2,5,7))$ smooth in $S^3\times_{\Theta}\mathbb{R}\subset R^4$. As argued in \parencite{AK2014,AK2019} on the base of Freedman result there are needed 3-stages of Ch: Every Ch is embeddable in its first 3-stages. Thus the shortest possible time change, coordinatized by the levels $n$, reads \parencite{AK2014,AK2019} 
\[ \Delta t= \big(1+ \theta + \frac{\theta^2}{2}+\frac{\theta^3}{6}\big)\cdot t_{\rm Planck} \]
which is the shortest change lowering the Planck energy. This gives rise to the first, below Planck, energy scale
\[E_{\rm GUT}= \frac{E_{\rm Planck}}{1+ \theta + \frac{\theta^2}{2}+\frac{\theta^3}{6}}\,. \] 
Calculating $\theta$ as a function of Chern-Simons invariant, $\theta=\frac{140}{3}$, gives rise to the GUT scale, i.e. $E_{\rm GUT}\simeq 0,67\cdot 10^{15}\, {\rm GeV}$. But this time it is a purely topologically determined energy scale (up to the initial Planck energy). 

Let us turn to the second topology transition in (\ref{S3}), i.e. $\Sigma(2,5,7)\to P\#P$. To make it smooth we need to glue in three Ch's. This is due to the topological decomposition of the boundary of $E8\oplus E8$ as K3 surface $E(2)$ \parencite{AK2018,AK2019}. Thus the second topology change follows from the embedding of exotic $R^4$ into $E(2)\# \overline{CP^2}$. Taking into account the entire infinite stages of these Ch's, and relating the result to the initial Planck energy, gives rise to the result 
\[E_2=\frac{E_{\rm Planck}\cdot \exp\big(-\frac{1}{2\cdot CS(P\# P)}\big)}{1+\theta +\frac{\theta^2}{2}+\frac{\theta^3}{6}}\,. \]
This energy $E_2\simeq 63\, {\rm GeV}$ is of the order of the electroweak energy scale (or the half of the Higgs mass). Again, the result is supported topologically and the exotic smoothness of $R^4$ is the main reason for this support. 

Given the two topological energy scales we are using them to evaluate the neutrino masses. Applying the simplest seesaw mechanism we are taking the mass matrix 
\[
\left(\begin{array}{cc}
0 & M\\
M & B
\end{array}\right)
\] with energy scales $M\sim {E\rm GUT}\simeq 0,67\cdot 10^{15}\, {\rm GeV}$ and $B\sim E_2\simeq 63\, {\rm GeV}$ introduced. Then the mechanism relies on calculating the eigenvalues which read \[
\lambda_{1}\approx B\qquad\lambda_{2}\approx-\frac{M^{2}}{B},
\] thus the neutrino mass is predicted as 
\[m=\lambda_2\simeq 0,006\, {\rm eV}.   \] The value agrees with the current experimental upper-bounds (e.g. \cite{PlanckCosmoParam2015,Neutrino2015}). It is, however, the value protected by the topology of the unverse which underlies certain nonstandard (exotic) smooth differentiable structure of $R^4$. 

It is quite interesting that the formula (\ref{exp1}) allows for determining the $e$-folds number $N$ for the inflation in such a topological model \parencite{AK2014,AK2019}. Namely
\[N=\frac{3}{2\cdot CS(\Sigma(2,5,7))} + \ln 8 \pi^2 \simeq 51 \] which is experimentally acceptable and, as the model explains, it is again topologically supported value.
\section{Very early universe and the evolution of the models of ZFC}
We have seen that
%exotic smoothness on $S^3\times \mathbb{R}$ replacing the standard one
the replacement of the standard smooth structure by the exotic one $S^3\times \mathbb{R}$
has tremendous impact on understanding of the cosmological evolution of the universe and leads to determining neutrino masses along with GUT and electroweak energy scales. All this relies upon the initial conditions settled as quantum Planck energy scale and the Planck length radius of $S^3$ so that the question Q4 from the Introduction can be addressed. Namely, does QM formalism determine the large scale smoothness of the evolving universe? Is this smoothness in dimension 4 indeed exotic or rather standard? If the answers indeed support an exotic geometry that would be a strong indication in favour of the entire model. The key tool to attack this problem is {\em formalization}: one goes back to a formal level where set theoretic constructions of QM and differential geometry become important. Especially instead of working in undetermined formally universal space one starts working in specific models of Zermelo-Fr\"ankel set theory with the axiom of choice (ZFC). The models can vary along the evolution of the universe and their relations serve as additional physical degrees of freedom. 

Let ${\cal H}$ be an infinite dimensional separable complex Hilbert space representing states of the quantum world at the Planck era. Infinite dimensionality is enforced by the need to refer to spacetime with momentum and position operators. Then let $\{\bf{L},\wedge,\vee,\neg, 0,1\}$ be the lattice of projections of ${\cal H}$. Local Boolean frames of the lattice are given by maximal complete Boolean algebras of projections $B$'s. If $\dim {\cal H}=\infty $ then each such $B$ is, in general, decomposed into the atomic, $B_a$, and atomless, $B_c$, parts \parencite{Kappos1969}
\begin{equation}\label{B} B=B_a \otimes B_c\,. \end{equation}
$B_c$ is the atomless measure Boolean algebra, i.e. $B_c\simeq {\rm Bor}([0,1])/{\cal N}$ where ${\rm Bor}([0,1])$ is the Borel algebra of subsets of $[0,1]\subset \mathbb{R}$ and ${\cal N}$ ideal of Lebesgue measure zero subsets of $[0,1]$. $B_c$ is homogenous, i.e. $B_c \overset{\rm iso}{\simeq} B_c(p)$ for every $p>0$ where $B_c(p)$ is the algebra of all $q\leq p$ with the unit $p$. It follows that $B_c$ is the universal algebra for all $B$'s which means that 
\[ \forall_{B\subset {\bf L}}\ B{\text{ is completely embeddable in }} B_c \,.\]
Thus in what follows we will use a single symbol $B$ for this atomless, complete, universal measure algebra, replacing the variety of $B$'s in (\ref{B}). 

Let $V$ be a transitive standard universe of set theory and $V^B$ the Boolean-valued class of ZFC (e.g. \cite{Jech2003}). Such a transitive standard model $V$ exists provided ZFC is consistent due to the Mostowski collapsing theorem \parencite{Jech2003}. The construction of $V^B$ is also well recognized and described in a variety of textbooks (see e.g. \cite{Jech2003,Bell2005}). For us $V^B$ is a Boolean-valued model which gathers together Boolean frames $B$'s derived from ${\bf L}$. Maximal complete Boolean algebras $B$'s determine maximal sets of commuting observables of QM based on ${\cal H}$ by virtue of the spectral theorem. Let $\{A_i,i\in I\}$ be a set of commuting observables on ${\cal H}$. The maximality of such a family is equivalent to the existence of a single self-adjoint operator $A$ with the spectral measure $\mu_{\sigma}$ on $S(A)$---the Stone spectrum of the projection algebra $B$. To every set $\{A_i,i\in I\}$ as above there exists a Boolean algebra of projections $B$ with the spectrum $S(A)$. The algebra $B$ generates all observables in the family in the sense that the projections being the values of the spectral measures live in $B$. In the case $B$ of being maximal the family of observables is also maximal (complete) set of observables. Let $L^2(S(A),\mu_{\sigma})$ be the Hilbert space of square $\mu_{\sigma}$-integrable complex-valued functions on $S(A)$. 
\begin{Lemma}\parencite{Boos1996}
The following statements are equivalent:
\begin{itemize}
    \item[i.] There exists a unitary isomorphism $U:{\cal H}\to L^2(S(A),\mu_{\sigma})$ such that 
    \[UAU^{-1}(\psi(x))=x\psi(x)
    \] is the self-adjoint position operator $Q$ on $L^2(S(A),\mu_{\sigma})$.
\item[ii.]  $A$ is maximal.
\item[iii.] $B$ is complete and maximal.
\item[iv.] Every self-adjoint operator $C$ on ${\cal H}$ commuting with $B$, fulfills $C=f(A)$ for some Borel function $f:S(A)\to \mathbb{R}$.
\end{itemize}
\end{Lemma}
We see that there is a strict $1:1$ correspondence between frames of complete sets of self-adjoint operators and maximal Boolean algebras $B$'s. Now let us assign the maximal operator $A$ to every self-adjoint $C$ on ${\cal H}$.
%Subsequently there corresponds to $A$ the maximal complete Boolean algebra $B$.
Subsequently, there is a maximal complete Boolean algebra $B$, which corresponds to $A$.
The important, though obvious, consequence is the following
\begin{Corollary}\label{corr1}
In QM one cannot reduce the resulting family of $B$'s as above to the single-element family $\{B\}$.
\end{Corollary}
The reason is that there exist noncommuting observables determining different maximal families which correspond to different maximal complete Boolean measure algebras $B$'s. Otherwise the correspondence would not be $1:1$.

Now let us assign the family of copies of $V^B$'s to $B$'s according to the lemma above. Again the assignment is irreducible to a single model $V^B$ even though the models $V^B$'s are isomorphic. Each $V^B$ is a Boolean-valued model of ZFC. To reduce it to a 2-valued model $V^{\{0,1\}}$ one should make use of certain homomorphisms \[h_B:B\to \{0,1\}\,. \] 
We want to preserve as much of the structure of $B$ as possible since these algebras are local frames of QM. Each $B$ is atomless complete maximal measure algebra. In particular given a subset $S\subset B$ there always exist maximum $\bigvee S\in B$. We say that $h_B$ preserves completeness of $B$ if for every family $S\subset B$ with $\bigvee S \in B$ ($B$ is complete)
\[h_B(\bigvee S)=\bigvee \{h_B(a):a\in S\}\,. \]
Moreover we want to preserve dense families in $B$. A subset $X\subset B$ is dense in $B$ when 
\begin{gather}
q\in X \wedge p\in B \wedge q\leq p \Rightarrow p\in X\\
\forall p\in B \exists q\in X (p\leq q)\,.    
\end{gather}
Particularly important families in $B$ are generic ultrafilters. A subset $X\subset B$ is a filter on $B$ when
\begin{gather}
  q_1, q_2\in X \Rightarrow \exists z\in X \; {\rm such\; that}\; q_1\leq z \wedge q_2 \leq z \\
 q\in X \wedge p\in B \wedge p\leq q \Rightarrow p\in X\,. 
\end{gather}

\begin{Definition}
A generic ultrafilter on $B$ is a filter ${\cal U}\subset B$ such that for any family $X\subset B$ dense in B
\[X\cap {\cal U}\neq \emptyset \,.  \]
\end{Definition}
Then, the following result hold.
\begin{Lemma}\parencite[p.35]{Solovay1970}\label{hB}
Let $h_B:B\to \{0,1\}$ be a complete homomorphism. Then 
\[h_B^{-1}(1)= {\cal U} \] is a generic ultrafilter on $B$.
\end{Lemma}
Let $V$ be the universe of sets as before and ${\cal U}\subset B$ a generic ultrafilter on $B$ i.e. ${\cal U}\cap X\neq \emptyset$ for every dense subfamily $X$ on $B$ in~$V$.
\begin{Lemma}\parencite[p.7]{Jech1986}\label{UB} $B$ is atomless iff ${\cal U} \notin V$.
\end{Lemma}
From Lemmas \ref{hB} and \ref{UB} it follows
\begin{Lemma}
In $V$: There does not exist any complete $h_B:B\to \{0,1\}$ for the measure algebra $B$.
\end{Lemma}
\begin{proof}$B$ is atomless so there does not exist any generic ${\cal U}$ in $B$ in~$V$.\end{proof}
One way to overpass this no-go property is to relativize ZFC into models of ZFC or allow for changing the universe of sets. Let $V$ be a standard transitive model of ZFC as above. We need two important conditions imposed on $B$ defined in $V$. If $B$ is a complete atomless Boolean algebra in $V$ and $P$ a dense partial order $P\subset B$ in $V$. Then, following \parencite{Solovay1970}, we assume that:
\begin{itemize}
    \item[1.] There exist only countably many subsets of $P$ in $V$.
    \item[2.] $h_B:B\to \{0,1\}$ is said to be $V$-complete if for every family $S\subseteq B$ living in $V$, i.e. $S\in V$, and $\bigvee S\in V$ then 
    \[ h_B(\bigvee S)=\bigvee\{h_B(s):s\in S \} .\]
    \item[3.] A filter ${\cal U}$ on $B$ is $V$-generic when ${\cal U}$ has nonempty intersection with every dense family in $V$.  
\end{itemize}
Then one proves
\begin{Lemma}\parencite[p.35]{Solovay1970}
For every $V$-complete homomorphism $h_B:B\to \{0,1\}$ the set 
\begin{equation}\label{hC} {\cal U}=\{x:h_B(x)=1\} \end{equation} is an $V$-generic filter on $B$.

Conversely, for any $V$-generic filter ${\cal U}$ there exists unique $h_B$ fulfilling (\ref{hC}).
\end{Lemma}
Note that for $B$ atomless still Lemma \ref{UB} forbids the existence of ${\cal U}$ in $V$ so that \[h_B\notin V \text{ and } {\cal U}\notin V\,. \] 
However, due to the relativization of models of ZFC in models of ZFC we can now indicate the model $V'$ extending the $V$ where there live both $h_B$ and ${\cal U}$. This is the random forcing extension of $V$.
\begin{Theorem}\parencite[p.36]{Solovay1970}\label{Th1}
There is a canonical 1:1 correspondence between the reals random over $V$ and $V$-complete homomorphisms of $B$, $h_B\to \{0,1\}$.
\end{Theorem}
As we noted before the Boolean-valued models $\{V^B: B\in {\cal B}\}$ are isomorphic. On the other hand these models cannot be reduced to a single-element family $V^B$. Given the procedure above reducing $B$ to $\{0,1\}$ we are faced with a family of trivially isomorphic algebras $\{0,1\}$ so that they are distinguished by different $V$-generic filters ${\cal U}$'s. As a result we have a family of pairs $\{(V^{\{0,1\}},{\cal U}_{\alpha})_{\alpha\in I}\} $. There exist, however, corresponding reductions of $V^B$ to 2-valued models as in the above family of pairs. This follows from Theorem \ref{Th1}.
\begin{Lemma}
The family $\{(V^{\{0,1\}},{\cal U}_{\alpha})_{\alpha\in I}\} $ is given by the random forcing extensions $\{ V[{\cal U}_{\alpha}], \alpha\in I$\}.  
\end{Lemma}
In this way we avoid just to duplicate isomorphic copies of $V^{\{0,1\}}$. Rather there are 2-valued forcing extensions $V[{\cal U}_{\alpha}]^{\{0,1\}}=V[{\cal U}_{\alpha}], \alpha\in I$ respecting 2-valued algebras and ultrafilters ${\cal U}_{\alpha}$.
Now we can give the construction of a spacetime manifold $M^4$ via local coordinate frames supported by the models of ZFC. Let $M^4$ be a smooth 4-manifold with a smooth atlas $\{ U_{\alpha}\simeq \mathbb{R}^4:\alpha \in J \}$.
\begin{Definition}\label{def1}
We call an atlas $\{ U_{\alpha}\simeq \mathbb{R}^4:\alpha \in J \}$ of $M^4$ {\bf L}-supported if for every $U_{\alpha}$ there exists $V$-generic ultrafilter ${\cal U}_{\alpha}$ and the model $V[{\cal U}_{\alpha}]$ such that the formalisations of $U_{\alpha}$ read
\begin{equation}\label{F} U_{\alpha}\simeq R^4_{V[{\cal U}_{\alpha}]}\,. \end{equation}
If for every local QM frame $B=B_{\alpha}\in {\cal B}$ and its corresponding 2-valued forcing reduction $V[{\cal U}_{\alpha}]$ every smooth atlas of $M^4$ contains all formalisations as in (\ref{F}), then we say that $M^4$ covers smoothly {\bf L}.  
\end{Definition}
Here $R_{V[{\cal U}_{\alpha}]}$ is the unique model of complete algebraically closed field of real numbers in the model $V[{\cal U}_{\alpha}]$.
\begin{Theorem}
If every atlas of a smooth $\mathbb{R}^4$ covers smoothly {\bf L} then such $\mathbb{R}^4$ cannot be the standard smooth $\mathbb{R}^4$.
\end{Theorem}
\begin{proof}
The family $\{B_{\alpha}\}$ of QM frames is not any single-element family (see Corollary \ref{corr1}) so thus the families of $\{V^{B_{\alpha}}\}$ and its 2-valued reductions $\{V[{\cal U}_{\alpha}]\}$. From the smooth covering of {\bf L} property as in Definition \ref{def1} every smooth atlas on $\mathbb{R}^4$ contains a family of $\{U_{\alpha}\}$ with the corresponding formalisations $\{U_{\alpha}\simeq R^4_{V[{\cal U}_{\alpha}]} \}$. Thus every smooth atlas cannot be a single-chart one. 
So we have smooth $\mathbb{R}^4$ whose none smooth atlas is single-chart. Now it is enough to note that any smooth $\mathbb{R}^4$ which is diffeomorphic to the standard $\mathbb{R}^4$ assumes 1-chart smooth atlas. Otherwise it would not be diffeomorphic to the standard $\mathbb{R}^4$.
\end{proof}
So to make agreement between smooth structure on $\mathbb{R}^4$ and QM lattice {\bf L} such that {\bf L} supports this structure requires referring to exotic $R^4$. The standard $\mathbb{R}^4$ cannot cover {\bf L}. If such an agreement took place in the real evolution of the universe the phenomenon of changing models of set theory from $V$ to the forcing extension $V[{\cal U}_{\alpha}]$ should also be a physical process. This more that as we saw in Sec. \ref{sec:2} certain exotic $R^4$ considered as input of the cosmological model allows for predicting the values of important physical parameters like GUT and electroweak energy scales and the neutrino masses. 

\section{Discussion}
One disturbing feature of the presented model is that the exotic $R^4$ generating reliable values of physical parameters is determined by the embedding 
\[R^4\subset K3\# \overline{CP^2}\,. \] What is a physical role ascribed to $K3\# \overline{CP^2}$? One possible answer is to see $R^4$ as a small part of the entire universe which remains outside of our observational capabilities. It is not excluded by current experiments (cf. \cite{AK2018}). However, accepting this point of view there remains the question about the origins of shape and compactness of the 4-dimensional large universe like $K3\# \overline{CP^2}$. One indication is the uniqueness of the $K3$ surface as Ricci flat Callabi-Yau manifold, without closed time-like loops. Possibly certain minimality conditions imposed from general relativity would enforce such structure of the universe. Moreover this is the peculiar and important prediction of our model that the universe at largest scales is compact and based on smooth (exotic) $K3$ surface.

Another possibility is the fundamental role ascribed to exotic \mbox{4-smoothness} on open 4-manifolds in the evolution of the universe, especially exotic $R^4$ and $S^3\times_{\Theta} \mathbb{R}$. This indicates rather technical and purely mathematical appearance of $K3\# \overline{CP^2}$ which, however, determines both spatial topological transitions supporting physical results. Anyway the model shows that exotic smoothness on open \mbox{4-manifolds} appears as new and fundamental tool for physics. Many unanswered so far questions of physics gain new formulations resulting in their resolutions.

It seems quite important to derive these exotic smooth manifolds directly from QM formalism. If succeeded the model would show very strong indication that the differential structure of 4-dimensional spacetime regions \emph{must} be exotic. We showed that the standard structure on large scales of the universe does not agree with QM. Similar approach, though using somewhat different techniques, have been proposed and developed already in \parencite{JKuniverse17,JK2017a}. 
The proposal here is an important step into this direction. There remains, however, to determine precisely this unique exotic $R^4\subset K3\# \overline{CP^2}$ from QM. 



\end{artengenv2auth}

