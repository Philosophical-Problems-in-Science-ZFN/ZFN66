\begin{recplenv}{Kamil Trombik}
	{Czy można dwóm panom służyć?}
	{Czy można dwóm panom służyć?}
	{Dominique Lambert, \textit{Ryzykowne spotkanie teologii z~nauką}, przeł. P.~Korycińska, Copernicus Center Press, Kraków 2018,
		ss.~268.}




Dominique Lambert być może nie jest postacią dobrze znaną większości polskich filozofów, o~ile nie należą do grona
czytelników \textit{Zagadnień Filozoficznych w~Nauce}
\parencite{lambert_relativites_2005}.
%\label{ref:RND4nFmuMtOI8}(D. Lambert, 2005).
Choć belgijski myśliciel ma już w~swoim dorobku kilka uznanych publikacji książkowych, a~także gościł w~naszym
kraju na przestrzeni ostatnich lat\footnote{Dominique Lambert odwiedzał nasz kraj wielokrotnie: 25 kwietnia 2001 r.
wygłosił w~ramach Ośrodka Badań Interdyscyplinarnych wykład pt. \textit{Science and Theo\-logy: Towards New
Interactions}; w~listopadzie 2006 r. brał udział w~konferencji ,,Nauka~-- wiara. Rola filozofii'' w~KUL z~referatem
\textit{The plasticity of Life. Towards philosophical and theological interpretations}; brał również
udział w~międzynarodowym seminarium \textit{Language Logic Theology}, które odbyło się w~Krakowie w~dniach 9–10 grudnia 2011 r.
W ramach posiedzeń Komisji ,,Fides et Ratio'' PAU 15 grudnia 2011 r. wygłosił odczyt: \textit{Towards an articulation of
the science-theology relationship}. W~roku akademickim 2011/2012 Lambert gościł również w~Politechnice Warszawskiej
jako profesor wizytujący. }, pozostaje dla wielu przedstawicieli rodzimej nauki postacią nieco enigmatyczną.
\textit{Ryzykowne spotkanie teologii z~nauką} jest pierwszą jego książką, która doczekała się przekładu na j. polski.
Wprawdzie w~2015 roku wydawnictwo Copernicus Center Press zdecydowało się opublikować jego pracę \textit{The Atom of
the Universe. The Life and work of Georges Lemaître}
\parencite{lambert_atom_2015},
%\label{ref:RNDSW3c1asqkN}(D. Lambert, 2015),
jednak dzieło to
ukazało się na naszym rynku wyłącznie w~j. angielskim. 

D. Lambert jest wychowankiem prof. J. Ladrière’a~-- filozofia nauki, znanego w~Polsce między innymi z~książki
\textit{Nauka, świat i~wiara}
%\label{ref:RNDXHkGC7dOLL}(J. Ladrière, 1978)
\parencite{ladriere_nauka_1978}\footnote{Pierwsze wydanie tej książki w~j.
francuskim ukazało się w~1972 roku. Polski przekład pojawił się w~naszym kraju dzięki Instytutowi
Wydawniczemu PAX w~1978 roku. }, w~której podjęto zagadnienie relacji między naukami przyrodniczymi a~teologią\footnote{Poglądy J.
Ladrière’a były w~Polsce propagowane m.in. przez M. Hellera
\parencite*{heller_wyzwanie_1978}.
%\label{ref:RND7C8ACognRO}(1978).
Sam J. Ladrière gościł
także w~naszym kraju, był m.in. jednym z~uczestników konferencji ,,The Interplay of Scientific and Theological World
Views'', zorganizowanej przez OBI w~dniach 26–30 marca 1996 r. w~Krakowie.}. Podobną tematykę porusza w~omawianej tu
pracy D. Lambert. Belgijski uczony stawia w~niej za cel próbę ,,filozoficznego wyjaśnienia trafności racjonalnych modeli
dialogu nauka-teologia'' (s.~12), przy czym warto od razu nadmienić, że nie jest to próba wyczerpująca. W~związku z~tym
nie sposób traktować tej pozycji jako ,,podręcznika'' do studiowania złożonych relacji między nauką i~religią z~punktu
widzenia historycznego czy problemowego. Nie takie zresztą były ambicje Autora. D. Lambert już we wstępie zaznaczył, że
kwestie podjęte na kartach książki zostały ,,poruszone w~celu wzbudzenia dyskusji i~pokazania na bieżąco, «w trakcie
pracy», że powiązanie nauka-teologia jest do pomyślenia'' (s.~12).

Książka składa się z~części wstępnej, pięciu głównych rozdziałów (Trzy poziomy działalności naukowej; Rozum,
nauka i~stworzenie w~świetle teologii; Trzy tryby interakcji nauki z~teologią; Sakramenty stworzenia; Kilka problemów
współczesnych), oraz wniosków wieńczących pracę. Choć struktura pracy nie wzbudza zastrzeżeń, to duże zaskoczenie może
wywołać fakt, iż książka nie została zaopatrzona w~indeks nazwisk oraz bibliografię. Jest to o~tyle zadziwiające, że
\textit{Ryzykowne spotkanie teologii z~nauką} zostały obficie zaopatrzone w~przypisy, w~których Autor odwołuje się do
dorobku licznych przedstawicieli filozofii, teologii oraz nauki. Nawiasem mówiąc, osadzenie rozważań na temat możliwych
oddziaływań nauki na religię w~tak szerokim kontekście jest jedną z~zalet tej książki\footnote{D. Lambert odwołuje
się w~książce przede wszystkim do literatury francuskojęzycznej, choć zdarza mu się cytować także prace w~j. angielskim. Co
ciekawe, choć nie zaskakujące w~świetle powyższych informacji, znane są mu również dzieła krakowskich
filozofów: M. Hellera (zob. np. s.~68 i~180) i~J. Życińskiego (s.~67).}. 

Z drugiej strony, odczuwalny jest brak odniesień do najnowszej literatury. Omawiany tu przekład bazuje na pierwszym,
francuskim wydaniu z~roku 1999\footnote{Skoro poruszono już kwestię przekładu, to warto odnotować, iż nie wszystkie
cytaty, które przytacza D. Lambert na kartach swojej książki, zostały przełożone na j. polski (zob. np. s.~96, 186--187,
266). }. Problem konfrontacji teologii z~osiągnięciami nauki i~filozofii z~ostatnich dwóch dekad nie został więc
uwzględniony. Nie przesądza to wprawdzie o~wartości książki D. Lamberta, jednak może budzić wątpliwości ze strony tych
czytelników, którzy spodziewali się po tej pozycji syntetycznego spojrzenia na relacje teologii do nauk
szczegółowych w~kontekście najnowszych problemów, jakie generuje rozwój filozofii, przyrodoznawstwa czy technologii. 

Autor proponuje w~swojej książce przyjrzeć się teologii i~nauce z~perspektywy filozoficznej. Analiza interakcji między
naukami a~teologią została poprzedzona refleksją nad fenomenem naukowym i~próbą uściślenia, co wchodzi w~zakres
katolickiej teologii stworzenia i~tzw. teologii rozumu, akcentującej znaczenie poznania naturalnego w~kontekście wiary
religijnej. W~refleksjach tych D. Lambert sięga do instrumentarium różnych dyscyplin filozoficznych. Działalność
naukową rozpatruje więc z~płaszczyzny ontologicznej, epistemologicznej i~etycznej, odwołując się przy tym do
zagadnień z~zakresu filozofii przyrody i~filozofii nauki. Na gruncie przeprowadzonych analiz uznaje, że interakcje między
nauką a~teologią powinny być podejmowane z~zachowaniem kilku reguł metodologicznych, które można streścić w~czterech punktach:
1) twierdzenia nauki nie przeczą dogmatom, a~dogmaty nie uchylają twierdzeń nauki, 2) dialog nauki z~teologią jest
możliwy, ale również konieczny~-- zarówno dla teologii, jak i~dla nauki, 3) dialog nauki z~teologią należy
rozpatrywać w~kontekście etycznym, 4) płaszczyznę dialogu nauki z~teologią tworzy szeroko rozumiana filozofia,
nierezygnująca z~wymiaru metafizycznego i~,,mądrościowego'' (s.~77–91).

W świetle zaproponowanych reguł metodologicznych D. Lambert odrzuca dwa podejścia do ujmowania relacji między
nauką a~wiarą: konkordyzm, zacierający różnice między nauką a~teologią, oraz dyskordyzm (separacjonizm), który przeciwstawia
naukę teologii w~oparciu o~tezę, iż teologia oraz nauki szczegółowe znajdują się na dwóch, całkowicie odmiennych
płaszczyznach ontologicznych i~epistemologicznych. Według belgijskiego uczonego adekwatną analizę interakcji między
naukami a~teologią umożliwia natomiast koncepcja zespojenia. W~świetle tej idei możliwy jest ,,dialog między
teologią i~nauką na gruncie filozoficznym: na płaszczyźnie metafizycznej z~punktu widzenia ontologii, na polu filozofii przyrody z
punktu widzenia epistemologii oraz w~obszarze moralnych pytań o~działalność naukową~-- z~punktu widzenia etyki''
(s.~160). Według D. Lamberta, dzięki koncepcji zespojenia możliwy staje się m.in. opis relacji Boga do świata, który byłby
spójny z~religijną wizją stworzenia, respektowałby regułę naturalizmu metodologicznego i~nie
pozostawałby w~sprzeczności z~wynikami nauk szczegółowych o~świecie. 

Koncepcja zespojenia miałaby okazać się także skuteczna w~podejmowaniu refleksji nad współczesnymi problemami,
pojawiającymi się na styki nauki i~teologii. Autor wskazuje i~analizuje w~tym miejscu rozmaite problemy filozofii
przyrody, antropologii filozoficznej i~filozofii Boga, m.in. zagadnienia związane z~początkiem i~końcem wszechświata,
pochodzeniem życia, uduchowieniem człowieka, genezą zła i~cierpienia. Podejmuje również krótką refleksję nad związkami
matematyki i~teologii, a~także zastanawia się nad rolą i~znaczeniem uniwersytetów katolickich, w~obrębie których
koegzystują~-- obok wydziałów teologicznych~-- ośrodki naukowe prowadzące badania w~dziedzinie nauk
przyrodniczych i~humanistycznych.

Choć zakres tematyczny pracy jest bardzo szeroki, można wskazać kilka kluczowych idei, które tkwią u~podstaw analiz
podejmowanych na kartach tej książki. Gdyby pokusić się o~syntetyczne ujęcie propozycji filozoficznej D. Lamberta, to
można sprowadzić ją do dwóch generalnych tez: 1) między nauką a~teologią zachodzi wiele interakcji, które można
skutecznie analizować przy użyciu narzędzi filozoficznych, 2) możliwy jest owocny dialog nauki i~teologii, w~ramach
którego nie potrzeba naruszać autorytetu nauki, ani przeczyć twierdzeniom teologii.

\enlargethispage{-.5\baselineskip}

Pierwsza teza~-- bazująca wprawdzie na kilku istotnych założeniach~-- bywa kwestionowana, jednak jej akceptacja nie
przesądza jeszcze na korzyć żadnej ze stron, które biorą udział w~tytułowych ,,ryzykownych spotkaniach''. Dopiero
przyjęcie drugiej tezy niesie ze sobą perspektywę możliwego rozwiązania problemu relacji między nauką a~religią, przy
czym powstaje pytanie, na ile będzie to próba akceptowalna ze strony naukowców, teologów i~filozofów reprezentujących
odmienne stanowiska epistemologiczne i~ontologiczne. W~przypadku książki D. Lamberta pozostaje to kwestią otwartą, choć
można odnieść wrażenie, że Autor nie dość wystarczająco poruszył w~swojej pracy fundamentalny problem możliwości
dialogu teologii z~nauką, z~góry przyjmując, że taki dialog jest nie tylko możliwy, ale również owocny\footnote{W tym
kontekście zaskakujące mogą wydać się słowa Autora, który napisał, iż ,,ta książka opiera się na \textbf{nadziei}
[pogrubienie moje], że możliwy jest dialog, który nie okalecza ani prawdy naukowej, ani prawdy teologicznej'' (s.~9).}.
Takie podejście może narazić Autora na dwa zarzuty. Pierwszy z~nich dotyczy tendencyjnego doboru środków do realizacji
celu, jaki powziął belgijski uczony na kartach swojej książki. Czy wizja nauki i~teologii, zaproponowana przez Autora,
nie została przemyślana w~taki sposób, aby odpowiadała logice wywodu, opartym na ,,nadziei, że możliwy jest dialog,
który nie okalecza ani prawdy naukowej, ani prawdy teologicznej''? Drugi zarzut wiąże się z~pytaniem, na ile rozumowanie
Autora~-- w~kontekście odgórnie przyjętego założenia o~możliwym owocnym dialogu nauki i~teologii~-- stało się obarczone
ryzykiem błędu \textit{petitio principii}.

Inna rzecz, że omawianą książkę można potraktować jako próbę spojrzenia na relację nauki i~teologii niejako z~,,wnętrza
chrześcijaństwa''. Zmiana perspektywy nakazywałaby jednak zastanowić się, czy \textit{Ryzykowne spotkanie
teologii z~nauką} to pozycja pisana z~punktu widzenia filozofii, czy już raczej teologii. A~może jest to przykład filozofii
chrześcijańskiej w~praktyce, jeśli oczywiście zaakceptujemy ten termin? Niezależnie od odpowiedzi, tytułowe spotkania
teologii z~nauką w~zamierzeniu Autora miały być nie tylko ryzykowne, ale także owocne dla osób wierzących. W~tym sensie
książka D. Lamberta może spełnić swoją rolę. Jest to bowiem wszechstronna próba analizy istotnych problemów, jakie
rodzą się na styku nauk szczegółowych i~teologicznej refleksji o~Bogu, człowieku oraz otaczającym nas świecie. Co
ciekawe, próba ta momentami zbliża się do propozycji podejmowanych na gruncie tzw. teologii nauki, zainicjowanej przez
M.~Hellera i~J.~Życińskiego, a~rozwijanej po dzień dzisiejszy w~środowisku filozofów krakowskich
\parencite{polak_teologia_2015,obolevitch_problem_2012}.
%\label{ref:RNDXIf9jvSSc3}(P. Polak, 2015; T. Obolevitch, 2012). 

Według D. Lamberta człowiek może służyć dwóm panom; oddawać należną cześć nauce, a~jednocześnie uważać, że teologia
formułuje wartościowe poznawczo sądy. Nie wszyscy podpiszą się pod takim stanowiskiem, jednak nie zmienia to faktu, że
prowokuje ono do stawiania kolejnych pytań dotyczących możliwych interakcji pomiędzy nauką a~wiarą. ,,Kto nie ryzykuje,
ten nie pije szampana''~-- mówi znane powiedzenie. Nawiązując do poruszanego w~recenzji problemu~-- być może jeszcze nie
nadeszła pora, by otwierać butelki, ale im więcej ryzykownych spotkań, tym lepiej. Zarówno dla nauki, jak i~dla
teologii. 



\autorrec{Kamil Trombik}


\subsubsection{Bibliografia}\nopagebreak[4]
\end{recplenv}


