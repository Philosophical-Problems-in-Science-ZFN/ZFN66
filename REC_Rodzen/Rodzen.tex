\begin{recplenv}{Jacek Rodzeń}
	{W stronę odnowionego wizerunku Newtona}
	{W stronę odnowionego wizerunku Newtona}
	{Niccolò Guicciardini, \textit{Isaac Newton and Natural Philosophy}, Reaktion Books, London 2018, ss.~268.}
		




Od dziesięcioleci nie słabnie wśród historyków i~filozofów nauki zainteresowanie życiem oraz dziełem Isaaca Newtona.
Ostatnie dwie dekady przyniosły nowe ożywienie w~badaniach nad dokonaniami autora \textit{Zasad}, z~racji zarówno
pogłębionego wglądu w~mało znane uprzednio, nieopublikowane jego manuskrypty (niemal w~całości udostępnione już w~formie
cyfrowej w~profesjonalnych projektach internetowych), jak i~nieuprzedzonego otwarcia się na niepodejmowane dotąd
szerzej nowe obszary tematyczne. Wzorem trudu włożonego w~latach 70. i~80. ubiegłego wieku przez takich wybitnych
historyków, jak I.B. Cohen, A.R.~Hall, D.T. Whiteside czy R.S. Westfall, odkrywających głównie meandry newtonowskiej
matematyki i~fizyki, wysiłek współczesnych nam autorów koncentruje się na pracach alchemicznych uczonego z~Woolsthorpe
(B.J.T. Dobbs, W.R. Newman, L. Principe), pismach teologicznych (R. Iliffe, S. Snobelen), pracach z~zakresu chronologii
(J.~Buchwald, M. Feingold), czy tematach filozoficznych (H. Stein, S. Ducheyne, A. Janiak). Studia te zdają się
dopełniać ,,portret Isaaca Newtona'' (idąc za tytułem znanej jego biografii pióra F.E. Manuela) nie tylko jako
matematycznego geniusza, lecz także człowieka swojej epoki, żywotnie zainteresowanego wieloma obszarami ówczesnej
wiedzy i~stawiającego te same pytania, co jemu współcześni.

Z kilku najnowszych opracowań dorobku badawczego Newtona na uwagę zasługuje książka autorstwa włoskiego historyka
nauki z~Uniwersytetu w~Bergamo, znawcy newtonowskich metod matematycznych, Niccola Guicciardiniego, zatytułowana
\textit{Isaac Newton and Natural Philosophy}. Stanowi ona próbę syntetycznego przedstawienia, w~ujęciu chronologicznym,
najważniejszych dokonań badawczych autora \textit{Zasad}. Książka Guicciardiniego stanowi, wzbogaconą o~wyniki
najnowszych badań, wersję włoskojęzycznej pracy \textit{Newton} (rzymskie wydawnictwo Carocci) z~2011
roku. Z~wymienionych w~bibliografii wydań literatury przedmiotu (także dat dostępu autora do obszernego materiału
źródłowego z~baz cyfrowych) oraz samej treści pracy wynika, że reprezentuje ona stan badań nad myślą
uczonego z~Woolsthorpe w~przybliżeniu na rok 2015. Jest to ważne zwłaszcza w~świetle wspomnianego
znacznego przyrostu ilościowego literatury
newtonologicznej w~ciągu ostatnich lat. 

Jak już zostało wspomniane, narracja Guicciardiniego ma charakter chronologiczny z~akcentem na reprezentatywne dla
danego okresu aktywności Newtona jego dokonania badawcze. Nie brakuje też odniesień autora do burzliwego tła
epoki w~dziejach Anglii XVII i~początków XVIII wieku, w~której przyszło autorowi \textit{Zasad} wzrastać, dokonywać swoich
odkryć, wreszcie włączyć się samemu w~nurt działań propaństwowych (jako posłowi do parlamentu angielskiego,
nadzorcy i~kuratorowi Mennicy Królewskiej, wreszcie prezesowi Towarzystwa Królewskiego). Tak więc pod piórem Guicciardiniego
widzimy drogę zainteresowań młodego Newtona jako kilkunastoletniego budowniczego prostych zabawek
mechanicznych i~bardziej skomplikowanych zegarów słonecznych, studenta wchodzącego twórczo w~meandry ówczesnej
matematyki w~Trinity College i~prowadzącego pierwsze eksperymenty optyczne, następnie młodego profesora
matematyki w~Cambridge. Możemy śledzić newtonowskie analizy tekstów alchemicznych i~jego prace w~domowym
laboratorium, budzące się
zainteresowania teologią i~wczesną historią Kościołów chrześcijańskich, a~także losy Newtona jako
urzędnika i~,,pokornego sługi Korony''.

Ślad bardziej osobistych zainteresowań Giucciardiniego jest zauważalny w~obszerniejszym przybliżeniu czytelnikowi
drogi Newtona do sformułowania matematycznej teorii ruchu. Towarzyszy temu, siłą rzeczy, tok rozumowania, który
odsłania nie tylko meandry myśli matematycznej uczonego z~Woolsthorpe, ale także preferencje interpretacyjne samego
autora recenzowanej książki. W~krótkiej prezentacji koncepcji ruchu u~Newtona, Guicciardini wychodzi poza tradycyjne
odniesienia do jego znanych rozpraw matematycznych (opublikowanych dopiero po 1700 roku) oraz konstrukcji
argumentacyjnej \textit{Matematycznych zasad filozofii naturalnej}, w~których ewoluowały pojęcia fluent, fluksji czy
,,metody stosunków wielkości pierwszych i~ostatnich'', charakterystyczne dla newtonowskiego wariantu rachunku
różniczkowego i~całkowego. Włoski historyk zwraca uwagę na rzadko przywoływaną przez badaczy \textit{Przedmowę autora}
do \textit{Zasad}, następnie wydaną przez J. Whistona w~1707 roku pracę \textit{Arithmetica Universalis}, a~także mniej
znane manuskrypty, jak np. \textit{Geometriae Libri Duo }(z ok. 1693 roku). 

Guicciardini zwraca uwagę na, zawarte w~\textit{Przedmowie autora} do \textit{Zasad}, kluczowe dla Newtona,
rozróżnienie między mechaniką a~geometrią (s. 154-155). Dla uczonego z~Woolsthorpe geometria, choć zajmuje się liniami
prostymi i~okręgami ,,nie uczy nas, jak rysować te linie, lecz zakłada, że są narysowane''. Kwestia wykreślania prostych
i okręgów to zadanie (konstrukcyjno-rzemieślnicze) mechaniki. W~ten sposób geometria w~ujęciu Newtona jest
podporządkowana mechanice (zauważmy, że odwrotnie aniżeli w~przypadku wielowiekowej tradycji arystotelesowskiej). Jak
zauważa włoski historyk, Newton wyjaśnia w~\textit{Przedmowie} zadania szczegółowe mechaniki w~kontekście teorii ruchu:
,,Mechanika racjonalna będzie nauką o~ruchach wynikających z~jakichkolwiek sił oraz o~siłach potrzebnych do wytworzenia
jakichkolwiek ruchów''. Wiąże się z~tym, według Newtona, zadanie nowej filozofii przyrody: obserwując ruchy wyznaczać
powodujące je siły, a~następnie na podstawie tych sił przewidywać nowe zjawiska ruchu. 

Zdaniem Guicciardiniego nowe ujęcie filozofii przyrody wyraźnie nawiązuje nie tylko do tradycji geometryzacyjnej
starożytnych, ale także stanowi polemikę z~poglądem Kartezjusza, według którego mechanika (podobnie jak w~szkole
perypatetyckiej) podporządkowana jest geometrii. Problemy geometrii, w~szczególności dotyczące krzywych,
zgodnie z~programem autora \textit{Rozprawy o~metodzie}, powinny być rozwiązywane przede wszystkim z~wykorzystaniem metod
algebraicznych. Nie zgadza się z~tym poglądem Newton, który pod koniec pracy \textit{Arithmetica Universalis} wyraźnie
zaznacza, że krzywe powinny być wytyczane raczej przez ruch, aniżeli definiowane równaniami. Myśl tę
rozwija w~manuskrypcie \textit{Geometriae Libri Duo}. Newton stwierdza w~nim, że różne bryły geometryczne,
takie jak kula, stożek
czy walec najlepiej przedstawiać nie przez równania, lecz przez ,,racje ich powstania''. Guicciardini podsumowuje:
,,Geometra, który poznał mechaniczną genezę krzywych, ma epistemologiczną przewagę nad algebraikiem: on zna naturę
krzywych ponieważ opanował ich konstrukcję. Newton zdaje się sugerować, iż wiemy, co skonstruowaliśmy, a~nie co
obliczyliśmy'' (s.~118).

Autor recenzowanej książki zwraca także uwagę na kolejne konsekwencje strategii geometryzacyjnej przyjętej przez
Newtona. Autor manuskryptu \textit{Geometriae Libri Duo} pisze w~nim: ,,wszelkie figury płaskie, czy to wytworzone przez
Boga, przyrodę czy jakiegoś mechanika, są mierzone przez geometrię, zgodnie z~założeniem, że są one dokładnie
skonstruowane''. Tak więc krzywe, które ma na myśli Newton, mogą być wykreślone nie tylko przez rzemieślnika przy pomocy
odpowiedniego przyrządu, ale także przez Boga (przychodzi tu na myśl wyrażenie z~\textit{Przedmowy autora} do
\textit{Zasad}: ,,najdoskonalszego mechanika ze wszystkich mechaników'') lub przyrodę. W~związku z~tym Guicciardini pyta:
,,Czy więc tak jak cyrkiel znajduje się w~rękach rzemieślnika, tak siły [przyrody~-- J.R.] są w~‘rękach’ Boga?'' (s. 156).
W tym przypadku mogą nasuwać się różne odpowiedzi. Można więc sądzić, iż matematyczna teoria ruchu i~siły miała nie
tylko cele teoretyczne i~praktyczne (np. wyznaczanie orbit planet, księżyców i~komet). Stanowiła dla Newtona swoiste
odniesienie do refleksji sięgającej kwestii filozoficznych i~teologicznych. W~tym kontekście łatwiej jest zrozumieć nie
tylko jego polemikę z~poglądami Kartezjusza, ale także krytykę Leibniza oraz znaną powszechnie
korespondencję z~R. Bentleyem.

Oprócz syntetycznego omówienia najważniejszych dokonań Newtona, Guicciardini stawia także
tezy i~pytania o~charakterze bardziej historiograficznym, odnoszące się do aktualnego stanu naszej
wiedzy o~autorze \textit{Zasad}.
Jednym z~tych pytań jest to, jak badacze i~zwykli odbiorcy tekstów popularnonaukowych mają sobie poradzić z~obrazem
,,dwóch Newtonów'' (s. 19), powstałym po spektakularnej sprzedaży nieznanych uprzednio jego manuskryptów
alchemicznych i~teologicznych w~londyńskim domu aukcyjnym \textit{Sotheby’s} w~1936 roku, a~następnie
po blisko osiemdziesięciu latach
ich skrupulatnych badań przez historyków? Zdaniem Guicciardiniego obraz Newtona-matematyka i~fizyka jest do
pogodzenia z~obrazem
Newtona-\mbox{-alchemika} %notabene
i~teologa, choć nie za cenę jakiejś nadrzędnej unifikacji jego myśli.
Dopowiada też, że
,,Newton był człowiekiem swojego czasu […]. Jego pasje i~badania […] nie wydawały się czymś niezwykłym dla
współczesnych, chociaż jego wnioski i~metody czasami mogły ich dziwić. Z~pewnością nie był to człowiek, który lubił
powtarzać to, co nauczył się od innych, ale oryginalny myśliciel nieustannie zaangażowany w~odkrywanie czegoś nowego,
czy to twierdzenia matematycznego, czy też interpretacji jakiegoś fragmentu biblijnego'' (s. 20). 

Czymś w~rodzaju odpowiedzi na powyższe pytanie, choć w~podtekstach przewijającej przez całą pracę, włoski historyk
podzielił się dopiero przy jej końcu: ,,[Newton~-- J.R.] był przede wszystkim rozwiązywaczem problemów (\textit{a
problem-solver}), dumnym z~tak skutecznego posługiwania się technikami matematycznymi, alchemicznymi oraz z~zakresu
hermeneutyki biblijnej. […] Jego postawa anty-filozoficzna (\textit{anti-philosophical stance}) powodująca
beznadziejność wszelkich prób określenia go jako platonika lub empirystę, socyniana lub deistę, bierze się
zarówno z~dumy przynależności do wysoce wyspecjalizowanych cechów praktyków, jak i~z~podzielanego przez niego minimalizmu
religijnego'' (s.~229-230). Teza ta jest niewątpliwie interesująca i~pobudzająca do dalszej dyskusji. Zachęcają do tego
co najmniej dwie, poruszone w~niej przez autora recenzowanej książki kwestie. Chodzi o~rozumienie przez niego statusu
Newtona jako praktyka oraz jego faktyczny stosunek do filozofii. 

Praktyczny wymiar prac autora \textit{Zasad} może być kwestią umowy, choć w~jednym miejscu swojej książki (s. 47)
Guicciardini zdaje się nawet niedwuznacznie sugerować ,,pokrewieństwo'' Newtona z~grupą angielskich tzw. matematyków
praktyków (termin spopularyzowany przez Evę Taylor w~latach 50. XX wieku, która \textit{notabene} wprost zaliczyła
Newtona do tej grupy \parencite[zob.][]{taylor_mathematical_1954}), a~więc m.in. mierniczych, nawigatorów, artylerzystów, budowniczych fortyfikacji itd. Dziś
niektórzy historycy zaliczają do tej grupy m.in. Galileusza, S. Stevina i~C. Huyghensa. Jednak w~odróżnieniu od
nawigatorów i~mierniczych Newton nie pobierał uposażenia pieniężnego np. za budowę przyrządów
obserwacyjnych i~pomiarowych (choć skonstruował samodzielnie m.in. przyrząd do wykreślania krzywych,
teleskop zwierciadłowy i~oktant
żeglarski) lub za wykonywanie zleconych pomiarów lub obserwacji. Był zatrudniony na Uniwersytecie w~Cambridge na
stanowisku profesora w~Katedrze Lucasa, a~potem został wysokim urzędnikiem Mennicy Królewskiej. Nie znaczy to
oczywiście, co podkreśla także Guicciardini, że Newtonem zupełnie nie powodowały cele praktyczne lub quasi-praktyczne.
Istotnie, powodowały zarówno w~przypadku jego zainteresowań szeregami nieskończonymi, fluentami i~fluksjami,
jak i~w~kwestii, opartej na idei powszechnego ciążenia, teorii ruchu Księżyca, która m.in. miała służyć do wypracowania
skutecznej metody określania długości geograficznej na morzu (sam E. Halley stwierdził kiedyś, iż \textit{Zasady}
Newtona mogą być przydatne w~praktyce żeglarskiej).

O ile jednak praktyczna strona zainteresowań Newtona jest raczej kwestią faktów historycznych, o~tyle jego
zaangażowania filozoficzne mogą być bardziej uzależnione od interpretacji lub nastawienia
historiograficznego. W~przytoczonym powyżej cytacie włoski historyk wspomniał o~,,postawie antyfilozoficznej''
uczonego z~Wools\-thorpe, co
oczywiście w~ścisłym rozumieniu tego wyrażenia nie jest zgodne z~prawdą, zważywszy choćby na jego zwięzłą
polemikę z~Kartezjuszem na temat argumentu ontologicznego na istnienie Boga, zapisaną w~notatniku studenckim
\textit{Questiones
quaedam philosophicae}, czy też obszerniejszą metafizyczną refleksję na temat Boga w~rękopiśmiennym eseju \textit{De
gravitatione}, spisanym przez Newtona przed wydaniem \textit{Zasad}. Guicciardini faktycznie łagodzi swoją opinię
twierdząc, iż ,,jego [Newtona~-- J.R.] wplątanie się w~kwestie filozoficzne było raczej natury obronnej. Nie wydaje się,
by te pisma [jak np. cytowane przez Guicciardiniego \textit{Scholium Generale}~-- J.R.] powstały z~autentycznego
(\textit{genuine}) zainteresowania [filozofią~-- J.R.] ze strony Newtona, lecz były efektem […] potrzeby obrony jego
budowli matematycznej i~eksperymentalnej przed krytykami'' (s.~179). A~w~innym miejscu stwierdza: ,,filozofia była dla
Newtona raczej koniecznością, aniżeli powołaniem (\textit{vocation}), raczej strategią obronną, aniżeli przyjętą linią
badań'' (s. 180). Wobec takiego \textit{dictum} włoskiego historyka być może interesująca byłaby jego konfrontacja ze
stanowiskiem Andrew Janiaka, autora ważnej monografii zatytułowanej
\textit{Newton as Philosopher} \parencite*{janiak_newton_2008}.

Dzisiejsza myśl humanistyczna (w tym historyczna), zgodnie z~panującymi modami i~odchodząc od ujęć hagiograficznych,
niejednokrotnie stawia w~swoich interpretacjach na elementy różnorodności w~miejsce ujęć unitarnych, na wymiar zmiennej
praktyki życiowej w~miejsce poszukiwania aletycznych fundamentów. Jest interesujące, jak w~tej
perspektywie, w~niedalekiej przyszłości zostaną potraktowane postać i~dzieło Isaaca Newtona, zważywszy
na ogrom pracy badawczej
wykonanej w~ciągu ostatnich kilkudziesięciu lat nad przyswojeniem i~zrozumieniem jego piśmiennictwa. Zdaniem
Guicciardiniego jesteśmy świadkami sytuacji jakby rozdwojonego wizerunku Newtona, matematyka i~alchemika jednocześnie,
fizyka-\mbox{-filozofa} %notabene
i~biblisty, profesora Trinity College i~żelaznego zarządcy Mennicy Królewskiej,
myśliciela-\mbox{-teoretyka} %notabene
i~wrażliwego na potrzeby otoczenia praktyka. Czy chcąc uniknąć uproszczonego, zunifikowanego
obrazu uczonego z~Wools\-thorpe nie rozmyjemy jednak sensu tych dokonań w~puzzlopodobnych elementach stanowiących wyraz
jego zadziwiającej wszechstronności? 



\autorrec{Jacek Rodzeń}


\subsubsection{Bibliografia}\nopagebreak[4]
\end{recplenv}
