
\begin{artengenv}{Michael Heller}
	{What does it mean ‘to exist’ in physics?}
	{What does it mean ‘to exist’ in physics?}
	{What does it mean ‘to exist’ in physics?}
	{Copernicus Center for Interdisciplinary Studies;\\
		Pontifical University of John Paul II in Kraków}
	{Physical theories give us the best available information about what there exists. Although physics is not ontology, it
		can be ontologically interpreted. In the present study, I~propose to interpret physical theories \textit{à la} Quine,
		i.e. not to speculate about what really exists, but rather to identify what a~given physical theory presupposes that
		exists. I~briefly suggest how Quine’s program should by adapted to this goal. To put the idea to the test, I~apply it
		to the famous Hartle–Hawking model of the quantum creation of the universe from nothing, and try to discover what kind
		of nothingness the model presupposes. I~also make some remarks concerning ontological commitments of the method of
		physics itself.}
	{existence, ontology in the sense of Quine, ontology of physics, Hartle-Hawking quantum creation model, nothingness in
		physics.}
	




\section{Introduction}
\lettrine[loversize=0.13,lines=2,lraise=-0.05,nindent=0em,findent=0.2pt]%
{E}{}xistence is an ontological issue. How then could one ask: ``What does it mean ‘to exist’ in physics?'' There is a~general
opinion that physics does not discuss the problem of existence; it simply presupposes that the subject-matter of its
investigation does exist. But in what sense does it exist? And what about claims of some physicists that they have
constructed a~self-creating world model\footnote{Formally speaking, ``to construct a~self-creating model'' is a
contradiction or very close to a~contradiction. This is a~typical situation when we speak about a~self-creating
universe. Our language is then at the limits of its applicability.}. Are these claims only engines for better selling
their works or do they indeed explain the existence of the physical universe?

The problem of existence is indeed an ontological issue, but the history of philosophy abounds in so many and so
different ontologies that this statement means nothing until we clarify what do we mean by ontology. To clarify
something means to restrict the variety of possible meanings to a~smaller subset of our choices. Since we are asking
about ``to exist'' in physics, the natural direction of our preferences should point toward physics itself. Physics is
not ontology, but it is often interpreted in an ontological way. This seems to be justified at least as far as the
deepest results of physics are concerned. And the deepest, and at the same time the most far-reaching, its result is –
this is my claim – the method physics has elaborated in investigating the world. Since this method is so effective, it
says something about the structure of the world, namely that in the structure of the world there is something that
makes this method so effective. And this ``something'' has certainly an ontological bearing.

This is essentially the standpoint so persuasively advocated by John Worrall who wrote:

\myquote{
It would be a~miracle, a~coincidence on a~near cosmic scale, if a~theory made as many correct empirical predictions as,
say, the general theory of relativity or as the photon theory of light without what that theory says about the
fundamental structure of the universe being correct or ‘essentially’ or ‘basically’ correct
\parencite[p.~101]{worrall_structural_1989}\footnote{Worrall’s paper initiated a~long discussion on scientific
realism. Some references to this discussion can be found in \parencite{heller_discovering_2006}.}.
}
The only difference between Worrall and myself is the difference in emphasis: he puts an accent on predictions whereas I
emphasize the method within which the predictions are possible.

Another project that is close to my idea is Tim Maudlin’s proposal of ``ontology based in physics''. In his
\textit{The Metaphysics within Physics} he writes:

\myquote{
[\ldots] metaphysics, insofar as it is concerned with the natural world, can do no better than to reflect on physics.
Physical theories provide us with the best handle we have on what there is, and the philosopher’s proper task is the
interpretation and elucidation of these theories. In particular, when choosing the fundamental posit of one’s ontology,
one must look to scientific practice rather than to philosophical prejudice \parencite[p.~1]{maudlin_metaphysics_2007}.
}

This is an ambitious project. In the present study, to make it workable, I~constrain it to a~very specific meaning
of ontology, the so-called ontology in the sense of Quine. In his ``ontological'' investigations, Quine did not try to
answer the question of what there is, but rather of what a~given theory or utterance presupposes there is (which are
its ``ontological commitments''), and to make his program precise he limited it to its strictly logico-formal aspects. In
section 2, I~briefly sketch Quine’s approach. To apply it to the ontology of physics, the original Quine’s program has
to be broadened; one should stick to its basic idea, rather than to its technicalities. In section 3, I~suggest how
this could be done, and ask, more generally, about ontological commitments of the method of physics itself. The best
way to put general doctrines to the test is to see how do they work in concrete instances. In section 4, I~apply the
broadened version of Quine’s ontological program to the famous Hartle-Hawking model of quantum creation of the universe
from nothingness in view of identifying its ontological commitments. Finally, in section 5, I~try to find out what kind
of nothingness is that the model presupposes.

\section{À la Quine}
A good starting point for our analysis is the famous Quine’s criterion of existence. The question he faced was: Which
ontological commitments a~given language enforces on its user? His celebrated answer is encapsulated in the short
formulation: ``To be is to be the value of a~variable'' \parencite[p.~15]{quine_what_1964}. This means that

\myquote{
the theory is committed to those and only those entities to which the bound variables of the theory must be capable of
referring in order that the affirmations made in the theory be true \parencite[p.~13–14]{quine_what_1964}.
}
Although in this particular place Quine speaks on mathematical theories, his idea remains valid if the ``theory'' is
replaced by any statement formulated in a~language capable of being logically analyzable. The goal of such an analysis
is to disclose ``ontological commitments'' of a~given statement:

\myquote{
We look to bound variables in connection with ontology not in order to know what there is, but in order to know what a
given remark or doctrine, ours or someone else’s, \textit{says} there is; and this much is quite properly a~problem
involving language \parencite[p.~15–16]{quine_what_1964}.
}

Physical theories are expressed in a~language, and Quine’s criterion of existence refers to them as well. In fact,
Quine, in his essay, makes numerous references to physical theories. However the language of physics is very peculiar.
To be precise, we should speak about the particular languages of various physical theories rather than about a~language
of physics in general. The language of a~given physical theory consists of mathematical formulae and a~text
accompanying them, and both these elements are essential. In more advanced physical theories, the content of the theory
is contained in its formulae, and the text provides an interpretation without which the formulae were at most a~part of
mathematics. If we aspire to make an analysis \textit{à la Quine}, we should look for ``bound variables'' in both these
layers of the language which, of course, would make the analysis more complicated, but still in principle possible.
This would give us a~knowledge not about ``what there is'', but rather about what a~given theory ``says there is''.

In practice, we could use simplified version of this approach, which I~would call an \textit{exegesis of the
mathematical structure of a~given physical theory}. To see what I~have in mind, let us distinguish three types of
comments or interpretations of a~physical theory:

\begin{enumerate}
\item A~comment which is inconsistent or even contradictory with the mathematical structure of the theory; for instance,
Bergson’s interpretation of the special theory of relativity \parencite{bergson_duree_1922}. Of course, such
an interpretation has no value at all.
\item A~comment that is neutral with respect to the mathematical structure of a~given physical theory. For instance, the
space-time of special relativity can be interpreted as a~``block universe'', i.e., as a~totality existing ``all at once'',
or as ``now'' flowing in time. Both these interpretations can be reconciled with special relativity\footnote{Roger
Penrose \parencite*{hawking_singularities_1979} has demonstrated, somewhat against a~common view, that the idea of flowing time
can be reconciled with special relativity.}. In such a~case, we may freely choose among such
possibilities.
\item A~comment could so closely follow the mathematical structure of a~physical theory that any its ``perturbation''
would result into inconsistencies or contradictions with the theory’s formalism. This I~call exegesis of the
mathematical structure of this theory. A~good example is provided by the interpretation of theorems on the geodesic
incompleteness of space-time as space-time singularities; see \parencite{hawking_large_1973}. Such
an exegesis is a~practical way (often unconsciously done by physicists) of disclosing what a~given theory ``says there
is''.
\end{enumerate}

\section{Beyond Quine}
Let us again quote from Quine:

\myquote{
We commit ourselves to an ontology containing numbers when we say there are prime numbers larger than a~million; we
commit ourselves to an ontology containing centaurs when we say there are centaurs; and we commit ourselves to an
ontology containing Pegasus when we say Pegasus is \parencite[p.~8]{quine_what_1964}.
}
And to which ontology we commit ourselves when we are doing physics? I~do not have in mind any particular theory or
model but rather physics as such. By asking this question we are going beyond Quine since we are leaving a~relatively
secure domain of logico-linguistic analyses; nevertheless we can learn from Quine to look for those elements without
which doing physics would be impossible. We should look for such elements in the very method of physics. If it would be
a miracle ``on a~cosmic scale'' provided theories, such as the general theory of relativity or the photon theory of
light, were so successful without being ``basically true'', then the success of the physical method without its reference
to ``what there is'', should be qualified as a~coincidence on the mega-cosmic scale. Successes of all particular physical
theories hang on these ``ontological commitments'' of the method.

How to identify ``ontological commitments'' of the method? To do this in a~precise way, at least partially paralleling
preciseness of Quine’s approach, would certainly go beyond the bounds of the present essay but, on the other hand, the
method of physics has been subject to so many analyses that to do this in a~sketchy way does not seem too difficult and
is quite sufficient for our purposes.

Roughly speaking, method of physics presupposes three things:

\begin{enumerate}
\item[(A)] a~certain mathematical structure;
\item[(B)] a~part or the aspect of the world which a~given mathematical structure is supposed to model;
\item[(C)]  ``bridge rules'' interpreting (A) in terms of (B); owing to these rules (A) serves as a~mathematical model of (B).
\end{enumerate}

Every particular physical theory (or model) is an implementation of this scheme. Also making empirical predictions
following from the theory (or model) and testing them by confrontation with experimental data is done within the
context of this scheme; independently of it the entire procedure would have no meaning at all.

There is no need to enter now into many philosophical discussions related to the above scheme, such as: How mathematical
structures do exist? What is the relationship between mathematical structures and mathematical objects? Does the above
scheme presupposes structuralist view on physics? Etc., etc.\footnote{Some of these problems are discussed in
\parencite{heller_discovering_2006}.}. All these problems are now irrelevant. What interests us at the moment is
what the method of physics (as represented, in a~sketchy way, by the above scheme) says there is. We are not asking
about the ``absolute ontology of reality'', we are only looking for the ontology of the \textit{univers de discourse} of
physics. And the answer is as follows. There exist: mathematical structures, a~domain to which they refer, and rules
establishing this reference. Without presupposing these three elements nothing can be done in physics; or even – no
physics could be possible.

\section{A case study -- The Hartle-Hawking Quantum Creation model}
In this section, I~apply, as an example, the above interpretational proposal to a~particular model. Since we are
concerned with the existence problem in physics I~have chosen the model the authors of which claim that they have
mathematically modeled the creation of the universe from nothing (one speaks also about a~``quantum tunneling out of
nothingness''). The model was published by James Hartle and Stephen Hawking \parencite*{hartle_wave_1983},
and was later on developed by others; see, for instance \parencite{wu_zhong_no-boundary_1993}.

In quantum field theory there is a~method, due to Richard Feynman, to calculate the transition probability for a
quantum system to go from a~state \textit{S1} to a~state \textit{S2}. This is not a~theoretical subtlety satisfying
esthetical predilections of theoretical physicists, but an essential procedure, a~way of computing the dynamical
evolution of a~quantum system. To do so one must take into account all possible paths from \textit{S1} to \textit{S2},
and to calculate a~certain integral along all of them. The extremal value of all these integrals is related to the
transition probability we want to know.

The idea of Hartle and Hawking was to transfer this strategy to the conceptual environment of quantum cosmology.
This required a~chain of bold hypotheses. A~state of the universe is unlike the state in quantum theory that can be
visualized as a~point in a~space, called phase space. Hartle and Hawking assumed that the universe is spatially closed
and, consequently, its state (at a~given time\footnote{The problem of time is another subtle issue in quantum gravity
and in this model, in particular.}) can be represented as a~three-dimensional surface of a~hyper-sphere (``3-geometry'')
equipped with suitable quantum fields. All such states of the universe are elements of a~space, called superspace which
is mathematically much more complicated than the usual phase space.

How to compute all possible paths from one of such states to another? This is a~difficult task both from the
conceptual and technical point of view. Hartle and Hawking showed their mastership dealing with it. In order to
overcome some technical difficulties they introduced a~bold conceptual innovation – an imaginary time, i.e., a~time
that has acquired all properties of a~fourth space dimension. All this (with some other important simplifying
assumptions) served to calculate the probability for the universe to find itself in the state \textit{S2} if it was
before in the state \textit{S1}.

The standard tool for calculating probabilities in quantum theory is the wave function that is defined on the space
of states; here it must be defined on the superspace of all possible states of the universe and is called the ``wave
function of the universe''. It is another investment of the Hartle-Hawking model that is involved in some conceptual
problems, but it is indispensable to calculate transition probabilities.

Hartle and Hawking went a~step further. Let us assume that the state \textit{S1} is ``empty'': no 3-geometry, no
quantum fields. What is the probability for the universe to find itself in the state \textit{S2} if the state
\textit{S1} was ``empty''? Not only this question turned out to be meaningful, but the calculated probability for such a
transition from a~``no-state'' to \textit{S2} could be different from zero. And this allows one to speak about a~quantum
creation of the universe from nothing\footnote{This is evidently a~very simplified description of the Hartle-Hawking
model; for a~slightly more detailed discussion, see \parencite[pp.~68–73]{heller_ultimate_2009}.}.

Which are ontological commitments of the Hartle-Hawking model? To answer this question we put aside the future
developments of this model and a~criticism it has provoked \parencite[see for instance][]{mccabe_structure_2005},
and take into account the model as it was originally presented by Hartle and Hawking. Of course, the precise analysis
\textit{à la Quine} should go into technical details which cannot be done in this essay. We must be satisfied with a
rather superficial dealing with the problem which, however, should be enough for grasping the main idea.

What the model says there is? Two levels of existence should be distinguished in it. First, the level of a
\textit{potential existence}. The ``potentialities'' in the model are severely limited by many factors. The wave function
of the universe must be a~solution to a~differential functional equation called DeWitt-Wheeler equation. Moreover,  to
overcome some technical difficulties Hartle and Hawking consider only a~``small'' subspace of the superspace, called
mini-superspace. Everything that goes beyond this limitations has no even potential existence in this model.

The second level of existence is an actual existence. This is a~delicate question. Since the model is a~quantum
model, probabilities in it play the essential role. To states of the universe, before they are instantiated, only a
certain probability of coming to existence can be ascribed. In this sense, the model’s ontology admits a~situation in
which there is (a different from zero) probability for some states of the universe to emerge from a~no-state. At least
one of such probabilities has been realized, and this is why the universe actually exists. We should not forget that
all the time we are speaking about the universe as an element of the model and about its existence as presupposed by
the model (in the sense \textit{à la Quine}). Whether this model corresponds to reality, i.e., to which degree is it
verified experimentally – this is another story\footnote{In fact, the Hartle-Hawking model, because of its many
simplifications and \textit{ad hoc} assumptions, was never seriously considered as describing the real universe.}.

We now should go beyond the analysis \textit{à la Quine} and ask about ontological commitments of the method of
physics on which the Hartle-Hawking model is based. In agreement with what was said in the preceding section, the model
must assume everything without which the method of physics cannot work, i.e., certain mathematical structures that are
interpreted as structures of the physical world or of some of its aspects. One says sometimes that every model
presupposes certain laws of physics. We may adopt this way of speaking as a~simplifying convention without going into a
dispute concerning the status of laws of physics, they semantic denotations, etc. In the case of the Hartle-Hawking
model three collections or systems of physical laws (mathematical structures with suitable interpretations) are
assumed. First, laws taken from quantum field theory, such as Feynman’s path integrals or the method of calculating
probabilities with the help of wave function. Second, the laws taken from general relativity, e.g., everything related
to closed cosmological models, and some approaches to quantum gravity, e.g., DeWitt-Wheeler equation. And third, some
new mathematical tools, suitably interpreted, e.g., imaginary time, that have turned out to be indispensable to make
the above two kinds of laws work together.

The Hartle-Hawking quantum creation model is ontologically committed to the existence of these three systems of
physical laws. Without them the model is unthinkable.

\section{The analysis of ``nothing''}
Is the claim of Hartle and Hawking justified that they have succeeded in constructing a~model of quantum creation of the
universe from nothing? Assuming that their model is both mathematically and physically correct and taking into account
our \textit{à la Quine} analysis, we are entitled to say that, in their model, there is indeed a~(different from zero)
probability for the process of an emergence of the universe from nothingness to occur. But what does it mean
``nothingness'' in this context? Let us notice that in the mathematical structure of the model there is nothing (and
rightly so) that could be interpreted as ``nothingness''. ``Nothingness'' is outside of the model. In this sense,
nothingness is what model says nothing about.

However, if we look at the problem not from the perspective of the model itself, but rather from the perspective of
the method of physics, the situation looks different. The model is based on a~rich mathematical structure equipped with
a rich physical interpretation. The model itself, with all its structural elements (quantum creation included), is made
out of this physically interpreted mathematical structure which is far from being nothingness.

If we attempted to construct a~physical model from absolute nothing: the zero of existence, no mathematical
structure and nothing to interpret, we would not be able to move one step forward. This is why the Leibniz question is
so persistent: ``Why is there something rather than nothing?'', and his short comment'': ``For nothing is simpler and
easier than something'' \parencite[p.~303]{leibniz_principles_1908}\label{ref:RNDdi2I3ZddDY}. Why then is there something that is neither
easy nor simple?

\paragraph{Acknowledgement} 
This publication was made possible through the support of a~grant from the John Templeton Foundation (Grant No. 60671).

\end{artengenv}
