\begin{artplenv}{Zbigniew Liana}
	{Nauka jako racjonalna \textit{doxa}. Józefa Życińskiego koncepcja nauki i~filozofii nauki -- poza internalizmem i~eksternalizmem}
	{Nauka jako racjonalna \pagname{doxa}\ldots}
	{Nauka jako racjonalna \textit{doxa}. Józefa Życińskiego koncepcja nauki i~filozofii nauki -- poza internalizmem i~eksternalizmem}
	{Uniwersytet Papieski Jana Pawła II w~Krakowie}
	{Science as rational \textit{doxa}. J. Życiński's understanding of science and philosophy of science -- beyond internalism and externalism}
	{Philosophical interests of Joseph Życiński (1948-2011) in the domain of the philosophy of science were focused on the
		debate concerning the nature of science and philosophy of science that followed the Einstein-Planck revolution in
		science. The unexpected discovery of the philosophical, extra-scientific  presuppositions in science, as well as of the
		extra-rational factors determining the way these presuppositions are accepted in science were to be explained within
		the meta-scientific framework. It is the aim of this paper to present Życiński's diagnosis of this post-revolutionary
		situation in the philosophy of science as well as his critique of the metascientific answers to this challenge. The
		reasons will be given why all those answers are put under two dichotomous rubrics of \textit{internalism} and
		\textit{externalism}. It will be also explained how Życiński intends to supersede this false in his opinion opposition
		with a~new concept of doxatic rationality. However the details of the metascientific proposal of Życiński will be given
		only in the subsequent paper. In order to perform the aim of the paper the metatheoretic tools set out by Popper
		\parencite*{popper_beiden_1979}
		%(1979)
		will be used.}
	{rationalism, skepticism, internalism, externalism, scientific revolution, metascientific revolution, philosophical
		presumptions in science, commitment to the research tradition.}


\section*{Wstęp}

\lettrine[loversize=0.13,lines=2,lraise=-0.05,nindent=0em,findent=0.2pt]%
{W}{}przekonaniu Życińskiego dwudziestowieczny spór filozoficzny o~charakter nauki należy interpretować jako kolejny,
historyczny przejaw bardziej fundamentalnego sporu między racjonalizmem a~sceptycyzmem o~racjonalność poznania.

W dwudziestowiecznej tradycji filozoficznej spór o~rozumienie tego, czym jest \textit{nauka}, wiąże się ściśle ze
sporem o \textit{racjonalność} rozwoju wiedzy naukowej. W~swej najbardziej radykalnej wersji -- ,,zewnętrznej'' --
jest to spór racjonalizmu ze sceptycyzmem o~\textit{istnienie}, względnie \textit{nieistnienie} specyficznego, ,,racjonalnego
elementu'' poznania, różnego od zmysłowego postrzeżenia, czyli tak czy inaczej rozumianego \textit{rozumu}. W~swej wersji
,,wewnętrznej'' -- to znaczy wewnątrz tradycji racjonalistycznej, uznającej istnienie \textit{elementu racjonalnego} -- jest
to spór o~charakter tego elementu oraz o~jego miejsce w~procesie zmiany teoretycznej\footnote{Na zasadzie \textit{pars
pro toto} Życiński
\parencite*[s.~184]{zycinski_jezyk_1983}
%\label{ref:RND1yjaBGa3S8}(1983, s.~184)
utożsamia naukę z~,,rozumem naukowym'', a~filozofię nauki
określa na sposób Kanta mianem \textit{Krytyki rozumu naukowego}. Ponieważ w~celu metanaukowej analizy rzeczywistej nauki
tego typu emblematyczne określenie jest mało przydatne ze względu na swą nadmierną ogólność, Życiński wprowadza
bardziej realistyczną kategorię `elementu racjonalnego'
\parencite[s.~142n]{zycinski_jezyk_1983}.
%\label{ref:RNDZsYVuB7lRL}(J. Życiński, 1983, s.~142n).
}.

W diagnozie Życińskiego dwudziestowieczny spór o~\textit{naukę} jest sporem fundamentalnym, sporem o~rozumienie
racjonalności jako takiej, i~to zarówno w~jego wersji ,,zewnętrznej'', jak i~,,wewnętrznej''. W~przypadku sporu
,,zewnętrznego'' -- i~tutaj tkwi \textit{proprium} rozwiązania Życińskiego -- obie strony sporu, czyli rozwiązania zarówno
racjonalistyczne, jak i~sceptyczne, mają wspólny rodowód. Łączy je błędna, tradycyjna koncepcja racjonalności, a~tym
samym błędna koncepcja nauki. W~tezie tej ujawnia się ogólniejsza presupozycja metafilozoficzna postulująca relatywny
charakter sceptycyzmu filozoficznego. Jest to sceptycyzm \textit{względem} określonej koncepcji racjonalności lub
naukowości, a~nie sceptycyzm \textit{w ogóle}. Jest on specyficzną teorią metanaukową odrzucającą \textit{określoną
historycznie} koncepcję naukowości i~racjonalności. W~przypadku sceptycyzmu dwudziestowiecznego, przedmiotu analiz
Życińskiego, negacji podlega koncepcja racjonalności, jaką żywił się dwudziestowieczny racjonalizm metanaukowy,
racjonalizm normatywno-demarkacjonistyczny Koła Wiedeńskiego, Poppera i~Lakatosa\footnote{Zob.
\parencites[s.~228]{zycinski_elementy_1996}[s.~186]{zycinski_jezyk_1983}.
%\label{ref:RNDM0IkZEEMNb}(J. Życiński, 1996, s.~228, 1983, s.~186).
W~tym drugim tekście jest mowa o~porzuceniu ostrej
dychotomii kontekstu odkrycia i~uzasadnienia typowej dla demarkacjonistycznych metodologii.}.

Motywem, który skłonił Życińskiego do postawienia takiej tezy metafilozoficznej, była konstatacja pewnego istotnego
faktu historycznego: to, że oba te przeciwstawne rozwiązania metanaukowe powstały w~reakcji na dwudziestowieczną
rewolucję naukową, która zanegowała dotychczasowy wzorzec naukowości i~racjonalności, wywołując efekt ,,intelektualnego
szoku'' i~prowadząc do ,,rewolucji metanaukowej''. Ale konstatacja faktu z~historii nauki jest jednocześnie interpretacją
sytuacji filozoficznej w~perspektywie metanaukowego sporu o~racjonalność naukową. Jest to zatem ,,konstrukcja''
specyficznego faktu metanaukowego, w~którym już zawiera się \textit{implicite} możliwe rozwiązanie spornego problemu
filozoficznego\footnote{Życiński nie wierzył oczywiście w~istnienie czystych faktów. Zob. niżej przypis \ref{lia-foo-12}. W~tym
wypadku ma swoje zastosowanie teza Lakatosa o~zaślubinach filozofii nauki z~historią nauki
\parencites[zob.][s.~121n]{zycinski_jezyk_1983}[s.~26]{zycinski_structure_1988}[s.~47]{zycinski_struktura_2013}.
%\label{ref:RNDS7T4vadlBl}(zob. J. Życiński, 1983, s.~121n, 1988b, s.~26, 2013, s.~47).
}. Rozwiązanie proponowane przez
Życińskiego jest bezpośrednią konsekwencją przyjętej diagnozy metafilozoficznej: przezwyciężenie problemów związanych z~tradycyjną
koncepcją racjonalności powinno skutkować przezwyciężeniem współczesnego sporu racjonalizmu ze sceptycyzmem.
Konieczna jest zatem istotna modyfikacja tradycyjnego rozumienia racjonalności i~naukowości. Taka, która rozwiąże
problemy nierozwiązywalne na gruncie koncepcji tradycyjnej, a~będące bezpośrednim powodem pojawienia się nowej wersji
sceptycyzmu. W~ujęciu Życińskiego problemy te sprowadzają się do wspólnego mianownika, jakim jest problem pogodzenia
racjonalności naukowej z~występowaniem w~nauce i~w~jej rozwoju \textit{elementów pozanaukowych} i~\textit{czynników
pozaracjonalnych}.

Celem niniejszego artykułu jest analiza wewnętrznej logiki filozofii nauki Życińskiego, jaka prowadzi go od konstatacji
faktów metanaukowych do nowego metateoretycznego rozwiązania kwestii racjonalności naukowej. Panuje dość powszechna
opinia wśród czytelników prac Józefa Życińskiego, że są one trudne w~percepcji ze względu na swą zawiłość,
wielowątkowość, a~przede wszystkim ze względu na specyficzny, retoryczny styl. W~jego tekstach, zwłaszcza pozycjach
książkowych, roi się od przykładów, polemik i~dygresji. Analizy o~wysokim stopniu abstrakcji i~metateoretycznej
precyzji przeplatają się z~argumentami retorycznymi, a~subtelny język analiz miesza się z~literackim językiem ciętej
polemiki. Lekturze jego prac towarzyszy jednak trudna do wyartykułowania intuicja zasadniczej poprawności jego
argumentacji. To właśnie ta intuicja stanowiła dla mnie motywację poszukiwania teoretycznej spójności w~filozofii nauki
Życińskiego.

Metoda poszukiwania wewnętrznej logiki i~ukrytych przedzałożeń jest w~dużej mierze metodą racjonalnej rekonstrukcji.
Wszelkie próby nazbyt deskryptywnego podejścia do filozofii nauki Życińskiego musiałyby ponieść fiasko ze względu na
wspomniany charakter jego prac. Element deskryptywny musi zostać zracjonalizowany za pomocą choćby tymczasowej
metateoretycznej perspektywy interpretacyjnej. Najbardziej zasadne wydaje się wybranie perspektywy wyznaczanej przez
tradycję epistemologiczną, z~jaką Życiński wiąże swoje filozofowanie o~nauce. Pomimo wszystkich jego krytycznych uwag i~zastrzeżeń,
jest to szeroko rozumiana racjonalistyczna tradycja \textit{episteme} (por. \textit{Elementy}, s.~126) w~jej
dwudziestowiecznej wersji zapoczątkowanej przez logiczne badania nad językiem, matematyką i~samą logiką oraz przez
logiczno-metodologiczne badania nad nauką i~jej rozwojem w~ujęciu Poppera i~Lakatosa. Tradycja ta dostarczy
odpowiednich kategorii, przy pomocy których będę mógł rozpocząć analizę tekstów Życińskiego. 

Ogólny, racjonalno-rekonstrukcyjny cel mojego tekstu usprawiedliwia ograniczenie zakresu podejmowanych w~niniejszym
artykule analiz i~wykorzystywanych tekstów źródłowych. Ponieważ nie jest moim celem szczegółowa prezentacja wszystkich
wątków i~wszystkich możliwych niuansów proponowanych przez Życińskiego rozwiązań metanaukowych, dlatego w~artykule
wykorzystane zostaną jedynie jego książki z~zakresu filozofii nauki z~pominięciem licznych, szczegółowych
artykułów\footnote{Są to następujące prace: \textit{Język i~metoda}
\parencite*{zycinski_jezyk_1983},
%\label{ref:RND1LhiGh0y5p}(1983);
\textit{Teizm i~filozofia analityczna}, tom~1
\parencite*{zycinski_teizm_1985},
%\label{ref:RND68nQZEVcjZ}(1985),
\textit{Structure of the Metascientific Revolution}
\parencite*{zycinski_structure_1988},
%\label{ref:RNDgXcIEkiEEw}(1988b),
\textit{Granice racjonalności}
\parencite*{zycinski_granice_1993}
%\label{ref:RND9m82QoDque}(1993)
oraz \textit{Elementy filozofii nauki}
\parencite*{zycinski_elementy_1996}.
%\label{ref:RNDBnf6pOKYn0}(1996).
Wiele wątków powtarza się przez wszystkie prace. Różni je bardziej
sposób prezentacji niż charakter proponowanych rozwiązań. Zauważalna jest zasadnicza ciągłość w~myśli metanaukowej
Życińskiego, począwszy od pierwszej pracy. Tylko w~wybranych przypadkach wskażę pewną ewolucję myśli Życińskiego, inne
pominę, jako nieistotne dla celu mojej pracy. Najbardziej rozwinięte analizy metafilozoficzne Życińskiego można
odnaleźć w~jego \textit{Strukturze rewolucji metanaukowej}
\parencite*{zycinski_structure_1988,zycinski_struktura_2013}.
%\label{ref:RNDPOjzmhEMek}(1988b, 2013).
Tutaj pojawiają się
istotne z~punktu widzenia rozumienia racjonalności nauki koncepcje \textit{ideatów} oraz \textit{ideologicznych programów
badawczych}. Pisząc ten tekst korzystałem z~oryginału angielskiego
\parencite{zycinski_structure_1988}.
%\label{ref:RNDFZPJ2Z2OCH}(J. Życiński, 1988b).
Tekst
ten został przetłumaczony na język polski już po śmierci J. Życińskiego w~2013~r. Cytując fragmenty tej publikacji
korzystam z~przekładu M. Furmana, dokonując wszakże pewnych modyfikacji w~celu dostosowania terminologii do terminologii
niniejszego artykułu. Odnośniki do stron podaję zarówno dla tekstu angielskiego, jak i~dla polskiego przekładu.
Istnieje również drugie, poszerzone wydanie \textit{Elementów} filozofii nauki
\parencite{zycinski_elementy_2015}. \par
%\label{ref:RNDL62LQXHrez}(J. Życiński, 2015). \par
Czytelnikowi, który chciałby poznać najważniejsze myśli całego dorobku Józefa Życińskiego, a~nie tylko z~zakresu
filozofii nauki, polecam przeglądowy artykuł Michała Hellera
\parencite*{heller_filozoficzny_2011}.
%\label{ref:RNDeScsQTkW9D}(2011).
Metanaukowe poglądy Życińskiego omawiam również w~\parencite{evers_can_2016}.
%\label{ref:RNDsr0gJlWoso}(Z. Liana, 2016).
}. Ze względu na obszerność koniecznych analiz zostały one podzielone na dwie odrębne części\footnote{\label{lia-foo-5}Podział ten
dokonany został na prośbę Redakcji.}. Część pierwsza -- zawarta w~niniejszym artykule -- obejmuje sobą najogólniejsze,
metateoretyczne analizy koncepcji Życińskiego filozofii nauki. Część druga zawierać będzie szczegółowe rozwiązania
metanaukowe zaproponowane przez Życińskiego w~celu wypracowania nowego rozumienia nauki i~racjonalności. Są to
rozwiązania najczęściej kojarzone z~jego nazwiskiem, takie jak zasada aracjonalności, zasada epistemologicznej
niepewności, naturalności interdyscyplinarnej, czy też kwestia różnych typów racjonalności i~ewolucji pojęcia
racjonalności.

\section{Perspektywa metateoretyczna}

Nowego rozumienia racjonalności Życiński nie poszukuje na drodze apriorycznych analiz językowo-logicznych, lecz przez
pełniejsze uwzględnienie faktów z~historii nauki. Jego podstawowy postulat metafilozoficzny głosi, że filozofia nauki
powinna zrezygnować z~uproszczonych i~wyidealizowanych koncepcji nauki i~wypracowywać koncepcje bardziej realistyczne,
uwzględniające w~większym stopniu analizę rzeczywistych zachowań naukowców. Tego typu deklaracja metaepistemologiczna
stanowi wyraz określonych przekonań ontologicznych i~metametodologicznych Życińskiego. Element racjonalny ma charakter
\textit{obiektywny}, jest obecny w~nauce realnie, aczkolwiek nierzadko \textit{implicite}. Racjonalność obiektywna ujawnia
swój charakter stopniowo w~meandrach historii nauki. Filozofowi nie pozostaje nic innego, jak uważnie obserwować tę
historię i~próbować \textit{odkrywać }racjonalność obiektywną za pomocą hipotez metanaukowych i~testować te
ostatnie w~oparciu o~faktyczne i~faktualne przejawy racjonalności w~dziejach nauki. Widać z~tego, że pod względem metodologicznym
filozofia nauki Życińskiego bliższa jest nauce empirycznej niż analitycznej. Pod tym względem Życiński staje w~jednej,
\textit{transcendentalnej} tradycji metafilozoficznej obok Poppera i~Lakatosa\footnote{Życiński poddaje jednak tę metodę
reinterpretacji, gdyż inaczej rozumie on \textit{rzeczywistość} nauki niż Popper czy Lakatos. Kwestia ta zostanie
przedstawiona pod koniec artykułu.}.

Swoje koncepcje metafilozoficzne, w~tym koncepcję metody transcendentalnej, Popper przedstawił
najpełniej w~\textit{Die beiden Grundprobleme der Erkenntnistheorie}
\parencite*{popper_beiden_1979}.
%\label{ref:RNDZjoA35m8Sx}(1979).
To właśnie tam broni tezy, że
metodologia jest specyficzną metanauką posługującą się specyficzną metodą transcendentalną\footnote{Koncepcję Poppera
filozofii jako metanauki omawiam szczegółowo w~\parencite{liana_uzasadnienie_2006}.
%\label{ref:RNDfY6Fw7eCi8}(Z. Liana, 2006).
}. Wprawdzie nazwa metody
zaczerpnięta jest od Kanta, to jednak jej rozumienie różni się znacząco od znaczenia, jakie temu wyrażeniu nadawał
Kant. Nie jest to metoda dedukcji transcendentalnej, lecz, jak podkreśla Popper, metoda \textit{analogiczna} do metody
empirycznej. Także w~przypadku metodologii mamy do czynienia z~określonymi faktami filozoficznymi, które wymagają
filozoficznego wyjaśnienia. Wyjaśnienia filozoficzne są wartościowe o~tyle, o~ile wytrzymują konfrontację z~faktami
filozoficznymi. W~pracy tej Popper nie nazywa wyjaśnień metodologicznych hipotezami, a~ich obalenia przez fakty
metodologiczne falsyfikacją. Oba te wyrażenia rezerwuje wyłącznie dla nauk empirycznych. W~metodologii i~w~metodzie
transcedentalnej mamy do czynienia z~tak zwaną \textit{transcedentalną sprzecznością}, analogiczną do empirycznej
falsyfikacji. Metoda ta została zmodyfikowana przez Lakatosa i~poszerzona o~wykorzystanie faktów z~historii nauki w~celu
quasi-empirycznej konfrontacji rozwiązań metanaukowych. Życiński przyjmuje w~filozofii identyczną transcendentalną
metodę, z~tym że terminy `falsyfikacja' i~`hipoteza' odnosi on bez żadnego rozróżnienia zarówno do metody nauk
empirycznych, jak i~do metody metanauki lub szerzej filozofii nauki\footnote{W jego tekstach można znaleźć wiele
wypowiedzi o~\textit{falsyfikacji }koncepcji filozoficznych
\parencites[s.~177]{zycinski_teizm_1985}[s.~9,12,97,135]{zycinski_structure_1988}[s.~17,22,174,239]{zycinski_struktura_2013}.
%\label{ref:RNDlMUuFKby8S}(J. Życiński, 1985, s.~177, 1988b, s.~9,12,97,135, 2013, s.~17,22,174,239).
W~\textit{Elementach}
\parencite*[s.~166]{zycinski_elementy_1996}
%\label{ref:RNDCkkQxwcbWF}(1996, s.~166),
mówi o~wymogu
\textit{falsyfikowalności} rozwiązań metanaukowych. Życiński odróżnia falsyfikowalne hipotezy metanaukowe od
niefalsyfikowalnych idei metafizycznych. Za Popperem uznaje metafizykę za \textit{zasadniczo niefalsyfikowalną}
\parencite[s.~130]{zycinski_elementy_1996}.
%\label{ref:RNDA2elxmxDEm}(J. Życiński, 1996, s.~130).
}.

Należy też zauważyć, że Życiński rozróżnia wąsko rozumianą metanaukę od szerszej filozofii nauki
\parencite[s.~14–16]{zycinski_elementy_1996}.
%\label{ref:RNDSXNKAbqry1}(J. Życiński, 1996, s.~14–16).
Metanauka bada kwestie logiczne,
metodologiczne i~epistemologiczne w~nauce. Filozofia nauki zajmuje się z~kolei wypracowaniem całościowej wizji nauki w~jej ujęciu
diachronicznym, czyli strukturą rewolucji naukowych, związkami między czynnikami racjonalnymi i~socjologicznymi,
ewolucją pojęcia racjonalności. Nie należy ich wszakże sobie przeciwstawiać, gdyż obie zajmują się ,,tą samą
rzeczywistością nauki'' a~granice między nimi są do pewnego stopnia \textit{rozmyte}
\parencite[s.~16]{zycinski_elementy_1996}.
%\label{ref:RNDJqZXflfCwo}(J. Życiński, 1996, s.~16).
Metoda transcedentalnej falsyfikacji stosuje się do nich obu.

Z perspektywy analizy koncepcji Życińskiego szczególnie przydatne wydaje się wprowadzone przez Poppera pojęcie
\textit{faktu metodologicznego} lub \textit{epistemologicznego}, czyli \textit{faktu metanaukowego}\footnote{Dla ułatwienia
lektury w~całym tekście treść idei, podobnie jak treść pojęć i~koncepcji, piszę zazwyczaj kursywą. Kursywa służy
również do podkreślenia istotnych elementów znaczeniowych.}. W~kontekście filozofii Życińskiego można pojęcie to
rozszerzyć do pojęcia \textit{faktu filozoficznego}. Pojęcia te, jakkolwiek budzą w~niektórych kręgach filozoficznych
sprzeciw i~opór, to jednak w~perspektywie współczesnej filozofii nauki odrzucającej ,,zgodnie'' -- podobnie jak Popper --
istnienie \textit{czystych faktów}, są one jak najbardziej na miejscu. Skoro fakt rozumiany jako pewne zdanie bazowe
nauki lub metanauki jest w~sposób nieunikniony teoretyczną interpretacją obserwacji\footnote{O ile w~czasach Poppera i~Koła
Wiedeńskiego teza ta była przedmiotem sporu, o~tyle w~czasach Życińskiego była ona już podstawowym i~,,niekontrowersyjnym''
założeniem filozofii nauki. Życiński uznaje ten fakt to za skutek rewolucji metanaukowej
\parencite[zob.][s.~127]{zycinski_elementy_1996}.
%\label{ref:RNDNNdGM9DYkK}(zob. J. Życiński, 1996, s.~127).
}, to staje się on pojęciem analogicznym. W~zależności od
typu interpretacji mamy do czynienia z~faktem empirycznym, filozoficznym, naukowym lub metanaukowym,
etc. W~,,socjologicznej'' koncepcji faktu wprowadzonej przez Poppera\footnote{Termin `socjologiczna' oznacza według Poppera
konieczność intersubiektywnej zgody. Warto zauważyć, że według niego chodzi o~zgodę ,,negatywną'', a~nie pozytywną. Ta
ostatnia implikowałaby swoisty indukcjonizm, który, jak wiadomo, jest dla Poppera niemożliwy do zaakceptowania z~racji
logicznych. Zgodę faktualną uznaje się za obowiązującą wtedy, gdy nikt \textit{kompetentny} (\textit{sic!}) nie wyraża
sprzeciwu, i~to tylko tymczasowo, do chwili, gdy pojawi się ktoś \textit{kompetentny} zgłaszający sprzeciw. Taki sprzeciw
zmusza do rewizji zdań faktualnych. Najpełniejsze przedstawienie ,,negatywnej'' koncepcji \textit{faktu} można odnaleźć w~pracy Poppera
\parencite*[s.~122–135, zwł. 131n.]{popper_beiden_1979}
%\label{ref:RNDFSE3yxffmO}(1979, s.~122–135, zwł. 131n.).
} stopień obiektywności faktu mierzy się stopniem
jego intersubiektywności. Z~tego względu fakty filozoficzne cechują się znacznie mniejszym stopniem obiektywności niż
fakty empiryczne i~w~konsekwencji możliwość obalenia koncepcji metanaukowych przez fakty filozoficzne jest dużo
mniejsza niż możliwość falsyfikacji hipotez empirycznych. O~ile Popper już w~\textit{Die beiden Grundprobleme} zaczął
wątpić w~możliwość rozstrzygającego obalenia na gruncie metodologii, o~tyle u~Życińskiego niełatwo znaleźć artykulacje
podobnych wątpliwości\footnote{\label{lia-foo-12}Mówi on o~falsyfikacji \textit{tout court} metanaukowej koncepcji teorii nauki
obiektywnej
\parencites[s.~9]{zycinski_structure_1988}[s.~17]{zycinski_struktura_2013},
%\label{ref:RNDwQZn2Zd98X}(J. Życiński, 1988b, s.~9, 2013, s.~17),
albo o~\textit{praktycznej} falsyfikacji
intuicjonizmu w~matematyce
\parencites[s.~97]{zycinski_structure_1988}[s.~174]{zycinski_struktura_2013}.
%\label{ref:RNDF7UBvcuK26}(J. Życiński, 1988b, s.~97, 2013, s.~174).
Jednocześnie jednak Życiński
\parencites*[s.~140]{zycinski_structure_1988}[s.~248]{zycinski_struktura_2013}
%\label{ref:RNDhVrobTs8jG}(1988b, s.~140, 2013, s.~248)
krytykuje Poppera i~jego tezę o~możliwości obiektywnej
oceny wartości schematów pojęciowo-metodologicznych. W~świetle \textit{Die beiden Grundprobleme} nie wydaje się, by teza
ta faktycznie była głoszona przez Poppera, a~po drugie, jej krytyka wydaje się niezbyt spójna z~niekrytycznym użyciem
przez Życińskiego terminu `falsyfikacja'.}.

\section{Rewolucja naukowa a~rewolucja metanaukowa -- narzędzia metateoretycznej analizy}

Punktem wyjścia dla Życińskiego do poszukiwania nowych rozwiązań metanaukowych i~z~zakresu filozofii nauki jest swoista
metahistoryczna i~metanaukowa teza. W~filozofii nauki doszło do \textit{metanaukowej rewolucji}, która została wywołana
przez wcześniejszą rewolucję naukową\footnote{Termin `rewolucja metanaukowa' pojawia się już
w~\parencite[s.~99]{zycinski_jezyk_1983}
%\textit{Języku i~metodzie} \label{ref:RNDX1VVM27OlW}(J. Życiński, 1983, s.~99)
w~tytule drugiej części dzieła: ,,Struktura rewolucji
metanaukowych''. Ten sam tytuł nosi angielska książka Życińskiego
\parencite*{zycinski_structure_1988},
%\label{ref:RNDZhnNjGxqOR}(1988b)
będąca rozwinięciem
idei zawartych w~\textit{Języku i~metodzie}. Życiński mówi o~odstępie ,,półwiecza'' między rewolucją naukową a~rewolucją
metanaukową. Wyrażenie pojawia się
w~\parencite[s.~101]{zycinski_jezyk_1983}
%\textit{Języku i~metodzie} \label{ref:RNDIDOT0gvEdV}(J. Życiński, 1983, s.~101)
i~zostaje powtórzone w~\parencite[s.~126]{zycinski_elementy_1996}.
%\textit{Elementach...} \label{ref:RNDTq4UOwaMNQ}(J. Życiński, 1996, s.~126).
}.
Jedna i~druga były
równie gwałtowne, powiązane z~doświadczeniem rzeczonego intelektualnego szoku. W~celu rekonstrukcji logiki ukrytej w~rozwiązaniu
Życińskiego konieczne wydaje się poddanie analizie specyficznej relacji, jaka zachodzi w~jego koncepcji
między dwoma faktami: między faktem rewolucji naukowej a~faktem rewolucji metanaukowej.

Język, jakim Życiński operuje w~kontekście omawiania relacji między rewolucją metanaukową a~rewolucją naukową, jest
zarówno językiem logiki, jak i~językiem psychologii. Jest językiem logiki, gdyż mówi o~\textit{implikowaniu} rewolucji
metanaukowej przez rewolucję naukową oraz o~metanaukowych \textit{konsekwencjach} rewolucji naukowej
%\label{ref:RND7WtbaMWBYd}(J. Życiński, 1988b, s.~8.13, 2013, s.~15.24)
\parencites[s.~8.13]{zycinski_structure_1988}[s.~15.24]{zycinski_struktura_2013}\footnote{Polski przekład `rezultat' nie oddaje
wiernie angielskiego `implied'.}. Język ten sugeruje, iż Życiński postuluje istnienie silnych związków
merytoryczno-\mbox{-logicznych} %notabene
pomiędzy tymi rewolucjami. Teoretyczne rozwiązania zaproponowane w~ramach rewolucji
metanaukowej nie miały zatem genezy czysto apriorycznej, lecz były w~dużej mierze  zdeterminowane rzeczywistą nauką.
Ale z~drugiej strony Życiński mówi o~\textit{szoku} wywołanym przez rewolucję naukową wśród filozofów nauki i~o~ich
reakcji na ten szok. Także sam termin `rewolucja' użyty przez Życińskiego niesie ze sobą silne konotacje pozalogiczne.
Został wprowadzony do filozofii nauki przez Kuhna w~celu podkreślenia istotnej roli czynników socjologicznych,
psychologicznych i~kulturowych w~rozwoju nauki. Odrzucając dychotomię skrajnego internalizmu i~skrajnego eksternalizmu,
Życiński odrzuca zarówno koncepcję relacji czysto logicznej, jak i~koncepcję relację czysto ,,zewnętrznej'',
przyczynowej. W~konsekwencji przedstawia tę relację dwoiście, zarówno jako związek logiczny, racjonalny, jak i~jako związek
przyczynowo-\mbox{-skutkowy}\footnote{Nie %notabene
jest możliwe przedstawienie koncepcji Życińskiego w~sposób liniowy, niejako ,,cegiełka
po cegiełce''. Ponieważ to, w~jaki sposób dokonuje on metanaukowej interpretacji faktów historycznych i~relacji między
nimi, jest już warunkowane \textit{implicite} przez jego metanaukowe koncepcje, dlatego nieunikniona jest pewna kolistość
prezentacji i~przyjęcie na początku pojęć, które będą wyjaśniane dopiero z~czasem. Pojęcia \textit{internalizmu} i~\textit{eksternalizmu}
zostaną przedstawione w~punkcie \ref{lia-sec-5}, a~psychologiczne pojęcie \textit{szoku} w~punkcie \ref{lia-sec-4}b.}.

Życiński podejmuje studium i~analizę historii nauki i~historii filozofii nauki w~celu zidentyfikowania różnego typu
faktów metanaukowych, zarówno racjonalnych, jak i~pozaracjonalnych, które pozwolą mu na właściwą interpretację zależności
rewolucji metanaukowej od rewolucji naukowej. Z~jednej strony poszukuje faktów pozaempirycznych metanaukowych z~historii
nauki, które pozwolą mu na uchwycenie racjonalnych (logiczno-merytorycznych) relacji zachodzących pomiędzy
zmianami na poziomie nauki teoretycznej ze zmianami na poziomie teorii metanaukowych (zmian metateoretycznych). Z~drugiej
strony poszukuje faktów ,,psychologicznych'' usprawiedliwiających (potwierdzających) użycie psychologicznych
kategorii `rewolucji' i~`intelektualnego szoku'\footnote{Ponieważ pojęcie \textit{kategorii} traktuję jako pojęcie
metateoretyczne i~metajęzykowe zarazem, dlatego treść kategorii będę zapisywał w~cudzysłowie metajęzykowym.} dla opisu
stanu umysłu filozofów nauki.

Poniżej przedstawiam w~formie metateoretycznych uwag wyjaśnienie różnych typów faktu metanaukowego. Uwagi te są
przydatne do lepszego zrozumienia dalszych analiz, ale nie są konieczne. Jako bardziej abstrakcyjne i~niekonieczne do
dalszej lektury zostały wydzielone z~całości tekstu.\footnote{W całym artykule bardziej abstrakcyjne analizy
metateoretyczne lub metafilozoficzne przedstawiam w~formie oddzielnych uwag. Mam nadzieję, że ułatwi to lekturę
tekstu. W~obecnym artykule nie podejmuję szczegółowej analizy metody, jaką Życiński posługuje się w~filozofii nauki. Byłaby to
niewątpliwie niezwykle interesująca strategia poszukiwania ukrytych presupozycji Życińskiego na temat racjonalności
metodologicznej, ale zaciemniłoby to dodatkowo i~tak stosunkowo skomplikowany tekst artykułu. Dlatego ograniczam się w~tym
względzie do rzeczonych uwag.}


\begin{uwaga}
Fakty metanaukowe to quasi-empiryczne fakty z~historii nauki. będące przedmiotem wyjaśnień
metanaukowych. Idea wyróżnienia faktów empirycznych od quasi-empirycznych faktów metanaukowych pochodzi od Poppera.
Wprawdzie Życiński nie stosuje tego typu terminologii, niemniej mówi o~faktach z~historii nauki będących przedmiotem
wyjaśnień metanaukowych. Odróżnienie Poppera ma charakter demarkacjonistyczny. Na gruncie metodologicznym odróżnia on
wyraźnie fakty (zdania) empiryczne od faktów epistemologicznych, czy metodologicznych. Jego teoria
\textit{metodologicznych typów} faktów bazuje na specyficznej \textit{zasadzie analogii}: fakty epistemologiczne są
analogiczne do faktów empirycznych, podobnie jak metoda filozofii nauki (metoda transcedentalna) jest analogiczna do
metody empirycznej. W~przypadku Życińskiego, jego teksty sugerują, że operuje on raczej jednoznacznym niż analogicznym
pojęciem faktu we wszystkich kontekstach metodologicznych. Brak wystarczających danych, by rozstrzygnąć, czy jest to
świadoma i~krytyczna postawa metodologiczna, czy jedynie spontaniczna. Z~tego względu zastosowaniu terminologii Poppera
do interpretacji tekstu Życińskiego musi towarzyszyć daleko posunięta ostrożność hermeneutyczna. Tego typu ostrożna
interpretacja terminu `fakt metanaukowy' abstrahuje od meta-metodologicznych idei Poppera i~ogranicza znaczenie tego
terminu do własności \textit{bycia faktem wyjaśnianym przez metanaukę}.
\end{uwaga}

\begin{uwaga}
Fakt metanaukowy, analogicznie do faktu empirycznego, stanowi językowe przedstawienie i~zarazem
teoretyczną interpretację konkretnego, postrzeganego zmysłami ,,zdarzenia''. W~nie-demarkacjonistycznym ujęciu
Życińskiego zdarzeniem ujmowanym przez fakt metanaukowy może być zarówno zjawisko fizyczne (np. psychologiczne), jak i~zjawisko
,,językowe''. Takim zjawiskiem językowym jest pojawienie się określonego twierdzenia lub teorii o~określonych
cechach ontologicznych lub logicznych. Może nim być także zachowanie metajęzykowe i~zarazem metanaukowe
(epistemologiczne, metodologiczne) naukowca: to, co robi on ze swoimi wypowiedziami, by je uzasadnić, obalić,
sprawdzić, a~także to, do jakich celów poznawczych ich używa.

Ze względu na dwa typy faktów metanaukowych: empiryczne fakty metanaukowe i~pozaempiryczne ,,językowe'' fakty metanaukowe,
te ostatnie będę nazywał `faktami metateoretycznymi'.

W perspektywie metajęzykowej fakt metateoretyczny należy utożsamić ze zdaniem egzystencjalnym orzekającym występowanie
\textit{typowych} (powtarzających się) cech metateoretycznych (np.: ,,Naukowcy dokonują predykcji'', ,,Fakty naukowe są
obciążone teoretycznie'', ,,Nauka nie potwierdza ostrego odróżnienia kontekstu odkrycia od kontekstu uzasadnienia'',
,,Teorie się zmieniają'', ,,Mechanika kwantowa odchodzi od tradycyjnego ideału poznania jednoznacznego'', ,,Twierdzenia
limitacyjne metalogiki ukazują granice poznania matematycznego i~logicznego'' itp.), a~konkretne zdarzenia z~historii
nauki z~desygnatami spełniającymi lub nie to zdanie. 

Stwierdzenie powyższe pokazuje, że wyrażenia `wyrażenie metateoretyczne' i~`wyrażenie metajęzykowe' nie są tożsame.
Fakty metateoretyczne można (i należy) traktować jako język przedmiotowy metanauki. Metanauka ma swój własny język
przedmiotowy i~swój własny metajęzyk, różny od języka i~metajęzyka nauk przedmiotowych. Osobnym problemem jest pytanie,
czy język przedmiotowy metanauki jest metajęzykiem nauki przedmiotowej i, w~konsekwencji, czy metajęzyk metanauki jest
meta-metajęzykiem nauki przedmiotowej. Szczegółowa odpowiedź na tak postawione pytanie wykracza poza ramy niniejszego
artykułu. Ale w~metodologistycznym podejściu Poppera do metanauki zdania metanauki mają za przedmiot wyrażenia
metajęzykowe nauki przedmiotowej tylko w~ich ujęciu `materialnym', a~nie `formalnym'. Metanauka tworzy własny język
przedmiotowy na bazie metajęzyka nauki przedmiotowej. Ten język przedmiotowy nie utożsamia się z~metajęzykiem nauk
przedmiotowych, a~jedynie powstaje na drodze odpowiedniej interpretacji metodologicznej metajęzykowych (i zarazem
metanaukowych) zachowań naukowców (względem faktów, teorii etc.) jako zachowań metateoretycznych lub innych (np.
psychologicznych). Nie należy również mylić metanauki jako dyscypliny filozoficznej z~metanaukowymi \textit{zachowaniami}
naukowców. Te ostanie to te zachowania naukowców, które są przedmiotem metanaukowej interpretacji w~filozofii nauki.

Warto przy tym zauważyć, że samo odróżnienie dwóch dziedzin faktów metanaukowych: dziedziny metateoretycznej (historia
nauki rozumianej intersubiektywistycznie) i~dziedziny empirycznej (świat, w~tym także subiektywne reakcje i~uwarunkowania
naukowców) jest już wyrazem określonej interpretacji metanaukowej (albo jeszcze wyższego rzędu
metanaukowego) określonych intuicji poznawczych.
\end{uwaga}

\begin{uwaga}
Faktów metateoretycznych nie należy utożsamiać z~\textit{konkretnymi} przypadkami z~historii nauki.
Fakty te są przedstawieniem ogólnie ujętych, określonych cech -- ontologicznych lub metateoretycznych -- teorii naukowych
i zachowań epistemologiczno-metodologicznych naukowców. Konkretne przypadki z~historii nauki stanowią co najwyżej
\textit{ilustrację}, \textit{przykład} lub \textit{potwierdzenie} występowania tego typu cech w~nauce. Swego czasu August
Comte rozróżnił \textit{fakty jednostkowe} od \textit{faktów ogólnych}. Naukę empiryczną i~filozofię o~wiele bardziej
interesują fakty ogólne -- czyli to, co jest powtarzalne -- niż konkretne jednostkowe przypadki (zdarzenia). Nawet
w~procedurze potwierdzenia lub obalenia empirycznego pojedyncze zdarzenia są bezwartościowe metodologicznie. Muszą być
powtarzalne i~muszą się pewną ilość razy powtórzyć -- najlepiej niezależnie -- by mogły zostać \textit{zaakceptowane} w~sposób
intersubiektywny przez wspólnotę badaczy. Jak wiadomo, Popper nazywa te \textit{fakty ogólne} zdaniami bazowymi i~hipotezami
niskiego rzędu.

Z tego względu fakt metateoretyczny -- fakt pozaempiryczny, czyli fakt w~perspektywie metody innej niż metoda empiryczna
-- to sąd egzystencjalny, który dotyczy nauki (względnie metanauki) rozumianej zarówno w~jej aspekcie funkcjonalnym,
jak i~przedmiotowym. W~pierwszym przypadku fakt metateoretyczny dotyczy zachowań metodologicznych naukowców (względnie
filozofów nauki), w~drugim wystąpienia określonych sądów, twierdzeń, teorii, hipotez, idei, a~także ich własności
metateoretycznych (względnie meta-metateoretycznych).
\end{uwaga}

\begin{uwaga}
Osobną kwestią jest problem zaklasyfikowania faktów psychologicznych odnotowujących reakcje
psychologiczne w~obliczu nowej nauki. W~demarkacjonistycznym ujęciu Poppera należy je uznać za fakty \textit{stricte}
empiryczne nienależące do metanauki. W~przypadku nie-demarkacjonistycznego ujęcia Życińskiego -- gdzie element
\textit{stricte} racjonalny współwystępuje z, i~jest dookreślany przez element przyczynowy -- należy przypuszczać, że
podział nie jest tak ostry i~także fakty psychologiczne z~historii nauki są specyficznymi faktami metanaukowymi.

Inna sprawa, że w~historii nauki te dwa typy ,,faktów'' nie występują ,,oddzielnie'', lecz tworzą \textit{jeden}
\textit{złożony} fakt metanaukowy: bezpośredni przedmiot obserwacji. Fakt złożony (to znaczy fakt ,,surowy'' --
uteoretyzowany w~mniejszym stopniu) można ,,rozłożyć'' -- to znaczy zinterpretować i~wyjaśnić dedukcyjnie -- na wspomniane
dwa \textit{typy} lub\textit{ aspekty} metanaukowe (na fakt empiryczny i~na fakt \textit{stricte} metateoretyczny, to znaczy
pozaempiryczny) dopiero za pomocą odpowiedniej analizy metanaukowej.

Niezależnie od \textit{faktycznej} struktury faktów metanaukowych i~niezależnie od \textit{faktycznego} charakteru
metanaukowych faktów psychologicznych, rozróżnienie empirycznych i~pozaempirycznych (metateoretycznych) faktów
metanaukowych jest pragmatycznie użyteczne. Nie należy traktować tego rozróżnienia skrajnie demarkacjonistycznie, jak
chciał Popper, lecz wyłącznie jako pierwsze przybliżenie metanaukowe tego, czym jest fakt metanaukowy. Jego użyciu, w~perspektywie
metafilozofii Życińskiego, musi towarzyszyć zawsze odpowiednie ograniczenie, że wszelkie demarkacje mają
charakter wyłącznie idealizacyjny i~nigdy nie stanowią ostatecznego ujęcia rzeczywistości racjonalnej.
\end{uwaga}

\begin{uwaga}
Życiński nie przeprowadza tego typu analiz metateoretycznych w~odniesieniu do stosowanej przez siebie
terminologii metanaukowej. W~konsekwencji nie rozróżnia on \textit{explicite} pod względem metodologicznym ani faktu
naukowego od metanaukowego, ani tym bardziej obu aspektów tego ostatniego. Konkretne znaczenie metateoretyczne,
jakie wiąże z~terminem `fakt', określane jest przez kontekst użycia.
\end{uwaga}
	
Przyjmując wraz Życińskim, że w~rekonstrukcji historii nauki i~historii filozofii nauki związki logiczne są ważniejsze
od uwarunkowań psycho-społecznych, racjonalna rekonstrukcja jego rozumienia relacji między rewolucją naukową a~rewolucją
metanaukową należy przyjąć, że najważniejsze są w~tym wypadku związki pomiędzy specyficznymi faktami
metateoretycznymi z~dziedziny nauki przedmiotowej (np. teoriami, założeniami) a~specyficznymi faktami metateoretycznymi z~dziedziny
metanauki (np. wyjaśnieniami)\footnote{W perspektywie przyjętych rozróżnień między poziomami naukowości i~teoretyczności
należałoby nazwać te ostatnie fakty \textit{faktami meta-metateoretycznymi}. Ale takie rozróżnianie nic
nie wniesie do dalszych analiz poza zbędnym balastem precyzji. Życiński wydaje się nie przywiązywać dużej wagi do tego
typu precyzyjnych rozróżnień poziomów analiz.
W~\parencite[s.~123]{zycinski_jezyk_1983}
%\textit{Języku i~metodzie} \label{ref:RNDqQTlKikSxt}(J. Życiński, 1983, s.~123)
przytacza sceptyczną uwagę J. Watkinsa odnośnie do tego typu rozróżnień u~Poppera i~Lakatosa, ale nie wydaje na
ten temat własnej oceny.}. Metanaukowych faktów psychologicznych w~takiej rekonstrukcji nie można pominąć, jeśli nie
chcemy przypisać Życińskiemu koncepcji nazbyt uproszczonej i~wyidealizowanej, ale też nie mogą być one, jak zobaczymy
poniżej, dominujące\footnote{Uzasadnienie tych warunków zostanie przedstawione niżej przy okazji omawiania własnego
rozwiązania metanaukowego Życińskiego.}.

Wedle takiej rekonstrukcji Życiński najpierw ustala odpowiednie zbiory faktów metateoretycznych obu typów, a~następnie
poszukuje właściwych relacji merytoryczno-logicznych pomiędzy nimi. Wyznaczenie odpowiednich zbiorów faktów
metateoretycznych jest stosunkowo proste. Czerpie je odpowiednio z~historii nauki przedmiotowej
(empirycznej i~formalnej) z~jednej strony i~z~historii metanauki z~drugiej. Problematyczny jest natomiast sposób poszukiwania
powiązanych merytoryczno-logicznie \textit{uporządkowanych par} faktów metateoretycznych: 

\smallskip
{\centering
	{\textless}fakt z~historii nauki; fakt z~historii metanauki{\textgreater}.
\par}
\smallskip


W tym celu Życiński wydaje się odwoływać do idei powiązania faktów psychologicznych z~faktami metateoretycznymi w~postaci
surowych faktów metanaukowych i~meta-metanaukowych\footnote{Fakty meta-metanaukowe to fakty tworzące
rzeczywistość metanauki jako pewnej działalności poznawczej człowieka.}. Idea ta kieruje \textit{implicite} jego
heurystyczną strategią metanaukową: psychologiczna idea \textit{intelektualnego szoku} i~kategorii jemu równoważnych
nadaje się na stosunkowo łatwe kryterium wyróżniania tych faktów metateoretycznych z~dziejów nauki, które doprowadziły
do zmian w~metanauce, czyli do nowych faktów metateoretycznych w~dziejach metanauki. Ponieważ psychologiczny aspekt
faktów metanaukowych jest łatwiejszy do konstatacji, zatem może służyć jako wygodne narzędzie wyróżniania odpowiednich
pozaempirycznych faktów metateoretycznych obu typów i~ich związków. Z~jednej strony należy poszukiwać takich teorii
naukowych i~takich cech tych teorii, które wywołały szok wśród filozofów nauki, a~z~drugiej strony takich postaw
metanaukowych i~teorii metanaukowych, które były odpowiedzią na ów szok. Oczywiście, zgodnie z~zasadniczo
racjonalistycznym stanowiskiem Życińskiego, warunkowanie psychologicznie w~żadnym wypadku nie może wyjaśniać ani
zastępować związków logicznych, jakie występują między obu typami faktów metateoretycznych. W~tym celu konieczna jest
klasyczna analiza znaczeń i~struktur logicznych.

Oprócz \textit{szoku intelektualnego} Życiński wskazuje także na inne psychologiczne fakty metanaukowe w~kontekście
powiązania rewolucji naukowej z~metanaukową. Jednym z~nich jest odczucie \textit{paradoksalności} nowych teorii. W~\textit{Strukturze
rewolucji metanaukowej} pisze, że szok intelektualny towarzyszył zarówno odkryciu
\textit{paradoksalnych} geometrii nieeuklidesowych, antynomii w~podstawach matematyki, jak i~\textit{paradoksów} teorii
kwantów oraz teorii względności. W~okresie rewolucji naukowej doświadczenie paradoksalności nauki było do tego stopnia
powszechne, że niektórzy, jak na przykład Niels Bohr, uznali ją za synonim poprawności i~prawdziwości rozwiązań
teoretycznych\footnote{Zob.
\parencites[s.~9,132]{zycinski_structure_1988}[s.~16,233]{zycinski_struktura_2013},
%\label{ref:RNDsmv0TXVWdS}(J. Życiński, 1988b, s.~9,132, 2013, s.~16,233),
gdzie mówi o~wymogu
stawianym teorii naukowej przez Bohra, by była ,,dostatecznie szalona''.}. Podobną reakcją psychologiczną było
\textit{zaskoczenie} spowodowane odkryciem nieoczekiwanych cech nauki, takich jak istnienie wewnętrznych, niepokonalnych,
logicznych ograniczeń. I~to zarówno w~obrębie nauk przedmiotowych (formalnych i~empirycznych), jak i~w~obrębie
metanauki: odkrycie niezupełności bogatych systemów logicznych, zasady losowości w~fizyce czarnych dziur czy zasady
nieoznaczoności Heisenberga oraz odkrycie metanaukowej zasady niedookreślenia teorii przez obserwacje
\parencites[s.~11]{zycinski_structure_1988}[s.~20]{zycinski_struktura_2013}.
%\label{ref:RND6Rd4uQGI6n}(J. Życiński, 1988b, s.~11, 2013, s.~20).

Wewnętrzna logika kierująca przejściem od rewolucji naukowej do rewolucji metanaukowej wydaje się być zatem w~rozumieniu
Życińskiego następująca. Pojawienie się nowej teorii naukowej o~określonych cechach (fakt metateoretyczny) wywołuje
szok intelektualny (fakt psychologiczny), a~ten z~kolei \textit{przyczynowo} prowadzi do zaproponowania nowych koncepcji
metanaukowych (fakt metateoretyczny wyższego rzędu), przy czym treść tych ostatnich \textit{z zasady }nie zależy od
uwarunkowań psycho-społecznych, lecz od treści wyjściowego faktu metateoretycznego i~od przyjmowanej tradycji
badawczej\footnote{Na temat związania z~tradycją badawczą lub z~paradygmatem będzie mowa niżej w~punkcie \ref{lia-sec-4}b.}. Jest to
zatem relacja logiczno-merytoryczna. W~perspektywie analizy metanaukowej najbardziej fundamentalne wydają się być zatem
fakty metateoretyczne zachodzące w~obrębie nauk przedmiotowych. Ich zajście (np. pojawienie się określonej teorii
naukowej) staje się \textit{przyczyną} określonych faktów psychologicznych, takich jak intelektualny szok filozofów. Ten
ostatni z~kolei okazuje się możliwą \textit{przyczyną} nowych faktów metateoretycznych, tym razem jednak w~obrębie
metanauki: pojawienie się nowych rozwiązań metanaukowych.

Życiński mówi w~tym kontekście o~potrzebie `racjonalnej \textit{reakcji}' filozofów
\parencites[s.~143]{zycinski_structure_1988}[s.~254; podkreślenie moje]{zycinski_struktura_2013}
%\label{ref:RNDTgN8tnj5Jz}(J. Życiński, 1988b, s.~143, 2013, s.~254; podkreślenie moje)
na zaskakujące implikacje rewolucji naukowej. Ukazując znacząco
istotniejszą obecność \textit{elementu niepewności i~subiektywizmu} w~nauce od powszechnie zakładanego, rewolucja naukowa
uświadomiła filozofom \textit{fundamentalne ograniczenia} racjonalności naukowej. Ale już treść tak psychologicznie
uwarunkowanych nowych rozwiązań metanaukowych jest zasadniczo określana związkami
logiczno-\mbox{-merytorycznymi.} %notabene
Reakcje te
powinny być zdaniem Życińskiego \textit{racjonalne}\footnote{Życiński nie ma tutaj na myśli \textit{racjonalności} w~sensie
\textit{źródła} tej reakcji, czyli że jest to reakcja intelektu, względnie rozumu, lecz w~znaczeniu \textit{normatywnym}.
Normatywny charakter wynika z~faktu, iż Życiński podaje jednocześnie kryteria tej racjonalności. Należy dodać, że
Życiński zna i~racjonalizuje na gruncie swej metanauki wyjątki od normatywnej reguły \textit{racjonalnych reakcji}. W~metanauce
Życińskiego rządzi nimi tak zwana zasada aracjonalności. Zostanie ona omówiona w~kolejnej części tego tekstu
(zob. wyżej tekst i~przypis nr \ref{lia-foo-5}).}.

Przykładem takiej metanaukowej reakcji jest bez wątpienia wspomniana reakcja N. Bohra na doświadczenie paradoksalności
nowych teorii naukowych. Była to intelektualna próba wyjścia z~impasu w~metanauce i~sprowadzała się do przeformułowania
koncepcji naukowości jako takiej. W~celu obrony idei naukowości Bohrowi wystarczało proste uznanie cechy
paradoksalności za kryterium naukowości. Życiński nie mówi jednak, czy była to reakcja racjonalna, czy nieracjonalna.
Oceny takiej podejmuje się natomiast w~odniesieniu do metanaukowych koncepcji internalizmu i~eksternalizmu (zob.
tamże).

W swej metanaukowej analizie postawy Bohra Życiński nie poprzestaje na stwierdzeniu zajścia prostej reakcji metanaukowej
ze strony Bohra. Jego analiza idzie jeszcze głębiej. Poszukuje on w~postawie Bohra \textit{jeszcze} \textit{ogólniejszych
}faktów metateoretycznych, a~zatem ukrytych na jeszcze wyższym poziomie dedukcyjnych warunków możliwości tej postawy.
Pisze, że przywiązanie prezentowane przez Bohra do idei paradoksalności samo w~sobie jest już ,,wyrazem odejścia od
tradycyjnych założeń na temat roli zdroworozsądkowych kryteriów racjonalności w~nauce''
\parencites[s.~9]{zycinski_structure_1988}[s.~16]{zycinski_struktura_2013}.
%\label{ref:RND7AT9mZ6IoG}(J. Życiński, 1988b, s.~9, 2013, s.~16).
Metanaukowa postawa Bohra jest dla Życińskiego swego rodzaju antycypacją przyszłej
rewolucji metanaukowej, jaka w~filozofii nauki nastąpi dopiero w~drugiej połowie XX wieku. Jednocześnie jednak postawa
ta stanowi wyraźny przykład racjonalnego związku faktów metateoretycznych z~poziomu nauki (teza o~paradoksalnym
charakterze nowych teorii naukowych) z~faktem metateoretycznym na poziomie metanauki (nowa koncepcja naukowości). Można
jedynie się zastanawiać, na ile związek ten jest zapośredniczony przez pozaracjonalny element podmiotowy, czyli przez
jakiś fakt psychologiczny\footnote{\label{lia-foo-26}Rozwijana przez Życińskiego koncepcja Polanyiego \textit{wiedzy milczącej} i
\textit{wiedzy osobowej }pozwala przyjąć, że w~jego przekonaniu oba te fakty metateoretyczne muszą być powiązane ze sobą
podmiotowymi uwarunkowaniami Bohra. Metanaukowe rozwiązanie Bohra w~najmniejszym bowiem stopniu nie jest konieczną,
logiczną konsekwencją odkrytej przezeń paradoksalności nowych teorii naukowych. Na temat Polanyiego zob.
\parencites[s.~169n]{zycinski_jezyk_1983}[s.~156–166]{zycinski_teizm_1985}[s.~144.202]{zycinski_structure_1988}%
[s.~218.351]{zycinski_struktura_2013}[s.~179–191]{zycinski_elementy_2015},
%\label{ref:RNDgm9vg5CtBx}(J. Życiński, 1983, s.~169n, 1985, s.~156–166, 1988b, s.~144.202, 2013, s.~218.351, 2015, s.~179–191);
zob. też niżej przypis \ref{lia-foo-55}.}.

\enlargethispage{1\baselineskip}
Przykład Bohra pokazuje jednak tylko jedną z~wielu możliwych \textit{rewolucyjnych zależności intelektualnych} pomiędzy
zmianami na poziomie nauki a~zmianami na poziomie metanaukowym, czy -- ogólniej -- na poziomie filozoficznym. Zdaniem
Życińskiego bogactwo zmian, jakie zaszły i~ciągle zachodzą w~nauce w~wyniku rewolucji naukowej, jest tak wielkie, że
trudno przedstawić ich wszystkie możliwe \textit{konsekwencje} filozoficzne, w~szczególności \textit{metanaukowe}
\parencites[s.~8]{zycinski_structure_1988}[s.~15]{zycinski_struktura_2013}.
%\label{ref:RNDJXAyHhUuS3}(zob. J. Życiński, 1988b, , 2013, s.~15).
Z~konieczności ogranicza się więc do
wyartykułowania tylko tych, które uważa za najbardziej istotne z~punktu widzenia problemu racjonalności.

\section[Rewolucja metanaukowa: \textit{doxa} zamiast \textit{episteme}]{Rewolucja metanaukowa: \textit{doxa} zamiast \textit{episteme}\footnote{Zob. tytuł rozdziału w~\parencite[s.~101]{zycinski_jezyk_1983}.
		%\textit{Języku i~metodzie} \label{ref:RNDDNOb8iCJVi}(J. Życiński, 1983, s.~101).
	}
}

Po omówieniu wewnętrznej logiki metanaukowych analiz Życińskiego, czas na przedstawienie bardziej deskryptywnego elementu
jego rozwiązania, a~mianowicie punktu wyjścia, jakim jest stwierdzenie metanaukowego faktu rewolucji naukowej i~metanaukowego
faktu rewolucji metanaukowej. Nie zachodzi tutaj oczywiście symetria. Idea \textit{rewolucji naukowej}
stała się częścią języka współczesnej kultury. Bliższego przedstawienia i~uzasadnienia wymaga natomiast stwierdzenie
faktu \textit{rewolucji metanaukowej}. 

Zdaniem Życińskiego
\parencites*[s.~7]{zycinski_structure_1988}[s.~13]{zycinski_struktura_2013}
%\label{ref:RNDm1e4fQpTDw}(1988b, s.~7, 2013, s.~13)
termin `rewolucja' bywa nadużywany w~analizach z~zakresu
historii nauki i~filozofii. Nie każde psychologiczne odczucie nowości czy zmiany usprawiedliwia metanaukowe
użycie tego terminu. By mówić o~rewolucji w~nauce, oprócz psychologicznego faktu zaskoczenia i~nowości konieczne jest
spełnienie jakiegoś racjonalnego kryterium. W~ujęciu Życińskiego kryterium takim jest przełomowy charakter zmian w~podstawowych
założeniach. Nie dziwi zatem, że utożsamia on dwudziestowieczną rewolucję naukową z~pojawieniem się teorii
względności i~mechaniki kwantowej, a~jako nazwy własnej tej rewolucji używa wyrażenia `rewolucja
Einsteina-Plancka'\footnote{Wyrażenie to pojawia się przykładowo
w~\parencites[s.~13.25]{zycinski_structure_1988}[s.~24.45]{zycinski_struktura_2013}
%\textit{Strukturze} \label{ref:RNDnNQlvKNfJU}(J. Życiński, 1988b, s.~13.25, 2013, s.~24.45)
i~w~\parencite[s.~228.258]{zycinski_elementy_2015}.
%\textit{Elementach} \label{ref:RNDwUSTjxSagp}(J. Życiński, 2015, s.~228,258).
}. Osobno mówi też o~rewolucji metamatematycznej związanej z~pojawieniem się twierdzeń
limitacyjnych\footnote{Wyrażenie `rewolucja metamatematyczna' pojawia się
w~\parencites[s.~101]{zycinski_structure_1988}[s.~181]{zycinski_struktura_2013}.
%\textit{Strukturze } \label{ref:RND06p9KBsVwr}(J. Życiński, 1988b, s.~101, 2013, s.~181).
Epistemologiczną interpretację twierdzeń
limitacyjnych Życiński przedstawia szczegółowo
w~\parencites*[s.~118–126]{zycinski_teizm_1985}[s.~18–46]{zycinski_teizm_1988}.
%\textit{Teizmie i~filozofii analitycznej} \label{ref:RNDromKVdEplH}(J. Życiński, 1985, s.~118–126, 1988a, s.~18–46).
Tematyka ta jest także obecna
w~\parencites[rozdział IV]{zycinski_structure_1988,zycinski_struktura_2013}[s.~61–68]{zycinski_granice_1993}[s.~262–276]{zycinski_elementy_1996}.
%\textit{Strukturze} (rozdział IV),
%w~\parencite[s.~61–68]{zycinski_granice_1993}
%\textit{Granicach racjonalności} \label{ref:RNDVhjIsTMBSi}(J. Życiński, 1993, s.~61–68)
%oraz
%w~\parencite[s.~262–276]{zycinski_elementy_1996}.
%\textit{Elementach} \label{ref:RNDoBnYdXZe0T}(J. Życiński, 1996, s.~262–276).
} oraz ogólniej o~rewolucji w~podstawach matematyki,
zapoczątkowanych zakwestionowaniem piątego postulatu Euklidesa
\parencite[zob.][s.~196]{zycinski_jezyk_1983}.
%\label{ref:RNDjSFh4bpxsh}(zob. J. Życiński, 1983, s.~196).

Analogiczne kryterium rewolucyjności ma zastosowanie także w~odniesieniu do rewolucji metanaukowej. Zdaniem Życińskiego
\parencites*[s.~126]{zycinski_elementy_1996}[por.][s.~101]{zycinski_jezyk_1983}
%\label{ref:RND9iw6GUDcsc}(1996, s.~126, por. 1983, s.~101)
radykalne zmiany, jakie zaszły w~filozoficznej refleksji nad
nauką od lat trzydziestych XX wieku zasługują na miano \textit{rewolucji metanaukowej}. W~wyniku rewolucji naukowej
konieczne okazało się porzucenie tradycyjnych, wygórowanych \textit{założeń epistemologicznych} oraz naiwnego pojmowania
samej natury poznania naukowego. Rewolucja ta wywołała ,,kryzys wiary'' w~przyrodoznawstwo jako pewną, niekwestionowaną i~doskonałą
wiedzę o~rzeczywistości i~jej prawach
\parencite[zob.][s.~102]{zycinski_jezyk_1983}.
%\label{ref:RNDbGHN8nMTuu}(zob. J. Życiński, 1983, s.~102).

Porzucenie tradycyjnych założeń epistemologicznych definiujących naukowość w~kategoriach `wiedzy pewnej' i~`doskonałej'
Życiński przedstawia jako odejście od tradycyjnego ideału \textit{episteme}\footnote{Określenie to pojawia się
w~\parencite[s.~102]{zycinski_jezyk_1983}.
%\textit{Języku i~metodzie} \label{ref:RNDVZe2UQr77Q}(J. Życiński, 1983, s.~102).
Kategoria `ideału \textit{episteme}' jest
analogiczna do ukutej w~mniej więcej tym samym czasie kategorii J. Watkinsa `ideału Bacona-Kartezjusza'. Watkins w~przeciwieństwie
do Życińskiego pomija jednak całkowicie starożytne źródła nowożytnych ideałów poznawczych, zob.
\parencite[s.~31–36]{watkins_nauka_1989}.
%\label{ref:RNDFZ4rQltK9B}(J.W.N. Watkins, 1989, s.~31–36).
Angielski oryginał pochodzi z~1984 r.}.
Wszyscy teoretycy
nauki, jacy nastali po Carnapie, zarówno indukcjoniści, jak i~dedukcjoniści, porzucili ideał \textit{episteme} na rzecz
\textit{doxa}
\parencite[s.~107]{zycinski_jezyk_1983},
%\label{ref:RNDFYiS161msr}(J. Życiński, 1983, s.~107),
a~jeśli uwzględni się dodatkowo odkrycie nieusuwalnej
niedoskonałości poznania matematycznego, to jedyna możliwa konkluzja, jaka się narzuca Życińskiemu, jest następująca:
,,\textit{epist\=em\=e} kurczy się gwałtownie, ustępując wszechwładnej \textit{doxa}''
\parencite[s.~109]{zycinski_jezyk_1983}.
%\label{ref:RND0ssJjiFTXJ}(J. Życiński, 1983, s.~109).
Identyczna interpretacja pojawia się
w~\parencites[s.~12n]{zycinski_structure_1988}[s.~22n]{zycinski_struktura_2013},
%\textit{Strukturze} \label{ref:RNDOxi36USGCm}(J. Życiński, 1988b, s.~12n, 2013, s.~22n),
tyle, że nieco inaczej wyartykułowana i~poparta innymi przykładami. Życiński mówi o~końcu
,,epistemetycznej'' teorii nauki, w~której rozwijano platońsko-arystotelesowską tradycję \textit{episteme} rozumianej jako
wiedza pewna i~niepodważalna. Epistemetyczna teoria nauki została zastąpiona teorią ,,doksatyczną'', kontynuującą
platońską tradycję \textit{doxa}, wiedzy prawdopodobnej. Dokonało się to na skutek odkrycia \textit{istotnych ograniczeń}
poznawczych w~różnych dziedzinach poznania. Twierdzenie o~niezupełności w~logice, zasada losowości w~fizyce czarnych
dziur, zasada nieoznaczoności w~fizyce kwantowej i~epistemologiczna zasada niedookreśloności to tylko niektóre z~takich
,,limitacyjnych'' odkryć przytaczanych przez Życińskiego
%\label{ref:RNDP7Ju9f9P3l}(zob. 1988b, s.~11, 2013, s.~22n)
\parencites*[zob.][s.~11]{zycinski_structure_1988}[s.~22n]{zycinski_struktura_2013}\footnote{W
\parencite[s.~129]{zycinski_elementy_1996}
%\textit{Elementach} \label{ref:RNDGof4TZDE7e}(J. Życiński, 1996, s.~129)
pojawia się stwierdzenie, że
odejście od ideału \textit{episteme} było \textit{metanaukowym odpowiednikiem }rewolucji Einsteina-Plancka i~że współczesna
epistemologia nauk przyrodniczych stała się de facto \textit{doxalogią}. Tematyka rewizji \textit{episteme} występuje
również
w~\parencite[s.~55]{zycinski_granice_1993}.
%\textit{Granicach racjonalności} \label{ref:RNDEK9raCkAFp}(J. Życiński, 1993, s.~55).
}.

Porzucony ideał \textit{episteme} pochodzi z~arystotelesowskiej tradycji \textit{episteme}
\parencites[s.~102]{zycinski_jezyk_1983}[s.~46–55]{zycinski_granice_1993}.
%\label{ref:RNDRmhyRD3ZvW}(zob. J. Życiński, 1983, s.~102, 1993, s.~46–55).
Życiński nie utożsamia jednak tego ideału z~oryginalnym pojęciem wypracowanym
przez samego Arystotelesa. Wprawdzie takie cechy jak pewność i~niepodważalność występują również w~Arystotelesowskiej
definicji \textit{episteme}, niemniej są one niewystarczające, by ukonstytuować jego oryginalne pojęcie. Dodatkowe
określenia takie jak konieczność i~przyczynowość
\parencite[s.~46]{zycinski_granice_1993}
%\label{ref:RNDUffssKrdla}(J. Życiński, 1993, s.~46)
oraz bazowanie na
niewzruszonych podstawach
%\label{ref:RNDxd5cHMOpD0}(J. Życiński, 1988b, s.~143, 2013, s.~253)
\parencites[s.~143]{zycinski_structure_1988}[s.~253]{zycinski_struktura_2013}\footnote{Pisze na
przykład o~frustrującym (\textit{disappointing})
odkryciu, że u~podstaw nauki nie znajdują się jakieś \textit{niewzruszone (unshakable) podstawy}, lecz rozmyty zbiór
(arbitralnych) przedzałożeń.} także nie wyczerpują cech istotnych oryginalnej koncepcji
Arystotelesa\footnote{Arystoteles podaje i~omawia je w~pierwszej księdze \textit{Analityk wtórych} w~rozdziale 2 i~4.}. Z
wypowiedzi Życińskiego
\parencite*[s.~46]{zycinski_granice_1993}
%\label{ref:RNDawumYBBRbe}(1993, s.~46)
wynika, że chodzi mu raczej o~to, co z~oryginalnego
pojęcia \textit{episteme} przetrwało w~nowożytnej tradycji filozoficznej i~co inspirowało twórców terminu `epistemologia'
rozumianej jako teoria wiedzy\footnote{\label{lia-foo-32}Warto zauważyć, że termin ten (ang.: \textit{epistemology}) został ukuty,
według historyków, dopiero w~wieku XIX przez filozofa angielskiego J.F. Ferriera
pozostającego pod silnym wpływem Fichtego i~jego
koncepcji \textit{Wissenschaftslehre}. Ferrier wzorował się na innym, wcześniejszym terminie `ontology'. Termin ten
oznaczał u~niego właśnie \textit{theory of knowledge}
\parencite[zob.][s.~44]{ferrier_institutes_1854}.
%\label{ref:RNDLNW5r1yicU}(zob. J.F. Ferrier, 1854, s.~44).
Na temat
historii tego terminu zob.
\parencite[s.~63–66]{sinacoer_lepistemologie_1973}.
%\label{ref:RNDfrxMechpbI}(M.A. Sinacœr, 1973, s.~63–66).
}. Jest to ideał wiedzy pewnej
wyznawany wspólnie przez nowożytnych autorów o~bardzo różnych poglądach filozoficznych: Kartezjusza, Keplera,
Galileusza, Newtona czy Leibniza. W~przypadku Leibniza
\parencite[s.~51]{zycinski_granice_1993}
%\label{ref:RNDgl3cIrrlVL}(zob. J. Życiński, 1993, s.~51)
tradycja arystotelesowska została dodatkowo połączona z~pewną wersją świata idei Platona. Ten specyficzny platonizm
epistemologiczny przedostał się z~kolei za pośrednictwem prac Fregego i~młodego Russella do dwudziestowiecznej
filozofii z~kręgu Koła Wiedeńskiego\footnote{Zapewne to ten leibnizjański związek arystotelesowskiej \textit{episteme} ze
światem idei Platona ma na myśli Życiński, gdy rozszerza pojęcie \textit{tradycji arystotelesowskiej} na pojęcie
\textit{tradycji platońsko-arystotelesowskiej}, zob.
\parencites[s.~12]{zycinski_structure_1988}[s.~22]{zycinski_struktura_2013}[s.~126]{zycinski_elementy_1996}.
%\label{ref:RND3PMPQPUS6i}(J. Życiński, 1988b, s.~12, 2013, s.~22, 1996, s.~126).
}. W~tak rozumianej tradycji \textit{episteme} obrona rozumu naukowego wymaga odwołania się do Platońskiej
ontologii głoszącej ,,niesprowadzalność idei do procesów psychofizycznych''
\parencite[s.~126]{zycinski_elementy_1996}.
%\label{ref:RNDhHlzendpnr}(J. Życiński, 1996, s.~126).
Życiński mówi również
\parencite[s.~102n]{zycinski_jezyk_1983},
%\label{ref:RNDynpLuVGokl}(1983, s.~102n),
że w~wieku XIX, gdy zakwestionowano naukowy
charakter filozofii, ideał \textit{episteme} został bezkrytycznie przeniesiony na nauki empiryczne. W~epistemologii
nastała era scjentyzmu, której epigonem było Koło Wiedeńskie, w~szczególności Carnap, wraz ze swym pozytywizmem
logicznym.

Odrzucenie nierealistycznego ideału \textit{episteme} nie oznacza jednak dla Życińskiego porzucenia idei
\textit{racjonalności wiedzy} i~przyjęcia sceptycznego punktu widzenia na naukę. Przeciwnie, uważa że możliwa jest
kontynuacja tradycji arystotelesowsko-platońskiej \textit{episteme} i~sam siebie do takiego nurtu filozoficznego zalicza.
Tradycja ta wymaga jednak istotnych zmian. W~miejsce nierealistycznych apriorycznych ideałów należy przedstawić
koncepcje wyjaśniające faktyczne uwarunkowania ,,\textit{rzeczywistości} określanej mianem nauki''
\parencite[zob.][s.~126 -- podkreślenie moje]{zycinski_elementy_1996}.
%\label{ref:RNDDRndo6NuDC}(zob. J. Życiński, 1996, s.~126 -- podkreślenie moje).

\section{Metanaukowe konsekwencje rewolucji naukowej}\label{lia-sec-4}

Za dwa najistotniejsze metanaukowe uwarunkowania rzeczywistej nauki Życiński uznaje obecność w~nauce \textit{elementów
pozanaukowych}\footnote{Jakkolwiek paradoksalnie może brzmieć to wyrażenie, to jest ono jednak wyrażeniem sensownym.
Pod warunkiem wszakże uzmysłowienia sobie faktu świadomej i~w~dużej mierze nieuniknionej odmienności znaczeniowej
terminu `nauka' (i jego przymiotnikowej odmiany) w~dwóch jego wystąpieniach w~tym wyrażeniu. Dla oddania tej
dwuznaczności należałoby zastosować, na przykład, indeksy dolne: `element
\textit{pozanaukowy\textsubscript{1}} w~\textit{nauce\textsubscript{2}}'.
`Nauka\textsubscript{1}' odnosiłaby się do intersubiektywnej,
racjonalnej rzeczywistości artykułowalnej za pomocą języka. Byłaby to
nauka rozumiana zarówno przedmiotowo, jak i~funkcjonalnie. Z~kolei
`nauka\textsubscript{2}' odnosiłaby się do pewnego złożonego,
obserwowalnego \textit{zjawiska}, na które składa się zarówno aspekt intersubiektywny nauki, jak i~jej aspekt podmiotowy.
Termin `nauka\textsubscript{2}' odpowiadałby temu, co Życiński
nazywa \textit{rzeczywistą} lub \textit{realnie istniejącą nauką}, względnie
\textit{rzeczywistością zwaną nauką};
`nauka\textsubscript{1}' byłaby natomiast pewną \textit{wyidealizowaną} i~\textit{uproszczoną} wizją
nauki filozofów, mniej lub bardziej zgodną z~nauką rzeczywistą.} oraz
\textit{czynników pozaracjonalnych}, a~w~przypadku
tych ostatnich ich istotną rolę rozwoju nauki\footnote{Od samego początku Życiński wiąże obecność czynników
pozaracjonalnych w~nauce z~kwestią jej rozwoju
\parencite[zob.][s.~143]{zycinski_jezyk_1983}.
%\label{ref:RNDGwyI8mc6od}(zob. J. Życiński, 1983, s.~143).
}. Odkrycie
tych uwarunkowań przez filozofów należy uznać za metanaukowe konsekwencje rewolucji naukowej. Odróżnienie to wydaje się
jak najbardziej na miejscu. W~nauce ujmowanej przedmiotowo, a~zatem jako pewien uporządkowany zbiór zdań lub sądów,
elementy pozanaukowe nauki to zdania (sądy), których nie można uzasadnić metodą właściwą nauce empirycznej lub
analitycznej. Życiński utożsamia je z~przekonaniami ideologicznymi i/lub filozoficznymi, najczęściej przyjmowanymi w~formie
milczących założeń\footnote{Na temat \textit{ideologii} i~jej odróżnienia od \textit{filozofii}
zob. \parencites[s.~18]{zycinski_structure_1988}[s.~33]{zycinski_struktura_2013}.
%\textit{Strukturę} \label{ref:RND65JRx0WUvQ}(J. Życiński, 1988b, s.~18, 2013, s.~33).
Wśród elementów pozanaukowych
Życiński odróżnia \textit{ideaty} od \textit{przedzałożeń} (\textit{presumptions}) \textit{filozoficznych}: ,,ogólnie rzecz
biorąc, nie można postrzegać ideatów jako wytworu filozoficznych przedzałożeń''
\parencites[s.~29]{zycinski_structure_1988}[s.~51]{zycinski_struktura_2013}.
%\label{ref:RNDNBJ2x09ZUn}(J. Życiński, 1988b, s.~29, 2013, s.~51).
Podobnie nie każde założenie filozoficzne jest godne miana (fundamentalnego) przedzałożenia
(\textit{presumption})
\parencites[s.~143]{zycinski_structure_1988}[s.~253]{zycinski_struktura_2013}.
%\label{ref:RNDKgGZVRdQ4Q}(J. Życiński, 1988b, s.~143, 2013, s.~253).
}. Funkcjonują one jako idee
kształtujące \textit{implicite} zarówno treść teorii naukowych, jak i~zachowania epistemologiczno-metodologiczne
naukowców. Z~kolei w~odniesieniu do nauki ujmowanej funkcjonalnie mowa jest o~pozaracjonalnych czynnikach lub faktorach
wpływających zarówno na sposób uprawiania nauki, jak i~na treść wysuwanych hipotez naukowych. Ich charakter
pozaracjonalny oznacza, iż nie są to uniwersalne\textit{ racje}, lecz konkretne \textit{przyczyny} sprawcze zachowań
naukowych i~metanaukowych. Najczęściej Życiński wymienia przyczyny socjologiczne i~psychologiczne\footnote{Zob. np. 
(\cites[s.~142nn]{zycinski_jezyk_1983}[s.~9]{zycinski_structure_1988}[s.~17,]{zycinski_struktura_2013}
\cite*[a~także][s.~190]{zycinski_elementy_1996}),
%\label{ref:RNDijWA45P0nD}(J. Życiński, 1983, s.~142nn, 1988b, s.~9, 2013, s.~17,  a~także 1996, s.~190),
gdzie
mówi o~\textit{czynnikach psycho-społecznych}.}.

\subsection{4a.~Założenia filozoficzne w~nauce\footnote{Zagadnieniu założeń filozoficznych w~nauce Życiński poświęca wiele miejsca,
np. \textit{Język i~metoda}
\parencite*[s.~246–261]{zycinski_jezyk_1983};
%\textit{Język i~metoda} \label{ref:RNDXAivjQttgc}(J. Życiński, 1983, s.~246–261);
znaczna część pierwszego tomu
\textit{Teizmu} \parencite[s.~156–232]{zycinski_teizm_1985};
%\label{ref:RND6qvog2NMv7}(J. Życiński, 1985, s.~156–232);
pierwszy rozdział \textit{Struktury} oraz część druga
\textit{Granic racjonalności} \parencite{zycinski_granice_1993}.
%(1993).
Niezwykle istotne wypowiedzi na ten temat zawarte są również
w~\textit{Strukturze} przy okazji omawiania epistemologicznej zasady niepewności (\textit{uncertainty}) lub nieoznaczoności
\parencites[s.~143n]{zycinski_structure_1988}[s.~254–256]{zycinski_struktura_2013}.
%\label{ref:RND7zd8YYNkba}(J. Życiński, 1988b, s.~143n, 2013, s.~254–256).
Zob. wypowiedź przytoczoną w~przypisie \ref{lia-foo-32}.}}

Jednym z~istotnych elementów metanaukowej rewolucji było uświadomienie sobie przez naukowców, a~jeszcze bardziej przez
filozofów, filozoficznego zakorzenienia nowej nauki:

\myquote{
nie można zaprzeczyć, że konsekwencje odkryć związanych z~teorią ewolucji wszechświata, z~fizyką czarnych dziur, z~twierdzeniami
limitacyjnymi w~metalogice, czy z~pojawieniem się wielu nowych dyscyplin naukowych są blisko związane z~wielkimi
filozoficznymi pytaniami nurtującymi ludzkość od niepamiętnych czasów
\parencites[s.~8]{zycinski_structure_1988}[s.~16]{zycinski_struktura_2013}.
%\label{ref:RNDlgeSSJugD6}(J. Życiński, 1988b, s.~8, 2013, s.~16).
}
Podobne metateoretyczne implikacje dla filozofii nauki ma metanaukowy fakt stosowania przez naukowców określonej metody
naukowej: 
\myquote{
nawet ci z~przyrodników, którzy są nastawieni niechętnie wobec filozofii muszą -- przynajmniej \textit{implicite} --
przyjmować jakieś założenia filozofii nauki. W~przeciwnym razie przyrodoznawstwo byłoby uprawiane przez nich w~stylu
żywiołowo-\mbox{-naiwnym} %notabene
\parencite[s.~254]{zycinski_jezyk_1983}.
%\label{ref:RNDzLB3079QHH}(J. Życiński, 1983, s.~254).
}

Odkrycia te zachwiały głęboko zakorzenioną modą na antymetafizyczne tendencje pozytywizmu. W~wyniku rewolucji naukowej
jedna z~podstawowych metanaukowych tez pozytywizmu, teza o~mitologicznym charakterze metafizyki i~filozofii w~ogóle,
sama okazała się niekrytycznym mitem
%\label{ref:RNDw5Xs923WOa}(zob. J. Życiński, 1996, s.~228)
\parencite[s.~228]{zycinski_elementy_1996}\footnote{W sposób
szczególny Życiński poddaje krytyce koncepcję pozytywizmu logicznego, pokazując jego milczące założenia metafizyczne
\parencite[s.~246–252]{zycinski_jezyk_1983}.
%\label{ref:RNDk96fcXirdq}(zob. J. Życiński, 1983, s.~246–252).
Życiński obraca metodę racjonalistycznych ,,mistrzów
podejrzeń'' przeciw nim samym. Podobną taktykę stosował Popper, gdy mówił o~\textit{przesądzie empirystycznym} lub
\textit{indukcjonistycznym} w~epistemologii. Miał przy tym na myśli niekrytyczną wiarę empirystów nowożytnych w~indukcję.
Przesąd (mit, poezja) wskazywany przez Życińskiego jest jeszcze bardziej fundamentalną niekrytyczną presupozycją
nowożytnego empiryzmu. Do wyznawców tego mitu zalicza się także Popper, gdyż głęboko wierzył on, przynajmniej we
wczesnym okresie swej twórczości, w~ostrą demarkację nauki i~metafizyki
\parencites[s.~11]{zycinski_structure_1988}[ s.~19n]{zycinski_struktura_2013}[s.~228nn]{zycinski_elementy_1996}.
%\label{ref:RNDdtZAuOTDND}(zob. J. Życiński, 1988b, s.~11, 2013, s.~19n, 1996, s.~228nn).
}.

W ujęciu mniej poetyckim odkrycie elementów pozanaukowych zostaje powiązane przez Życińskiego z~rewolucją naukową
Einsteina-\mbox{-Plancka} %notabene
za pomocą idei \textit{zmiany teoretycznej} i~ideą \textit{zmiany wizji świata} implikowanej przez
teorie. Zmiany te uświadomiły filozofom, że praktyka naukowa nie polega na protokołowaniu czystych faktów
\parencites[s.~34]{zycinski_structure_1988}[s.~59]{zycinski_struktura_2013}
%\label{ref:RNDi0g21xRjLd}(zob. J. Życiński, 1988b, s.~34, 2013, s.~59)
i~że jej rozwój nie jest prostą kumulacją
kolejnych odkryć
\parencite[zob.][s.~229]{zycinski_elementy_1996}.
%\label{ref:RNDl7y894RdpX}(zob. J. Życiński, 1996, s.~229).
W~nowym paradygmacie metanaukowym
neokantowskie w~swej wymowie idee uteoretyzowania obserwacji i~obecności elementu poza-empirycznego w~hipotezach
teoretycznych stały się niekontrowersyjnymi faktami metateoretycznymi
\parencite[por.][s.~127]{zycinski_elementy_1996}.
%\label{ref:RNDLhcTxbq259}(por. tamże J. Życiński, 1996, s.~127).

Radykalne zmiany w~wizji świata implikowane przez nowe teorie wyraźnie wskazują na nieusuwalny element pozanaukowy w~rozwoju
nauki, choćby przez to, że przeczą \textit{zdrowemu rozsądkowi}\footnote{O sprzeczności ze zdrowym rozsądkiem
zob.
\parencites[s.~249]{zycinski_jezyk_1983}[s.~176n]{zycinski_teizm_1985}.
%\label{ref:RND3LfNDVSWRv}(J. Życiński, 1983, s.~249, zob. też 1985, s.~176n).
Według Życińskiego teza Einsteina i~jego
współpracowników o~załamywaniu się w~mechanice kwantowej pojęć języka potocznego posiada istotne implikacje
filozoficzne dotyczące racjonalności ontycznej i~realizmu poznawczego. O~nieusuwalności z~nauki elementów pozanaukowych
zob.
\parencites[s.~34]{zycinski_structure_1988}[s.~60]{zycinski_struktura_2013},
%\label{ref:RNDhn4h2RTiN2}(J. Życiński, 1988b, s.~34, 2013, s.~60),
gdzie pisze, że wszelkie próby wyeliminowania
ideatów z~nauki prowadziły do wprowadzenia nowych, ukrytych ideatów.}. Teoria względności i~teoria kwantów zadały kłam
niejednej zdroworozsądkowej oczywistości determinującej zachowania naukowców. Niezwykle wymowny jest w~tym kontekście 
przykład Einsteina, który nie miał problemu z~zaakceptowaniem nowatorskich idei relatywistycznych, a~mimo to nie
potrafił zaakceptować idei rozszerzającego się wszechświata, idei, która była logiczną konsekwencją jego równań pola.
Silne przywiązanie do ,,oczywistej'' idei statyczności wszechświata skłoniło go do wprowadzenia do równania całkowicie
\textit{ad hoc} członu lambda zapewniającego oczekiwaną statyczność rozwiązania. Bez wątpienia przywiązanie Einsteina do
idei statycznego wszechświata wbrew racjonalnym argumentom stanowi silny argument za tezą o~pozaracjonalnych
uwarunkowaniach decyzji \mbox{Einsteina}\footnote{Na temat Einsteina zob.
\parencites[s.~249]{zycinski_jezyk_1983}[s.~73]{zycinski_structure_1988}%
[s.~130]{zycinski_struktura_2013}[s.~189]{zycinski_elementy_1996}.
%\label{ref:RNDOEPytE2GkR}(J. Życiński, 1983, s.~249, 1988b, s.~73, 2013, s.~130, 1996, s.~189).
Zachowanie Einsteina pokazuje, że idea statyczności wszechświata nie była
ideą ,,wyprowadzoną'' z~faktów, też ani empiryczną hipotezą poddawaną empirycznemu sprawdzaniu. Pojęcie
\textit{przywiązania} lub \textit{związania} (\textit{commitment}) Życiński stosuje w~sposób analogiczny do Kuhna.}.

Innym faktem metanaukowym świadczącym dobitnie o~obecności pozanaukowego elementu filozoficznego w~praktyce naukowej
jest to, iż praktyka falsyfikowania hipotez teoretycznych we współczesnej nauce nierzadko odbiega od idei empirycznej
falsyfikacji w~jej \textit{uproszczonej} i~\textit{wyidealizowanej} wersji przedstawianej przez Poppera
%\label{ref:RNDLar942AozJ}(zob. J. Życiński, 1996, s.~230)
\parencite[zob.][s.~230]{zycinski_elementy_1996}\footnote{Życiński nie ma raczej tutaj na myśli naiwnej wersji
falsyfikacjonizmu przypisywanej wczesnemu Popperowi, czyli takiej, w~której falsyfikatory empiryczne miałyby charakter
bezwzględnie rozstrzygający. W~\parencite[s.~121]{zycinski_jezyk_1983}
%\textit{Języku i~metodzie} \label{ref:RNDyLCTuOy1b8}(J. Życiński, 1983, s.~121)
mówił
bowiem, że naiwnego falsyfikacjonizmu Popper właściwie nigdy nie głosił.}. Zaistnienie logicznej sprzeczności pomiędzy
hipotezą teoretyczną a~faktem nie musi prowadzić do obalenia wyjściowej hipotezy, i~to nie tylko dlatego, że brak
innej, lepszej teorii. Życiński wskazuje na liczne przypadki z~mechaniki kwantowej, gdy wystąpienie empirycznej
niezgodności z~podstawowymi założeniami teoretycznymi tych teorii nie doprowadziło do odrzucenia mechaniki kwantowej, a~jedynie
do prób zrewidowania głębokich założeń filozoficznych tychże teorii. Naukowe reakcje na eksperyment EPR i~na
odkrycie nierówności Bella prowadziły nierzadko do prób negacji założeń realizmu ontologicznego i~epistemologicznego
oraz tradycyjnej idei racjonalności, założeń konstytuujących tradycyjną koncepcję ,,rzeczywistości obiektywnej''
%\label{ref:RND9B582PV6dW}(J. Życiński, 1996, s.~230)
\parencite[s.~230]{zycinski_elementy_1996}\footnote{Przykłady te Życiński omawia szeroko
w~\parencite*[s.~175–180]{zycinski_teizm_1985}.
%\textit{Teizmie} \label{ref:RNDeDVxzhZtXS}(J. Życiński, 1985, s.~175–180).
Wykorzystuje je również metanaukowo
w~\parencites[s.~129n.137]{zycinski_structure_1988}[s.~228nn.243]{zycinski_struktura_2013}[s.~230]{zycinski_elementy_1996}.
%\textit{Strukturze} \label{ref:RNDYDcMfhVjkQ}(J. Życiński, 1988b, s.~129n.137, 2013, s.~228nn.243)
%i~w~\parencite[s.~230]{zycinski_elementy_1996}.
%\textit{Elementach} \label{ref:RNDMGOXueF4r7}(J. Życiński, 1996, s.~230).
}.

Równie istotna z~punktu widzenia implikacji metanaukowych okazała się rewolucja w~pojmowaniu procedury obserwacyjnej.
Już Pierre Duhem wskazywał na głębokie uteoretyzowanie narządzi obserwacji i~eksperymentu
\parencite[zob.][s.~77–81; 85–89]{duhem_pierre_1991}.
%\footnote{Zob. przekłady jego tekstów w: \label{ref:RND7H3en34uaC}(P. Duhem, 1991, s.~77–81; 85–89).}.
Życiński idzie dalej i~mówi o~ich istotnym
obciążeniu ,,bagażem tez ontologicznych'', niezależnie od stopnia świadomości tego faktu przez ich użytkowników
\parencite[s.~249]{zycinski_jezyk_1983}.
%\label{ref:RND1c8rmkjZxD}(J. Życiński, 1983, s.~249).
 Szczególnie wymowne są tutaj jednak radykalne zmiany w~pojmowaniu
przedmiotu obserwacji, jakie pociąga za sobą teoretyczny rozwój fizyki czarnych dziur. W~kontekście tej fizyki traci
swój obiektywny sens takie tradycyjne pojęcie, jak pojęcie \textit{obiektu materialnego}. Jedynym uzasadnieniem jego
użycia mogą być dzisiaj jedynie względy sentymentalne
\parencites[s.~77]{zycinski_structure_1988}[s.~137n]{zycinski_struktura_2013}.
%\label{ref:RNDdNrJ37wYMr}(zob. J. Życiński, 1988b, s.~77, 2013, s.~137n).

Metanaukowy fakt odkrycia obecności nieusuwalnego\footnote{Mówiąc o~nieusuwalności elementu
pozanaukowego z~nauki, a~w~szczególności założeń filozoficznych,
należy wspomnieć o~polemice Józefa Życińskiego
\parencite*{zycinski_czy_2009}
%\label{ref:RND1yIXITCd0q}(2009)
z~Janem Woleńskim
\parencite*{wolenski_odpowiedz_2009}
%\label{ref:RNDgHTjy4YLsK}(2009)
na temat relacji nauki do filozofii i~\textit{vice versa}.
} elementu
filozoficznego w~języku nauki eksploatowany jest przez Życińskiego do różnych celów filozoficznych. Jednym z~nich jest
jego żywe zainteresowanie nową metafizyką, która pozwoliłaby pogodzić religijny obraz świata z~obrazem implikowanym
przez współczesną naukę. Wiele ze swych publikacji poświęcił on opracowaniu nowej wersji teizmu, określanego mianem
\textit{panenteizmu} \parencite[zob. np.][]{zycinski_teizm_1988}.
%\footnote{Tej problematyce poświęcony jest na przykład drugi tom \textit{Teizmu i~filozofii
%analitycznej}
%\parencite{zycinski_teizm_1988}.
%\label{ref:RNDIkjJ8iRdco}(J. Życiński, 1988a).
%}.
Na pytanie ,,Czy można żyć bez metafizyki?'' -- to znaczy
czy można uprawiać naukę bez założeń filozoficznych -- odpowiada, że na metafizykę tak czy inaczej
,,jesteśmy skazani'' i~to niezależnie od składanych na ten temat
deklaracji
\parencite[s.~246.249]{zycinski_jezyk_1983}.
%\label{ref:RNDBwmyHW0qHJ}(J. Życiński, 1983, s.~246.249).
Jej całkowita eliminacja z~języka nauki musiałaby skutkować \textit{zupełnym milczeniem} naukowców. Z~tego punktu widzenia
odkrycie elementów pozanaukowych w~nauce rozumiane jest przez Życińskiego jako rewolucyjna, w~kontekście dominującego
wcześniej pozytywizmu, rehabilitacja metafizyki. Jednocześnie jest też ono wezwaniem do poddania tradycyjnej metafizyki
istotnym modyfikacjom w~obliczu rewolucyjnych zmian w~założeniach filozoficznych wprowadzonych przez nowe teorie
naukowe. 

\subsection{4b.~Czynniki pozaracjonalne w~nauce\footnote{Kwestia czynników pozaracjonalnych jest omawiana przez Życińskiego równie
szeroko, jak kwestia założeń filozoficznych, zob.
\parencites[s.~127–154]{zycinski_jezyk_1983}[s.~156–166]{zycinski_teizm_1985}[rozdz. 5]%
{zycinski_structure_1988}[rozdz. 5]{zycinski_struktura_2013}%
[rozdz. 7]{zycinski_elementy_1996}.
%\label{ref:RND5w30wnfua2}(J. Życiński, 1983, s.~127–154, 1985, s.~156–166, 1988b, rozdz. 5, 2013, rozdz. 5, 1996, rozdz. 7).
	}
}

Z punktu widzenia rewolucji metanaukowej o~wiele ważniejsze są jednak inne, filozoficzno-naukowe implikacje, jakie
Życiński wyprowadza z~faktu odkrycia w~nauce nieusuwalnego elementu filozoficznego. Warto podkreślić, że nie chodzi
tutaj o~implikacje \textit{stricte} metanaukowe, czyli w~terminologii Życińskiego dotyczące logicznej struktury nauki i~jej
procedur uzasadniania, lecz o~implikacje filozoficzno-naukowe, wykraczające poza teren analiz
logicznych i~wkraczające na teren opisu nauki rzeczywistej oraz jej rzeczywistego procesu rozwoju. 

Dla filozofii nauki szczególnie doniosłe okazało się jednoczesne odkrycie rewolucyjnej \textit{zmienności} przyjmowanych w~nauce
założeń filozoficznych. Jak w~przypadku Einsteina, nowe idee stały w~,,rażącej sprzeczności ze zdrowym rozsądkiem''
\parencite[s.~249]{zycinski_jezyk_1983},
%\label{ref:RND5BoFYxeq4l}(J. Życiński, 1983, s.~249),
z~utrwaloną powszechnie \textit{wizją} świata
\parencite[s.~229]{zycinski_elementy_1996}
%\label{ref:RND3mHQnO08gW}(zob. J. Życiński, 1996, s.~229)
i~z~,,epistemetyczną'' \textit{tradycją} rozumienia tego,
czym jest nauka lub racjonalność.  Doświadczenie tego typu sprzeczności świadczy nie tylko o~obecności elementu
filozoficznego w~nowych teoriach, ale także o~jego radykalnej zmianie trudnej do wytłumaczenia na gruncie
racjonalistycznych koncepcji metanaukowych. Tego typu doświadczenie towarzyszyło zarówno porzuceniu arystotelesowskiej
idei doskonałego świata nadksiężycowego za czasów Galileusza, jak i~pojawieniu się idei rozszerzającego się
wszechświata jako zupełnie niespodziewanej przez Einsteina implikacji jego równań pola. Podobne sytuacje dotyczyły
pojawienia się idei zbiorów nieskończonych Cantora, czy idei podzielności atomu, o~twierdzeniach limitacyjnych nie
wspominając\footnote{Podane przykłady pochodzą
z~\parencite[s.~187]{zycinski_jezyk_1983}
%\textit{Języka i~metody} \label{ref:RNDrtmG57I1dG}(J. Życiński, 1983, s.~187)
oraz
z~\parencite[s.~247]{zycinski_elementy_1996}.
%\textit{Elementów} \label{ref:RNDorQuglTydR}(J. Życiński, 1996, s.~247).
W~\textit{Elementach}
\parencite[s.~267n]{zycinski_elementy_1996}
%\label{ref:RNDYo0f9hcHkW}(J. Życiński, 1996, s.~267n)
mówi o~emocjonalnych reakcjach na Gödlowskie twierdzenie o~możliwej
wewnętrznej sprzeczności arytmetyki. Zob. też wcześniejsze uwagi na temat \textit{intelektualnego szoku} i~\textit{paradoksalności}.}.

Zderzenie nowych teorii ze zdrowym rozsądkiem, z~milczącymi oczywistościami, z~niewyartykułowaną intuicją tego, co
normalne i~milcząco oczekiwane, musi dawać do myślenia. I~to nie tyle na temat samych teorii, ile na temat naukowców i~filozofów
reagujących w~ten lub inny sposób na wspomniane zdroworozsądkowe sprzeczności. Zachowania naukowców
wykraczające poza tradycyjne wzorce \textit{racjonalności} czy \textit{normalności} muszą być warunkowane przez coś, co
wykracza poza racjonalną artykulację. To coś Życiński nazywa \textit{czynnikami pozaracjonalnymi} lub \textit{zewnętrznymi
uwarunkowaniami}\footnote{Określenie `czynniki pozaracjonalne' oraz równoważne `elementy aracjonalne' pojawia
się
w~\parencite[s.~142]{zycinski_jezyk_1983}.
%\textit{Języku i~metodzie} \label{ref:RNDQuqexwGAtV}(J. Życiński, 1983, s.~142).
Nieco wcześniej
%\label{ref:RND8JMQ2LGnoG}(J. Życiński, 1983, s.~127)
(s.~127)
mowa jest o~`czynnikach pozalogicznych'. Wyrażenia te Życiński
wzoruje \textit{explicite} na wyrażeniu Satosi Watanabego ,,les éléments arationnels''
%\label{ref:RNDDibcJHr6Ng}(J. Życiński, 1983, s.~142).
(s.~142).
Wyrażenie `uwarunkowania zewnętrzne' pojawia się
w~\parencites[s.~130]{zycinski_structure_1988}[s.~230]{zycinski_struktura_2013}.
%\textit{Strukturze} \label{ref:RNDmQCXMWkgmK}(J. Życiński, 1988b, s.~130, 2013, s.~230).
Stosuje też inne określenia, takie jak `czynniki
podmiotowe', `czynniki subiektywno-osobowościowe', `pozaracjonalne elementy\textit{ }(składniki)' nauki, czy w~końcu
`elementy nieskonceptualizowane'.}. Wybór terminologii nie jest przypadkowy. Życiński przeciwstawia termin
`pozaracjonalny' terminowi `irracjonalny' 
\parencites[zob.][s.~137]{zycinski_structure_1988}[s.~242]{zycinski_struktura_2013}\footnote{%
%\footnote{Zob. \label{ref:RND4GDp5Hkb8p}(J. Życiński, 1988b, s.~137, 2013, s.~242).
Nie tłumaczy jednak bliżej różnicy. Z~kontekstu można wnioskować, że irracjonalność ma charakter
\textit{wewnętrzny} w~stosunku przekonań człowieka, natomiast czynniki pozaracjonalne są wobec nich zewnętrzne, ale je mogą determinować,
na przykład w~kwestii wyboru spośród alternatywnych i~równoważnych teorii}.

Życiński poddaje analizie na przykład ,,szokujący'' fakt zmiany oceny racjonalności hipotezy tachionów we współczesnej
fizyce. Mówi, iż jest to ,,niezwykle interesujący przykład przejścia od fantastycznej koncepcji \textit{science-fiction}
do racjonalnej hipotezy''
\parencites[s.~129]{zycinski_structure_1988}[s.~228]{zycinski_struktura_2013}.
%\label{ref:RNDOVbNCXy3az}(J. Życiński, 1988b, s.~129, 2013, s.~228).
To, co wcześniej
uchodziło za \textit{absurdalne} i~wywoływało \textit{psychologiczną awersję}, w~stosunkowo krótkim czasie zostało
\textit{znormalizowane }w wyniku odpowiedniej \textit{reinterpretacji} znaczeń. Dla wyjaśnienia tego faktu metanaukowego
proponuje on, by rozróżnić racjonalność wewnętrzną teorii od \textit{racjonalności zdroworozsądkowej}, odpowiedzialnej za
doświadczenie szoku, absurdalności i~potrzeby normalizacji. Mówi też, że zdroworozsądkowa racjonalność zależy od
\textit{uwarunkowań zewnętrznych} wobec nauki
%\label{ref:RNDDERgYAAw4H}(J. Życiński, 1988b, s.~130, 2013, s.~229n)
\parencites[s.~130]{zycinski_structure_1988}[s.~229n]{zycinski_struktura_2013}\footnote{Życiński omawia hipotezę tachionów już w~\parencite*[s.~175–176]{zycinski_teizm_1985},
%\textit{Teizmie} \label{ref:RNDGRaNeNBaa8}(J. Życiński, 1985, s.~175–176),
ale nie przeprowadza tam jej analizy metanaukowej i~nie wyprowadza implikacji na temat czynników
pozaracjonalnych. Przykład ten wykorzystuje wyłącznie w~celu analizy kwestii obecności filozoficznych
przedzałożeń w~nauce, w~szczególności realizmu ontologicznego i~epistemologicznego.}.

Wspomniany fakt modyfikacji \textit{ad hoc} wprowadzonej do równań pola przez Einsteina
\parencite[zob.][s.~189]{zycinski_elementy_1996}
%\label{ref:RNDuYYfgAN5No}(zob. J. Życiński, 1996, s.~189)
podlega analogicznej interpretacji. Określenie \textit{ad hoc} wskazuje na irracjonalny --
przynajmniej w~odniesieniu do \textit{racjonalności wewnętrznej} teorii -- charakter modyfikacji Einsteina. A~skoro idea
rozszerzającego się wszechświata stała w~sprzeczności ze \textit{zdrowym rozsądkiem} determinującym zachowania naukowe
Einsteina
\parencite[s.~249]{zycinski_jezyk_1983},
%\label{ref:RNDcdQNYoHSWz}(por. J. Życiński, 1983, s.~249),
to jasno z~tego wynika, że modyfikacja określana
jako \textit{ad hoc} musiała być determinowana przez jakieś czynniki \textit{zewnętrzne} względem samej
teorii\footnote{Podana tutaj metanaukowa interpretacja faktu modyfikacji \textit{ad hoc} nie wyczerpuje wszystkich
metanaukowych konsekwencji, jakie Życiński wyciąga z~tego faktu. Hipotezy \textit{ad hoc} posiadają ambiwalentny status
metanaukowy. Mogą być szkodliwe, irracjonalne, gdy są wprowadzane dla \textit{ratowania} programu badawczego, albo
korzystne, gdy przyczyniają się do rozwoju nauki. Na ten temat zob.
\parencite[s.~162n]{zycinski_jezyk_1983}.
%\label{ref:RNDbdDBcwL7tT}(J. Życiński, 1983, s.~162n).
}.

W \textit{Języku i~metodzie} Życiński ograniczył się do ogólnego utożsamienia czynników pozaracjonalnych z~uwarunkowaniami
socjologicznymi i~psychologicznymi. Z~czasem poddał je jednak dokładniejszej analizie w~perspektywie koncepcji
\textit{wiedzy osobowej} M. Pola\-nyiego %notabene
i~wprowadził szersze pojęcie \textit{elementów nieskonceptualizowanych}, obejmujące
sobą oprócz czynników psycho-społecznych także elementy quasi-racjonalne \textit{wiedzy milczącej}, \textit{wiedzy osobowej}
oraz \textit{intuicji twórczej}
\parencite[zob.][s.~179–191]{zycinski_elementy_1996}\footnote{Przez
%\footnote{Zob. \label{ref:RNDTM62j2cxS8}(J. Życiński, 1996, s.~179–191). Przez
\textit{wiedzę milczącą} Życiński rozumie nie tylko milczące przedzałożenia teorii, lecz także wiedzę zawartą
\textit{implicite} w~równaniach pola czy w~formalizmie mechaniki Newtonowskiej
\parencite[zob.][s.~189]{zycinski_elementy_1996}.
%\label{ref:RNDBBgSxRheRf}(zob. tamże J. Życiński, 1996, s.~189).
Idee Polanyiego Życiński wykorzystuje już wcześniej (zob. wyżej przypis \ref{lia-foo-26}), ale nie czyni
tego w~sposób tak systematyczny jak w~\parencite*{zycinski_elementy_1996}.}.

Niezależnie jednak od ewolucji w~pojmowaniu czynników pozaracjonalnych Życiński niezmiennie od samego początku wskazuje
na Kuhnowską ideę \textit{związania z~paradygmatem }(\textit{commitment}) jako niezwykle użyteczne, a~może nawet najlepsze
narzędzie metanaukowej interpretacji faktu nieuniknionej obecności czynników
pozaracjonalnych w~nauce, a~także w~metanauce\footnote{W
\parencite[s.~153]{zycinski_jezyk_1983}
%\textit{Języku i~metodzie} \label{ref:RNDYoSNyQG6OT}(J. Życiński, 1983, s.~153)
mówi o~niewątpliwej
zasłudze Kuhna ukazania paradygmatycznych uwarunkowań filozofii nauki.
W~\parencite[s.~160]{zycinski_teizm_1985}
%\textit{Teizmie} \label{ref:RNDJ9iyAcBeRp}(J. Życiński, 1985, s.~160)
mówi o~subiektywnym charakterze \textit{commitment} -- przywiązania do jednej z~wielu możliwych
teoretycznie koncepcji poznania i~o~podmiotowym związaniu z~określoną tradycją badawczą, nazywanym \textit{commitment to
paradigms}
\parencite*[s.~164]{zycinski_teizm_1985}.
%\label{ref:RND7uoHzvi4cv}(1985, s.~164).
W~\parencite[s.~191–200]{zycinski_elementy_1996}
%\textit{Elementach} \label{ref:RNDiRufhvdQjS}(J. Życiński, 1996, s.~191–200)
terminy `\textit{commitment}' oraz `paradygmat' poddaje szczegółowej analizie, by pokazać, jak
\textit{rzeczywiście} funkcjonują one w~\textit{realnej} nauce.}.
W~\parencites[s.~136n]{zycinski_structure_1988}[s.~241n]{zycinski_struktura_2013}
%\textit{Strukturze }\label{ref:RNDlo3gRg9nYs}(J. Życiński, 1988b, s.~136n, 2013, s.~241n)
mówi o~potrzebie uzupełnienia braków w~czysto racjonalnym wyjaśnianiu
podejmowanych rozstrzygnięć kwestii naukowych w~różnych dziedzinach, od fizyki, przez matematykę po filozofię nauki,
odwołaniem się do czynnika osobowego \textit{commitment} (\textit{zaangażowanie}) charakterystycznego dla poszczególnych
programów badawczych. Jest to konieczne, gdyż wbrew maksymalistycznym postulatom racjonalistów (dosł. wbrew
,,oczekiwaniom marzycieli'') nie było możliwe wskazanie w~filozofii nauki powszechnie akceptowanych kryteriów
rozstrzygania problemów w~poszczególnych dziedzinach. Również przyjęciu najbardziej fundamentalnych przedzałożeń
filozoficznych w~programach lub tradycjach badawczych musi towarzyszyć element ,,częściowej arbitralności,
wyboru i~zobowiązania'' (\textit{commitment}).
Jest tak, gdyż ich uzasadnienie może być jedynie \textit{częściowe}
\parencites[zob.][s.~143]{zycinski_structure_1988}[s.~253n]{zycinski_struktura_2013}\footnote{\label{lia-foo-55}Oba
%\label{ref:RNDNqJzVjNuqB}(J. Życiński, 1988b, s.~143, 2013, s.~253n).
ostatnie przykłady \textit{commitment} w~nauce
Życiński podciąga pod ogólną nazwę \textit{epistemologicznej zasady niepewności} (\textit{uncertainty}). Taka nazwa pada
również w~\parencite[s.~159]{zycinski_teizm_1985}.
%\textit{Teizmie} \label{ref:RNDTYIhxUDXK4}(J. Życiński, 1985, s.~159).
Polski przekład angielskiej
\textit{Structure} oddaje tę nazwę zarówno jako \textit{zasada nieoznaczoności}
\parencite[s.~240]{zycinski_struktura_2013},
%\label{ref:RNDcZvyKiHaKS}(J. Życiński, 2013, s.~240),
jak \textit{zasada niepewności}
%\label{ref:RNDwSXdH0AKWY}(J. Życiński, 2013, s.~245).
(s.~245).
Faktem jest, że
Życiński wzoruje ją na słynnej zasadzie nieoznaczoności Hei\-senberga. Szczegółowe omówienie tej zasady będzie możliwe
dopiero w~kolejnym artykule. Zob. tekst i~przypis nr \ref{lia-foo-5}.}.

Jednakże koncepcja związania z~paradygmatem, jak zobaczymy niżej, wymaga odpowiednich modyfikacji w~duchu tradycji
\textit{episteme} tak, by je oczyścić z~nazbyt eksternalistycznej interpretacji samego Kuhna.

\section{Metanaukowy spór o~czynniki pozaracjonalne w~nauce: internalizm \textit{vs} eksternalizm}\label{lia-sec-5}

Sytuacja metanaukowa u~progu rewolucji metanaukowej w~ujęciu Życińskiego przedstawiała się zatem następująco. W~wyniku
rewolucji naukowej filozofia nauki odkryła nieusuwalną obecność w~nauce elementów pozanaukowych oraz determinujących te
elementy czynników pozaracjonalych\footnote{Ideaty często zależą wyłącznie od uwarunkowań
psychologicznych i~socjologicznych, a~nie od bardziej podstawowej ontologii, zob.
\parencites[s.~29]{zycinski_structure_1988}[s.~51]{zycinski_struktura_2013}.
%\label{ref:RNDZ7VFgO3ocx}(J. Życiński, 1988b, s.~29, 2013, s.~51).
}. Ponieważ odkrycia te stały w~sprzeczności z~dotychczasowym ideałem \textit{episteme}, konieczna była
reakcja ze strony filozofów i~w~celu ich ,,normalizacji''. Zgodnie ze swym związaniem z~szeroko pojmowaną tradycją
\textit{episteme}, Życiński stawia wymóg, by były to rozwiązania \textit{racjonalne}, wykluczając tym samym rozwiązania
sceptyczne czy anarchiczne
\parencites[s.~143]{zycinski_structure_1988}[s.~254]{zycinski_struktura_2013}.
%\label{ref:RNDlHYNmrMqD0}(zob. J. Życiński, 1988b, s.~143, 2013, s.~254).
Krytyczne
związanie z~tradycją \textit{episteme} pozwala Życińskiemu na racjonalną ocenę wszystkich analizowanych rozwiązań.

Nowe rozwiązania metanaukowe w~ujęciu Życińskiego były całkowicie rozbieżne. W~tym miejscu Życiński z~historyka
metanauki zamienia się w~teoretyka metanauki. Rozbieżne tendencje rozwiązywania problemu obecności czynników
pozaracjonalnych w~nauce klasyfikuje za pomocą dwóch przeciwstawnych kategorii. W~pierwszych pracach
mówi o~rozwiązaniach \textit{internalistycznych} i~\textit{eksternalistycznych}. Z~czasem wprowadza specyficzne, teoretyczne
kategorie epistemologiczne `internalizmu' i~`eksternalizmu'\footnote{Przeciwstawienie to funkcjonuje od samego
początku w~tekstach Życińskiego
\parencites*[s.~141nn]{zycinski_jezyk_1983}[s.~12.16]{zycinski_structure_1988}[s.~21.29]{zycinski_struktura_2013}.
%\label{ref:RNDhYHdk6Gn9I}(J. Życiński, 1983, s.~141nn, 1988b, s.~12.16, 2013, s.~21.29).
Termin
`eksternalizm' pojawia się w~\parencite[s.~120n]{zycinski_teizm_1985},
%\textit{Teizmie} \label{ref:RNDqnijmYbBgM}(J. Życiński, 1985, s.~120n),
ale oznacza tam
jedynie specyficzne stanowisko metamatematyczne, uznające matematykę za wynik uwarunkowanych genetycznie ludzkich
intuicji logicznych oraz kulturowo uzależnionych zdolności do łączenia abstrakcyjnych tez. W~sensie ogólnych kategorii
metanaukowych oba terminy pojawiają się już
w~\parencite*[s.~242]{zycinski_granice_1993},
%\textit{Granicach racjonalności} \label{ref:RND37pLqxd4Aa}(J. Życiński, 1993, s.~242),
ale dopiero
w~\parencite*[s.~133nn]{zycinski_elementy_1996}
%\textit{Elementach filozofii nauki} \label{ref:RNDY1V0zunmmK}(J. Życiński, 1996, s.~133nn)
Życiński wykorzystuje je do opisu ogólnej sytuacji problemowej w~metanauce i~w~filozofii nauki.}. Wydaje się, że
właśnie te kategorie dostarczają klucza do najlepszej interpretacji filozofii nauki Życińskiego. Pełnią one w~jego
filozofii rolę uogólnionych teoretycznych kategorii meta-metanaukowych, pozwalających poddać analizie merytorycznej
(treściowej) wszystkie dwudziestowieczne teorie (rozwiązania) metanaukowe. Za ich pomocą Życiński odkrywa i~nazywa
najogólniejsze presupozycje teoretyczne kształtujące \textit{implicite} charakter dwudziestowiecznych rozwiązań
metanaukowych. Można powiedzieć, że kategorie te służą mu do hipotetyczno-teoretycznej interpretacji \textit{racjonalnej}
\textit{rzeczywistości} (\textit{sic!}) metanauki\footnote{Od dziewiętnastego wieku, od czasów J. Herschela, albo jeszcze
wcześniej, od czasów Kanta, wiadomo jednak, że wszelkie kategorie opisowe i~klasyfikacyjne mają jednocześnie charakter
aprioryczno-teoretyczny, a~zatem w~dużej mierze idealizujący. \textit{Racjonalną rzeczywistość }metanauki można rozumieć
jako \textit{historyczną artykulację racjonalności obiektywnej} typowej dla metanauki. Ideę \textit{racjonalności
obiektywnej} przedstawią analizy zawarte w~części drugiej mojego artykułu.}.

\begin{uwaga}
Heurystyczna funkcja procedury klasyfikacyjnej na gruncie nauk przedmiotowych stosunkowo wyczerpująco
została przedstawiona przez Herschela \parencite*[s.~131–139]{herschel_wstep_1955}.
%\label{ref:RND1u8fbCVzxR}(J.F. Herschel, 1955, s.~131–139).
Klasyfikacji zjawisk
dokonuje się z~nadzieją na ,,pośrednie'' odkrycie głębszych, ukrytych przyczyn względnie praw przyrody. Klasyfikacja jest
hipotezą teoretyczną. Ewentualne potwierdzenie empirycznych predykcji klasyfikacji uwiarygodnia jej status jako
poprawnej hipotezy wyjaśniającej. Analogiczna sytuacja zachodzi w~odniesieniu do klasyfikacji teoretycznych, czyli
klasyfikacji różnego typu teorii naukowych. Pozwala ona na odkrycie ukrytych, głębszych i~bardziej ogólnych (w sensie
dedukcyjnym) przesłanek względnie presupozycji teoretycznych warunkujących treść określonych rozwiązań
teoretycznych. W~fizyce przykładem takiej klasyfikacji może być podział rozwiązań problemu natury światła na rozwiązania
korpuskularne i~rozwiązania falowe, z~nadzieją na pokonanie tego podziału na ,,głębszym'' poziomie teoretycznym.
\end{uwaga}

Oba typy klasyfikacji, klasyfikacje zjawisk (faktów) i~klasyfikacje teoretyczne, funkcjonują zarówno na poziomie nauki
przedmiotowej, jak i~na poziomie metanauki, a~przynajmniej na poziomie metanauki w~rozumieniu Poppera, Lakatosa czy
Życińskiego, której postulowana metoda (specyficznie transcendentalna) jest analogiczna do metody empirycznej.

Przez \textit{internalizm} Życiński rozumie stanowisko metanaukowe
\parencite*[s.~134]{zycinski_elementy_1996},
%\label{ref:RNDCyQqUrHYuu}(zob. J. Życiński, 1996, s.~134),
według którego ,,treść teorii i~twierdzeń naukowych jest determinowana przez wewnętrzną zawartość ich racjonalnych
uzasadnień, natomiast czynniki pozaracjonalne mogą okazać się istotne dla nauki dopiero w~sytuacjach, gdy nie można
przedstawić merytorycznych uzasadnień dla proponowanych tez''.  \textit{Eksternalizm} jest określany natomiast jako
stanowisko przeciwne względem internalizmu. Tym, co łączy wszystkie koncepcje eksternalistyczne jest szczególne
,,wyakcentowanie'' związków między treściową zawartością nauki a~zewnętrznymi uwarunkowaniami jej rozwoju
\parencite[s.~134]{zycinski_elementy_1996}\footnote{W
%\label{ref:RNDuAc97ED1Y1}(J. Życiński, 1996, s.~134)
\parencite[s.~242]{zycinski_granice_1993}
%\textit{Granicach racjonalności} \label{ref:RND1EgQJ3J1nD}(J. Życiński, 1993, s.~242)
pada bardziej lakoniczne określenie \textit{internalizmu} jako
ujęcia kładącego nacisk na racjonalność nauki. Z~kolei określenie \textit{eksternalizmu} jest bardziej radykalne: oznacza
ujęcia traktujące naukę jako ,,wynik \textit{zewnętrznych} pozaracjonalnych czynników''.}. Do rozwiązań internalistycznych
Życiński zalicza koncepcje Koła Wiedeńskiego, Poppera i~Lakatosa, z~kolei do rozwiązań eksternalistycznych koncepcje
wczesnego Kuhna, Feyerabenda, Szkoły Edynburskiej i~różne koncepcje postmodernistyczne. Kategorie
`internalizmu' i~`eksternalizmu' podlegają jednak gradacji orzekalności w~zależności od stopnia uwzględnienia czynników
pozaracjonalnych. Sytuacja jest analogiczna jak w~przypadku orzekalności kategorii `racjonalizmu' i~`irracjonalizmu'.

\begin{uwaga}
W~toczonym w~neokantowskiej perspektywie sporze o~charakter referencjalny pojęć teoretycznych (zarówno
nauki przedmiotowej, jak i~metanauki) Życiński opowiada się po stronie umiarkowanego realizmu. Opowiada się za
umiarkowanym realizmem teoretycznym nie tylko w~odniesieniu do języka nauk empirycznych, lecz także w~odniesieniu do
języka metanauki. Umiarkowany realizm teoretyczny oznacza, że terminy teoretyczne oprócz aspektu konstruktywnego
posiadają także aspekt realno-opisowy (referencyjny). W~przypadku kategorii metanaukowych znaczy to, że odnoszą się one
referencjalnie do \textit{racjonalnej rzeczywistości} metanauki. Referencja terminów teoretycznych według realizmu
umiarkowanego jest jednak tylko pośrednia. Nie można \textit{wskazać }bezpośrednio na ich referenty. W~przypadku terminów
metanaukowych można jedynie wskazać ich historycznie zmienne \textit{przejawy} pod postacią \textit{różnych teorii}
lub \textit{koncepcji}.
\end{uwaga}

\begin{uwaga}
Istnienie racjonalności obiektywnej jest zatem \textit{odmiennym sposobem istnienia} od istnienia
przedmiotów empirycznych. To już kwestia ontologii przyjmowanej przez Życińskiego. Wydaje się, że jego koncepcja
sposobów istnienia ,,przedmiotów'' będących desygnatami teoretycznych terminów metanauki bliska jest partycypacyjnej
ontologii neoplatońskiej będącej próbą pogodzenia ontologii Arystotelesa i~Platona. Racjonalność obiektywna mająca swe
źródło w~samym Bogu obecna jest \textit{implicite} w~świecie i~w~ludzkiej myśli, a~ujawnia się poznawczo nie wprost,
lecz w~sposób uwikłany w~historii tejże myśli ludzkiej. Jej poznanie wymaga odpowiedniej hermeneutyki.
\end{uwaga}

W tej perspektywie koncepcja Lakatosa jest mniej radykalnie internalistyczna niż koncepcja Poppera, a~ta z~kolei mniej
niż koncepcje Koła Wiedeńskiego, gdyż w~różnym stopniu uwzględniają one rolę czynników zewnętrznych. Podobnie koncepcja
Kuhna wydaje się być mniej radykalnie eksternalistyczna niż mocny program socjologii wiedzy\footnote{Życiński pisze
przykładowo
\parencites*[s.~135]{zycinski_structure_1988}[s.~239]{zycinski_struktura_2013}:
%\label{ref:RNDMFFbOpIuQe}(1988b, s.~135, 2013, s.~239):
,,Asercje o~\textit{różnym stopniu racjonalności}
[podkreślenie moje] pojawiają się w~obrębie poszczególnych programów, a~nawet w~obrębie poszczególnych teorii''. Pisze
też, że wiele umiarkowanych tez Kuhna zawartych w~jego \textit{Strukturze rewolucji naukowych} zostało
zradykalizowanych w~socjologicznej interpretacji nauki
\parencite[zob.][s.~207n]{zycinski_elementy_1996}.
%\label{ref:RND7QncKOKY2H}(zob. J. Życiński, 1996, s.~207n).
}.

W przypadku eksternalizmu Życiński rozróżnia
\parencite*[por.][s.~135]{zycinski_elementy_1996}
%\label{ref:RNDAJVAwcgRgv}(por. J. Życiński, 1996, s.~135)
między
eksternalizmem umiarkowanym a~skrajnym. Pierwszy nie neguje elementu racjonalnego w~rozwoju nauki, lecz traktuje rozwój
nauki jako wypadkową elementów racjonalnych i~czynników pozaracjonalnych. Z~kolei eksternalizm skrajny traktuje
wewnętrzną racjonalność nauki jako funkcję samych tylko zewnętrznych czynników psycho-społecznych. W~\textit{Strukturze}
funkcjonuje nieco inny podział na trzy możliwe rodzaje rozwiązań eksternalistycznych
\parencites*[s.~123n]{zycinski_structure_1988}[s.~218]{zycinski_struktura_2013}.
%\label{ref:RNDcwjdYvX9ov}(J. Życiński, 1988b, s.~123n, 2013, s.~218).
Pierwszy typ rozwiązań ogranicza się do uznania czynników pozaracjonalnych za
pozaracjonalne \textit{inspiracje} występujące wyłącznie w~kontekście odkrycia, drugi mówi o~\textit{niezwykle ważnej roli}
tychże czynników (elementów) \textit{także} w~okresie rewolucji naukowej, natomiast trzeci uznaje \textit{dominującą} rolę
tych czynników nad elementami racjonalnymi zarówno w~okresie powstania, jak i~rozwoju programu badawczego. To ostatnie
rozwiązanie Życiński z~góry odrzuca jako nieracjonalne.

Życiński poddaje szczegółowej analizie krytycznej wszystkie najważniejsze dwudziestowieczne rozwiązania racjonalistyczne
(internalistyczne) i~sceptyczne (eksternalistyczne). W~jednym i~drugim przypadku pokazuje ich wewnętrzne
ograniczenia, a~wielokrotnie wewnętrzne sprzeczności. Tego typu analizy zajmują znaczną część jego twórczości metanaukowej. Nawet
pobieżna ich prezentacja musiałaby zająć zbyt wiele miejsca. Ograniczmy się zatem do konkluzji, jakie wyciąga on z~tej
analizy oraz do kilku przykładów. Konkluzje formułuje następująco:

\myquote{
Długie dyskusje na temat natury wiedzy naukowej pokazały, że nauka nie jest ani tak racjonalna, jak chciał tego młody
Popper, ani tak socjologicznie uwarunkowana, jak twierdził Kuhn w~pierwszym wydaniu \textit{Struktury rewolucji naukowej}
\parencites[s.~145]{zycinski_structure_1988}[s.~256]{zycinski_struktura_2013}.
%\label{ref:RNDelOLSePcR6}(J. Życiński, 1988b, s.~145, 2013, s.~256).
}
W innym miejscu tę samą ideę rozwija w~typowym dla siebie retorycznym stylu:
\myquote{
Świadomość epistemologicznych ograniczeń nauki oraz uwarunkowania historyczne wpływające na niektóre determinanty
racjonalności nie dają zatem żadnych podstaw do stwierdzenia, że wszystkie przekonania dotyczące wewnętrznej
racjonalności nauki są przejawem nierealnych marzeń. Niewątpliwie takimi marzeniami były maksymalistyczne idee
filozofów epoki Wiktoriańskiej, którzy na podobieństwo wcześniejszych metafizyków poszukiwali racjonalności
sekretnej i~tajemnej. Z~faktu, że ich nadzieje okazały się płonne, nie wynika jednak, że nauka jest irracjonalna. Można jedynie
powiedzieć, że [sama] racjonalność jest różna od tego, czego oczekiwano. Upraszczające zastępowanie racjonalności
socjologią może okazać się równie nierealnym, marzycielskim przedsięwzięciem
\parencites*[s.~123]{zycinski_structure_1988}[s.~217]{zycinski_struktura_2013}.
%  (1988b, s.~123; 2013, s.~217).
}

W dychotomicznym ujęciu stanowisk metanaukowych po rewolucji naukowej Popper zajmuje miejsce internalizmu, Kuhn miejsce
eksternalizmu. Krytykę pierwotnej koncepcji dedukcjonizmu Poppera jako teorii nazbyt optymistycznej Życiński przejmuje
od Lakatosa
\parencite[zob.][s.~120n]{zycinski_jezyk_1983}.
%\label{ref:RNDCRLw5MmQwf}(zob. J. Życiński, 1983, s.~120n).
Od siebie dodaje, że  ,,wierze w~racjonalność
nauki i~w~potęgę falsyfikacji był również zawarty element uproszczenia idealizujący procedury stosowane w~rzeczywistej
nauce'' 
\parencite[s.~230]{zycinski_elementy_1996}.
%\label{ref:RNDR9nCa1gaG1}(J. Życiński, 1996, s.~230).
Przykładowo, w~historii nauki istnieje wiele modyfikacji
\textit{ad hoc}, które podobnie jak modyfikacja Einsteina miały pozytywny wpływ na rozwój nauki, i~to wbrew
\textit{irracjonalności} postulowanej przez Popperowską teorię falsyfikacji
\parencite[zob.][s.~108]{zycinski_elementy_1996}.
%\label{ref:RNDq9oApJerP4}(zob. J. Życiński, 1996, s.~108).
Przypadek Poppera i~jego koncepcję modyfikacji \textit{ad hoc} można zatem zinterpretować jako specyficzny
fakt metafilozoficzny, który daje Życińskiemu do myślenia. Popperowska teza o~irracjonalności modyfikacji \textit{ad hoc}
nie wynika logicznie z~faktów opisujących rzeczywiste zachowania naukowców, lecz okazuje się wyrazem nazbyt
wyidealizowanej i~nazbyt apriorycznej koncepcji racjonalności. Koncepcja Poppera jest zatem wyrazem silnych założeń
filozoficznych i~jako taka wskazuje na obecność zewnętrznych uwarunkowań Poppera. W~terminologii Życińskiego ten
subiektywny czynnik wyboru i~związania z~tradycją filozoficzną przyjmuje postać optymistycznej \textit{wiary}
Poppera w~racjonalność nauki i~w~potęgę falsyfikacji\footnote{W tym kontekście Życiński przywołuje też przykład naukowych reakcji
na wynik eksperymentu EPR czy na  odkrycie nierówności Bella, których nie potraktowano jako falsyfikatory, jak to
powinno się stać w~myśli koncepcji Poppera.}.

Analogicznej krytyce poddana zostaje koncepcja Kuhna i~jego radykalnie eksternalistyczna interpretacja
\textit{związania z	paradygmatem} z~1962 roku. W~perspektywie tej interpretacji, jak zauważa Życiński 
\parencite*[s.~106]{zycinski_jezyk_1983},
%\label{ref:RNDaETzNFGzG9}(J. Życiński, 1983, s.~106),
logika nauki zostaje niemal całkowicie zastąpiona socjologią wiedzy. Tymczasem bliższa analiza
historyczna pokazuje, że w~dłuższej perspektywie możliwe jest racjonalne wykazanie wyższości jednego programu
badawczego nad innym, gdyż wewnętrzna logika rozwoju nauki jest silniejsza od emocjonalnych przywiązań do
poszczególnych teorii
\parencite*[s.~165]{zycinski_jezyk_1983}.
%\label{ref:RND184R61G3Ny}(J. Życiński, 1983, s.~165).
Z~kolei w~\parencite*[s.~195]{zycinski_elementy_1996}
%\textit{Elementach}
stwierdza, że
rzeczywista nauka obala tezę o~dogmatycznym charakterze związania z~paradygmatem, gdyż ,,zmiana paradygmatu jest nie
tylko możliwa teoretycznie, lecz również zachodzi rzeczywiście w~nauce''.
%\label{ref:RND8iXlMhyTs3}(J. Życiński, 1996, s.~195).

Najsilniejszym argumentem przeciw skrajnie eksternalistycznej interpretacji związania z~paradygmatem jest argument na
rzecz epistemologicznej \textit{konieczności} takiego związania w~sytuacji, gdy niemożliwe jest precyzyjne
rozstrzygnięcie pomiędzy alternatywnymi tradycjami badawczymi. W~takiej sytuacji jedyną alternatywą dla idei
podmiotowego \textit{commitment} byłby jedynie sceptycyzm. Ten natomiast, jak pamiętamy, jest z~góry wykluczony, gdyż nie
jest reakcją racjonalną na fakt czynników pozaracjonalnych w~nauce. Zatem zgoda na podmiotowy \textit{commitment} jest
jedynym racjonalnym rozwiązaniem w~tej sytuacji\footnote{Jest to argument rozwijający ideę epistemologicznej zasady
niepewności. Zob. \hyperref[lia-sec-4]{poprzedni punkt}.}.

Skoro ani skrajny internalizm, ani skrajny eksternalizm nie wytrzymują krytyki, to Życiński konkluduje o~konieczności
zanegowania sensu zaistniałej dychotomii wśród tych rozwiązań. Co więcej, uważa że rozwianie marzycielskich tendencji
metanaukowych skrajnego internalizmu i~skrajnego eksternalizmu dokonało się nie tyle za sprawą wewnętrznych analiz
metanaukowych, ile za sprawą samej nauki, która nie dała się wtłoczyć w~sztywne ramy ideologiczne\footnote{Jest to
przykład sprzeczności transcendentalnej rozumianej jako sprzeczność metanaukowych hipotez z~faktami metanaukowymi.}:

\myquote{
W tej sytuacji [gdy programy badawcze i~teorie cechują się różnym stopniem racjonalności] próby absolutyzowania pojęcia
racjonalności, jak również skłonność do dychotomicznego dzielenia interpretacji [metanaukowych] na
racjonalne i~irracjonalne stanowią wyraz pewnej filozofii bronionej w~sposób dogmatyczny, niemniej sfalsyfikowanej przez samą naukę
\parencites[s.~135]{zycinski_structure_1988}[s.~239]{zycinski_struktura_2013}.
%\label{ref:RNDmPUxCCAI7w}(J. Życiński, 1988b, s.~135, 2013, s.~239).
}

Konfrontacji z~rzeczywistą nauką nie oparł się ani internalizm, ani eksternalizm. Dokładniejsza obserwacja faktycznej
nauki pokazuje, ,,że rzeczywisty rozwój nauki przebiega w~odmienny sposób niż sugerowali to przedstawicie normatywnych
metodologii''
\parencite[s.~230]{zycinski_elementy_1996}.
%\label{ref:RNDWCF6lU0sbi}(J. Życiński, 1996, s.~230).
Z~kolei eksternalizm radykalny wikła
się w~nieusuwalne ,,antynomie'' w~wyjaśnianiu faktów metateoretycznych z~dziejów nauki\footnote{Poświęcony im jest cały
podrozdział w~\parencite[s.~147–156]{zycinski_elementy_1996}.
%\textit{Elementach} \label{ref:RNDrpsugHpH8D}(J. Życiński, 1996, s.~147–156).
}.

Dychotomia rozwiązań metanaukowych będących reakcją na rewolucję naukową jest faktem. Pytanie, czy jest nieunikniona?
Życiński podejmuje się pokazać, że ma ona jedynie charakter historyczny, a~zatem niekonieczny. Możliwe jest
unieważnienie sporu internalizmu z~eksternalizmem. Powstaje pytanie, jak tego dokonać?

W myśl nadrzędnej zasady racjonalności -- występującej \textit{implicite} choćby w~przytoczonym przed chwilą argumencie na
rzecz racjonalności podmiotowego \textit{commitment} w~sytuacjach wyborów alternatywnych -- skoro rzeczywista nauka
unieważnia oba człony metanaukowej alternatywy \textit{internalizm versus eksternalizm}, to racjonalną reakcją na tę
sytuację nie może być sceptycyzm, lecz nowe zmodyfikowane rozwiązanie racjonalne.

Z powyższych tekstów można wnioskować (poszukując ich ukrytych presupozycji), że kategorie
`internalizmu' i~`eksternalizmu' służą do sproblematyzowania metanaukowej idei \textit{normatywnej demarkacji
elementów wewnętrznych
(racjonalnych) i~zewnętrznych (pozaracjonalnych) w~rozwoju nauki}\footnote{Przypomnienie: treść idei, podobnie jak
treść pojęć i~koncepcji, dla ułatwienia lektury, podaję kursywą.} jako szczególnego przypadku problemu
\textit{racjonalności jako takiej. }Nowe rozwiązanie może potraktować ten spór jako ślepą odnogę ewolucji pojęcia
racjonalności.

Falsyfikacja teorii metanaukowych przez fakty metateoretyczne z~rzeczywistej nauki nie implikuje ich całkowitej
bezwartościowości i~konieczności ich odrzucenia oraz zaproponowania całkiem nowych teorii metanaukowych. Życiński
preferuje bardziej umiarkowaną strategię rozwoju teoretycznego w~metanauce, zaczerpniętą od późnego Poppera i~od
Lakatosa. Wystarczy dokonać jedynie niezbędnych modyfikacji w~dotychczasowych rozwiązaniach metanaukowych. 

Modyfikacje w~przekonaniu Życińskiego powinny polegać na ,,liberalizacji'' radykalnych postaw metanaukowych po obu
stronach sporu. Wprawdzie należy przyznać rację Lakatosowi, że ,,żadne kryterium demarkacji nie ma charakteru
absolutnego''
%\label{ref:RNDlahtfGg0vv}(J. Życiński, 1996, s.~230)
\parencite[s.~230]{zycinski_elementy_1996}\footnote{Ale w~tym samym rozdziale
\parencite*[s.~244]{zycinski_elementy_1996}
%\label{ref:RNDU4zrzQHMRM}(J. Życiński, 1996, s.~244)
Życiński uznaje Lakatosa za przedstawiciela nazbyt
wyidealizowanej, normatywnej koncepcji nauki.}, ale nie znaczy to, że należy odejść od metodologii normatywnej, lecz
jedynie to, że należy rozwinąć i~zliberalizować te normy, które doprowadziły do powstania wyidealizowanej koncepcji
nauki. Podobnie, chcąc uniknąć antynomii eksternalizmu radykalnego, ,,trzeba przyjąć jego umiarkowany wariant, który
głosi, iż elementy racjonalne oraz zewnętrzne uwarunkowania psychospołeczne `przenikają się'
wzajemnie i~uzupełniają w~procesie rozwoju nauki''
\parencite[s.~156]{zycinski_elementy_1996}.
%\label{ref:RND0qiCuFReRd}(J. Życiński, 1996, s.~156).

W obu przypadkach przyczyną (a zatem czynnikiem pozaracjonalnym) pojawienia się błędnych rozwiązań radykalnych była
nieuświadomiona tendencja do nadmiernego \textit{upraszczania} procedur naukowych na poziomie wyjaśniania
metanaukowego i~do idealizacji\footnote{Życiński zarzuca mu nadmierną idealizację kwestii związania z~paradygmatem
\parencite*[zob.][s.~152]{zycinski_jezyk_1983}.
%\label{ref:RNDIrA1jDg11P}(zob. 1983, s.~152).
}. Tendencja ta stanowi cechę charakterystyczną ,,koncepcji niedojrzałych'',
mitologicznych, opierających się na jednym uniwersalnym czynniku (przyczynie) mającym wyjaśnić złożoną rzeczywistość.
Zarówno skrajny internalizm, jak i~skrajny eksternalizm można zatem uznać w~perspektywie analiz Życińskiego za
metanaukową wersję myślenia mitologicznego na kształt mitu ostrej demarkacji przypisywanej Popperowi. Jako takie mają
one jednak charakter wyłącznie historyczny. Według Życińskiego, długie spory o~kryterium demarkacji doprowadziły do
uświadomienia sobie idealizacyjnych uproszczeń i~do sukcesywnego ,,odkrycia złożonej prawdy o~bogactwie procedur
badawczych i~wzajemnych uwarunkowań występujących w~\textit{realnej} nauce''
\parencite[s.~230]{zycinski_elementy_1996}.
%\label{ref:RNDT7FuhQK2nh}(por. J. Życiński, 1996, s.~230).
Dzięki temu odkryciu możliwe jest obecnie ,,wypracowanie \textit{bardziej realistycznych} wzorców
metanaukowych'' (s.~237)\footnote{W obu cytatach kursywa
pochodzi ode mnie. Wskazuje ona na specyficzny charakter metodologii metanauki Życińskiego.}.

Najlepszym kandydatem teoretycznym do poprawnego ujęcia racjonalnego rozwoju nauki jest zdaniem Życińskiego odpowiednio
zliberalizowana koncepcja Lakatosa \textit{naukowych programów badawczych}
%\label{ref:RNDR4AQgoKDGC}(zob. J. Życiński, 1996, s.~244n)
\parencite[s.~244n]{zycinski_elementy_1996}\footnote{Za przykład pozytywnych modyfikacji podaje on prace J.~Waralla, E.~Zahara,
P.~Urbacha czy J.~Watkinsa.}. Liberalizacja polega na rezygnacji z~absolutyzowania norm metodologicznych wskazanych przez
Lakatosa i~uznaniu ich tylko za \textit{przybliżony wzorzec} racjonalnych zachowań naukowych, adekwatny
jedynie w~odniesieniu do
\textit{określonych okoliczności}:

\myquote{
Przejawem metanaukowego dogmatyzmu byłoby traktowanie normatywnej metodologii Lakatosa w~tym samym stylu, w~jakim przed
laty narzucano nauce kanony Koła Wiedeńskiego. [...] Jego [tzn. Lakatosa] propozycje zdają się zawierać najbardziej
rozwiniętą syntezę krytycyzmu i~racjonalności na poziomie refleksji metanaukowej. Niezależnie od emocjonalnych
deklaracji autora [...] metodologia programów badawczych, w~porównaniu z~konkurencyjnymi propozycjami, stwarza
najlepszą szansę racjonalnej rekonstrukcji nauki i~racjonalnego wyboru rywalizujących teorii. Metodologię tę można
odrzucić a~priori, np. z~racji aksjomatycznego przyjęcia tezy o~irracjonalnych mechanizmach rozwoju nauki. Jeśli jednak
unika się równie radykalnej aksjomatyki, można z~ostrożnością i~krytycyzmem F. Suppego «skonkludować, iż Lakatos ukazał
częściowo pewien wzorzec rozumowania w~rozwoju wiedzy naukowej, który to wzorzec -- w~określonych okolicznościach --
charakteryzuje poprawne rozumowanie naukowe» [\textit{Afterword}, 670]. Konkluzja ta, a~zwłaszcza położony w~niej akcent
na cząstkowy charakter opracowań Lakatosa, nie oznacza absolutyzowania przedstawionych opracowań metanaukowych, lecz
wskazuje na potrzebę dalszych poszukiwań umożliwiających pełniejszą charakterystykę racjonalnych struktur nauki
%\label{ref:RND6tjtqNjtZ0}(J. Życiński, 1996, s.~245n)
\parencite[s.~245n]{zycinski_elementy_1996}\footnote{Cytat w~tekście pochodzi z~posłowia Fredericka Suppego
do znanego zbioru tekstów
\parencite{suppe_structure_1974}.
%\label{ref:RNDtowH5ssMQB}(F. Suppe, 1974).
}.
}

Zgodnie z~tymi deklaracjami, proponowane przez Życińskiego rozwiązanie metanaukowe nie posiada charakteru
prostej i~precyzyjnej, uporządkowanej logicznie (dedukcyjnej) teorii metanaukowej wychodzącej od kilku prostych aksjomatów
metanaukowych. Raczej jest to pewien niedookreślony logicznie obraz metodologiczny nauki, składający się z~wielu
stosunkowo luźno powiązanych reguł metanaukowych. Do istotnych elementów rozwiązania Życińskiego należy zaliczyć jego
koncepcję ideatów i~ideologicznych programów badawczych, mających na celu wykazać zasadniczą ciągłość i~tym samym
racjonalność w~historycznym rozwoju nauki, pomimo występowania rewolucji naukowych i~związania naukowców z~tradycją
badawczą. Do istotnych elementów należą również trzy zasady: zasada aracjonalności i~naturalności interdyscyplinarnej
oraz epistemologiczna zasada niepewności. Ich celem jest racjonalna normalizacja funkcji czynników
pozaracjonalnych w~nauce tak, by zachowana została nadrzędna funkcja elementu racjonalnego.
Wśród tych elementów nie może zabraknąć
również rozróżnień wielu typów racjonalności oraz idei ewolucji pojęcia racjonalności. Dopiero omówienie wszystkich
tych elementów pozwoli poznać doksatyczny charakter proponowanego przez Życińskiego rozwiązania. Zagadnienia te ze
względu na swą obszerność zostaną przedstawione w~kolejnym artykule (zob. przypis nr \ref{lia-foo-5} niniejszej pracy).

\pagebreak %notabene
\medskip
{\centering$\ast\ast\ast$\par}
\smallskip


Na zakończenie tej części chciałbym powrócić do fundamentalnej idei metafilozoficznej Życińskiego, jaką jest postulat,
by metanaukowe i~filozoficzne koncepcje na temat natury i~struktury nauki odnosiły się do \textit{rzeczywistej} nauki
względnie do \textit{rzeczywistości} nauki
\parencite*[s.~16 i~126]{zycinski_elementy_1996},
%(zob. 1996, ss.~16 i~126)
a~nie do wyidealizowanych i~uproszczonych
konstrukcji filozofów. Idea ta pozwala zrozumieć nacisk, jaki nasz autor kładzie na rolę czynników pozaracjonalnych w~nauce
oraz na rolę tej kategorii metanaukowej w~rozwiązaniach metanaukowych, a~jednocześnie pozwala zrozumieć jego
racjonalistyczny optymizm w~obliczu różnych skrajnych interpretacji eksternalistycznych.

Mówiąc o~tradycji badania realnej lub rzeczywistej nauki Życiński wskazuje Kuhna i~Polanyiego jako prekursorów tego
trendu
\parencites[s.~153.169]{zycinski_jezyk_1983}[s.~231]{zycinski_elementy_1996}.
%\label{ref:RNDi6TPvuVUK4}(J. Życiński, 1983, s.~153.169, 1996, s.~231).
Jeden i~drugi kojarzą się wszakże
jednoznacznie z~ideą czynników pozaracjonalnych w~nauce. To pokazuje, że w~jego rozumieniu \textit{rzeczywista nauka} lub
\textit{rzeczywistość nauki} to nie tylko logiczna konstrukcja językowa w~perspektywie metody uzasadnienia, lecz także
cały kontekst osobowy, w~jakim konstrukcje teoretyczne funkcją. Co więcej, nie wystarczy tradycyjne rozwiązanie
racjonalistyczne, by kontekst osobowy utożsamić z~psychologicznym kontekstem odkrycia. Bez uwzględnienia czynników
pozaracjonalnych także w~kontekście uzasadnienia i~akceptacji nie jest w~ogóle możliwe pełne zrozumienie samej natury
nauki.

Radykalny charakter tych stwierdzeń najlepiej ilustrują same wypowiedzi Życińskiego. W~\textit{Elementach} pisze, że
,,pomijanie roli czynników pozaracjonalnych w~nauce prowadzi do wyidealizowanego i~nierealistycznego obrazu struktur
nauki'',
%\label{ref:RND07iSFRLobf}(J. Życiński, 1996, s.~189),
a~nieco dalej, że ,,uwzględnienie ich roli stanowi
\textit{konieczny warunek} pełniejszego \textit{zrozumienia istoty} nauki''
\parencite[s.~189.190, podkreślenia moje]{zycinski_elementy_1996}.
%\label{ref:RNDVV3U1jSC9D}(J. Życiński, 1996, s.~190, podkreślenia moje).
 Czynniki pozaracjonalne zostają powiązane w~sposób nierozerwalny z~naturą nauki. Jest to
rzeczywista rewolucja metanaukowa.

W tej sytuacji trudno dziwić się praktyce Życińskiego podkreślania obecności czynników pozaracjonalnych w~nauce i~w~metanauce
za pomocą specyficznej, psychologicznie naznaczonej terminologii, jak choćby wspomnianej kategorii `szoku'.
Życiński nie obawia się jednak posądzenia o~psychologizm, gdyż wypowiada te słowa z~wnętrza szeroko rozumianej tradycji
\textit{episteme}, z~którą uważa się związany\footnote{Zob.
\parencite*[s.~126]{zycinski_elementy_1996},
%\textit{Elementy} (1996, s.~126)\label{ref:RNDxQIHHr1tkz}(J. Życiński, 1996, s.~126),
gdzie Życiński zalicza siebie do nurtu epistemologii kontynuującego platońsko-arystotelesowską
tradycję \textit{episteme}.}. A~ponieważ dotychczasowa tradycja \textit{episteme} cechowała się nazbyt racjonalistycznym i~przeidealizowanym
podejściem do nauki, to podkreślanie roli czynników pozaracjonalnych wydaje się w~pełni uzasadnione.

Także związanie z~tradycją \textit{episteme} przejawia się u~Życińskiego specyficzną praktyką językową. Tym razem
przeciwną w~stosunku do poprzedniej. Gdy wypowiada się na temat nieusuwalnej obecności czynników pozaracjonalnych w~nauce,
dodaje od razu  dla przeciwwagi, że obecność ta nie implikuje w~żadnym wypadku konkluzji o~\textit{dominacji
subiektywizmu} w~nauce
%\label{ref:RND5RKyXkJozz}(J. Życiński, 1996, s.~187).
\parencite[ s.~187]{zycinski_elementy_1996}.
Podobne zastrzeżenia czyni  mówiąc o~roli
czynników pozaracjonalnych w~poznaniu natury nauki:

\myquote{
Obecność w~nauce wiedzy osobowej oraz nieskonceptualizowanych intuicji nie upoważnia do kwestionowania wewnętrznej
racjonalności nauki. Uświadamia ona natomiast, iż racjonalna refleksja, która inspirowała zarówno \textit{Elementy} Euklidesa,
jak i~\textit{Principia} Newtona, łączy się w~kontekście rozwoju nauki z~pozaracjonalnymi czynnikami psycho-społecznymi.
Uwzględnienie ich roli stanowi konieczny warunek pełniejszego zrozumienia istoty nauki
\parencite[s.~190]{zycinski_elementy_1996}.
%\label{ref:RND6rPInFRHqu}(J. Życiński, 1996, s.~190).
}

Natura nauki z~perspektywy tradycji \textit{episteme} jest zatem zasadniczo racjonalna, ale element racjonalny w~rzeczywistym
świecie zawsze występuje w~kontekście osobowego \textit{commitment}. Czysty rozum naukowy jest filozoficznym
mitem. Nie jest to jednak w~żadnym wypadku kapitulacja rozumu. Zgodnie z~epistemologiczną zasadą niepewności w~sytuacji,
gdy mamy do czynienia z~wyborem alternatywnym albo -- albo, gdzie jednym z~członów jest uznanie istotnej, ale
nie dominującej roli czynników pozaracjonalnych w~nauce, a~drugim eksternalistyczny sceptycyzm, to wybór rozwiązania
uwzględniającego rolę czynników pozaracjonalnych w~nauce jest jedynym racjonalnym rozwiązaniem.






\end{artplenv}
