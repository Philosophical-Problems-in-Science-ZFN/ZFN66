\begin{artengenv}{Michael Heller}
	{How is \textit{philosophy in science} possible?}
	{How is \textit{philosophy in Science} possible?}
	{How is \textit{philosophy in science} possible?}
	{Translated by Bartosz Brożek and Aeddan Shaw\edtfootnote{In this edition some quotations have been replaced by the translations of their original sources (if available). The references have been adjusted to the standards of the journal.}\label{heller-start}}
	{The Michael Heller's article entitled ``How is \textit{philosophy in science} possible?'' was originally published in Polish in 1986
		\parencite[see][]{heller_jakmozliwa_1986} and then translated into English by Bartosz Brożek and Aeddan Shaw and published in 2011 in the collection of essays entitled \textit{Philosophy in Science. Methods and Applications} \parencite{heller_howpossible_2011}. This seminal paper has founded further growth of the `philosophy in science' and become the reference point in the methodological discussions, especially in Poland. On the 40\textsuperscript{th} anniversary of \textit{Philosophical Problems in Science} we wanted to make this paper freely available to the international public by reprinting its English version. In this issue it is followed by two additional articles-commentaries (by Paweł Polak and Kamil Trombik).}
	{philosophy in science, philosophy of science, metaphilosophy, interdisciplinary research, science and religion, analytic philosophy.}



\section{Introduction}






\lettrine[loversize=0.13,lines=2,lraise=-0.05,nindent=0em,findent=0.2pt]%
{P}{}hilosophy in science' grew out of practice.\label{heller-out-of} Its most significant example is the phenomenon of the `philosophizing
physicists'. And even though the philosophical reflection of the representatives of the empirical sciences often falls
short of the professional philosophical standards, it does not change the fact that the sciences are filled with
philosophical contents.

\AddToShipoutPictureFG*{% Add <stuff> to current page foreground
	\put(\LenToUnit{22mm},\LenToUnit{168.5mm}){\begin{minipage}[t][23mm][t]{\textwidth}`\end{minipage}}%
}%

In the recent years in the Polish philosophical literature such terms as `philosophical issues in science' have
appeared on the covers of several publications.\footnote{Cf. \textit{Zagadnienia Filozoficzne w Nauce}, a periodical
published in Kraków since 1978; see also
\parencite{heller_zagadnienia_1980}.
%\label{ref:RNDeyUVXQ9KCX}(Heller, Lubański and Ślaga, 1980).
} The English
`philosophy in science', through its contrast with, and similarity to `philosophy of science', has been `sanctioned' in
the title of a new periodical.\footnote{\textit{Philosophy in Science} is published by Pachart Publishing House,
Tuscon. The first volume appeared in 1983.} The paper by W.H. Stoeger, published in the first volume of
\textit{Philosophy in Science}, may be considered a manifesto of the editorial board, as well as an attempt to provide
a theory of `philosophy \textit{in} science'.\label{heller-stoeger}

I am against any planning what kind of philosophy should be practised, i.e. determining \textit{a priori} the method
of analysis and its consequent application. It is more natural when the methodological reflection follows the period of
abundant, sometimes instinctive or even chaotic research in a new discipline. I believe, however, that the time has
come for an attempt to systematize what \textit{de facto} is `philosophy in science'.\label{heller-defacto}


\section{Philosophy in science and philosophy of science}

Among the philosophers of nature (in particular those belonging to the neo-thomistic school) there is a commonly
accepted doctrine of the non-intersecting planes. Generally speaking, it says that philosophical cognition lies at a
totally different epistemological plane than the empirical sciences; they use different methods and operate with
mutually untranslatable languages.\footnote{This is a kind of philosophy advanced in two books:
	\parencite{mazierski_prolegomena_1969,klosak_z_1980}.
%\label{ref:RNDDukAXvxyUz}(Mazierski, 1969; Kłósak, 1980).
Both these authors seem to see the need for the mutual
influences of philosophy and the sciences and develop subtle distinctions in order to open the way for such influences
despite the non-intersecting planes.} In order to justify this view the theories developed within the contemporary
methodology are cited. It is sometimes tempting to say that the major motive behind such stances is to safeguard one's
philosophy against any conflict with the sciences, as well as the theoretical justification of one's incompetence in
the sciences.

The proponents of the two planes doctrine may protest against the `philosophy in science' project as
methodologically flawed and epistemological nonsense, an attempt at a comparison of the incomparable. I recall those
objections not in order to dismiss them (the best way to reply to them is through the results already obtained in the
`philosophy in science' field), but to underline the relationship between `philosophy in science' and philosophy of
science. It is obvious that any philosophizing which is open for the dialogue with the empirical sciences must take
into account their achievements. Otherwise it would be subject to the objection of anachronism. It is equally difficult
to reject the claim that there exist serious differences between the `cognitive plane' of the empirical sciences and
some philosophical currents. I do not believe, however, in any strict isolationism: of the philosophy in relation to
the sciences, or \textit{vice versa}. The methodological bans will be breached anyway, and it is often through the
violation of the received canons that new paradigms emerge, i.e. some progress is made in our attempts to understand
the world: the two non-intersecting planes may turn out to be elements of the same stratification of a more-dimensional
space.

`Philosophy in science' has \textit{de facto} been practised from the beginnings of the empirical sciences. For
example, looking at the Newton's oeuvre, it is difficult to determine whether it is a case of science in philosophy, or
already of philosophy in science.\label{heller-newton} Thus, an attempt to categorize \textit{ex post }the problems of `philosophy in
science' is possibly realizable; however, in face of the richness of this problematic, I shall concentrate on a
succinct analysis of three exemplary issues. Although they do not exhaust the content of `philosophy in science', they
remain typical examples so that they enable to reconstruct its nature and methods. In what follows I shall present (A)
the influence of the philosophical ideas on the development and evolution of scientific theories; (B) the traditional
philosophical problems intertwined with empirical theories; (C) philosophical reflection over some assumptions of the
empirical sciences.\label{heller-three-points}


\section{The influence of philosophical ideas on the development and evolution of scientific theories}


Empirical science originated through the separation from the old, all-embracing philosophy and still bear the
imprint of this origin. Contemporaneously, various philosophical ideas often serve as an inspiration for developing new
conceptions in the empirical sciences. However, many methodologists defend the `purity' of science by introducing the
well-known distinction: indeed, in the context of discovery philosophical ideas often influence the development of
science, however it is not their role only---other factors, even irrational ones, may be influential in the process of
arriving at new discoveries; on the other hand, in the context of justification, i.e. the sphere of the proper
science-creating activities, philosophy has no bearing---it is an `alien body', effectively eliminated by the built-in
mechanisms of science. It is the disregard for this distinction that led to the phenomenon of the `philosophizing
physicists'---the representatives of the empirical sciences who, wrongly taking the context of discovery for the
discovery itself, believe to have something philosophically interesting to say, while in fact they reveal only their
psychological associations.

In the recent years, the distinction between the two contexts has been severely criticized. A case in point is the
following passage from Stefan Amsterdamski's study: %
%Zastąpiłem fragmentami z tłumaczenia książki Amsterdamski. 
%Oryginał z tłumaczenia Brożek\&Shaw: „Metaphysics, myths, prejudices are in a way an immanent element of science on par with the fact we try to incorporate into a rational reconstruction. Neoplatonic philosophy of Kepler or Copernicus constituted an aspect of the rational order of the universe they were trying to uncover in the same sense the strictly empirical statements of their astronomical systems did.'' (Amsterdamski, 1973, p.99)
%nieznany
%June 28, 2019 11:54 PM

\myquote{
Metaphysics, myths or superstitions are in some manner as immanently a part of science as the facts which we attempt to
include into the rational reconstruction. The neoplatonic metaphysic of Kepler and Copernicus were as much an element
of the rational organization of the universe which they attempted to reconstruct as the strictly empirical statements
of their astronomic systems.
\parencites[p.99]{amsterdamski_miedzy_1973}[pp.65--66]{amsterdamski_between_1975}
%\label{ref:RNDU9z9ka2JPW}(Amsterdamski, 1973, p.99, 1975, pp.65---66)
}
To put it more
succinctly: %
%Zastąpiłem fragment: „Science thus always comprises not only the claims about the studied universe, but also the assumptions pertaining to the nature of the subject who practices science.'' (Amsterdamski, 1973, p.100) 
%nieznany
%June 28, 2019 11:55 PM
\myquote{
Therefore, science consist not only of statements about the universe under study, but also of assumptions about the
knowing subject.
\parencites[p.100]{amsterdamski_miedzy_1973}[p.66]{amsterdamski_between_1975}
%\label{ref:RNDhNkCU0uqR3}(Amsterdamski, 1973, p.100, 1975, p.66)
}
If this line of argument is sound,
`philosophy of science' is simply a part of the science itself.

It is worth underlining, that the psychological or sociological accounts of the philosophy of science---which have
recently gained in strength and prestige---almost completely dispense with the distinction between the `logic of science'
and the `external circumstances' of that logic
\parencites{amsterdamski_miedzy_1983}{amsterdamski_between_1992}[cf.][]{zycinski_jezyk_1983_in_hell}.
%\label{ref:RND1eO94kZzna}(Amsterdamski, 1983, 1992; cf. Życiński, 1983).
It is not my goal to engage in a philosophical discussion. However, I personally consider the distinction between the
context of discovery and the context of justification useful under the condition that it is  understood in a flexible
way, which paves the way to a gradual passage from one  context to the other. All in all, the impossibility of drawing
a sharp demarcation line between `inspirations' and `justification' is a sufficiently strong argument in favour of the
`philosophy in science'.

Another conception of the contemporary methodology which clearly points towards some philosophical elements in
science is the so-called thematic analysis, proposed by Gerald Holton \label{ref:RNDQaGpAxlOZj}(cf. Holton, 1998). He
believes that in many concepts, methods, claims and hypothesis of science there are certain elements he calls
\textit{themata}, which as if from hiding influence or even determine the development of new scientific ideas.
\textit{Themata} often come in pairs (of opposites), sometimes in triplets, and have surprising durability over the
centuries---they are capable of surviving many scientific revolutions. Here are some examples of \textit{themata}:
unity---multiplicity; determinism---indeterminism; continuity---discontinuity; symmetry; invariance, complementarity, etc.
Holton is surprised with the relatively small number of \textit{themata}---in physics he identified some 100
thereof---and underlines their interdisciplinary and philosophical character. \textit{Themata} may constitute the
pivotal ideas for the studies in the history of science, but considered from the perspective of their philosophical
load they are nothing else but `philosophy in science'.


\section{Traditional empirical problems intertwined with empirical theories}


One can enumerate a number of such problems or rather clusters of problems. Here, I shall limit myself to examples
pertaining to time and space. It would be difficult to find a philosophical system that has nothing to say about time
and space; and it would be difficult to identify a relatively comprehensive contemporary physical theory that would
assume no theses pertaining to time and space. A classical objection against such bonding of philosophy with empirical
theories consists in stressing the fact that any doctrine which `migrates' from philosophy to the `specialized'
disciplines loses irrevocably its philosophical character, and the only thing that speaks to its philosophical origins
are words, which---even though they sound the same---have completely changed their old meanings. As elsewhere, the doctrine
of planes guards here the purity of philosophy. As I remarked earlier, it is not my goal to fight this doctrine; I
would like to show, however, that philosophy exercises much more direct influence over the development of empirical
theories than granted by the traditional wisdom.

Sometimes, in philosophy a view or a complex set of ideas---we shall say: a doctrine---is established which becomes a
kind of paradigm or a research programme for one or more empirical theories. It so happens that philosophical paradigms
are incorporated into some empirical theories (possibly in violation of the rule that forbids trespassing from one
`plane' to the other, while changing its `meaning content'); but it happens also that a paradigm resists all such
attempts, which leads to partial effects or side-effects only. When an empirical theory succeeds in realizing such a
philosophical programme, one may say that the given empirical theory is a model of the given philosophical doctrine.
The conception of empirical models of philosophical doctrines is still awaiting a more thorough analysis. Below, I
confine myself to examples pertaining to the philosophy of space and time.

In the famous \textit{Scholium} at the beginning of his \textit{Philosophiae Naturalis Principia Mathematica},
Newton formulated a philosophical doctrine of the absoluteness of time and space:

\myquote{
Absolute, true, and mathematical
time, of itself, and from its own nature flows equably without regard to anything external, and by another name is
called duration.---Absolute space, in its own nature, without regard to anything external, remains always similar and
immovable.
\parencite[Scholium B]{newton_philosophiae_1687}
%\label{ref:RND5nqR34354w}(Newton, 1687, Scholium B)
}

Today one would say that these definitions functioned
within the context of discovery of the classical mechanics. It is certainly true, but this was not their only role. It
was Newton's intent to incorporate the doctrine of the absolute time and space into the new mechanics. Newton himself,
as well as generations of physicists that followed him, believed that he had succeeded in doing so. However, a careful
analysis, with the use of the contemporary mathematical tools, reveals that---indeed---the absolute time plays an important
role in the structure of the classical mechanics, but the structure does not include an element that would correspond
to the philosophical intuitions pertaining to the absolute space
\parencite[pp.57--81]{raine_science_1981}.
%\label{ref:RND4xun9zJHBa}(cf. Raine and Heller, 1981, pp.57---81).
Thus, one must carefully distinguish between Newton's own views concerning space and time and the structure
of space and time presupposed by the Newtonian mechanics. The fact that Newton's views are incompatible with the
`views' of his mechanics is clear evidence that philosophical ideas are active not only in the contexts of discoveries,
but are also intimately linked to the history of justifications of scientific theories.

To sum up this stage of our reflection, one may succinctly say: the classical mechanics is a physical model of the
philosophical doctrine of the absolute time; however, it is not a physical model of the doctrine of the absolute
space.\footnote{In connection to the problem of the logical structure of the classical mechanics analysed with the use
of the contemporary mathematical techniques, it is also worth mentioning two studies
\parencite{friedman_foundations_1983,torretti_relativity_1983}.
%\label{ref:RNDIYv6tEO5va}(Friedman, 1983; Torretti, 1983).
}

The `other side' of this story is equally instructive. Long before Newton there was known a philosophical doctrine
rival to the conception of the absolute time and space. Its most famous incarnation was formulated by Leibniz:

\myquote{
As for
my own opinion, I have said more than once, that I hold \textit{space} to be something \textit{merely relative}, as
\textit{time} is; that I hold it to be an \textit{order of coexistences}, as time is an \textit{order of successions}.
\parencite[p.57]{leibniz_mr._1717}
%\label{ref:RNDRSL4adjdiS}(Leibniz, 1717, p.57)
}

Despite the clear attractiveness of the Leibnizian philosophy of time
and space, it belonged the philosophy textbooks only till the development of  the theory of relativity
\parencite[cf.][]{heller_physicists_1975}.
%\label{ref:RNDn9tYnMyJ9c}(cf. Heller and Staruszkiewicz, 1975).
The obvious reason for this was that neither Leibniz
nor any of his followers managed to create a physical model of the philosophical doctrine of the relative character of
space and time
\parencite[cf.][]{raine_science_1981}.
%\label{ref:RNDlDZpjVPVrx}(cf. Raine and Heller, 1981).
There is a deeply rooted conviction that such a
model was provided by the general relativity theory. This conviction proved essentially wrong,\footnote{The problem is
more subtle than the above considerations suggest. One would need to identify at least several senses of `relational'
and `absolute'. There is no place in this essay to go into the details, thus I recommend the cited works
\parencite{raine_science_1981},
%\label{ref:RNDRTE2UAbuq1}(Raine and Heller, 1981),
as well as
\parencite{friedman_foundations_1983}.
%\label{ref:RNDf9yHzW0dHY}(Friedman, 1983).
} but the
analysis led to a new, interesting observation. In the past, the doctrines of the absoluteness and relativeness of time
and space were treated as mutually exclusive; only one of them would turn out true, \textit{tertium non datur. }The
general relativity theory falsified this view: it is a model of a partially relational (as it depends on the bodies
that populate it), and a partially absolute (in the Newtonian sense) space-time
\parencite[cf.][chap.13]{raine_science_1981}.
%\label{ref:RNDxfuOAfIRMd}(cf. Raine and Heller, 1981, chap.13).

This example illustrates again in which way a philosophical doctrine reveals its presence (or absence) in empirical
theories; it is completely independent of the beliefs of the authors of these theories (i.e., the problem lies beyond
the context of discovery), and often in violation of such explicit beliefs. An empirical theory may be---or not---a
physical model of some philosophical doctrine: it is its fully objective feature, which may be analysed with the
contemporary formal means.\label{heller-model-phil}

The elements of the conception of absolute time and space stubbornly remain \textit{inside} the theories of the
contemporary physics, despite many attempts at their removal and creating a physical model of a doctrine of fully
relative space and time. One may even say that the drive towards such a model is one of the determinants of the
tendencies in the contemporary theoretical physics. It is in this sense also that philosophical doctrines are present
in the evolution of science.


\section{Philosophical reflection over some assumptions of the empirical sciences}

This type of analysis has long been applied in the contemporary philosophy. For example, it is the general framework
of the important part of Husserlian phenomenology. Here however, a different aspect of this problematic is interesting.
Again, it is suitable to use examples. I shall sketch the problems surrounding the following assumptions of the
empirical sciences: (a) the assumption of the mathematicity; and (b) of the idealizability of nature, as well as (c)
the assumption of the elementary character and (d) the unity of nature. These assumption may in a natural way be joined
in pairs (a-b and c-d), which should be analysed together. A number of remarks and short commentaries concerning these
assumption has already been formulated; however, they still await a more thorough, monographic study that would provide
a precise formulation of the fundamental questions to which the assumptions inevitably lead.

(a) \textbf{The assumption of the mathematicity of nature.} From the most  general point of view, the mathematicity
of nature boils down to the fact that nature can be described mathematically. It may be considered a fact since it is
`empirically' confirmed by the development of the sciences from the times of Galileo and Newton. Moreover, this
development is extremely efficient, documented with a sequence of successes, both theoretical and pertaining to the
`technical' conquest of nature.

The mathematicity of nature may be considered a counterpart of the medieval \textit{intelligibilitas entis}---the
comprehensiveness of being. In this context, %
%Tłumacz opuścił frazę: „Wigner mówił o „niezrozumiałej zrozumiałości świata'' a [Einstein...{]}
%nieznany
%July 1, 2019 12:47 AM
Wigner discusses ``incomprehensible comprehensibility of the universe'', and Einstein remarks that
``the most incomprehensible thing
about our universe is that it is comprehensible.'' In order to better grasp this problem one should distinguish between
at least three senses in which nature could have been non-mathematical:

\begin{enumerate}
\item Nature could have been amathematical, i.e. non-describable with the use of any mathematics. This would mean that
nature is irrational and would probably exclude it from existence.\footnote{It must be stressed that I am speaking of
the mathematicity of nature only. The complicated problem of the relationship between `mathematicity' and mental
phenomena cannot be addressed in this essay.}
\item Nature could have been mathematically transcendent in relation to our cognitive capacities, i.e. mathematics
needed to adequately describe nature would require such formal means that are in principle inaccessible to our
cognition. Simple models of universes that are non-mathematical in this sense were constructed by Kemeny
(\cite*{kemeny_philosopher_1959}; \cite[see also my study][pp.112--119]{heller_spotkania_1974})
%\label{ref:RNDyqLz7y4oHx}(Kemeny, 1959; see also my study Heller, 1974, pp.112---119)
and Staruszkiewicz
\parencite*{staruszkiewicz_co_1980}.
%\label{ref:RNDVbbltpjBRC}(1980).
\item Nature could have been mathematically too complicated in relation to our capacities, but not in principle---only
regarding the level of difficulty. Some level of difficulty would make impossible or very unlikely the rise and
development of the empirical sciences. For example, the fact that the Newtonian equation
$$
F = G \frac{m_1\  m_2}{r^2}
$$
%{\centering
%\textit{F} = \textit{G}(\textit{m}\textit{\textsubscript{1}}
%\textit{m}\textit{\textsubscript{2}})/\textit{r}\textit{\textsuperscript{2}}
%\par}
approximates well the gravitational force between two point masses, facilitated or even enabled the development of the
theory of universal gravitation at the end of the 16\textsuperscript{th} Century. If the exponent in the denominator
did not equal 2, but, say, 2.009, the orbits of planets would be so complicated that Kepler would most probably fail to
discover any significant regularities.
\end{enumerate}

This final understanding of mathematicity of nature is strictly connected to the next tacit assumption of the
contemporary empirical method, i.e.:

(b) \textbf{the assumption of the idealizability of nature.} It is worth noticing that the modern empirical method
proved successful not when it began its experimental game with nature, but when people learnt to ignore a number of
`inessential' factors of that game. The failure of the Aristotelian physics as an empirical science was connected to
its insistence on accounting for the entire complexity of nature (friction or drag were not ignored). One may even say
that the `creation' of `non-existent', but mathematically simple `entities' was a prerequisite of the success of the
empirical method, to mention but the class of inertial coordinate systems, energetically isolated systems, etc. The
possibility of approximating nature with sufficiently simple mathematical models is the mathematical manifestation of
the idealizability of nature.\footnote{Some aspects of this problem are discussed in my paper
	\parencite{heller_o_1983}.
%\label{ref:RNDAMyUgwcR21}(Heller, 1983).
}

The assumption of  the idealizability of nature accommodates also the assumption of its stability of a certain kind.
For example: if small perturbations of an observable measurement led to significantly different (non-equivalent in
certain respect) mathematical models of the studied domain, then---given the fact that observable parameters are always
measurable with some perturbations (measurement error)---the study of nature would be impossible. By excluding such
situations, one assumes the observational stability of nature. The observational stability of nature is a special case
of a more general concept, that of the structural stability of nature. By postulating such a kind of stability, one
needs to determine an equivalence class of structures, kinds and magnitude of their perturbations and assume that a
small perturbation does not exclude the given structure from the equivalence class.\footnote{On the subject of the
concept of structural stability and its applications in the methodology of the sciences see
\parencite{szydlowski_filozoficzne_1983}.
%\label{ref:RNDPm7wMAH8UM}(Szydłowski, 1983).
} The role of structural stability was stressed by René Thom
\parencite*{thom_stabilite_1977},
%\label{ref:RNDsgQb0egyxL}(1977),
but a systematic discussion of this problem in relation to the philosophy of science
is still missing.

In the contemporary empirical sciences a significant role is reserved for probabilistic models. When operating with
them, one needs to assume a special kind of stability, known as frequency stability. In the standard probability
calculus, the probability measure of the elementary events is taken to be represented by the numbers close to their
observed frequency. Such a definition of probability assumes that the future series of similar experiments shall, in
the long run, give relative frequencies substantially similar to the relative frequencies observed currently. This 
assumption---which is verified both in our ordinary experience and in the scientific practice---is called the assumption of
frequency stability. It attributes to the world a certain feature, thanks to which it can be studied probabilistically
\parencite[cf.][]{heller_kilka_1985}.
%\label{ref:RNDUm28JJfui2}(cf. Heller, 1985).

The problems of the mathematicity and idealizability of the universe are connected to one additional issue. Both
these assumptions attribute to nature a feature, which is responsible for the nature's mathematicity and
idealizability, but they also say something about the human mind, which is capable of accounting for nature as
mathematical or idealizable. Thus, the assumptions in question may be considered both from the ontological and the
epistemological perspectives. It is also possible that one cannot take one of the perspectives, while excluding the
other. This problem must also be scrutinized.

The assumptions of the mathematicity and idealizability of nature are strictly connected to:

(c) and (d) \textbf{the assumptions of the elementary character }and\textbf{ unity of nature.} These assumptions are
counterparts of two essential features of the mathematical method. Understanding in mathematics may proceed either in
the direction of analysis (towards axioms and primitive concepts of the given mathematical theory) or in the direction
of synthesis (i.e., towards `embedding' the given mathematical `entity' within some global structure, from which it can
be---artificially?---extracted). The reductionist and holistic explanations outside of mathematics have their sources in
the same two opposite tendencies of the human mind.

The assumption of the elementary character of nature urges us to uncover the `elementary level' in reality. At the
first sight, it seems that the process of descending towards more elementary levels never ends (as the drive `to
understand' requires to reduce any `data' to something more elementary) or must be `artificially' terminated by a
conventional acceptance of some `rudimentary' level. In the contemporary theoretical physics there is a strong tendency
to reduce physics to pure mathematical structures. In this sense, the `mathematical material' becomes elementary for
physics.\footnote{This is illustrated by the example of the concept of matter, which---during the evolution of
physics---was replaced by purely formal structures; cf. my paper
\parencite{heller_ewolucja_1982}.
%\label{ref:RNDhQcUluq6il}(Heller, 1982).
}

The problem of the unity of nature has been analysed in detail
\parencite[cf.][]{weizsacker_unity_1980}.
%\label{ref:RNDjAhBcetsBs}(cf. Weizsäcker, 1980).
Doubtless, it has many dimensions. One of them is the clearly visible tendency of the contemporary physics to develop
unification theories. However, from the philosophical point of view a deeper dimension of the problem is constituted by
the unity postulated by the very mathematical-empirical method of studying the universe.

In this context a question arises: may totality (i.e., unity in one of its meanings) turn out to be an elementary
category? Even if it is not the case, I believe that the assumptions of unity and the elementary character of nature
must be analysed together. Possibly, one has no definite sense without the other.


\section{A Proviso and an appeal}


It goes without saying that the above mentioned problems are only a preliminary catalogue of questions delineating
`philosophy in science'. Under no condition the above considerations should be considered an attempt to provide event
partial answers.

It was also not my intent to provide a theory of `philosophy in science', although I am not against such
undertakings. I would only protest against calling `philosophy in science' some meta-considerations which are not
rooted in the scientific practice.\label{heller-not-rooted} However, this proviso is barren: philosophical issues in science are so interesting
that they will be contemplated irrespective of any appeals or restrictions. They require interdisciplinary research and
thus only one appeal is in place---an appeal for a responsible cooperation between philosophers-methodologists and the
representatives of the empirical sciences. Only through expertise in both disciplines it may be guaranteed that
`philosophy in science' will not transform into commonsensical (and hence: naïve) considerations, but will become a
truly creative domain of knowledge, one indispensable in the contemporary intellectual ambience.\label{heller-conclusion}








\end{artengenv}\label{heller-stop}
