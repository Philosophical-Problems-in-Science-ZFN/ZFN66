\begin{recplenv}{Krzysztof Śleziński}
	{W trosce o~kulturę logiczną w~badaniach filozoficznych i~teologicznych}
	{W trosce o~kulturę logiczną w~badaniach filozoficznych\\i~teologicznych}
	{Stanisław Kamiński, Kazimierz Marek Wolsza (red.), ser. \textit{Polska filozofia chrześcijańska XX wieku,}
		Wydawnictwo Naukowe Akademii Ignatianum, Kraków 2019, ss.~232.
		
		Stanisław Kamiński, Kazimierz Marek Wolsza (ed.), ser. \textit{The Polish Christian Philosophy in the
			20\textsuperscript{th} Century}, Ignatianum University Press, Kraków 2019, ss.~236.}







W wersji polskiej i~angielskiej ukazał się kolejny tom z~serii wydawniczej \textit{Polskiej filozofii chrześcijańskiej
XX wieku} (ang. \textit{The Polish Christian Philosophy in the 20\textsuperscript{th} Century}.
Stanisław Kamiński), zrealizowanej w~ramach grantu \textit{Pomniki polskiej myśli filozoficznej,
teologicznej i~społecznej XX i~XXI wieku}. Opublikowana monografia stanowi opracowanie najważniejszych obszarów badań, które
wyznaczały drogę rozwoju naukowego Stanisława Kamińskiego (1919-1986), ściśle związanego ze środowiskiem akademickim
Katolickiego Uniwersytetu Lubelskiego. Kamiński pracę dydaktyczną oraz naukową rozpoczął od logiki, stopniowo
przechodząc do ogólnej metodologii nauk oraz metodologii filozofii i~teologii. Podjęty przez niego program badań wyrósł
z~umiejętnego połączenia i~rozwinięcia trzech tradycji: (1)~Szkoły Lwowsko-Warszawskiej~-- przez kontakty i~rozmowy
m.in. z~Kazimierzem Ajdukiewiczem, Tadeuszem Czeżowskim, Izydorą Dąmbską, Tadeuszem Kotarbińskim, Ludwikiem Borkowskim,
(2)~koła krakowskiego~-- przez znajomość prac Jana Salamuchy, Józefa Bocheńskiego, Jana Drewnowskiego,
(3)~tomizmu~-- rozwijanego w~ramach lubelskiej szkoły filozoficznej, do twórców której jest zaliczany.

Monografia składa się z~dziesięciu rozdziałów i~jest opracowana przez czterech autorów, wybitnych znawców filozofii
rodzimej: Tadeusza Szubkę, Kazimierza Marka Wolszę, Gabrielę Besler i~Marka Rembierza. Rozdziały
pierwszy i~trzeci~-- \textit{Biogram Stanisława Kamińskiego} oraz \textit{Stanisław Kamiński: logik,
historyk logiki i~filozof  logiki}~-- opracowała Gabriela Besler.
Marek Rembierz opracował rozdziały drugi i~piąty:  \textit{Jak (powinno się)
rozumieć i~uprawiać filozofię? O~koncepcji filozofii }\textit{wypracowanej przez Stanisława Kamińskiego} oraz \textit{Zagadnienia
metafilozoficzne i~kwestia kształtowania samoświadomości filozofii w~dociekaniach Stanisława Kamińskiego}. Rozdziały
czwarty i~dziewiąty~-- \textit{Metodologia i~filozofia w~ujęciu Stanisława Kamińskiego} oraz  \textit{Biografia
Stanisława Kamińskiego}~-- opracował natomiast Tadeusz Szubka. Z~kolei Kazimierz Marek Wolsza jest autorem rozdziałów
szóstego, siódmego i~ósmego: \textit{Udział Stanisława Kamińskiego w~dyskusjach filozoficznych}, \textit{Wpływ
Stanisława Kamińskiego na środowisko filozoficzne i~pozafilozoficzne} oraz \textit{Słownik podstawowych terminów
Stanisława Kamińskiego}. 

Całość monografii dopełnia rozdział dziesiąty: \textit{Wybór pism Stanisława Kamińskiego}, w~którym przedrukowano cztery
artykuły z~pracy Kamińskiego \textit{Jak filozofować? Studia z~metodologii filozofii klasycznej}, które w~niniejszej
edycji przygotował Tadeusz Szubka.

Monografia wieloautorska jest całościowym opracowaniem najważniejszych osiągnięć myśli filozoficznej i~metafilozoficznej
Stanisława Kamińskiego. Po zaprezentowaniu biogramu S. Kamińskiego, Gabriela Besler w~trzecim rozdziale przechodzi do
omówienia jego początkowych obszarów zainteresowań oraz działalności dydaktycznej dotyczących logiki jako narzędzia
badań filozoficznych. W~sposób przejrzysty i~wielowątkowy autorka przedstawia główne rezultaty prac
Kamińskiego z~zakresu semiotyki logicznej, sylogistyki, a~w~sposób szczególny zwraca uwagę na problematykę supozycji terminów oraz
wyszczególnienie i~uporządkowanie błędów (nie tylko logicznych). Kamiński dużo miejsca w~pracach poświęcił również
logice G. Fregego (ukazując jej aktualność i~wyjątkowość), a~także problematyce definicji oraz rodzajów i~podziału
rozumowań. Jak zauważa Besler pod koniec lat pięćdziesiątych obszar zainteresowań Kamińskiego przesunął się w~stronę
badań metodologicznych, co jest ukazane przez autorów kolejnych rozdziałów pracy. Należy stwierdzić, że Gabriela Besler
problematykę logiki i~zagadnienia logiczne prezentuje z~wielkim znawstwem tematyki badań Kamińskiego. 

Ważnym obszarem zainteresowań Kamińskiego była metodologia. Problematyka ta jest opracowana przez Tadeusza
Szubkę w~rozdziale czwartym. Autor z~niezwykłą swobodą wprowadza kolejne, szczegółowe zagadnienia tworząc
usystematyzowany i~spójny przegląd dorobku badań metodologicznych Kamińskiego. Niezwykle cenne są Tadeusza Szubki
uwagi poszerzające
horyzont spojrzenia na osiągnięcia lubelskiego filozofa. Ukazują one z~jednej strony trwałość jego osiągnięć
dotyczących czynności wiedzotwórczych i~logicznych aspektów nauki, ale z~drugiej~-- możliwość ich
korygowania i~uzupełniania o~tradycyjną epistemologię, metafizykę, historię nauki i~socjologię wiedzy.
Uwagi te są przestrogą przed
izolowaniem się od rozwoju danej dyscypliny wiedzy.

W rozdziale piątym Marek Rembierz podjął udaną próbę usystematyzowania poglądów metafilozoficznych Kamińskiego. Trafny
dobór fragmentów wypowiedzi Kamińskiego umożliwił mu wieloaspektowe przedyskutowanie podstawowych zagadnień, które
stanowią ogólne tło dla przedmiotowego uprawiania filozofii w~tzw. szkole lubelskiej. Marek Rembierz bardzo wyraźnie
podkreśla wpływ badań epistemologiczno-metodologicznych oraz metafilozoficznych na kształtowanie ,,samoświadomości
filozofii'' uprawianej przez Kamińskiego. W~opracowaniu tej problematyki Rembierz wskazał na ważność dociekań
metafilozoficznych Kamińskiego prowadzonych w~ramach konwersatorium metafilozoficznego, tworzonego przez
przedstawicieli środowiska uniwersyteckiego KUL oraz uczestnictwa w~licznych debatach ,,poza murami'' macierzystej
uczelni. Należy podkreślić, że problematyka poglądów metafilozoficznych Kamińskiego została przez Rembierza
zaprezentowana w~sposób przejrzysty i~usystematyzowany.

Stanisław Kamiński przejawiał dużą aktywność naukową w~środowisku filozoficznym biorąc udział w~licznych debatach
poświęconych m.in. relacji filozofii do nauk przyrodniczych, znaczeniu filozofii dla człowieka czy filozoficznej nauce
o Bogu. Kazimierz Wolsza opracował obszary dyskusji, w~które aktywnie włączał się Kamiński. Wolsza przedstawiając owe
debaty, a~niekiedy i~spory, z~lekkością stylu wypowiedzi niezatracającej ani głębi, ani powagi tematyki, nie tylko
przybliża je czytelnikowi, ale również wprowadza go w~świat żywej rozwijającej się myśli filozoficznej. Jak zauważa
Wolsza, aktywność organizacyjna, naukowa i~dydaktyczna Kamińskiego wywierała zauważalny wpływ na różne kręgi odbiorców
przez m.in. dążenie do kodyfikacji terminologii filozoficznej i~metodologicznej (Wolsza przedstawia tę kwestię
szczegółowo w~rozdziale ósmym) oraz rozszerzenie badań metodologicznych na teologię. Rezultaty tych działań są z~kolei
przenoszone w~świat życia codziennego ucząc kultury logicznej oraz wzbogacając wiedzę o~człowieku i~jego życiu. Wolsza
przyjmując od Andrzeja Bronka podział twórczości Kamińskiego na trzy etapy: niemetafizyczny,
prometafizyczny i~mądrościowy, systematyzuje i~ukazuje ciągłość rozwoju myśli filozofa lubelskiego.
Ważne i~cenne są uwagi Wolszy o
pracach Kamińskiego dotyczących mądrościowego wymiaru wiedzy. 

Poszczególne zagadnienia analizowane w~opublikowanej monografii dają wyczerpujący obraz dorobku Stanisława Kamińskiego.
W rozdziale drugim obszernie przedyskutowano jego koncepcję filozofii, która, z~jednej strony prezentuje wysoką kulturę
logiczno-metodologiczną, a~z~drugiej jest otwarta na problematykę metafizyczną. W~następnych rozdziałach przedstawiono,
zgodnie z~rozwijanymi, filozoficznymi zainteresowaniami Kamińskiego, zagadnienia dotyczące logiki i~problematyki
logicznej, szczegółowe kwestie dotyczące metodologii i~filozofii nauki oraz te kwestie, w~których Kamiński dał się
poznać jako aktywny współtwórca polskiego środowiska filozoficznego.

W monografii ukazano, iż miernikiem wartości myśli Kamińskiego jest prawda, do której myśl owa zmierza. Dążenie do
obiektywnej prawdziwości nie może jednak dokonać się bez pogłębianej refleksji dotyczącej metod i~środków, za których
pomocą umysł wznosi się do prawdy. Kamiński dbał, aby dochodzenie do niej dokonywało się w~ścisłym związku
metody z~przedmiotem. W~tej perspektywie daje się zrozumieć tok myślenia filozofa lubelskiego. Autorzy monografii z~dużym
znawstwem filozofii Kamińskiego i~erudycją wywodów przedstawili jego życie intelektualne dokonując próby nadania
kształtu jego myślowej konstrukcji.

Monografia opracowana przez znawców rodzimej filozofii ukazuje liczne walory poczynając od przejrzystego i~komunikatywnego,
nawet dla nie-filozofów, stylu wypowiedzi, a~kończąc na niebywałej erudycji i~twórczym (a nie
odtwórczym) spojrzeniu na dorobek filozoficzny i~metafilozoficzny Stanisława Kamińskiego.

Treść niniejszej monografii zachęca do kontunuowania badań dotyczących rodzimej myśli filozoficznej, ale także do
pogłębienia badań dotyczących twórczości Kamińskiego. Tym bardziej, że autorzy niniejszego opracowania zarysu całości
twórczości filozofa lubelskiego nie mieli możliwości, przy tak ograniczonej objętości tomu, zaprezentowania
szczegółowych kwestii, w~których najpełniej można byłoby dostrzec połączenie twórczej myśli z~wysoką kulturą
logiczno-metodologiczną prowadzonego wywodu. W~prezentacji myśli filozoficznej Kamińskiego cenne jest również to,
że w~monografii zachowano indywidualny styl i~język wypowiedzi każdego z~jej współautorów. 

Opublikowana praca jest świadectwem bogactwa rodzimej myśli filozoficznej. Jej treść zachęca także nie-filozofów do
zapoznania się z~często marginalizowanym dorobkiem polskiej humanistyki. Często formułowanym pytaniem jest to, na ile
rodzima myśl wpisuje się w~ogólnoświatowe dążenia do poznania prawdy o~człowieku i~o~otaczającej go rzeczywistości.
Odpowiedzią na to pytanie niech będzie publikacja angielskojęzycznej wersji omawianej pracy.

\autorrec{Krzysztof Śleziński}

\end{recplenv}
