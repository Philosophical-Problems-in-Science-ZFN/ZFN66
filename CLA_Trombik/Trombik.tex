\begin{artengenv}{Kamil Trombik}
	{The origin and development of the Center for Interdisciplinary Studies. A historical outline by 1993}
	{The origin and development of the Center for Interdisciplinary\ldots}
	{The origin and development of the Center for Interdisciplinary Studies.\\A historical outline by 1993}
	{Pontifical\label{trombik-start} University of John Paul II in Kraków\footnote{The author obtained funds for the
			preparation a doctoral dissertation. Source of funding is the Own Scholarship Fund of The Pontifical University of John
			Paul II in Kraków.}}
	{The paper concerns the origin and early stage of development of the Center for Interdisciplinary Studies at the
		Pontifical Academy of Theology in Kraków. Center for Interdisciplinary Studies was founded by Michał Heller and Józef
		Życiński in the late 1970s. It was an informal institution which focused on conducting scientific activity in the area
		of philosophy of nature, relationship between mathematical \& natural sciences and philosophy, history of science, as
		well as relationships between science and religion. In this paper I would like to present how this institution
		developed, I will discuss various forms of its activity and discuss---very generally---what kind of philosophy was promoted
		by M. Heller, J. Życiński as well as their pupils and close associates. An important part of the paper will also concern
		the Center for Interdisciplinary Studies as a unique institution, which has developed---in difficult historical
		period in Poland---philosophical research in the spirit of freedom and respect for the new achievements of science, and
		also promoted interdisciplinary dialogue between scientists and philosophers.}
	{Center for Interdisciplinary Studies, Michał Heller, Józef Życiński, history of Polish philosophy, philosophy of nature, philosophy in science.}




\lettrine[loversize=0.13,lines=2,lraise=-0.05,nindent=3pt,findent=0.2pt]%
{40}{} years ago the first issue of journal \textit{Philosophical Problems in Science
(Zagadnienia Filozoficzne w Nauce)} appeared. Over the past four decades, the magazine has obtained a reputation in Poland,
appearing continuously to the present days. Despite the transformations in science and technology, changing political
realities, and sometimes also debatable attempts to modify scientific standards by academic establishment, the journal
remained the mainstay of independent and creative thinking in the field of philosophy practiced in the context of
modern mathematics and natural sciences. Therefore, there is an opportunity to consider the future of the periodical,
and also to summarize and at least provide initially outline the development of the Center for Interdisciplinary
Studies (Polish name: Ośrodek Badań Interdyscyplinarnych, OBI), an institution closely related to this
journal.\footnote{Due to the length of this paper, I will provide the outline of the early period of activity of the
Center for Interdisciplinary Studies, illustrate the history of this institution until 1993. The regular organization
of interdisciplinary conversations in the form in which they were held since the end of the 1970s, has been stopped
this year. Due to the fact, that these seminars had a significant impact on the creation and development of the Center
for Interdisciplinary Studies (as will be discussed later in this article), it seems appropriate to adopt this date,
symbolically closing the early stage of the Center for Interdisciplinary Studies activity.} Regular
publication of \textit{Philosophical Problems in Science (Zagadnienia Filozoficzne w
Nauce)} was in fact one of the key factors of activity of this institution, and the content of
individual numbers testified to the area of research interests of the members of the Center, with the founders: Michał
Heller and Józef Życiński.

Center for Interdisciplinary Studies operated at the Faculty of Philosophy at the Pontifical Academy of Theology (PAT)
in Kraków. PAT was an ecclesiastical university established by pope John Paul II (\textit{motu proprio} ``Beata
Hedvigis'', 1981). In fact, this academic institution derived from the former Faculty of Theology of the Jagiellonian
University, erected in 1397, and formally liquidated by a unilateral decision of the communist authorities of the
Polish People's Republic in 1954. Despite the decision of the communist government, intellectual traditions in Kraków
have survived, as a result of which the former Faculty of Theology has not stopped trying to regain its rights and
conducting autonomous activities based on catholic church law. Over the years it has resulted, \textit{inter alia}, in
structural changes within the department, caused by the gradual development of the group of philosophers. The efforts
of catholic philosophers brought desirable effects, which resulted in the creation of the Faculty of Philosophy,
established in 1976, but for various reasons---mainly of a political nature---functioning as a separate scientific unit
exclusively from 1981.

The Center for Interdisciplinary Studies was not established as a completely autonomous and ideologically independent
institution devoid of references to important intellectual traditions---the origin and initial development of Center can
even be considered in the context of the specificity of the Kraków scientific milieu, including the philosophers
associated with the Faculty of Theology of the Jagiellonian University. So, before I begin to discuss the first stages
of the Center's development, I will briefly present how the interdisciplinary tradition in Kraków was born and shaped
in the period preceding the initiatives of M. Heller and J. Życiński.

\section{Interdisciplinary traditions in Kraków -- from S.~Pawlicki to K. Wojtyła}

Due to historical reasons, Kraków can be considered as the place of the greatest philosophical
traditions in Poland. These traditions were also cultivated and developed by the Faculty of Theology of the
Jagiellonian University. Christian philosophy was developed for several centuries in this Faculty, although its rapid
development when encyclic ``Aeterni Patris'' by pope Leon XIII was published in 1879. Then, the pioneer of modern
philosophy in Kraków became Stefan Zachariasz Pawlicki, who was the first scholar appointed in 1882 to the head of the
Department of Christian Philosophy at the Faculty of Theology at the Jagiellonian University. Pawlicki turned out to be
one of the most important Polish philosopher of the second half of the
19\textsuperscript{th} century, who tried to develop modern, non-thomistic
philosophy based on the models taken from ancient philosophy and acceptance of the results of modern science 
\parencite{polak_rola_2017}.
%\label{ref:RNDAF5eF7rVZ2}\textcolor{black}{(Polak, 2017b)}\textcolor{black}{.}

Another philosophers of the Department of Christian Philosophy---the neoscholastic philosopher of nature (Franciszek
Gabryl), historian of the philosophy (Konstanty Michalski) or famous logician, representative of the
Cracow Circle (Jan Salamucha), laid the foundations for the philosophy trying to show a coherent image of the broadly
understood modern culture and faith. Parallel to their activities, Kraków's philosophy also developed among
non-scholastic scientists and philosophers. These scientists were interested in philosophy, although many scholars at
that time shared the opinion of the positivists, questioning the cognitive value of the philosophical reflection on
nature. Meanwhile, in Poland, especially in Kraków, even in the interwar period (1918-1939), the philosophy of nature
had its representatives who addressed specific issues for it, e.g. during the Polish Philosophical Conventions (Polskie
Zjazdy Filozoficzne). It should be noted, however, that the Kraków milieu---starting from the initiatives of Władysław
Heinrich and Maurucy Straszewski---stressed the development of interdisciplinary dialogue between scientists and
philosophers. Philosophical issues were mainly taken up in the context of formal sciences, physics, biology and
psychology. Concepts and propositions developed at that time are considered to be crucial in the development of the
Kraków philosophy of nature
\parencite{heller_krakowska_2007,dziedzic_historia_2007,polak_tradycja_2018}.
%\label{ref:RNDtc2qUrfP3e}(Heller and Mączka, 2007; Mączka, 2007; Polak, 2018b).

After the Second World War (1945), Polish philosophers faced the challenges of Marxist ideology. In that difficult
period of the Polish history, Kazimierz Kłósak and Tadeusz Wojciechowski were active in the field of philosophy of
nature in Kraków\footnote{Both, Kłósak and Wojciechowski were philosophers from Faculty of Theology in Kraków. }. They
tried to develop the philosophy of nature in the neoscholastic context (methods, terminology etc.). However, they
differed from orthodox thomists, because they attached more importance to the achievements of natural science, trying
to reconcile new scientific theories with classical philosophy. As a result, they become key representatives of ``open
thomism'' in Poland.

The intellectual efforts of these philosophers coincided with the activity of Karol Wojtyła in Kraków. Already in the
1950s, Wojtyła readily participated in informal discussions among physicists from the Jagiellonian University. Over
time, these meetings began to take on a more organized form, which favored the exchange of ideas between Kraków's
philosophers and scientists. The year of 1963 was important, because Wojtyła was appointed a Kraków bishop. He quickly
began to develop the local milieu, initiating a series of interdisciplinary events. One of this event is worth mentioning
here. For example, the philosophical symposium devoted to the analysis of the kinetic point of departure and the
teleological argument for the existence of God, which took place in the residence of Cardinal K.~Wojtyła in Kraków
(January 10--11, 1968), and in which participated scientists from Faculty of Theology in Kraków, Academy of Catholic
Theology in Warsaw (Akademia Teologii Katolickiej w~Warszawie) and Catholic University of Lublin (Katolicki Uniwersytet
Lubelski), as well as professors of physics and biology from the Jagiellonian
University.\footnote{Extensive report from this event, see
\parencite{morawiec_sympozjum_1968}.
%\label{ref:RNDOKp52VRWKL}(Morawiec, 1968).
} Among the speakers were: K. Kłósak,
J. Iwanicki, S.~Kamiński, M.A. Krąpiec, M.~Lubański, A. Stępień, S.J. Zdybicka, Sz. Ślaga, B. Bejze, et al.. Scientist
were represented, among others, by S.J. Twarowska, L. Balczewski, Z.~Chyliński, J.~Janik.

This meeting of scholars and philosophers was not the only such event in Kraków in this period. Wojtyła initiated further
symposia---thanks to that, philosophy continued to develop in the spirit of interdisciplinary cooperation. One can
mention in this context the symposium, which was held on January 9--10, 1970 in the archbishop's residence of Cardinal
Wojtyła. The participants of the event were professors and lecturers of Catholic universities in Poland and scientists
from the Jagiellonian University, mainly physicists, though there were also philosophers---an active participant was, for
example, Roman Ingarden
\parencite{heller_poczatki_2006}.
%\label{ref:RND0cLhaGNzvH}(Heller and Mączka, 2006).
During the meeting a few topics emerged in
the field of natural philosophy, including metaphilosophical question concerning the way of doing the philosophy in the
context of modern scientific knowledge. The problem of theory of philosophy of nature was discussed, especially the
issue of the language of this discipline in the context of terminology used by physicists. The symposium was
interdisciplinary and constituted another attempt to maintain the Kraków ambience of cooperation between philosophers
and scientists. Cooperation, which did not always bring results in the form of working out a common position of
scientists and philosophers. Michael Heller, who participated in the symposia initiated by Wojtyła, mentioned, for
example, that ``one of the philosophical lectures rose up one of the physicists and said to the speaker: `I don't
understand this'. You must know that in the physicists language the sentence `I don't understand this' means that it is
impossible to understand, that this is nonsense. But the philosopher did not know the habits of physicists and began to
explain again what he meant. It struck me then that the languages spoken by them are so
divergent''
\parencite[pp.227--228]{heller_wierze_2016}.
%\label{ref:RNDltUtn6TtE2}(Heller et al., 2016, pp.227---228).

M. Heller was a young philosopher at the time and he has just begun his academic career in Kraków. Although he was
associated with the local community earlier, he was officially employed at Faculty of Theology in 1972. His activity in
Kraków was also related to the efforts of K. Wojtyła, who tried to develop the scientific and didactic facilities of
Faculty of Theology (since 1974 Pontifical Faculty of Theology) related to the philosophy of nature. Shortly after,
Józef Życiński (K. Kłósak's pupil) became a close associate of M. Heller.

Heller and Życiński got to know each other at the beginning of the 1970s, but they started a more profound cooperation a
few years later, when Życiński was preparing his doctoral dissertation
\parencite[p.230]{heller_wierze_2016}.
%\label{ref:RNDBdKmRAvZPP}(Heller et al., 2016, p.230).
Heller was in that time a philosopher with considerable scientific achievements, who was not interested in
developing the traditional, neoscholastic philosophy of nature. In contrast to the thomists, he emphasized to identify
the philosophical problems in the sciences and then analyze them using methods of contemporary logic and science. He
was looking for inspiration to work in the field of philosophy in natural science, not in closed philosophical systems
\parencite{polak_philosophy_2019}.
%\label{ref:RND6aWNzFbaUi}(Polak, 2019).
It is not excluded that the space of dialogue and freedom of scientific
research promoted by Wojtyła favored the development of the philosophical views of Heller and Życiński, who wanted to
maintain a dialogue between science and philosophy. In such circumstances, the idea of organizing interdisciplinary
seminars arose.

\section{Development of the Center for Interdisciplinary Studies}

Interdisciplinary seminars were initiated by M. Heller and J. Życiński in the autumn of 1978. These meetings between
philosophers and scholars inaugurated the activity of the Center for Interdisciplinary Studies, which aimed to conduct
research in the field of philosophy of nature, philosophy and history of science, the relationship between science and
religion, etc. The seminars organized by M. Heller and J.~Życiński in Kraków referred to the earlier initiatives of K.
Wojtyła.

Starting from the first seminar on October 27, 1978, many scientists participated in the meetings organized by M. Heller
and J.~Życiński---not only philosophers and theologians, but also biologists, physicists or mathematicians---who discussed
problems arises on the borderline of philosophy and science, faith and art. Speakers and listeners met initially at
Franciszkańska 3 Street on the first Friday after the fifteenth day of each month, and when the group of participants
began to grow (lectures were heard by up to 200 people sometimes), meetings were moved to the Augustinian monastery at
Augustiańska street. Despite the poor conditions, lack of financial resources and official statutes, and at the same
time against the reluctance of state authorities to the Church, the philosophical milieu at Pontifical Faculty of
Theology developed extremely well, with its initiatives acting as a ``clear primacy of the spirit over matter''
\parencite[p.13]{skoczny_spotkania_1999}.
%\label{ref:RNDqof40Fu51L}(Skoczny, 1999, p.13).
According to Włodzimierz Skoczny, a deponent of those meetings, ``the
ability to not notice shortcomings, and only to notice the positives has been perfected by the participants. Indeed, it
is difficult today to reflect the unique atmosphere of those days. Something imperceptible hovered in the air,
something that had a taste of self-denial, friendship and Truth''
\parencites[p.15]{skoczny_spotkania_1999}[see also][]{zycinski_kartki_1999}.
%\label{ref:RNDHDyldo4CpP}(Skoczny, 1999, p.15; see also Życiński, 1999).


The seminars were organized in two thematic series: ``Science---Faith'' (1978--1991 and 1992--1993) and
``Science---Philosophy---Art'' (1983--1985). The first meeting in October 1978 was opened by M.~Heller and J.
Życiński.\footnote{M. Heller gave a lecture entitled „The problem of extrapolation in theology'' („Problem
ekstrapolacji w teologii''), in turn J. Życiński---``Contemporary tendencies in the philosophy of science'' („Współczesne
tendencje w filozofii nauki''), see
\parencite[p.133]{liana_z_1999}.
%\label{ref:RNDPh1P7nkBwe}(Liana and Mączka, 1999, p.133).
} Over time, the seminars
began to gather speakers representing other scientific centers, also from outside Poland.\footnote{E.g. Ch. W. Misner,
J. Dougherty, W. Greenberg (USA), E. Barth (Netherlands), M.A. McCallum (Great Britian), L. Michel (France).} In the
mid of 1980s, the meetings organized by M. Heller and J. Życiński were permanently included in the calendar of
important scientific initiatives in the country. The lectures were given primarily by physicists and astronomers, among
others: Andrzej Staruszkiewicz, Konrad Rudnicki, Zygmunt Chyliński, Jerzy Rayski, Carl Friedrich von
Weizsäcker, Bronisław Średniawa, Jan Mozrzymas,
Jerzy Janik, Charles W. Misner, Andrzej Fuliński, Leszek M. Sokołowski, Małgorzata Głódź. Despite the significant
advantage of papers in the field of physics and the philosophy of inanimate nature, as part of the seminars lectures
were also delivered by mathematicians (including Krzysztof Maurin, Stanisław Krajewski), chemists (Zbigniew Grabowski)
and philosophers (including Jan Woleński, Stefan Amsterdamski, Barbara Tuchańska). The subject matter of the
presentations coincided with the interests of M. Heller and J. Życiński: lectures devoted to philosophical problems in
physics, relationship between science and religion, fundamental problems of the philosophy of mathematics, there were
also presentations addressing contemporary issues of the philosophy of science. Philosophy of biology generally played
a smaller role---over several years only a few papers devoted to the topic of the origin and evolution of life were
declaimed.\footnote{It is worth mentioning at this point that also the seminar entitled ``Science---Philosophy---Art''
(although some sources give a different title: ``Art---Religion---Science'', initiated in May 1983, was very popular. The
first session was attended by 130 participants. It is noteworthy that scientists such as K. Maurin and Z. Chyliński
also declaimed their papers as part of the seminars. }

In later years, seminars were held with less regularity, however, still attracted many interested scholars who could
participate in discussions on current problems occurring at the borders of science and philosophy or science and
religion. Guests representing other research centers from Poland and abroad continued to appear (including A.
Plantinga, S. Desjardinis, L. Kostro, and E. Mickiewicz), which allowed to strengthen Center's position on the national
and international arena. In 1990, the continuation of the seminars was stopped for some time, and then returned in
1992--1993 at Jagiellonian University. At that time, several meetings took place, initiated by M. Heller and the
physicist and philosopher Andrzej Fuliński. After 1993, the idea of interdisciplinary meetings was no longer carried
out in the same scope as before, and the role of seminars was taken over by the Methodological Conferences, organized
annually.

The seminars organized by M. Heller and J. Życiński resulted in the creation of the periodical \textit{Zagadnienia
Filozoficzne w Nauce} (now entitled \textit{Philosophical Problems in Science})---the first strictly
philosophical journal published by Pontifical Faculty of Theology in Kraków. The first issue of this journal appeared
in 1979 and contained eight articles. The journal, edited by M. Heller and J. Życiński as a yearbook, initially
contained primarily materials delivered as part of the seminars, although content independent of the
Kraków's meetings was also included.\footnote{Publishing of this periodical was initiated by M. Heller together with J. Życiński for two
reasons. First, both philosophers wanted to create their own independent journal on philosophy and science. The second
reason was practical---people who delivered papers as part of interdisciplinary seminars gave M. Heller and J. Życiński
texts that formed the basis of their lecture. There was, therefore, a need to publish the texts of the speeches,
although it was not possible to obtain official permission from the authorities to publish the periodical.
\textit{Philosophical Problems in Science (Zagadnienia Filozoficzne w Nauce)} appeared, therefore, as \textit{samizdat}
\parencite{heller_interview_2017}.
%\label{ref:RND3qsUM2PGth}(see Heller, 2017).
\textit{Samizdat} (Russian origin word, coined from \textit{samodielnoje
izdatielstwo} or \textit{sam izdaju}, means ``self-publish'') was a clandestine print, published as a form of dissident
activity in countries where communist censorship was active. Samizdats were non legal and not allowed to be
distributed, therefore individuals reproduced underground publications and passed them from person to person. } From
the first issue, the journal was focused on publishing papers from the frontiers of philosophy, formal and natural
sciences. An important part of the magazine were reviews of the scientific books, those published in Poland and abroad.
Philosophers and representatives of other scientific disciplines, interested in broadly understood philosophical
issues, published in this journal. Already in the first issues of the magazine, there were also translations of
well-known scholars, including A. Einstein or K.R. Popper. The scope of the journal reflected the research interests
and approach to practicing philosophy that began to emerge in Kraków in the vicinity of M. Heller and J. Życiński. This
is especially visible in relation to other periodicals in the field of natural philosophy issued by catholic centers in
Poland---in relation to the series \textit{Z zagadnień filozofii przyrodoznawstwa i filozofii przyrody} (Warsaw)
published by K. Kłósak, or journal \textit{Roczniki Filozoficzne }(Lublin), the Kraków periodical distinguished a
greater focus on contemporary philosophy of science (especially the philosophy of physics) and a departure from taking
classical problems in philosophy in the context of the methods and language of the neo-thomist philosophy of nature.

M. Heller and J. Życiński were focused on developing an interdisciplinary milieu also outside the Poland. An expression
of progressing internationalization was creation of the English-language equivalent of \textit{Philosophical Problems
in Science (Zagadnienia Filozoficzne w Nauce) }published in Polish in that time. From 1983, the Center for
Interdisciplinary Studies began publishing the journal \textit{Philosophy in Science} in cooperation with the Vatican
Observatory and the Pachart Publishing House in Tucson.\footnote{It is noteworthy that on the front-page of the
magazine the name ``Center for Interdisciplinary Studies'' was mentioned as the organization initiating the publication
of this periodical. This proves that the Center for Interdisciplinary Studies was already an organization operating in
the field of science and cooperating with other research institutions.} The editors of the newly created periodical
were M. Heller (affiliated to the PAT and the Vatican Observatory), J. Życiński (PAT) and William R.
Stoeger SJ (Vatican Observatory). The publisher of the magazine was a Polish astronomer living in the USA Andrzej G.
Pacholczyk (University of Arizona), founder and chairman of the Pachart Foundation and director of Pachart Publishing
House.\footnote{A few years later, the editors of \textit{Philosophy in Science} started to publish the book series
\textit{Philosophy in Science Library}, in which philosophical works of M. Heller, A.G. Pacholczyk, Józef Życiński and
G. Tanzella-Nitti were published in English.}

The aims of the editors of this periodical were convergent with the philosophical program of M. Heller and J. Życiński.
The main task was to develop an interdisciplinary dialogue between science and philosophy, especially the contemporary
philosophy of science
\parencite[p.8]{heller_introduction_1983}.
%\label{ref:RNDntZQ2V9XCR}\textstyleFootnoteSymbol{(Heller, Stoeger and }Życiński, 1983, p.8).
The
journal was conceived as a forum for discussion of philosophical issues emerging in the natural sciences. The article
that opened the first issue of the magazine was W. G. Stoeger's text entitled ``The Evolving Interaction Between
Philosophy and the Sciences: Towards a Self-Critical Philosophy'', in which the author argued that contemporary philosophy
must be internally open to changes generated by scientific achievements. In this sense, the constant nature of
self-criticism should be inscribed in the nature of philosophy
\parencite[pp.39--43]{stoeger_evolving_1983}.
%\label{ref:RNDbpNJI9xDEH}\textstyleFootnoteSymbol{(Stoeger, 1983, pp.39}---43).
Interestingly, this view was supposed to
be a sort of editorial program, and the article itself was considered one of the first attempts at the theory of
``philosophy in science''
\parencites[p.7]{heller_jak_1986}[p.\pageref{heller-stoeger}]{heller_how_2019}.
%\label{ref:RNDY3OAZaU25E}(Heller, 1986, p.7, 2019, p.ref11).

The journal \textit{Philosophy in Science} was not the only manifestation of the opening of Kraków philosophers to the
foreign audience.\footnote{Five volumes of this periodical had been published by 1993. Unfortunately, currently the
magazine is not published---the last (10\textsuperscript{th}) issue appeared in 2003.} Although the Center for
Interdisciplinary Studies operated at the Faculty of Philosophy of the Pontifical Academy of Theology in Kraków, M.
Heller and J. Życiński also cooperated with the Vatican Observatory at Castel Gandolfo. Thanks to this cooperation, the
organization of an international session was possible. On May 24--25, 1984, the symposium ``The Galileo
Affair: a Meeting on Faith and Science'' held in English took place in Kraków. Many scientists and philosophers from outside Poland
actively participated in this conference (W.A. Wallace, J. Dietz-Moss, J. Casanovas, G. Coyne, O.~Pedersen, U. Baldini,
F.M. Hetzler, P. Mitra), and also scholars representing domestic universities (apart from M. Heller and J. Życiński,
speeches were given by M. Lubański, J. Dobrzycki, and K. Rudnicki). The symposium was also to be attended by P. K.
Feyerabend, but finally he did not appear in Kraków.\footnote{P. K. Feyerabend was represented in Kraków by his
co-worker I. Sieb-Madeja, who recreated the contents of a paper by an Austrian philosopher from a tape recorder.}
Individual sessions were held in the archbishop's palace, made available by Cardinal Franciszek Macharski, as well as
in the rooms of the Faculty of Philosophy of the PAT at Augustiańska street and the Collegium Maius at Jagiellonian
University
\parencite{skoczny_sympozjum_1984}.
%\label{ref:RNDOiQxouM3UQ}(Skoczny, 1984).

It should be noted that the symposium was organized in connection with the anniversary of the Galilean trial. For
philosophers of nature from the PAT, it was also an opportunity to take up issues that coincided with the interests of
the local environment---the history of science and the issue of the relationship between science and faith are key
research areas of philosophers and scientists from the circles of M. Heller and J. Życiński. Finally, the symposium was
a great opportunity to manifest the existence on the international academic map of PAT as a scientific center in which
advanced research in the field of philosophy is conducted. The event turned out to be an organizational success, which
was confirmed by W. Skoczny, writing about the atmosphere of ``full care of the hosts and unconcealed admiration from
foreign guests. For many of them it was the very first encounter with Poland, which---although poor in material means---was able
to fascinate people''
\parencite[p.73]{skoczny_sympozjum_1984}.
%\label{ref:RNDQCLaBvaoTK}(Skoczny, 1984, p.73).
However, since the symposium was organized by an
illegal university---from the point of view of the Polish authorities at the time---post-conference materials could
only appear as samizdat, in this case issued in cooperation with Specola Vaticana
\parencite{coyne_galileo_1985}.
%\label{ref:RNDwqsqpPrSpR}(Coyne, Heller and Życiński, 1985).

In the following years, the Center for Interdisciplinary Studies organized many scientific conferences, as well as
national and international symposia. In the years 1987--1993 Center organized: international symposium ``Newton and the
New Direction in Science'' (1987) , the conference ``Why is Nature Mathematical?'' (``Dlaczego przyroda jest
matematyczna?'', 1989), the international conference ``Universals and Particulars in the Context of Modern Science''
(1990), conference ``Relationships Between Science and Religion in Catechesis. Problems of Biological Evolution''
(``Relacja nauka---wiara w katechezie. Problematyka ewolucji'', 1991), international conference ``Theology, Philosophy and
Cosmology: On West and East ``(1991), symposium ``Cosmos and Philosophy'' (``Kosmos i filozofia'', 1992), scientific session
``Theology and Science from Antiquity to the Renaissance'' (``Teologia a nauki od starożytności do renesansu'', 1993). Many
scholars outside Poland participated in these events, like W.B.~Dries, S. Jaki, J.D. Moss, M. Novak, D. Park,
O.~Pedersen. The associates of M. Heller and J. Życiński actively participated in the organization of scientific meetings:
W. Skoczny, A. Michalik, Z. Liana, J. Mączka, A. Fuliński. Post-conference materials appeared after almost every
scientific event organized by Center.

Surprisingly, despite the difficulties caused by the government, activity in the field of publishing turned out to be
an important element of the Center for Interdisciplinary Studies. M. Heller and J. Życiński also cooperated in this
field and published common books in early 1980s in the Polish Theological Society (Polskie Towarzystwo Teologiczne) in
Kraków
\parencites{heller_wszechswiat_1980,heller_drogi_1983}\footnote{%
%	\label{ref:RNDTf5nBbodTA}(Heller and Życiński, 1980, 1983).
M. Heller commented the way of making
common books: ``The method of writing was such that we first made a rough plan of the book and then shared it---this
chapter is written by me, this one by you. We didn't write together, but after we wrote the chapters, we read them to
each other and made minor adjustments. When you take our books in your hand, you can immediately see who wrote which
chapter, because the style of each of us is different
\parencite[p.231]{heller_wierze_2016}.
%\label{ref:RNDN8kH0OpdBK}(Heller et al., 2016, p.231).
}. Although
the beginnings were modest, in later years the publishing activity of Center was much greater---books and scripts signed
with the name of this research institution or issued in cooperation with other publishing houses were published
regularly every year
\parencite[e.g.][]{heller_filozofowac_1987,coyne_newton_1988,heller_matematycznosc_1990,%
	mcmullin_ewolucja_1990,heller_spor_1991,wolak_neotomizm_1993}.
%\label{ref:RND0fapg56Vka}(e.g. Heller, Michalik and Życiński, 1987; Coyne, Heller and Życiński,
%1988; Heller, Życiński and Michalik, 1990; McMullin and Rodzeń, 1990; Heller, Skoczny and Życiński, 1991; Wolak, 1993).

The early process of development the new approach to practicing the philosophy of nature in PAT was also related to the
scientific activity of students of M. Heller and J. Życiński. The diploma theses of their students concerned the topics
related to the research interests of the creators of the Center, i.e. contemporary problems of philosophy of nature,
philosophy of science, philosophy of language and the relationship between science and religion.\footnote{Various
aspects of the Center research activities have already been discussed
\parencite[see e.g.][]{krauze_jedna_2008,maczka_badania_2012,skoczny_filozofia_2012}.
%\label{ref:RNDp1PNQqixjX}(see e.g. Krauze, 2008; Mączka et al., 2012; Skoczny, 2012).
} It is significant that the first doctorate defended at the Faculty of Philosophy
of the PAT in January 1983 was prepared under the supervision of M. Heller. The dissertation entitled ``Strukturalne
relacje między językiem, myśleniem a rzeczywistością'' (``Structural relations between language, thinking and reality'')
prepared by Krzysztof Turek received positive reviews of J. Życiński and M. Lubański, and its author became the first
doctor appointed by the authorities of the newly established university.\footnote{PhD defense was preceded by the
publication of two articles in \textit{Philosophical Problems in Science (Zagadnienia Filozoficzne w Nauce)}:
\parencite{turek_filozoficzne_1978,turek_rozwazania_1981}.
%\label{ref:RNDKeSlQokX67}(Turek, 1978, 1981).
Recently, these works met with interest and were analyzed
\parencite{krzanowski_towards_2016}.
%\label{ref:RNDyOBxtnZucT}(Krzanowski, 2016).
} In the initial period of the Faculty of Philosophy of the PAT, the
bachelor's thesis---based on the work ``Argument kinetyczny za istnieniem Boga w ujęciu K. Kłósaka'' (``Kinetic argument for
the existence of God in K. Kłósak's concept'')---was defended in turn by W.~Skoczny, a student of J. Życiński. The interest in the
problems in philosophy of physics was continued by W. Skoczny in his doctoral dissertation from 1986, entitled
``Filozoficzne aspekty zasady antropicznej'' (``Philosophical aspects of the anthropic principle'').\footnote{W. Skoczny's
article was related to the subject of his doctoral dissertation
\parencite{skoczny_glowne_1985}.
%\label{ref:RNDrMO3pWFrGD}(Skoczny, 1985).
} In the same
year, Z. Liana graduated the title of MA in philosophy\footnote{The Master's thesis prepared by Z. Liana was entitled:
``Rola matematyki w poznaniu naukowym w ujęciu Rene Thoma'' (``The role of mathematics in scientific knowledge in the
thought of Rene Thom'').}, and the canonical licentiate was obtained by J. Dadaczyński\footnote{Canonical
licentiate was awarded owing to dissertation entitled ``Problem racjonalności teizmu chrześcijańskiego w \textit{Neues
Glaubensbuch}'' (``The problem of rationality of Christian theism in \textit{Neues Glaubensbuch}''). Paper related to this
dissertation see
\parencite{dadaczynski_poznanie_1984}.
%\label{ref:RNDw9cxs50dsK}(Dadaczyński, 1984).
}---another students of J. Życiński, who will become
permanently associated with Center and PAT in the following years.\footnote{It is worth mentioning here that the
academic degrees awarded by the PAT were not acclaimed by the state authorities at this time. It happened that doctoral
students defended dissertations at other universities. For example: Jan Woleński, then associated with the Jagiellonian
University, became the thesis supervisor of Krzysztof Gurba, who defended his dissertation entitled ``Methodological aspects of
the representation of linguistic knowledge'' in 1986. This dissertation was created at the seminar of J.~Życiński,
although his defense took place at the Jagiellonian University
\parencite{wolenski_interview_2017}.
%\label{ref:RNDFdEunmKTKD}(Woleński, 2017).
} Over the
years, a group of students gathered around M. Heller and J. Życiński began to grow.\footnote{It is worth mentioning
that students and collaborators of M. Heller and J. Życiński undertook many initiatives to build unity and develop the
scientific community in Kraków, e.g. they prepared reports on the activities of interdisciplinary seminars, discussed
important academic events or undertook initiatives aimed at reaching agreement between various currents philosophies
practised at the Pontifical Academy of Theology, see e.g.
\parencite{michalik_wstep_1984,glodz_miedzynarodowe_1987,samborski_intuicjonizm_1987,liana_w_1989,%
	dembek_matematycznosc_1990,wolak_interdyscyplinarnosc_1992,samborski_na_1992,wolak_neotomizm_1993}.
%\label{ref:RNDV2Ic4V5oZj}(Michalik, 1984; Głódź, 1987;
%Samborski and Wójcik, 1987; Liana, 1989; Dembek, 1990; Wolak, 1992; Samborski and Wójcik, 1992; Wolak, 1992).
} Academic
degrees were soon obtained, among others, J. Dembek, J. Mączka, A. Olszewski, T. Sierotowicz, Z. Wolak, W. Wójcik.

The early development of the Kraków philosophical circle reminds, in some respects, the early stage of the Lvov milieu
around Kazimierz Twardowski, from which the Lvov-Warsaw school emerged. Why do I make such an analogy? As in the case
of the Lvov{}-Warsaw school, one can point to the factors determining the intellectual formation of the trend, which I
would call initially the Kraków school of ``philosophy in science'': the genetic factor (activities of M. Heller and J.
Życiński), geographical factor (scientific activity in Kraków), temporary (development in the 1980s and 1990s) and
substantive\footnote{According to Woleński, all these four factors (genetic, geographical, temporal and factual) were
to determine the intellectual formation of the Lvov-Warsaw school
\parencite{wolenski_filozoficzna_1985}.
%\label{ref:RNDOFxwgkR1Zb}(Woleński, 1985).
It is
debatable whether these criteria are not too broad, considering also the difficulty of clearly defining what a
``philosophical school'' is.} (although students of M. Heller and J. Życiński conducted research in various fields---such
as philosophy of nature, philosophy of science, logic, philosophy of mathematics, history of science, and Polish
philosophy of nature---it combined their conviction, taken from their teachers, that philosophy should be practiced in a
strict, critical way and in connection with the scientific and methodological knowledge, which at the same time does
not exclude possible references to metaphysics and Christian theology). These criteria for belonging to the
``philosophical school'' obviously do not have to be exhaustive; however they suggest the existence of a specific, still
developing intellectual community in Kraków, which stands out in comparison with other philosophical trends in Poland
and abroad with a specific approach to analyzing philosophical issues that appear in the mathematical and natural
sciences
\parencite{heller_jak_1986,heller_how_2019}.
%\label{ref:RNDyaNTk3a3Od}(Heller, 1986, 2019).

\section{Summary and research perspectives}

The Center for Interdisciplinary Studies played a key role in developing the academic dialogue between science and
philosophy in Poland, and the range of its initiatives extended far beyond the local context. M. Heller and J. Życiński
proved to be not only continuators of intellectual traditions, developed in Kraków from at least the second half of the
nineteenth century
\parencite{polak_19th_2013},
%\label{ref:RND0GG3DgutKJ}(Polak, 2013),
but also proposed their own original scientific attitude,
which involved a number of research initiatives in the area of relations between science and philosophy, as well as
science and religion
\parencite{polak_philosophy_2019}.
%\label{ref:RNDqCSr6hAHOD}(Polak, 2019). Czy na pewno ta pozycja?
Large-scale publishing and organizing activities turned
out to be a unique phenomenon on a national scale, and we can still use the fruits of M. Heller and J.~Życiński's labor
to this day.

It should be noted that the organizational activity of M. Heller and J. Życiński has not yet been sufficiently examined
in terms of historical and philosophical aspects, including the political context and in reference to the activity of
other scientific centers in Poland and abroad. Undertaking such research seems justified---they could show and emphasize
the specificity and originality of Center on the national and international background, while providing answers to the
question of the role played by this institution in upholding a reliable scientific discussion at a time when the ideal
of interdisciplinary cooperation was not yet so common, and in some places---such as Poland---was even treated by the
authorities as undesirable.

Certainly, it is also worth considering whether the history of the Center---along with its later face in the form of the
Copernicus Center for Interdisciplinary Studies---no longer deserves a book study. 40~years of activity in the field of
science, organization and publishing is a extensive material for historical analysis, in which one could also take into
account the large-scale activity of students and associates of M. Heller and J. Życiński. Considering the significance
and scope of the Kraków interdisciplinary environment in Poland and abroad, it would be important not only from the
scientific point of view, but also for the promotion of Kraków as a special place in Poland, where interdisciplinary
intellectual traditions are still successfully nurtured and developed.\footnote{An example of the creative development
of ideas initiated by M. Heller and J. Życiński is, among others, conducting research in the area of the so-called
theology of science
\parencite{maczka_teologia_2015},
%\label{ref:RNDhmsjYwLHIc}(Mączka and Urbańczyk, 2015),
philosophy of computer science
\parencite{polak_computing_2016,polak_current_2017,polak_miedzy_2018,krzanowski_minimal_2017},
%\label{ref:RNDc93TU9FMDO}(Polak, 2016, 2017a, Polish version: 2018a; Krzanowski, 2017),
methodological aspects related
to the language of theology
\parencite{olszewski_negation_2018}
%\label{ref:RNDFqYwOGV4Gx}(Olszewski, 2018)
or evolutionary theology
\parencite{grygiel_what_2018}.
%\label{ref:RNDEOZ5nLNd0h}(Grygiel, 2018).
}


\end{artengenv}\label{trombik-stop}
